\documentclass[a4paper,14pt]{article}
\usepackage{float}
\usepackage{extsizes}
\usepackage{amsmath}
\usepackage{amssymb}
\everymath{\displaystyle}
\usepackage{geometry}
\usepackage{fancyhdr}
\usepackage{multicol}
\usepackage{graphicx}
\usepackage[brazil]{babel}
\usepackage[shortlabels]{enumitem}
\usepackage{cancel}
\columnsep=2cm
\hoffset=0cm
\textwidth=8cm
\setlength{\columnseprule}{.1pt}
\setlength{\columnsep}{2cm}
\renewcommand{\headrulewidth}{0pt}
\geometry{top=1in, bottom=1in, left=0.7in, right=0.5in}

\pagestyle{fancy}
\fancyhf{}
\fancyfoot[C]{\thepage}

\begin{document}
	
	\noindent\textbf{8FMA39~Matemática} 
	
	\begin{center}Valor numérico de um polinômio (Versão estudante)
	\end{center}
	
	\noindent\textbf{Nome:} \underline{\hspace{10cm}}
	\noindent\textbf{Data:} \underline{\hspace{4cm}}
	
	%\section*{Questões de Matemática}
	
    \begin{multicols}{2}
    	\begin{enumerate}
    		\item Seja $P(x) = 5x^2 - 2$. Calcule o valor numérico de $P(x)$ para:
    		\begin{enumerate}[a)]
    			\item $x = 0$\\\\\\\\\\\\\\
    			\item $x = 1$\\\\\\\\\\\\\\
    			\item $x = \frac{\sqrt{5}}{5}$\\\\\\\\\\\\
    			\item $x = \sqrt{2}$\\\\\\\\\\\\\\
    		\end{enumerate}
    	    \item Seja $T(x) = x^3 - 4$. Calcule:
    	    \begin{enumerate}[a)]
    	    	\item $T(0)$\\\\\\\\\\\\\\
    	    	\item $T(1)$\\\\\\\\\\\\\\
    	    	\item $T(\sqrt[3]{4})$\\\\\\\\\\\\\\
    	    	\item $T(\pi)$\\\\\\\\\\\\\\
    	    \end{enumerate}
            \item Seja $A(x) = x^3 - x^2 + x - 2$. Calcule o valor da expressão\\\\
            $\frac{A(0) + A(1)}{A(-1)}$\\\\\\\\\\\\\\\\\\\\\\\\\\\\
            \item Sendo $P(x) = 3x + 4$ e $T(x) = 4x - 5$, calcule $k$ tal que $P(k) = T(k)$.\\\\\\\\\\\\\\\\\\
            \item Seja $P(x) = 9x^2 + 3x + 4.$ Calcule o valor numérico de $P(x)$ para:
            \begin{enumerate}[a)]
            	\item $x = 0$\\\\\\\\\\
            	\item $x = \frac{1}{3}$\\\\\\\\
            	\item $x = -2$\\\\\\\\
            	\item $x = \sqrt{2}$\\\\\\\\
            	\item $x = -1$\\\\\\\\
            \end{enumerate}
            \item Seja $Q(x)$ o polinômio $x-4a$. Sabendo que $Q(3) = -5$, determine $a$.\\\\\\\\\\\\\\\\\\\\\\\\\\\\\\\\
            \item Seja $B(x) = x^3 - 2x^2 - x + 2$. Calcule o valor das expressões a seguir:
            \begin{enumerate}[a)]
            	\item $B(1) - B(-1)$\\\\\\\\\\\\\\\\\\\\\\
            	\item $(B(-1))^2 - (B(0))^2$\\\\\\\\\\\\\\\\\\\\\\
            	\item $\frac{B(2) + B(3) - B(4)}{B(0)}$\\\\\\\\\\\\\\\\\\\\\\\\
            \end{enumerate}
            \item Considere um polinômio $P(x)$ do primeiro grau tal que $P(3x - 1) = x + 1$. Determine $P(-1)-P(4)$.
    	\end{enumerate}
    $~$ \\ $~$ \\ $~$ \\ $~$ \\ $~$ \\ $~$ \\ $~$ \\ $~$ \\ $~$ \\ $~$ \\ $~$ \\ $~$ \\ $~$ \\ $~$ \\ $~$ \\ $~$ \\ $~$ \\ $~$ \\ $~$ \\ $~$ \\ $~$ \\ $~$ \\ $~$ \\ $~$ \\ $~$ \\ $~$ \\ $~$ \\ $~$ \\ $~$ \\ $~$ \\ $~$ \\ $~$ \\ $~$ \\ $~$ \\ $~$ \\ $~$ \\ $~$ \\
    \end{multicols}
\end{document}