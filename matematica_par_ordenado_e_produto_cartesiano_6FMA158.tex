\documentclass[a4paper,14pt]{article}

\usepackage{comment} % Para comentar várias linhas ao mesmo tempo

%matemática
\usepackage{amsmath}
\usepackage{amssymb}

%diagramação
\usepackage{extsizes}
\everymath{\displaystyle}
\usepackage{geometry}
\usepackage{fancyhdr}
\usepackage{multicol}
\usepackage{graphicx}
\usepackage[brazil]{babel}
\usepackage[shortlabels]{enumitem}
\usepackage{cancel}
\usepackage{textcomp}
\usepackage{tcolorbox}

%tabelas
\usepackage{array} % Para melhor formatação de tabelas
\usepackage{longtable}
\usepackage{booktabs}  % Para linhas horizontais mais bonitas
\usepackage{float}   % Para usar o modificador [H]
\usepackage{caption} % Para usar legendas em tabelas
\usepackage{wrapfig} % Para usar tabelas e figuras flutuantes
\usepackage{xcolor} % Para cores do fundo de tabelas
\usepackage{colortbl} % Para cores do fundo de tabelas
\usepackage{upgreek} % Para inserir caracteres gregos

%tikzpicture
\begin{comment}
	\usepackage{tikz}
	\usepackage{scalerel}
	\usepackage{pict2e}
	\usepackage{tkz-euclide}
	\usetikzlibrary{calc}
	\usetikzlibrary{patterns,arrows.meta}
	\usetikzlibrary{shadows}
	\usetikzlibrary{external}
\end{comment}


%pgfplots
\usepackage{pgfplots}
\pgfplotsset{compat=newest}
\usepgfplotslibrary{statistics}
\usepgfplotslibrary{fillbetween}

%colours
\usepackage{xcolor}



\columnsep=2cm
\hoffset=0cm
\textwidth=8cm
\setlength{\columnseprule}{.1pt}
\setlength{\columnsep}{2cm}
\renewcommand{\headrulewidth}{0pt}
\geometry{top=1in, bottom=1in, left=0.7in, right=0.5in}

\pagestyle{fancy}
\fancyhf{}
\fancyfoot[C]{\thepage}

\begin{document}
	
	\noindent\textbf{6FMA158 - Matemática} 
	
	\begin{center}Par ordenado e produto cartesiano (Versão estudante)
	\end{center}
	
	\noindent\textbf{Nome:} \underline{\hspace{10cm}}
	\noindent\textbf{Data:} \underline{\hspace{4cm}}
	
	%\section*{Questões de Matemática}
	
	\begin{multicols}{2}
	    \noindent Representamos por (a, b) ou (a; b) o \textbf{par ordenado $ab$}, onde $a$ é o primeiro elemento, e $b$ o segundo. \\
	    Se $a \neq b$, então $(a; b) \neq (b; a)$. Além disso, $(a; b) = (c; d) \Leftrightarrow a = c$ e $b = d$. \\
	    Dados $A$ e $B$, subconjuntos de um universo $U$, o \textbf{produto cartesiano} $A \times B$ é dado por $A \times B = \{(a; b) : a \in A $ e $b \in B\}$. \\
	    $A \times B$ lê-se $A$ cartesiano $B$, e indicamos $A \times A$ por $A^2$ ($A$ ao quadrado). \\
	    Se $A$ tem $m$ elementos e $B$ tem $n$ elementos, então $A \times B$ tem $m \cdot n$ elementos.
		\noindent\textsubscript{--------------------------------------------------------------------------}
		\begin{enumerate} 
			\item Determine $x$ e $y$ reais:
			\begin{enumerate}[a)]
				\item $(8; x) = (y; -7)$ \\\\\\\\\\\\\\\\
				\item $(x; y) = (3; x)$ \\\\\\\\\\\\\\\\
				\item $(x - 4; 5) = (6; y - 3)$ \\\\\\\\\\\\\\\\\\\\
				\item $(-9; 3x + 2) = (y + 4; 11)$ \\\\\\\\\\\\\\\\\\\\
				\item $(4x + 12; x) = (0; y + 2)$ \newpage
			\end{enumerate}
			\item Dados $A$ e $B$, calcular $A \times B$ e $B \times A$.
			\begin{enumerate}[a)]
				\item $A = \{1, 3\}$ e $B = \{5\}$. \\\\\\\\\\\\\\\\\\\\\\\\
				\item $A = \{1, 2\}$ e $B = \{3, 6\}$. \\\\\\\\\\\\\\\\\\\\\\
				\item $A = \{5, 6, 7\}$ e $B = \varnothing$. \\\\\\\\\\\\\\\\\\\\\\
			\end{enumerate}
			\item Dado $A$, apresente $A^2$.
			\begin{enumerate}[a)]
				\item $A = \varnothing$ \\\\\\\\\\\\\\\\\\\\
				\item $A = \{3, 8\}$ \\\\\\\\\\\\\\\\\\\\
				\item $A = \{3, 4, 5\}$ \newpage
			\end{enumerate}
			\item Se $A$ tem 4 elementos e $B$ tem 5 elementos, quantos elementos têm $A \times B, B \times A, A^2$ e $B^2$? \\\\\\\\\\\\\\\\\\\\\\\\\\\\\\\\\\\\\\\\
			% 57 a 60
			\item Complete: $(a; b) = (c; d) \Leftarrow .....$ . \\\\\\\\\\\\\\\\\\\\\\\\\\\\\\\\
			\item Determine $a$ e $b$:
			\begin{enumerate}[a)]
				\item $(a; 5) = (7; b)$ \\\\\\\\\\\\\\\\\\\\
				\item $(a - 3; b + 2 = (0; 0)$ \\\\\\\\\\\\\\\\\\\\
				\item $(a + 8; b - 6) = (3a; 4b)$ \\\\\\\\\\\\\\\\\\\\
				\item $(a; -1) = (-1; b)$ \newpage
			\end{enumerate}
			\item Dados $A = \{2, 7, 8\}$ e $B = \{1, 3, 6, 9\}$, calcule o número de elementos:
			\begin{enumerate}[a)]
				\item $A \times B$ \\\\\\\\\\\\\\\\\\\\\\\\
				\item $B \times A$ \\\\\\\\\\\\\\\\\\\\\\
				\item $A^2$ \\\\\\\\\\\\\\\\\\\\\\
				\item $B^2$ \\\\\\\\\\\\\\\\\\\\
			\end{enumerate}
			\item Dados $A$ e $B$, calcule o número de elementos de $A \times B$:
			\begin{enumerate}[a)]
				\item $A = \{1, 2\}$ e $B = \{4, 7, 8\}$. \\\\\\\\\\\\\\\\\\\\
				\item $A = \varnothing$ e $B = \{1, 2, 4, 8\}$. \newpage
				\item $A = \{3, 4, 6, 8\}$ e $B = \{1, 6\}$. \\\\\\\\\\\\\\\\\\\\
				\item $A = \{1, 5, 6\}$ e \\ $B = \{2, 4, 5, 7, 9\}$. \\\\\\\\\\\\\\\\\\\\
			\end{enumerate}
		\end{enumerate}
		$~$ \\ $~$ \\ $~$ \\ $~$ \\ $~$ \\ $~$ \\ $~$ \\ $~$ \\ $~$ \\ $~$ \\ $~$ \\ $~$ \\ $~$ \\ $~$ \\ $~$ \\ $~$ \\ $~$ \\ $~$ \\ $~$ \\ $~$ \\ $~$ \\ $~$ \\ $~$ \\ $~$ \\ $~$ \\ $~$ \\ $~$ \\ $~$ \\ $~$ \\ $~$ \\ $~$ \\ $~$ \\ $~$ \\ $~$ \\ $~$ \\ $~$ \\ $~$ \\ $~$ \\ $~$ \\ $~$ \\ $~$ \\ $~$ \\ $~$ \\ $~$ \\ $~$ \\ $~$ \\ $~$ \\ $~$ \\ $~$ \\ $~$ \\ $~$ \\ $~$ \\ $~$ \\ $~$ \\ $~$ \\ $~$ \\ $~$ \\ $~$ \\ $~$
	\end{multicols}
\end{document}