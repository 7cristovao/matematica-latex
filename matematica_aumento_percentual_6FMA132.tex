\documentclass[a4paper,14pt]{article}

\usepackage{comment} % Para comentar várias linhas ao mesmo tempo

%matemática
\usepackage{amsmath}
\usepackage{amssymb}

%diagramação
\usepackage{extsizes}
\everymath{\displaystyle}
\usepackage{geometry}
\usepackage{fancyhdr}
\usepackage{multicol}
\usepackage{graphicx}
\usepackage[brazil]{babel}
\usepackage[shortlabels]{enumitem}
\usepackage{cancel}
\usepackage{textcomp}
\usepackage{tcolorbox}

%tabelas
\usepackage{array} % Para melhor formatação de tabelas
\usepackage{longtable}
\usepackage{booktabs}  % Para linhas horizontais mais bonitas
\usepackage{float}   % Para usar o modificador [H]
\usepackage{caption} % Para usar legendas em tabelas
\usepackage{wrapfig} % Para usar tabelas e figuras flutuantes
\usepackage{xcolor} % Para cores do fundo de tabelas
\usepackage{colortbl} % Para cores do fundo de tabelas

%tikzpicture
\begin{comment}
	\usepackage{tikz}
	\usepackage{scalerel}
	\usepackage{pict2e}
	\usepackage{tkz-euclide}
	\usetikzlibrary{calc}
	\usetikzlibrary{patterns,arrows.meta}
	\usetikzlibrary{shadows}
	\usetikzlibrary{external}
\end{comment}


%pgfplots
\usepackage{pgfplots}
\pgfplotsset{compat=newest}
\usepgfplotslibrary{statistics}
\usepgfplotslibrary{fillbetween}

%colours
\usepackage{xcolor}



\columnsep=2cm
\hoffset=0cm
\textwidth=8cm
\setlength{\columnseprule}{.1pt}
\setlength{\columnsep}{2cm}
\renewcommand{\headrulewidth}{0pt}
\geometry{top=1in, bottom=1in, left=0.7in, right=0.5in}

\pagestyle{fancy}
\fancyhf{}
\fancyfoot[C]{\thepage}

\begin{document}
	
	\noindent\textbf{6FMA132 - Matemática} 
	
	\begin{center}Aumento percentual (Versão estudante)
	\end{center}
	
	\noindent\textbf{Nome:} \underline{\hspace{10cm}}
	\noindent\textbf{Data:} \underline{\hspace{4cm}}
	
	%\section*{Questões de Matemática}
	
	\begin{multicols}{2}
	    \noindent Suponha que um valor $y$ foi acrescido de $x\%$.
	    \begin{itemize}
	    	\item Para sabermos o valor do acréscimo, devemos fazer: \\\\ $\frac{x}{100} \cdot y$ $\longrightarrow$ valor aumentado
	    	\item Para sabermos o valor final, após o acréscimo, devemos fazer: \\\\ $\frac{100 + x}{100} \cdot y$ $\longrightarrow$ valor final
	    \end{itemize}
		\noindent\textsubscript{--------------------------------------------------------------------------}
		\begin{enumerate} 
			\item Um carro custa R\$ 56.000,00. Foi anunciado que esse carro terá um aumento de 2,3\% em seu preço. Quanto ele irá custar? \\\\\\\\\\\\\\\\\\\\\\\\\\\\\\\\\\
			\item Em 2018, havia 72 500 habitantes em uma certa idade. No ano seguinte, calculou-se que houve um aumento percentual de 7\% na população dessa cidade. Qual o número de habitantes em 2019? \\\\\\\\\\\\\\\\\\\\
			\item O preço de um livro em dezembro era R\$ 70,00. Em janeiro, estava custando R\$ 74,20. Em quantos por cento, aumentou seu preço? \newpage
			\item O preço de uma casa era R\$ 400.000,00 em agosto. No mês seguinte, sofreu um aumento de 8\% e, em outubro, um novo aumento de 8\%. Qual o valor da casa no dia 1º de novembro? \\\\\\\\\\\\\\\\\\\\\\\\\\\\\\
			%30 a 32
			\item Uma televisão custava R\$ 1.200,00 e sofreu um aumento de 15\%. Quanto a televisão passou a custar? \\\\\\\\\\\\\\\\\\\\\\\\\\\\\\
			\item Um par de tênis custava R\$ 140,00 e passou a custar R\$ 147,00. Em quantos por cento aumentou seu preço? \\\\\\\\\\\\\\\\\\\\\\\\\\
			\item Um livro sofreu um aumento de 30\% no ano de 2017. No ano seguinte sofreu um novo aumento de 30\%. É possível afirmar que o aumento total nesses dois anos foi de:
			\begin{enumerate}[a)]
				\item 38\%
				\item 47\%
				\item 52\%
				\item 60\%
				\item 69\%
			\end{enumerate}
		\end{enumerate}
		$~$ \\ $~$ \\ $~$ \\ $~$ \\ $~$ \\ $~$ \\ $~$ \\ $~$ \\ $~$ \\ $~$
	\end{multicols}
\end{document}