\documentclass[a4paper,14pt]{article}
\usepackage{float}
\usepackage{extsizes}
\usepackage{amsmath}
\usepackage{amssymb}
\everymath{\displaystyle}
\usepackage{geometry}
\usepackage{fancyhdr}
\usepackage{multicol}
\usepackage{graphicx}
\usepackage[brazil]{babel}
\usepackage[shortlabels]{enumitem}
\usepackage{cancel}
\usepackage{textcomp}
\usepackage{array}
\usepackage{longtable}
\usepackage{booktabs}
\usepackage{float}   % Para usar o modificador [H]

\columnsep=2cm
\hoffset=0cm
\textwidth=8cm
\setlength{\columnseprule}{.1pt}
\setlength{\columnsep}{2cm}
\renewcommand{\headrulewidth}{0pt}
\geometry{top=1in, bottom=1in, left=0.7in, right=0.5in}

\pagestyle{fancy}
\fancyhf{}
\fancyfoot[C]{\thepage}

\begin{document}
	
	\noindent\textbf{8FMA94 - Matemática} 
	
	\begin{center}Usando duas variáveis para resolver alguns problemas (III) (Versão estudante)
	\end{center}
	
	\noindent\textbf{Nome:} \underline{\hspace{10cm}}
	\noindent\textbf{Data:} \underline{\hspace{4cm}}
	
	%\section*{Questões de Matemática}
    \begin{multicols}{2}
    	\begin{enumerate}
			\item Numa loja de miçangas, é possível misturar diversos tipos, e o preço final é dado de acordo com uma pesagem. Maria misturou miçangas do tipo \textit{delica} e do tipo \textit{rocaille} em uma proporção de 4 para 1, respectivamente, e pagou R\$ 107,00 o quilograma. Já Tereza misturou miçangas do tipo \textit{delica} e do tipo \textit{rocaille} em uma proporção de 3 para 2, respectivamente, e pagou R\$ 104,00 o quilograma. Qual é o preço do quilograma da miçanga do tipo \textit{delica} e qual é o preço do tipo \textit{rocaille}? \columnbreak
			\item Um comerciante vende feijão de dois tipos diferentes. Misturando 2 partes do feijão tipo 1 com 4 partes do feijão tipo 2, uma parte da nova mistura custa R\$ 5,00. Misturando 12 partes do feijão tipo 1 com 6 partes do feijão tipo 2, uma parte da nova mistura custa R\$ 7,00. Quanto vale uma parte de cada tipo de feijão? \newpage
			\item Dois amigos lançam uma moeda 30 vezes cada, ganhando 2 pontos cada vez que tiram cara e perdendo 1 ponto cada vez que tiram coroa. Sabendo que, ao final, os dois obtêm, respectivamente, 24 e 12 pontos, qual é a diferença entre o número de vezes que cada um tirou cara?  \columnbreak
			\item Eric e Jade resolveram jogar baralho. Nesse jogo, ambos iniciaram com 50 pontos cada e estabeleceram que toda vez que Eric perdesse uma partida, daria 3 dos seus pontos à Jade, ao passo que toda vez que ganhasse, receberia 2 dos pontos de Jade. Depois de 10 partidas, o número de pontos de Jade era o triplo do de Eric. Quantas partidas Jade ganhou?  \\\\\\\\\\\\\\\\\\\\
			
    	\end{enumerate}
    $~$ \\ $~$ \\ $~$ \\ $~$ \\ $~$ \\ $~$ \\ $~$ \\ $~$ \\ $~$ \\ $~$ \\ $~$ \\ $~$ \\ $~$ \\ $~$ \\ $~$ \\ $~$ \\ $~$ \\ $~$ \\ $~$ \\ $~$
    \end{multicols}
\end{document}