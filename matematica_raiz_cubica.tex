\documentclass[a4paper,14pt]{article}
\usepackage{extsizes}
\usepackage{amsmath}
\usepackage{amssymb}
\everymath{\displaystyle}
\usepackage{geometry}
\usepackage{fancyhdr}
\usepackage{multicol}
\usepackage{graphicx}
\usepackage[brazil]{babel}
\usepackage[shortlabels]{enumitem}
\columnsep=2cm
\hoffset=0cm
\textwidth=8cm
\setlength{\columnseprule}{.1pt}
\setlength{\columnsep}{2cm}
\renewcommand{\headrulewidth}{0pt}
\geometry{top=1in, bottom=1in, left=0.7in, right=0.8in}

\pagestyle{fancy}
\fancyhf{}
\fancyfoot[C]{\thepage}

\begin{document}
	
	\noindent\textbf{EF08MA02-A~-~Matemática} 
	
	\begin{center}
		\textbf{Raiz Cúbica (Versão estudante)}
	\end{center}
	
	\bigskip
	
	\noindent\textbf{Nome:} \underline{\hspace{10cm}}
    \noindent\textbf{Data:} \underline{\hspace{4cm}}
	
	\bigskip
	%\section*{Questões de Matemática}
	
	\begin{multicols}{2}
	\begin{itemize}
		\item Para quaisquer $a$ e $b$, temos:
		\begin{center}
			$a^3 = b \Leftrightarrow \sqrt[3]{b} = a$
		\end{center}
	    \item Para quaisquer $a$, $b$, $\sqrt[3]{\frac{a}{b}} \in \mathbb{Q}$, temos:
	    \begin{center}
	    	$\sqrt[3]{\frac{a}{b}} = \frac{\sqrt[3]{a}}{\sqrt[3]{b}}$
	    \end{center}  
	\end{itemize}
	\begin{enumerate}
		
		\item Com base na informação fornecida, completar:
		\begin{enumerate}[a)]
			\item $2^3 = 8 \Leftrightarrow \sqrt[3]{8} = $
			\item $4^3 = 64 \Leftrightarrow \sqrt[3]{64} = $
			\item $0^3 = 0 \Leftrightarrow \sqrt[3]{0} = $
			\item $1^3 = 1 \Leftrightarrow \sqrt[3]{1} = $
			\item $3^3 = 27 \Leftrightarrow \sqrt[3]{~~~~} = 3$
			\item $5^3 = 125 \Leftrightarrow \sqrt[3]{~~~~} = 5$
	    \end{enumerate}
    	\item Completar:
        \begin{enumerate}[a)]
        	\item $\sqrt[3]{216} = $
        	\item $\sqrt[3]{343} = $
        	\item $\sqrt[3]{512} = $
        	\item $\sqrt[3]{~~~~~~~~~~~~~~} = 0$
        	\item $\sqrt[3]{~~~~~~~~~~~~~~} = 1$
        	\item $\sqrt[3]{~~~~~~~~~~~~~~} = 2$        	
        \end{enumerate}
		\item Com base na informação fornecida, completar:
    	\begin{enumerate}[a)]
    		\item $(-3)^3 = -27 \Leftrightarrow \sqrt[3]{-27} = $
    		\item $(-9)^3 = -729 \Leftrightarrow \sqrt[3]{-729} = $
    		\item $(-5)^3 = -125 \Leftrightarrow \sqrt[3]{-125} = $
    		\item $(-6)^3 = -216 \Leftrightarrow \sqrt[3]{-216} = $
    		\item $(-2)^3$ \\ $= -8 \Leftrightarrow \sqrt[3]{~~~~~~} = -2$
    		\item $(-7)^3$ \\ $= -343 \Leftrightarrow \sqrt[3]{~~~~~~} = -7$
    	\end{enumerate}
        \item Completar:
        \begin{enumerate}[a)]
        	\item $\sqrt[3]{-1} = $
        	\item $\sqrt[3]{-8} = $
        	\item $-\sqrt[3]{27} = $
        	\item $-\sqrt[3]{~~~~~~~~~} = -4$
        	\item $-\sqrt[3]{~~~~~~~~~} = -1$
        	\item $-\sqrt[3]{~~~~~~~~~} = -5$
        	\item $-\sqrt[3]{~~~~~~~~~} = 6$
        	\item $-\sqrt[3]{~~~~~~~~~} = 3$
        	\item $-\sqrt[3]{~~~~~~~~~} = -2$
        	\item $\sqrt[3]{\frac{8}{729}} = $
        	\item $\sqrt[3]{\frac{1}{125}} = $
        	\item $\sqrt[3]{\frac{-27}{343}} = $
        	
        	
        \end{enumerate}
		
		
    \end{enumerate}        
    \end{multicols}    

\end{document}