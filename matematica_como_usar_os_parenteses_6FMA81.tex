\documentclass[a4paper,14pt]{article}

\usepackage{comment} % Para comentar várias linhas ao mesmo tempo

%matemática
\usepackage{amsmath}
\usepackage{amssymb}

%diagramação
\usepackage{extsizes}
\everymath{\displaystyle}
\usepackage{geometry}
\usepackage{fancyhdr}
\usepackage{multicol}
\usepackage{graphicx}
\usepackage[brazil]{babel}
\usepackage[shortlabels]{enumitem}
\usepackage{cancel}
\usepackage{textcomp}
\usepackage{tcolorbox}

%tabelas
\usepackage{array} % Para melhor formatação de tabelas
\usepackage{longtable}
\usepackage{booktabs}  % Para linhas horizontais mais bonitas
\usepackage{float}   % Para usar o modificador [H]
\usepackage{caption} % Para usar legendas em tabelas
\usepackage{wrapfig} % Para usar tabelas e figuras flutuantes


%tikzpicture
\usepackage{tikz}
\usepackage{scalerel}
\usepackage{pict2e}
\usepackage{tkz-euclide}
\usetikzlibrary{calc}
\usetikzlibrary{patterns,arrows.meta}
\usetikzlibrary{shadows}
\usetikzlibrary{external}

%pgfplots
\usepackage{pgfplots}
\pgfplotsset{compat=newest}
\usepgfplotslibrary{statistics}
\usepgfplotslibrary{fillbetween}

%colours
\usepackage{xcolor}



\columnsep=2cm
\hoffset=0cm
\textwidth=8cm
\setlength{\columnseprule}{.1pt}
\setlength{\columnsep}{2cm}
\renewcommand{\headrulewidth}{0pt}
\geometry{top=1in, bottom=1in, left=0.7in, right=0.5in}

\pagestyle{fancy}
\fancyhf{}
\fancyfoot[C]{\thepage}

\begin{document}
	
	\noindent\textbf{6FMA81 - Matemática} 
	
	\begin{center}Como usar os parênteses (Versão estudante)
	\end{center}
	
	\noindent\textbf{Nome:} \underline{\hspace{10cm}}
	\noindent\textbf{Data:} \underline{\hspace{4cm}}
	
	%\section*{Questões de Matemática}
	
	\begin{multicols}{2}
		\noindent Quando, em uma expressão numérica, aparecem parênteses, estes têm prioridade no cálculo. \\
		Se na expressão não ocorrem parênteses, a multiplicação tem prioridade sobre a adição e a subtração, isto é, devemos fazer primeiro as multiplicações e, em seguida, somas e subtrações.
		\noindent\textsubscript{-----------------------------------------------------------------------}
		\begin{enumerate}
			\item Efetue.
			\begin{enumerate}[a)]
				\item $(8 \cdot 4) - (2 \cdot (3 + 4))$ \\\\\\\\\\
				\item $(3 \cdot (4 + 5)) - (4 \cdot 3) + 2$ \\\\\\\\\\
				\item $((-1)(6 + 8)) - (4 - (6 \cdot 5))$ \\\\\\\\\\
				\item $(5 - 3)(3 - (2 - 8))$ \\\\\\
				\item $(((7 - 12 - 31) \cdot 0) \cdot 6) + ((-3) \cdot 7)$ \\\\\\\\\\
				\item $-6 + (((4(-7 + 2)) + 8) \cdot 3)$ \\\\\\\\\\
				\item $4 \cdot 2 - 9 - 5 \cdot 3 + 16 - 23$ \\\\\\\\\\
				\item $(3 - 6) \cdot 4 + 7(2 - 4)$ \\\\\\\\\\
				\item $3 - 6 \cdot 4 + 7 \cdot 2 - 4$ \\\\\\\\\\
			\end{enumerate}
			\item Nos itens abaixo, acrescente os parênteses da forma como desejar. Depois, resolva-os e compare com os de seus colegas.
			\begin{enumerate}
				\item $3 + 2 \cdot 5 - 4 \cdot 6 + 7$ \\\\\\\\\\
				\item $12 - 9 + 7 \cdot 5 - 5 \cdot 6 + 1$ \\\\\\\\\\
				\item $25 \cdot 2 - 32 - 10 + 3 \cdot 21 - 20$ \\\\\\\\\\
				\item $-7 + 12 \cdot 10 - 9 \cdot 0 + 4$ \\\\\\\\\\
			\end{enumerate}
			%26 e 27
			\item Efetue.
			\begin{enumerate}[a)]
				\item $4 \cdot 8 - 5(1 + 5)$ \\\\\\\\\\\\
				\item $4(5 + 1) - (3 \cdot 2) + 5$ \\\\\\\\\\
				\item $(8 - (7 \cdot 8)) - 7$ \\\\\\\\\\\\
				\item $((5 \cdot 6) - (7 \cdot 2)) - (3 \cdot 4)$ \\\\\\\\\\\\
				\item $6(5 - 9) - (7 - 1) \cdot 2$ \\\\\\\\\\\\
				\item $(-4)(3 + 6) - (3 - (9 \cdot 5))$ \\\\\\\\\\\\
				\item $6 \cdot 2 - 4 - 9 + 21 - 11$ \\\\\\\\\\\\
				\item $(3(-7) - 2) \cdot 5 + 15$ \\\\\\
				\item $3 -16 + 5 \cdot 5 - 5 + 4 - 4 \cdot 4$ \\\\\\\\\\\\
				\item $(3 - 7) \cdot 4 + 9 \cdot 2 - 5$ \\\\\\\\\\\\
				\item $3 - 7 \cdot 4 + 9 \cdot 2 - 5$ \\\\\\\\\\\\
				\item $((3 - 7) \cdot 4 + 9)(2 - 5)$ \\\\\\\\\\\\
				\item $(2 \cdot 8 - 12)(-6) + 4(-5)$ \\\\\\\\\\\\\\
				\item $(4 + 3(-6)+7(-2)) \cdot 3$ \\\\
			\end{enumerate}
			\item Resolvendo uma expressão numérica, um garoto fez o seguinte: \\
			$(-2) \cdot 3 + 8 - 9 + 4 \cdot (4^2 : 2)$ \\
			$=(-2) \cdot - 1 + 4(2^2)$ \\
			$=(-2) \cdot 6 \cdot 4$ \\
			$=-48$
			\begin{enumerate}[a)]
				\item Quais foram os erros dele? Aponte-os. \\\\\\\\\\
				\item Resolva a expressão corretamente. \\\\\\\\\\
				\item É possível colocar parênteses na expressão dada para que a resolução do garoto fique correta? Justifique sua resposta. \newpage
				\item Considere a expressão numérica dada sem os parênteses, ou seja, $-2 \cdot 3 + 8 - 9 + 4 \cdot 4^2 : 2$.
				Acrescente alguns parênteses de forma que você obtenha duas expressões diferentes dos itens acima.
				Quais valores você obteve? Compare com seus colegas seu resultado. \\\\\\\\\\
			\end{enumerate}
		\end{enumerate}
		$~$ \\ $~$ \\ $~$ \\ $~$ \\ $~$ \\ $~$ \\ $~$ \\ $~$ \\ $~$ \\ $~$ \\ $~$ \\ $~$ \\ $~$ \\ $~$ \\ $~$ \\ $~$ \\ $~$ \\ $~$ \\ $~$ \\ $~$ \\ $~$ \\ $~$ \\ $~$ \\ $~$ \\ $~$ \\ $~$ \\ $~$ \\ $~$ \\ $~$ \\ $~$ \\ $~$ \\ $~$ \\ $~$ \\ $~$ \\ $~$ \\ $~$ \\ $~$ \\ $~$ \\ $~$ \\ $~$ \\ $~$ \\ $~$ \\ $~$ \\ $~$ \\ $~$ \\ $~$ \\ $~$ \\ $~$ \\ $~$ \\ $~$ \\ $~$ \\ $~$ \\ $~$ \\ $~$ \\ $~$ \\ $~$ \\ $~$ \\ $~$ \\ $~$ \\ $~$ \\ $~$ \\ $~$ \\ $~$ \\ $~$ \\ $~$ \\ $~$ \\ $~$
	\end{multicols}
\end{document}