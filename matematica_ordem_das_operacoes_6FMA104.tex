\documentclass[a4paper,14pt]{article}
\usepackage{float}
\usepackage{extsizes}
\usepackage{amsmath}
\usepackage{amssymb}
\everymath{\displaystyle}
\usepackage{geometry}
\usepackage{fancyhdr}
\usepackage{multicol}
\usepackage{graphicx}
\usepackage[brazil]{babel}
\usepackage[shortlabels]{enumitem}
\usepackage{cancel}
\usepackage{textcomp}
\usepackage{array} % Para melhor formatação de tabelas
\usepackage{longtable}
\usepackage{booktabs}  % Para linhas horizontais mais bonitas
\usepackage{float}   % Para usar o modificador [H]
\usepackage{caption} % Para usar legendas em tabelas
\usepackage{tcolorbox}

\columnsep=2cm
\hoffset=0cm
\textwidth=8cm
\setlength{\columnseprule}{.1pt}
\setlength{\columnsep}{2cm}
\renewcommand{\headrulewidth}{0pt}
\geometry{top=1in, bottom=1in, left=0.7in, right=0.5in}

\pagestyle{fancy}
\fancyhf{}
\fancyfoot[C]{\thepage}

\begin{document}
	
	\noindent\textbf{6FMA104 - Matemática} 
	
	\begin{center}Ordem das operações (Versão estudante)
	\end{center}
	
	\noindent\textbf{Nome:} \underline{\hspace{10cm}}
	\noindent\textbf{Data:} \underline{\hspace{4cm}}
	
	%\section*{Questões de Matemática}
	~ \\ ~
	\begin{multicols}{2}
		\noindent Se $a$ e $b$ são inteiros, com $b \neq 0$, e se existe $c$ inteiro tal que $a = bc$, então $c$ é a divisão ou o quociente de $a$ por $b$. \\
		As regras de sinais são as mesmas da multiplicação. \\
		Ordem das operações: \\
		1º) Potenciações; \\
		2º) Multiplicações e divisões (na ordem em que aparecem); \\
		3º) Adições e subtrações. \\
		\textbf{Observação:}~qualquer operação entre parênteses deve ser efetuada em primeiro lugar.
	\end{multicols}
	\noindent\textsubscript{~-----------------------------------------------------------------------------------------------------------------------------------------------------}
	\begin{multicols}{2}
    	\begin{enumerate}
    		\item Usar a definição para calcular, se possível, cada uma das divisões.
    		\begin{enumerate}[a)]
    			\item $42 : (-6)$ \\\\\\\\\\\\\\\\\\\\
    			\item $\frac{48}{-8}$ \\\\\\\\\\\\\\\\\\\\
    			\item $(-28)/(-7)$ \\\\\\\\\\\\\\\\\\\\
    			\item $4 : 5$ \\\\\\\\\\\\\\\\\\\\
    			\item $72/(-9)$ \\\\\\\\\\\\\\\\\\\\
    			\item $(-54)/6$ \\\\\\\\\\\\\\\\\\\\
    			\item $(-36)/7$ \\\\\\\\\\\\\\\\\\\\
    			\item $35 : 5$ \\\\\\\\\\\\
    		\end{enumerate}
    		\item Efetuar:
    		\begin{enumerate}[a)]
    			\item $15 \cdot 7^0 + 2 \cdot 3^2 - 9 - 4(-2)^1$ \\\\\\\\\\\\\\\\\\\\
    			\item $(3 - 6)^2 + 8 \cdot 0 \cdot (-5) -5 \cdot (-2)^3$ \\\\\\\\\\\\\\\\\\\\
    			\item $5(-1 \cdot 2)^2 - 7(5 - 3) - 48 : 2^3$ \newpage
    			\item $((18 - 6)^2 : 2^2) : (-3^2) \\ + 0 : (32 - 4^3)$ \\\\\\\\\\\\\\\\\\\\
    		\end{enumerate}
    		\item Use a definição para calcular, se possível, cada uma das divisões.
    		\begin{enumerate}[a)]
    			\item $(-42):2$ \\\\\\\\\\\\\\\\\\
    			\item $\frac{36}{-9}$ \\\\\\\\\\\\\\\\\\
    			\item $-24/9$ \\\\\\\\\\\\\\\\\\
    			\item $8 : 2$ \\\\\\\\\\\\\\\\\\
    			\item $-48/6$ \\\\\\\\\\\\\\\\\\
    			\item $-63/7$ \\\\\\\\\\\\\\\\\\
    			\item $\frac{13}{-3}$ \newpage
    			\item $56 : 8$ \\\\\\\\\\\\\\\\\\
    			\item $\frac{-42}{-14}$ \\\\\\\\\\\\\\\\\\
    			\item $\frac{(-45)}{9}$ \\\\\\\\\\\\\\\\\\
    			\item $108 : 8$ \\\\\\\\\\\\\\\\\\
    			\item $41/-5$ \\\\\\\\\\\\\\\\\\
    		\end{enumerate}
    		\item Efetuar:
    		\begin{enumerate}[a)]
    			\item $6 - 9 \cdot 3 - 7 + 4 \cdot 4$ \\\\\\\\\\\\\\\\\\
    			\item $4 - 7 \cdot 2 + 8 \cdot 3 - 5 + 6^2 - 6$ \\\\\\\\\\\\\\\\\\
    			\item $2 + 4 \cdot (-6) - 8 + 7 \cdot (-5)^2$ \newpage
    			\item $6 \cdot (-3) + 2 \cdot 5 - 8 \cdot 4 -(-2)^2$ \\\\\\\\\\\\\\\\\\
    			\item $3^3 - 2 \cdot 4 \cdot 3 + 9 + 5 \cdot (-3)$ \\\\\\\\\\\\\\\\\\
    			\item $-7 -12(-3) - (2 \cdot 4)^2 + 27 : (-3)$ \\\\\\\\\\\\\\\\
    			\item $21 \cdot 7^0 - 6 \cdot 2^3 + 9 - 5 \cdot (-6)^1$ \\\\\\\\\\\\\\\\\\
    			\item $(6 - 8)^3 - 9 \cdot (-2) \cdot 0 + 2 (-5)^2$ \\\\\\\\\\\\\\\\\\
    			\item $5(-4 \cdot 2)^2 - 7(4 + 9) -56 : 2^3$ \\\\\\\\\\\\\\\\\\
    			\item $((21 - 12)^2 : 3^2) \cdot (-2) + 0 : (42 - 5^2)$ \\\\\\\\\\\\\\\\\\
    			\item $(2^3 : (7 - 5)^2 - 6^2) \\ \cdot (-4 + 9 - 2^3) - 12$ \\\\\\\\\\\\\\
    		\end{enumerate}
    	\end{enumerate}
	\end{multicols}
\end{document}