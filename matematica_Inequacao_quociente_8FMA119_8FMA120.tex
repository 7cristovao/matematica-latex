\documentclass[a4paper,14pt]{article}
\usepackage{float}
\usepackage{extsizes}
\usepackage{amsmath}
\usepackage{amssymb}
\everymath{\displaystyle}
\usepackage{geometry}
\usepackage{fancyhdr}
\usepackage{multicol}
\usepackage{graphicx}
\usepackage[brazil]{babel}
\usepackage[shortlabels]{enumitem}
\usepackage{cancel}
\usepackage{textcomp}
\usepackage{array} % Para melhor formatação de tabelas
\usepackage{longtable}
\usepackage{booktabs}  % Para linhas horizontais mais bonitas
\usepackage{float}   % Para usar o modificador [H]
\usepackage{caption} % Para usar legendas em tabelas

\columnsep=2cm
\hoffset=0cm
\textwidth=8cm
\setlength{\columnseprule}{.1pt}
\setlength{\columnsep}{2cm}
\renewcommand{\headrulewidth}{0pt}
\geometry{top=1in, bottom=1in, left=0.7in, right=0.5in}

\pagestyle{fancy}
\fancyhf{}
\fancyfoot[C]{\thepage}

\begin{document}
	
	\noindent\textbf{8FMA119, 8FMA120 - Matemática} 
	
	\begin{center}Inequação quociente (Versão estudante)
	\end{center}
	
	\noindent\textbf{Nome:} \underline{\hspace{10cm}}
	\noindent\textbf{Data:} \underline{\hspace{4cm}}
	
	%\section*{Questões de Matemática}	
    \begin{multicols}{2}
    	\noindent Ao estudar o sinais das expressões $A$ e $B$ separadamente, pode-se obter os sinais da expressão $\frac{A}{B}$. \\
    	\noindent\textsubscript{~---------------------------------------------------------------------------}
    	\begin{enumerate}
    		\item Resolver as seguintes inequações.
    		\begin{enumerate}[a)]
    			\item $\frac{2x - 8}{x^2 - 4} < 0$ \\\\\\\\\\\\\\\\\\\\\\\\\\\\
    			\item $\frac{(x^2 - 8x + 7)(x^2 - 4)}{4x - 1} \leq 0$ \\\\\\\\\\\\\\\\\\\\\\\\
    			\item $\frac{-x(2x^2 - 18)(x^2 - 10)}{x + 5} \geq 0$ \\\\\\\\\\\\\\\\\\\\
    		\end{enumerate}
    		\item Resolver as seguintes inequações.
    		\begin{enumerate}[a)]
    			\item $\frac{(x - 2)^2}{x^2 - 5x + 6} \leq 2$  \\\\\\\\\\\\\\\\\\\\\\\\\\\\\\\\\\\\\\
    			\item $\frac{x^2 - 5}{2x} \geq 2$  \\\\\\\\\\\\\\\\\\\\\\\\\\\\\\\\\\\\\\\\
    			\item $\frac{13}{x} \geq x$  \\\\\\\\\\\\\\\\\\\\\\\\\\\\\\\\\\
    		\end{enumerate}
    		\item Resolver as inequações no universo dos reais.
    		\begin{enumerate}[a)]
    			\item $\frac{x^2}{x + 9} < 0$ \\\\\\\\\\\\\\\\\\\\\\\\\\\\\\\\\\\\\\\\
    			\item $\frac{(x^2 + 3x - 10)(x^2 - 7)}{3x + 4} \geq 0$ \\\\\\\\\\\\\\\\\\\\\\\\\\\\\\
    			\item $\frac{-x^2(2x^2 - 8)(x + 5)}{x - 1} \leq 0$ \\\\\\\\\\\\\\\\\\\\\\\\\\\\\\
    			\item $\frac{x^2 - 9x + 20}{x^2 - 9x + 8} \geq 0$ \\\\\\\\\\\\\\\\\\\\\\\\\\\\\\\\\\\\\\\\
    			\item $\frac{(x^2 - 6x + 9)(-x^2 + 16)}{x^2 + 50} \leq 0$ \\\\\\\\\\\\\\\\\\\\\\\\\\\\\\\\\\\\\\\\
    			\item $\frac{-x(-x^2 - 11x)(x - 5)}{x(x^2 - 10x)} > 0$ \newpage
    		\end{enumerate}
    	\end{enumerate}
    \end{multicols}
    		\noindent4. Resolver as inequações em $U = \mathbb{R}$
    		
    			a) $\frac{x^2}{(x^2 - 4x + 4)(x + 4)} \geq 0$ \\\\\\\\\\\\\\\\\\\\\\\\\\\\\\\\\\\\\\\\
    	
    
    			b) $\frac{(x^2 + 5x - 14)(x^2 + x + 7)(x^2 + 5x + 6)}{(x+ 1)(x^2 - 9)} \geq 0$ \\\\\\\\\\\\\\\\\\\\\\\\\\\\\\\\\\
    			c) $\frac{-x^2(x^2 - 13x + 36)(x^2 + x + 2)(x^2 - 4x + 4)}{(4x^2 - 8)(x^2 + 6x)} > 0$ \newpage
    \begin{multicols}{2}	
    		
    		\noindent5. Resolver as inequações no universo dos reais.\\
    		
    			a) $\frac{x + 1}{x} - \frac{2x - 6}{x - 1} \leq 0$ \\\\\\\\\\\\\\\\\\\\\\\\\\\\
    			b) $\frac{x}{x + 1} + \frac{3x}{x - 2} \geq 4$ \\\\\\\\\\\\\\\\\\\\\\\\\\\\
    			c) $\frac{x}{x + 2} + \frac{x + 2}{x} < 5$ \\\\\\\\\\\\\\\\\\
    			d) $\frac{x + 3}{5} \geq \frac{2}{x - 3}$  \\\\\\\\\\\\\\\\\\\\\\\\\\\\
    			e) $\frac{2 + x}{7} \leq \frac{7}{x - 2}$
    			\newpage
    \end{multicols}		
    		6. Resolver a equação $x - \frac{2}{x + 1} \geq 4$, em $U = \mathbb{R}$. \\\\\\\\\\\\\\\\\\\\\\\\\\\\\\\\\\\\\\\\
    		7. Resolver a equação $x - 7 > \frac{1}{x + 2}$, em $U = \mathbb{R}$.  \\\\\\\\\\\\\\\\\\\\\\\\\\\\\\\\\\\\\\\\
    		8. Resolver a equação $\frac{3x^2 - 4x + 2}{3x + 2} \leq 1$, em $U = \mathbb{R}$.
    
    	
    
\end{document}