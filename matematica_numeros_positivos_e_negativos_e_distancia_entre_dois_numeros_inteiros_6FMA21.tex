\documentclass[a4paper,14pt]{article}

\usepackage{comment} % Para comentar várias linhas ao mesmo tempo

%matemática
\usepackage{amsmath}
\usepackage{amssymb}

%diagramação
\usepackage{extsizes}
\everymath{\displaystyle}
\usepackage{geometry}
\usepackage{fancyhdr}
\usepackage{multicol}
\usepackage{graphicx}
\usepackage[brazil]{babel}
\usepackage[shortlabels]{enumitem}
\usepackage{cancel}
\usepackage{textcomp}
\usepackage{tcolorbox}

%tabelas
\usepackage{array} % Para melhor formatação de tabelas
\usepackage{longtable}
\usepackage{booktabs}  % Para linhas horizontais mais bonitas
\usepackage{float}   % Para usar o modificador [H]
\usepackage{caption} % Para usar legendas em tabelas
\usepackage{wrapfig} % Para usar tabelas e figuras flutuantes
\usepackage{xcolor} % Para cores do fundo de tabelas
\usepackage{colortbl} % Para cores do fundo de tabelas

%tikzpicture
\begin{comment}
	\usepackage{tikz}
	\usepackage{scalerel}
	\usepackage{pict2e}
	\usepackage{tkz-euclide}
	\usetikzlibrary{calc}
	\usetikzlibrary{patterns,arrows.meta}
	\usetikzlibrary{shadows}
	\usetikzlibrary{external}
\end{comment}


%pgfplots
\usepackage{pgfplots}
\pgfplotsset{compat=newest}
\usepgfplotslibrary{statistics}
\usepgfplotslibrary{fillbetween}

%colours
\usepackage{xcolor}



\columnsep=2cm
\hoffset=0cm
\textwidth=8cm
\setlength{\columnseprule}{.1pt}
\setlength{\columnsep}{2cm}
\renewcommand{\headrulewidth}{0pt}
\geometry{top=1in, bottom=1in, left=0.7in, right=0.5in}

\pagestyle{fancy}
\fancyhf{}
\fancyfoot[C]{\thepage}

\begin{document}
	
	\noindent\textbf{6FMA21 - Matemática} 
	
	\begin{center}Números positivos e negativos e distância entre dois números inteiros (Versão estudante)
	\end{center}
	
	\noindent\textbf{Nome:} \underline{\hspace{10cm}}
	\noindent\textbf{Data:} \underline{\hspace{4cm}}
	
	%\section*{Questões de Matemática}
	
	\begin{multicols}{2}
		\noindent Na representação dos números inteiros em uma reta, os que seguem a orientação dada, a partir do zero, são os números positivos, e os que seguem a orientação contrária à dada, a partir do zero, são os números negativos. \\
		Se um número inteiro é positivo, escrevemos $x > 0$ ($x$ é maior do que zero) ou $0 < x$ (zero é menor do que $x$). Se $x$ é um número negativo, escrevemos $x < 0$ ($x$ é menor do que zero) ou $0 < x$ (zero é maior do que $x$). Se $x$ é igual a zero, escrevemos $x = 0$ (x é nulo). \\
		A distância entre dois números inteiros em uma reta é igual ao número de unidades existentes entre esses dois números. Escrevemos $d (x, y)$ para a distância entre os inteiros $x$ e $y$.
	\end{multicols}
		\noindent\textsubscript{------------------------------------------------------------------------------------------------------------------------------------------------------------}
	\begin{multicols}{2}
		\begin{enumerate} 
			\item Complete com \textbf{$>$}, \textbf{$<$} ou \textbf{=}.
			\begin{enumerate}[a)]
				\item -2 \underline{~~~~~~~} 3
				\item 3 \underline{~~~~~~~} -3
				\item 8 \underline{~~~~~~~} 10
				\item 13 \underline{~~~~~~~} 0
				\item 0 \underline{~~~~~~~} -0
				\item 0 \underline{~~~~~~~} -4
				\item -15 \underline{~~~~~~~} 0
				\item -8 \underline{~~~~~~~} 8
			\end{enumerate}
			\item Suponhamos que $X$ é um número inteiro e $x < 0$.
			\begin{enumerate}[a)]
				\item $x$ pode ser 5? \\\\\\\\\\\\
				\item $x$ pode ser 0? \\\\\\\\\\\\
				\item $x$ pode ser -2? \\\\\\\\\\\\
				\item $x$ pode ser $\frac{1}{4}$? \newpage
				\item $x$ pode ser 200? \\\\\\\\\\
				\item $x$ pode ser -1 000? \\\\\\\\\\
			\end{enumerate}
			\item Simbolize usando \textbf{$>$}, \textbf{$<$} ou \textbf{=}.
			\begin{enumerate}[a)]
				\item $x$ é positivo. \\\\\\
				\item $x$ é negativo. \\\\\\
				\item $x$ é igual a zero. \\\\\\
				\item $a$ é maior que zero. \\\\\\
				\item $b$ é menor que zero. \\\\\\
				\item $k$ é positivo. \\\\\\
				\item $m$ é nulo. \\\\\\
				\item $y$ é menor que zero. \\\\\\
			\end{enumerate}
			\item Escreva três maneiras de ler $0 > x$. \\\\\\\\\\\\\\
			\item Complete.
			\begin{enumerate}[a)]
				\item $d(0, 3)$ =  \underline{~~~~~~~} 
				\item $d(-6, 2)$ =  \underline{~~~~~~~} 
				\item $d(3, 3)$ =  \underline{~~~~~~~} 
				\item $d(10, -8)$ = \underline{~~~~~~~} 
				\item $d(0, 4)$ = \underline{~~~~~~~} 
				\item $d(-7, 5)$ = \underline{~~~~~~~} 
				\item $d(-8, -2)$ = \underline{~~~~~~~} 
				\item $d(0, 0)$ = \underline{~~~~~~~} 
			\end{enumerate}
			%73 a 75
			\item Assinale \textbf{V} (verdadeiro) ou \textbf{F} (falso).
			\begin{enumerate}[a)]
				\item (~~) Dizer que $x$ é um número positivo é o mesmo que dizer $0 < x$.
				\item (~~) Se $a$ é um número inteiro e $a$ é negativo, então $0 > a$.
				\item (~~) Se $m \in \mathbb{Z}$ e $m$ é negativo, então $0 < m$.
				\item (~~) Se $x$ é um número inteiro, então $x$ é positivo.
				\item (~~) Se $y \in \mathbb{Z}$ e $y < 0$, então $y$ é negativo.
			\end{enumerate}
			\item Você está jogando em um computador. Há uma regra: após cada partida, você só pode continuar jogando se seu número de pontos não for negativo. Se $x$ é o número de pontos, escreva a condição para que você possa continuar jogando. \\\\\\\\\\\\\\\\\\\\
			\item Sabe-se que, no nível do mar, a água é sólida (na forma de gelo) para temperaturas negativas em graus Celsius. Se $x$ é a temperatura da água no nível do mar, escreva a condição para que a água seja sólida.
		\end{enumerate}
		$~$ \\ $~$ \\ $~$ \\ $~$ \\ $~$ \\ $~$ \\ $~$ \\ $~$ \\ $~$ \\ $~$ \\ $~$ \\ $~$ \\ $~$ \\ $~$ \\ $~$ \\ $~$ \\ $~$ \\ $~$ \\ $~$ \\ $~$ \\ $~$ \\ $~$ \\ $~$ \\ $~$ \\ $~$ \\ $~$ \\ $~$ \\ $~$ \\ $~$ \\ $~$ \\ $~$ \\ $~$ \\ $~$ \\ $~$ \\ $~$ \\ $~$ \\ $~$ \\ $~$ \\ $~$ \\ $~$ \\ $~$ \\ $~$ \\ $~$ \\ $~$ \\ $~$ \\ $~$ \\ $~$ \\ $~$ \\ $~$ \\ $~$ \\ $~$ \\ 
	\end{multicols}
\end{document}