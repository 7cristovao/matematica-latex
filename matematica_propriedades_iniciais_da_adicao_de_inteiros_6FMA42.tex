\documentclass[a4paper,14pt]{article}
\usepackage{float}
\usepackage{extsizes}
\usepackage{amsmath}
\usepackage{amssymb}
\everymath{\displaystyle}
\usepackage{geometry}
\usepackage{fancyhdr}
\usepackage{multicol}
\usepackage{graphicx}
\usepackage[brazil]{babel}
\usepackage[shortlabels]{enumitem}
\usepackage{cancel}
\usepackage{textcomp}
\usepackage{array} % Para melhor formatação de tabelas
\usepackage{longtable}
\usepackage{booktabs}  % Para linhas horizontais mais bonitas
\usepackage{float}   % Para usar o modificador [H]
\usepackage{caption} % Para usar legendas em tabelas
\usepackage{tcolorbox}

\columnsep=2cm
\hoffset=0cm
\textwidth=8cm
\setlength{\columnseprule}{.1pt}
\setlength{\columnsep}{2cm}
\renewcommand{\headrulewidth}{0pt}
\geometry{top=1in, bottom=1in, left=0.7in, right=0.5in}

\pagestyle{fancy}
\fancyhf{}
\fancyfoot[C]{\thepage}

\begin{document}
	
	\noindent\textbf{6FMA42 - Matemática} 
	
	\begin{center}Propriedades iniciais da adição de inteiros (Versão estudante)
	\end{center}
	
	\noindent\textbf{Nome:} \underline{\hspace{10cm}}
	\noindent\textbf{Data:} \underline{\hspace{4cm}}
	
	%\section*{Questões de Matemática}
	\begin{multicols}{2}
		\noindent Propriedades da adição de números inteiros:
		\begin{itemize}
			\item \textbf{A0:} A soma de dois números inteiro é sempre um número inteiro.
			\item \textbf{A1:} (a + b) + c = a + (b + c)(propriedade associativa).
			\item \textbf{A2:} a + b = b + a (propriedade comutativa).
			\item Podemos também, combinar A1 e A2 na mesma expressão. 
		\end{itemize}
	\textsubscript{---------------------------------------------------------------------}
    	\begin{enumerate}
    		\item Efetue:
    		\begin{enumerate}[a)]
    			\item 6 + [(-7) + 8] \\\\\\\\\\
    			\item $[6 + (-7)] + 8$ \\\\\\\\\\
    			\item 5 + (4 + 6) \\\\\\\\\\
    			\item (5 + 4) + 6 \\\\\\\\\\
    			\item -7 + [0 + (-6)] \\\\\\\\\\
    		\end{enumerate}
    		A partir desses cálculos é possível concluir alguma coisa? O quê? Tente expressar-se em português e usando linguagem matemática.
    		\item Efetue:
    		\begin{enumerate}[a)]
    			\item -18 + 6 +(-5) \\\\\\\\\\
    			\item 7 +(-8) +(-6) \\\\\\\\\\
    			\item 16 +(-4) + 2 \\\\
    			\item 12 + (-5) +(-4) \\\\\\\\\\
    			\item 7 + 4 +(-8) \\\\\\\\\\
    			\item -3 + 9 +(-4) \\\\\\\\\\
    			\item 16 +(-12) + (-5) \\\\\\\\\\
    			\item 32 +(-13)+(-15) \\\\\\\\\\
    		\end{enumerate}
    		\item No dia 1º de abril, o saldo da conta bancária de Maria em certo banco era R\$ 2.500,00. Durante o mês, ela emitiu cheques de R\$ 700,00, R\$ 1.200,00 e R\$ 1.400,00 e fez depósitos de R\$ 100,00, R\$ 300,00, R\$ 150,00 e R\$ 200,00.
    		\begin{enumerate}[a)]
    			\item Expresse o saldo atual da conta de Maria como uma soma de números positivos e negativos. \\\\\\\\
    			\item Qual o saldo de Maria no fim do mês? \\\\\\\\
    		\end{enumerate}
    		\item A temperatura do interior de uma geladeira era de -12°C. Alguém deixou a porta aberta por algum tempo e a temperatura subiu 8°C. Fechada a porta, a temperatura desceu 10°C. Houve corte de energia e a temperatura subiu novamente 3°C.
    		\begin{enumerate}[a)]
    			\item Apresente uma descrição da nova temperatura da geladeira usando adição. \\\\\\\\\\\\\\
    			\item Qual a nova temperatura da geladeira? \newpage
    		\end{enumerate}
    		\item Efetue:
    		\begin{enumerate}[a)]
    			\item 5 + (-7) \\\\\\\\
    			\item (-7) + 5 \\\\\\\\
    			\item (-7) + 0 \\\\\\\\
    			\item 1 + 8 \\\\\\\\
    			\item 8 + 1 \\\\\\\\
    		\end{enumerate}
    		A partir desses cálculos é possível concluir alguma coisa? O quê? Expresse-se em português e usando linguagem matemática. \\\\\\\\\\\\\\\\
    	\end{enumerate}
    	\textbf{Desafio olímpico} \\\\
    	(OBMEP) Pedro Américo e Candido Portinari foram grandes pintores brasileiros, e Leonardo da Vinci foi um notável artista italiano. Pedro Américo nasceu em 1843. Já Leonardo nasceu 391 anos antes de Portinari. Em que ano Portinari nasceu? \\\\
    	\begin{enumerate}[a)]
    		\item 1903
    		\item 1904
    		\item 1905
    		\item 1906
    		\item 1907
    	\end{enumerate}
    	$~$ \\ $~$ \\ $~$ \\ $~$ \\ $~$ \\ $~$ \\ $~$ \\ $~$ \\ $~$ \\ $~$ \\ $~$ \\ $~$ \\ $~$ \\ $~$ \\ $~$ \\ $~$ \\ $~$ \\ $~$ \\ $~$ \\ $~$ \\ $~$ \\
	\end{multicols}
\end{document}