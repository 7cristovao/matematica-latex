\documentclass[a4paper,14pt]{article}
\usepackage{float}
\usepackage{extsizes}
\usepackage{amsmath}
\usepackage{amssymb}
\everymath{\displaystyle}
\usepackage{geometry}
\usepackage{fancyhdr}
\usepackage{multicol}
\usepackage{graphicx}
\usepackage[brazil]{babel}
\usepackage[shortlabels]{enumitem}
\usepackage{cancel}
\usepackage{textcomp}
\usepackage{array} % Para melhor formatação de tabelas
\usepackage{longtable}
\usepackage{booktabs}  % Para linhas horizontais mais bonitas
\usepackage{float}   % Para usar o modificador [H]
\usepackage{caption} % Para usar legendas em tabelas

\columnsep=2cm
\hoffset=0cm
\textwidth=8cm
\setlength{\columnseprule}{.1pt}
\setlength{\columnsep}{2cm}
\renewcommand{\headrulewidth}{0pt}
\geometry{top=1in, bottom=1in, left=0.7in, right=0.5in}

\pagestyle{fancy}
\fancyhf{}
\fancyfoot[C]{\thepage}

\begin{document}
	
	\noindent\textbf{8FMA100 - Matemática} 
	
	\begin{center}Equações do tipo $F_1 \cdot F_2 = F_1 \cdot F_3$ (Versão estudante)
	\end{center}
	
	\noindent\textbf{Nome:} \underline{\hspace{10cm}}
	\noindent\textbf{Data:} \underline{\hspace{4cm}}
	
	%\section*{Questões de Matemática}	
    \begin{multicols}{2}
    	\noindent Para resolvermos equações do tipo $F_1 F_2 = F_1 F_3$, podemos utilizar $F_1 F_2 = F_1 F_3 \Leftrightarrow \begin{cases} F_1 = 0 \\ ou \\ F_2 = F_3 \end{cases}$
    	\noindent\textsubscript{~---------------------------------------------------------------------------}
    	\begin{enumerate}
    		\item Resolver as equações.
    		\begin{enumerate}[a)]
    			\item $(x^2 - 9)(x^2 - 5x + 9) = (x^2 - 9)(4x - 5)$ \\\\\\\\\\\\\\\\
    			\item $x^2(x^2 - 4) = (x^2 - 4)(-2x + 5)$ \\\\\\\\\\\\\\\\
    			\item $(x - 8)(x - 5) = (x - 6)(x - 8)$ \\\\\\\\\\\\
    			\item $(x^2 - 8x + 15)(x^2 - 2x - 3) = (-2x - 3)(x^2 - 8x + 15)$ \\\\\\\\\\\\\\\\
    			\item $(x^2 - 3x + 4)(x^2 - 2x) = (-2x - 3)(x^2 - 3x + 4)$ \\\\\\\\\\\\\\
    		\end{enumerate}
    		\item Para resolver a equação $x^2 - 7x = 0$, um aluno usou as seguintes passagens: \\
    		$x^2 - 7x = 0 \Leftrightarrow x^2 = 7x \Leftrightarrow x \cdot \cancel{x} = 7 \cdot \cancel{x} \Leftrightarrow x = 7$ \\
    		O raciocínio do aluno está correto? Por quê? \newpage
    		\item O produto da minha idade há cinco anos pela idade do meu tio é igual ao produto da minha idade daqui a trinta anos pela metade de minha idade. Sabendo que meu tio é trinta anos mais velho do que eu, quantos anos eu tenho?
    	\end{enumerate}
    $~$ \\ $~$ \\ $~$ \\ $~$ \\ $~$ \\ $~$ \\ $~$ \\ $~$ \\ $~$ \\ $~$ \\ $~$ \\ $~$ \\ $~$ \\ $~$ \\ $~$ \\ $~$ \\ $~$ \\ $~$ \\ $~$ \\ $~$ \\ $~$ \\ $~$ \\ $~$ \\ $~$ \\ $~$ \\ $~$ \\ $~$ \\ $~$ \\ $~$ \\ $~$ \\ $~$ \\ $~$ \\ $~$ \\ $~$ \\ $~$ \\ $~$ \\ $~$ \\ $~$ \\ $~$ \\ $~$ \\ $~$ \\ $~$ \\ $~$ \\ $~$ \\ $~$ \\ $~$ \\ $~$ \\ $~$ \\ $~$ \\ $~$ \\ $~$ \\ $~$ \\ $~$ \\ $~$ \\ $~$ \\ $~$ \\ $~$ \\ $~$ \\ $~$ \\ $~$ \\ $~$ \\ $~$ \\ $~$ \\ $~$ \\ $~$ \\ $~$ \\ $~$ \\ $~$ \\ $~$ \\ $~$ \\ $~$ \\ $~$ \\ $~$ \\ $~$ \\ $~$
    \end{multicols}
\end{document}