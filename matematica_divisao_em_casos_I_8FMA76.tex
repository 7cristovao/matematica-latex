\documentclass[a4paper,14pt]{article}
\usepackage{float}
\usepackage{extsizes}
\usepackage{amsmath}
\usepackage{amssymb}
\everymath{\displaystyle}
\usepackage{geometry}
\usepackage{fancyhdr}
\usepackage{multicol}
\usepackage{graphicx}
\usepackage[brazil]{babel}
\usepackage[shortlabels]{enumitem}
\usepackage{cancel}
\usepackage{textcomp}
\columnsep=2cm
\hoffset=0cm
\textwidth=8cm
\setlength{\columnseprule}{.1pt}
\setlength{\columnsep}{2cm}
\renewcommand{\headrulewidth}{0pt}
\geometry{top=1in, bottom=1in, left=0.7in, right=0.5in}

\pagestyle{fancy}
\fancyhf{}
\fancyfoot[C]{\thepage}

\begin{document}
	
	\noindent\textbf{8FMA76 - Matemática} 
	
	\begin{center}Divisão em casos (I) (Versão estudante)
	\end{center}
	
	\noindent\textbf{Nome:} \underline{\hspace{10cm}}
	\noindent\textbf{Data:} \underline{\hspace{4cm}}
	
	%\section*{Questões de Matemática}
	
	
    \begin{multicols}{2}
    	\noindent O princípio multiplicativo não resolve todos os problemas. Pode ser necessário dividir o problema em casos (você percebe isso se em algum momento o número de possibilidades de acontecer alguma coisa depender das escolhas anteriores). Você sempre deve tomar o cuidado de dividir o problema em casos, de modo que dois deles não possam conter uma mesma possibilidade.
    	\noindent\textsubscript{~---------------------------------------------------------------------------}
		\begin{enumerate}
			\item Quantos números de algarismos distintos existem de 10 a 10000? \\\\\\\\\\\\\\\\\\
			\item Quantos números ímpares de algarismos distintos existem entre 2000 e 9000? \\\\\\\\\\\\\\\\
			\item Quantos números pares de algarismos distintos existem entre 2000 e 9000? \\\\\\\\\\\\\\\\\\\\\\\\\\\\
			\item Usando as letras A, E, I, O, U, quantas sequências de 3 ou 4 letras distintas podem ser feitas? \\\\\\\\\\\\\\\\\\\\\\\\\\\\
			\item Quantos inteiros positivos podem ser formados com os algarismos 3, 4, 5, 6 e 7 sem repetir os algarismos? \\\\\\\\\\\\\\\\\\\\\\\\\\\\
			\item Quantos números de 3 algarismos têm pelo menos 2 algarismos repetidos? \\\\\\\\\\\\\\\\\\\\\\\\\\\\
        \end{enumerate}
    $~$ \\ $~$ \\ $~$ \\ $~$ \\ $~$ \\ $~$ \\ $~$ \\ $~$ \\ $~$ \\ $~$ \\ $~$ \\ $~$ \\ $~$ \\ $~$ \\ $~$ \\ $~$ \\ $~$ \\ $~$ \\ $~$ \\ $~$ \\ $~$ \\ $~$ \\ $~$ \\ $~$ \\ $~$ \\ $~$ \\ $~$ \\ $~$ \\ $~$ \\ $~$ \\ $~$ \\ $~$ \\ $~$ \\ $~$ \\ $~$ \\ $~$ \\ $~$ \\ $~$ \\ $~$ \\ $~$ \\ $~$ \\$~$ \\ $~$ \\ $~$ \\ $~$ \\ 
    \end{multicols}
\end{document}