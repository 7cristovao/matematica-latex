\documentclass[a4paper,14pt]{article}

\usepackage{comment} % Para comentar várias linhas ao mesmo tempo

%matemática
\usepackage{amsmath}
\usepackage{amssymb}

%diagramação
\usepackage{extsizes}
\everymath{\displaystyle}
\usepackage{geometry}
\usepackage{fancyhdr}
\usepackage{multicol}
\usepackage{graphicx}
\usepackage[brazil]{babel}
\usepackage[shortlabels]{enumitem}
\usepackage{cancel}
\usepackage{textcomp}
\usepackage{tcolorbox}

%tabelas
\usepackage{array} % Para melhor formatação de tabelas
\usepackage{longtable}
\usepackage{booktabs}  % Para linhas horizontais mais bonitas
\usepackage{float}   % Para usar o modificador [H]
\usepackage{caption} % Para usar legendas em tabelas
\usepackage{wrapfig} % Para usar tabelas e figuras flutuantes
\usepackage{xcolor} % Para cores do fundo de tabelas
\usepackage{colortbl} % Para cores do fundo de tabelas

%tikzpicture
\begin{comment}
	\usepackage{tikz}
	\usepackage{scalerel}
	\usepackage{pict2e}
	\usepackage{tkz-euclide}
	\usetikzlibrary{calc}
	\usetikzlibrary{patterns,arrows.meta}
	\usetikzlibrary{shadows}
	\usetikzlibrary{external}
\end{comment}


%pgfplots
\usepackage{pgfplots}
\pgfplotsset{compat=newest}
\usepgfplotslibrary{statistics}
\usepgfplotslibrary{fillbetween}

%colours
\usepackage{xcolor}



\columnsep=2cm
\hoffset=0cm
\textwidth=8cm
\setlength{\columnseprule}{.1pt}
\setlength{\columnsep}{2cm}
\renewcommand{\headrulewidth}{0pt}
\geometry{top=1in, bottom=1in, left=0.7in, right=0.5in}

\pagestyle{fancy}
\fancyhf{}
\fancyfoot[C]{\thepage}

\begin{document}
	
	\noindent\textbf{6FMA84 - Matemática} 
	
	\begin{center}Reduzindo frações a um mesmo denominador (Versão estudante)
	\end{center}
	
	\noindent\textbf{Nome:} \underline{\hspace{10cm}}
	\noindent\textbf{Data:} \underline{\hspace{4cm}}
	
	%\section*{Questões de Matemática}
	
	\begin{multicols}{2}
		\noindent Para reduzir duas frações ao mesmo denominador, encontramos o menor múltiplo comum dos denominadores de tais frações. \\
		Por exemplo, $\frac{3}{4}$ e $\frac{4}{5}$. O menor múltiplo comum de 4 e 5 é 20. Logo $\frac{3}{4} = \frac{3 \cdot 5}{4 \cdot 5} = \frac{15}{20}$ e $\frac{4}{5} = \frac{4 \cdot 4}{5 \cdot 4} = \frac{16}{20}$. \\
		\noindent\textsubscript{-----------------------------------------------------------------------}
		\begin{enumerate} 
			\item Transforme a fração $\frac{1}{4}$ numa equivalente de denominador 24. \\\\\\\\\\\\\\\\\\\\
			\item Transforme as frações $\frac{1}{5}$ e $\frac{1}{6}$ em frações equivalentes de mesmo denominador. \\\\\\\\\\\\\\\\\\\\
			\item Transformar as frações $\frac{1}{3}, \frac{1}{6}$ e $\frac{1}{9}$ em frações equivalentes de mesmo denominador. \\\\\\\\\\\\\\\\\\\\
			\item Ache uma fração equivalente a $\frac{3}{5}$ que tenha 45 como denominador. \\\\\\\\\\\\\\\\\\\\
			\item Reduza as frações ao mesmo denominador:
			\begin{enumerate}[a)]
				\item $\frac{5}{6}$ e $\frac{6}{7}$. \\\\\\\\
				\item $\frac{3}{4}, \frac{4}{5}$ e $\frac{5}{6}$. \\\\\\\\\\\\\\\\\\\\
			\end{enumerate}
			%39 a 42
			\item A fração $\frac{3}{4}$ é equivalente a uma fração cujo denominador é 48. Qual é o numerador dessa fração?
			\item Simplifique as frações a seguir, tornando-as irredutíveis:
			\begin{enumerate}[a)]
				\item $\frac{14}{112}$ \\\\\\\\\\\\\\\\\\\\
				\item $-\frac{264}{484}$ \\\\\\\\\\\\\\\\\\\\
				\item $\frac{328}{738}$ \\\\\\\\\\\\\\\\\\\\
				\item $\frac{964}{1687}$ \\\\\\\\\\\\\\\\\\\\
				\item -$\frac{2960}{2220}$ \newpage
			\end{enumerate}
			\item Transforme as frações $\frac{3}{4}, \frac{5}{8}, \frac{7}{10}, \frac{12}{13}$ em frações respectivamente equivalentes, de mesmo denominador.  \\\\\\\\\\\\\\\\\\\\\\\\\\\\\\\\\\\\\\\\\\\\\\\\\\\\\\\\\\\\\\\\\\\\\\\\
			\item É possível obter uma fração de numerador e denominador inteiros que seja equivalente a $\frac{4}{9}$ e cujo denominador seja 59? Justifique sua resposta.  \\\\\\\\\\\\
		\end{enumerate}
		$~$ \\ $~$ \\ $~$ \\ $~$ \\ $~$ \\ $~$ \\ $~$ \\ $~$ \\ $~$ \\ $~$ \\ $~$ \\ $~$ \\ $~$ \\ $~$ \\ $~$ \\ $~$ \\ $~$ \\ $~$ \\ $~$ \\ $~$ \\ $~$ \\ $~$ \\ $~$ \\ $~$ \\ $~$ \\ $~$ \\ $~$ \\ $~$ \\ $~$
	\end{multicols}
\end{document}