\documentclass[a4paper,14pt]{article}
\usepackage{float}
\usepackage{extsizes}
\usepackage{amsmath}
\usepackage{amssymb}
\everymath{\displaystyle}
\usepackage{geometry}
\usepackage{fancyhdr}
\usepackage{multicol}
\usepackage{graphicx}
\usepackage[brazil]{babel}
\usepackage[shortlabels]{enumitem}
\usepackage{cancel}
\columnsep=2cm
\hoffset=0cm
\textwidth=8cm
\setlength{\columnseprule}{.1pt}
\setlength{\columnsep}{2cm}
\renewcommand{\headrulewidth}{0pt}
\geometry{top=1in, bottom=1in, left=0.7in, right=0.5in}

\pagestyle{fancy}
\fancyhf{}
\fancyfoot[C]{\thepage}

\begin{document}
	
	\noindent\textbf{8FMA60~Matemática} 
	
	\begin{center}Revisão: dízimas periódicas - Fração geratriz (Versão estudante)
	\end{center}
	
	\noindent\textbf{Nome:} \underline{\hspace{10cm}}
	\noindent\textbf{Data:} \underline{\hspace{4cm}}
	
	%\section*{Questões de Matemática}
	
	
    \begin{multicols}{2}
    	Transformando dízimas periódicas em frações: \\
    	\begin{itemize}
    		\item $x = 0,3$
    		\begin{center}
    			$~~~~~~10x = 3,333...$\\
    			$-~~~~~~~x = 0,333...$\\
    			\textsubscript{\textbf{--------------------------------------}}\\
    			$9x = 3$
    		\end{center}
    	    \begin{center}
    	    	$~~~~~~~~x = \frac{3}{9} = \frac{1}{3}$
    	    \end{center}
    		\item $x = 0,16$ \\
    		\begin{center}
    			$~~~~~~100x = 16,666...$\\
    			$-~~~~10x = 1,666...$\\
    			\textsubscript{\textbf{--------------------------------------}}\\
    			$90x = 15$ \\
    		\end{center}
    	    \begin{center}
    	    	$~~~~~~~~~x = \frac{15}{90} = \frac{1}{6}$
    	    \end{center}
        Um método para transformar dízima periódica em fração:
        \item $0,\overline{5} = \frac{5}{9}$
        \item $0,\overline{37} = \frac{37}{99}$
        \item $2,\overline{3} = \frac{23 - 2}{9} = \frac{21}{9} = \frac{7}{3}$
        \item $0,1\overline{6} = \frac{16 - 1}{90} = \frac{15}{90} = \frac{1}{6}$
        \item $2,1\overline{3} = \frac{213 - 21}{90} = \frac{192}{90} = \frac{32}{15}$
    	\end{itemize}
    	
    	\textsubscript{---------------------------------------------------------------------}
    	\begin{enumerate}
    		\item Determine as geratrizes das dízimas periódicas a seguir:
    		\begin{enumerate}[a)]
    		    \item $0,777...$ \\\\\\\\\\
    		    \item $3,1818...$ \\\\\\\\\\
    		    \item $4,\overline{3}$ \\\\\\\\\\
    		    \item $0,\overline{25}$ \\\\\\\\\\\\\\\\\\\\
    	    \end{enumerate}
            \item Ache as geratrizes de:
            \begin{enumerate}[a)]
            	\item $0,\overline{46}$ \\\\\\\\\\
            	\item $1,2\overline{8}$ \\\\\\\\\\
            	\item $0,4\overline{125}$ \\\\\\\\\\
            	\item $3,\overline{213}$ \\\\\\\\\\
            \end{enumerate}
            \item Sejam $a = 3,777...$ e $b = 5,333...$~. Calcule:
            \begin{enumerate}[a)]
            	\item $a + b$ \\\\\\\\\\\\\\\\\\\\
            	\item $a - b$ \\\\\\\\\\\\\\
            	\item $\frac{a}{b}$ \\\\\\\\\\
            \end{enumerate}
            \item A fração geratriz da dízima periódica 0,92 é:
            \begin{enumerate}[a)]
            	\item $\frac{41}{99}$
            	\item $\frac{83}{90}$
            	\item $\frac{92}{99}$
            	\item $\frac{89}{90}$
            	\item $\frac{79}{99}$
            \end{enumerate}
            \item O resultado da divisão $\frac{b}{a}$, com $a = 0,272727...$ e $b = 0,4222...$ é:
            \begin{enumerate}[a)]
            	\item $\frac{204}{181}$
            	\item $\frac{301}{125}$
            	\item $\frac{193}{125}$
            	\item $\frac{209}{135}$
            	\item $\frac{145}{121}$
            \end{enumerate}
        \end{enumerate}  
    \end{multicols}
\end{document}