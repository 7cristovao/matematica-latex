\documentclass[a4paper,14pt]{article}

\usepackage{comment} % Para comentar várias linhas ao mesmo tempo

%matemática
\usepackage{amsmath}
\usepackage{amssymb}

%diagramação
\usepackage{extsizes}
\everymath{\displaystyle}
\usepackage{geometry}
\usepackage{fancyhdr}
\usepackage{multicol}
\usepackage{graphicx}
\usepackage[brazil]{babel}
\usepackage[shortlabels]{enumitem}
\usepackage{cancel}
\usepackage{textcomp}
\usepackage{tcolorbox}

%tabelas
\usepackage{array} % Para melhor formatação de tabelas
\usepackage{longtable}
\usepackage{booktabs}  % Para linhas horizontais mais bonitas
\usepackage{float}   % Para usar o modificador [H]
\usepackage{caption} % Para usar legendas em tabelas
\usepackage{wrapfig} % Para usar tabelas e figuras flutuantes
\usepackage{xcolor} % Para cores do fundo de tabelas
\usepackage{colortbl} % Para cores do fundo de tabelas

%tikzpicture
\begin{comment}
	\usepackage{tikz}
	\usepackage{scalerel}
	\usepackage{pict2e}
	\usepackage{tkz-euclide}
	\usetikzlibrary{calc}
	\usetikzlibrary{patterns,arrows.meta}
	\usetikzlibrary{shadows}
	\usetikzlibrary{external}
\end{comment}


%pgfplots
\usepackage{pgfplots}
\pgfplotsset{compat=newest}
\usepgfplotslibrary{statistics}
\usepgfplotslibrary{fillbetween}

%colours
\usepackage{xcolor}



\columnsep=2cm
\hoffset=0cm
\textwidth=8cm
\setlength{\columnseprule}{.1pt}
\setlength{\columnsep}{2cm}
\renewcommand{\headrulewidth}{0pt}
\geometry{top=1in, bottom=1in, left=0.7in, right=0.5in}

\pagestyle{fancy}
\fancyhf{}
\fancyfoot[C]{\thepage}

\begin{document}
	
	\noindent\textbf{6FMA23 - Matemática} 
	
	\begin{center}Ordem em $\mathbb{Z}$ (Versão estudante)
	\end{center}
	
	\noindent\textbf{Nome:} \underline{\hspace{10cm}}
	\noindent\textbf{Data:} \underline{\hspace{4cm}}
	
	%\section*{Questões de Matemática}
	
	\begin{multicols}{2}
		\noindent São dados dois números inteiros, $a$ e $b$. Dizemos que $a$ é maior que $b$ (ou que $b$ é menor que $a$) se o número $a$ está à direita do número $b$. Indicamos isso por $a > b$ ou $b < a$. \\
		\begin{itemize}
			\item $a \geq b$ significa que $a > b$ ou $a = b$.
			\item $a \leq b$ significa que $a < b$ ou $a = b$.
			\item $a < b < c$ significa que $a < b$ e $b < c$.
			\item A negação de $a \leq b$ é $a > b$.
			\item A negação de $a < b$ é $a \geq b$.
			\item A negação de $a \geq b$ é $a < b$.
			\item A negação de $a > b$ é $a \leq b$.
		\end{itemize}
		\noindent\textsubscript{-----------------------------------------------------------------------}
		\begin{enumerate} 
			\item Assinale \textbf{V} (verdadeiro) ou \textbf{F} (falso).
			\begin{enumerate}[a)]
				\item (~~) $4 \geq 6$
				\item (~~) $-2 \leq 7$
				\item (~~) $3 > -4$
				\item (~~) $-8 \leq -2$
				\item (~~) $7 > 10$
				\item (~~) $-12 \geq 12$
				\item (~~) $21 < -10$
				\item (~~) $-1 < -5$
				\item (~~) $-5 > 0$
				\item (~~) $-6 \leq 3$
				\item (~~) $2 > -4$
				\item (~~) $-15 < -8$
				\item (~~) $0 \geq 0$
				\item (~~) $0 \leq 0$
				\item (~~) $-5 \geq -5$
			\end{enumerate}
			\item Assinale \textbf{V} (verdadeiro) ou \textbf{F} (falso).
			\begin{enumerate}[a)]
				\item (~~) Para todo $x$ inteiro, $x \geq$ x.
				\item (~~) Dados $x$ e $y$ inteiros, então ocorre uma das situações: $x = y$ ou $x > y$ ou $y > x$.
				\item (~~) Existem $x$ e $y$ inteiros tais que $x < y$ e $x > y$.
				\item (~~) Se $x$ e $y$ são inteiros e $x$ é negativo e $y$ é positivo, então $x < y$.
				\item (~~) Existem números inteiros $x$ e $y$ tais que $x \leq y$ e $x \geq y$.
				\item (~~) Se $a, b$ e $c$ são números inteiros, temos que se $a \leq b$ e $b \leq c$, então $a \leq c$.
			\end{enumerate}
			\item Assinale \textbf{V} (verdadeiro) ou \textbf{F} (falso).
			\begin{enumerate}[a)]
				\item (~~) $-1 < 2 < 7$
				\item (~~) $-2 \leq -8 \leq 5$
				\item (~~) $8 < 1 < -2$
				\item (~~) $-3 < -1 \leq 2$
				\item (~~) $-5 > -3 > 1$
				\item (~~) $-3 \leq 2 \leq 7$
				\item (~~) $-2 > 0 \geq 4$
				\item (~~) $0 \geq 1 > -2$
				\item (~~) $3 \geq 3 \geq -1$
				\item (~~) $-6 > 1 \geq -2$
			\end{enumerate}
			\item Complete.
			\begin{enumerate}[a)]
				\item A negação de $x \leq 2$ é $\underline{~~~~~~~~~~~~~~~~~~~~~~~~~~~~~~~~}$.
				\item A negação de $3 > y$ é $\underline{~~~~~~~~~~~~~~~~~~~~~~~~~~~~~~~~}$.
				\item A negação de $b \leq -2$ é $\underline{~~~~~~~~~~~~~~~~~~~~~~~~~~~~~~~~}$.
				\item Não é verdade que $a > 1$ quando $\underline{~~~~~~~~~~~~~~~~~~~~~~~~~~~~~~~~}$.
				\item A negação de $x > 0$ é $\underline{~~~~~~~~~~~~~~~~~~~~~~~~~~~~~~~~}$.
				\item Não é verdade que $8 < x$ quando $\underline{~~~~~~~~~~~~~~~~~~~~~~~~~~~~~~~~}$.
				\item A negação de $x < x$ é $\underline{~~~~~~~~~~~~~~~~~~~~~~~~~~~~~~~~}$.
				\item Não é verdade que $x \geq -3$ quando $\underline{~~~~~~~~~~~~~~~~~~~~~~~~~~~~~~~~}$.
			\end{enumerate}
			\item Use letras para simbolizar:
			\begin{enumerate}[a)]
				\item os números que não são menores do que 6. \\\\\\\\\\
				\item a idade, em anos, no Brasil, em que é obrigatório votar. \\\\\\\\\\
				\item a idade, em anos, no Brasil, em que você pode, mas não é obrigado a votar. \\\\\\\\\\
			\end{enumerate}
			%81 a 84
			\item Assinale \textbf{V} (verdadeiro) ou \textbf{F} (falso).
			\begin{enumerate}[a)]
				\item (~~) $-1 < 2 < 3$
				\item (~~) $-7 \leq 1 < 2$
				\item (~~) $-5 > 2 \geq 2$
				\item (~~) $9 < -1 < 8$
				\item (~~) $7 > 14 > 0$
				\item (~~) $-5 < -1 \leq 5$
				\item (~~) $8 \geq 6 > 0$
				\item (~~) $-3 < -4 < -5$
				\item (~~) $7 > 5 \geq 0$
				\item (~~) $-3 < 8 \leq 8$ \newpage
			\end{enumerate}
			\item Para o próximo feriado, a previsão do tempo diz que a temperatura mínima será de -2°C e a máxima de 15°C. Sendo $t$ a temperatura prevista, em °C, escreva os valores que $t$ pode assumir, usando desigualdades. \\\\\\\\\\\\\\\\\\\\
			\item Complete.
			\begin{enumerate}[a)]
				\item A negação de $-5 \leq -10$ é .....
				\item A negação de $-5 > -10$ é .....
				\item A negação de $8 < 8$ é ..... .
				\item A negação de $-1 \geq 2$ é ..... .
				\item A negação de $x \leq 0$ é ..... .
				\item A negação de $-6 > x$ é ..... .
				\item A negação de $x \geq 1 200$ é ..... .
				\item A negação de $x \leq x$ é ..... .
			\end{enumerate}
			\item Assinale \textbf{V} (verdadeiro) ou \textbf{F} (falso) (se você sentir dificuldade em algum item, desenhe a reta). Para $a, b, c$ inteiros:
			\begin{enumerate}[a)]
				\item (~~) se $a \leq b$ e $b \leq c$, então $a \leq c$.
				\item (~~) sempre ocorre $a \leq a$.
				\item (~~) se $a \leq b$ e $b \leq a$, então $a = b$.
				\item (~~) se $a \leq b, b \leq c$ e $c \leq a$, então $a = b = c$.
				\item (~~) se $a \geq b$ e $a \geq c$, então $b \geq c$.
			\end{enumerate}
		\end{enumerate}
		$~$ \\ $~$ \\ $~$ \\ $~$ \\ $~$ \\ $~$ \\ $~$ \\ $~$ \\ $~$ \\ $~$ \\ $~$ \\ $~$ \\ $~$ \\ $~$ \\ $~$ \\ $~$ \\ $~$ \\ $~$ \\ $~$ \\ $~$ \\ $~$ \\ $~$ \\ $~$ \\ $~$ \\ $~$ \\ $~$ \\ $~$ \\ $~$ \\ $~$ \\ $~$ \\ $~$ \\ $~$ \\ $~$ \\
	\end{multicols}
\end{document}