\documentclass[a4paper,14pt]{article}
\usepackage{extsizes}
\usepackage{amsmath}
\usepackage{amssymb}
\everymath{\displaystyle}
\usepackage{geometry}
\usepackage{fancyhdr}
\usepackage{multicol}
\usepackage{graphicx}
\usepackage[brazil]{babel}
\usepackage[shortlabels]{enumitem}
\usepackage{cancel}
\columnsep=2cm
\hoffset=0cm
\textwidth=8cm
\setlength{\columnseprule}{.1pt}
\setlength{\columnsep}{2cm}
\renewcommand{\headrulewidth}{0pt}
\geometry{top=1in, bottom=1in, left=0.7in, right=0.5in}

\pagestyle{fancy}
\fancyhf{}
\fancyfoot[C]{\thepage}

\begin{document}
	
	\noindent\textbf{EF08MA08-A~-~Matemática} 
	
	\begin{center}
		\textbf{Revisão: problemas que contam histórias ou apresentam situações (Versão estudante)}
	\end{center}
	
	
	\noindent\textbf{Nome:} \underline{\hspace{10cm}}
    \noindent\textbf{Data:} \underline{\hspace{4cm}}
	
	%\section*{Questões de Matemática}
	
	\begin{multicols}{2}
	\begin{enumerate}	
		\item Duas pessoas farão conjuntamente a pintura de um muro, cada uma trabalhando a partir de uma das extremidades. Se uma delas pintar 2/5 do muro e a outra os 15 m restantes, a extensão deste muro é de: 
		\begin{enumerate}[a)]
			\item 25 m
			\item 35 m
			\item 42 m
			\item 45 m
			\item 20 m
	    \end{enumerate}
        \item A bilheteria de um teatro só trabalha com ingressos "lugares A" e "lugares B" com preços de R\$ 16,00 e R\$ 10, respectivamente. Uma pessoa adquiriu, por R\$ 192,00, 15 ingressos. Quantos ingressos de "lugares A" e quantos de "lugares B" ela adquiriu, respectivamente?
        \begin{enumerate}[a)]
        	\item 3 e 12.
        	\item 6 e 9.
        	\item 7 e 8.
        	\item 5 e 10.
        	\item 1 e 14.
        \end{enumerate}
        \item Na criação de uma placa, um funcionário tem espaço de 9 cm de largura para cada letra do título. Se no título houvesse mais dez letras, o espaço seria reduzido para 6 cm. O número de letras que formam esse título é:
        \begin{enumerate}[a)]
        	\item 20
        	\item 25
        	\item 15
        	\item 10
        	\item 30
        \end{enumerate}
        \item Descubra um número, de acordo com as informações dadas a seguir:
        \begin{itemize}
        	\item É um número de dois algarismos.
        	\item O algarismo das dezenas é o triplo do algarismo das unidades.
        	\item Trocando os dois algarismos de lugar, obtemos um segundo número. Se subtraio o segundo número do primeiro, o resultado é 54.
        \end{itemize}
    \end{enumerate}        
    \end{multicols}    

\end{document}