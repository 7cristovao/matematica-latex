\documentclass[a4paper,14pt]{article}

\usepackage{comment} % Para comentar várias linhas ao mesmo tempo

%matemática
\usepackage{amsmath}
\usepackage{amssymb}

%diagramação
\usepackage{extsizes}
\everymath{\displaystyle}
\usepackage{geometry}
\usepackage{fancyhdr}
\usepackage{multicol}
\usepackage{graphicx}
\usepackage[brazil]{babel}
\usepackage[shortlabels]{enumitem}
\usepackage{cancel}
\usepackage{textcomp}
\usepackage{tcolorbox}

%tabelas
\usepackage{array} % Para melhor formatação de tabelas
\usepackage{longtable}
\usepackage{booktabs}  % Para linhas horizontais mais bonitas
\usepackage{float}   % Para usar o modificador [H]
\usepackage{caption} % Para usar legendas em tabelas
\usepackage{wrapfig} % Para usar tabelas e figuras flutuantes
\usepackage{xcolor} % Para cores do fundo de tabelas
\usepackage{colortbl} % Para cores do fundo de tabelas

%tikzpicture
\begin{comment}
	\usepackage{tikz}
	\usepackage{scalerel}
	\usepackage{pict2e}
	\usepackage{tkz-euclide}
	\usetikzlibrary{calc}
	\usetikzlibrary{patterns,arrows.meta}
	\usetikzlibrary{shadows}
	\usetikzlibrary{external}
\end{comment}


%pgfplots
\usepackage{pgfplots}
\pgfplotsset{compat=newest}
\usepgfplotslibrary{statistics}
\usepgfplotslibrary{fillbetween}

%colours
\usepackage{xcolor}



\columnsep=2cm
\hoffset=0cm
\textwidth=8cm
\setlength{\columnseprule}{.1pt}
\setlength{\columnsep}{2cm}
\renewcommand{\headrulewidth}{0pt}
\geometry{top=1in, bottom=1in, left=0.7in, right=0.5in}

\pagestyle{fancy}
\fancyhf{}
\fancyfoot[C]{\thepage}

\begin{document}
	
	\noindent\textbf{6FMA131 - Matemática} 
	
	\begin{center}Revisão: porcentagem (Versão estudante)
	\end{center}
	
	\noindent\textbf{Nome:} \underline{\hspace{10cm}}
	\noindent\textbf{Data:} \underline{\hspace{4cm}}
	
	%\section*{Questões de Matemática}
	
	\begin{multicols}{2}
	    \noindent 
	    \begin{itemize}
	    	\item $x\%$ de $y$ é $\frac{x \cdot y}{100}$.
	    	\item Se queremos saber quanto $a$ representa de $b$, em porcentagem, fazemos $\frac{a}{b} \cdot 100\%$.
	    \end{itemize}
		\noindent\textsubscript{--------------------------------------------------------------------------}
		\begin{enumerate} 
			\item Escreva em forma de fração irredutível as porcentagens:
			\begin{enumerate}[a)] 
				\item 85\% = \\\\\\
				\item 17\% = \\\\\\
				\item 2\% = \\\\\\
				\item 30\% = \\\\\\
				\item 260\% = \\\\\\
				\item 100\% = \\\\\\
				\item 2.8\% = \\\\\\
			\end{enumerate}
			\item Escreva em forma de porcentagem:
			\begin{enumerate}[a)] 
				\item $\frac{9}{100} = $ \\\\\\\\
				\item $0,03 = $ \\\\\\\\
				\item $\frac{7}{25} = $ \\\\\\\\
				\item $1,41 = $ \\\\\\\\
				\item $0,83 = $ \newpage
			\end{enumerate}
			\item Responda:
			\begin{enumerate}[a)] 
				\item Quanto $a$ representa de $b$ nos casos:
				\begin{enumerate}[I.] 
					\item $a = 6$ e $b = 10$. \\\\\\\\\\
					\item $a = 4$ e $b = 12$. \\\\\\\\\\
					\item $a = 17$ e $b = 250$. \\\\\\\\\\
				\end{enumerate}
				\item Calcule quanto representa $a\%$ de $b$ para:
				\begin{enumerate}[I.] 
					\item $a = 44$ e $b = 20$. \\\\\\\\
					\item $a = 37$ e $b = 243$. \\\\\\\\
					\item $a = 71$ e $b = 12$. \\\\\\\\
				\end{enumerate}
			\end{enumerate}
			\item Renata ganhou R\$ 260,00 e gastou 30\% deste valor com o seu presente de aniversário. Sabendo que ela guardou 55\% do que ganhou para o próximo mês, quanto sobrou para ela gastar durante este mês? \\\\\\\\\\\\\\\\\\\\\\\\
			\item Pedro conseguiu economizar R\$ 800,00 ao longo de um ano. No início do ano seguinte, ele gastou 20\% deste valor com o relógio que tanto queria e comprou um presente para seu pai com 30\% do restante. Com quantos reais ele ficou após essas compras? \\\\\\\\\\\\\\\\\\\\\\\\\\
			%21 a 29
			\item Escreva na forma de fração irredutível:
			\begin{enumerate}[a)] 
				\item $0,4\%$ \\\\\\
				\item $35\%$ \\\\\\
				\item $42\%$ \\\\\\
				\item $75\%$ \\\\\\
				\item $190\%$ \\\\\\
				\item $276\%$ \\\\\\
			\end{enumerate}
			\item Escreva na forma de decimal:
			\begin{enumerate}[a)] 
				\item $0,3\%$ \\\\\\
				\item $1,2\%$ \\\\\\
				\item $60\%$ \\
				\item $85\%$ \\\\\\
				\item $216\%$ \\\\\\
				\item $320\%$ \\\\\\
			\end{enumerate}
			\item Escreva na forma de porcentagem:
			\begin{enumerate}[a)] 
				\item $\frac{40}{100}$ \\\\\\
				\item $\frac{2}{10}$ \\\\\\
				\item $\frac{18}{20}$ \\\\\\
				\item $\frac{180}{600}$ \\\\\\
				\item $\frac{24}{40}$ \\\\\\
				\item $\frac{21}{70}$ \newpage
			\end{enumerate}
			\item Escreva os decimais em forma de porcentagem:
			\begin{enumerate}[a)] 
				\item $0,084$ \\\\\\
				\item $0,213$ \\\\\\
				\item $0,572$ \\\\\\
				\item $0,4$ \\\\\\
				\item $2,67$ \\\\\\
				\item $3,8$ \\\\\\
			\end{enumerate}
			\item Calcule:
			\begin{enumerate}[a)] 
				\item $30\%$ de 73. \\\\\\
				\item $45\%$ de 865. \\\\\\
				\item $36\%$ de 56. \\
				\item $120\%$ de 70. \\\\\\
				\item $12\%$ de 25\% de 340. \\\\\\
				\item $7\%$ de 60\% de 3 000. \\\\\\
			\end{enumerate}
			\large\item Uma pesquisa recente aponta que 8 em cada 10 homens brasileiros dizem cuidar de sua beleza, não apenas de sua higiene pessoal. \\
			\footnotesize Adaptado de M. Caetano; R. Soreiro; R. Davino. \\
			\footnotesize "Cosméticos". \textit{Superinteresssante}, n. 304, maio/2012. \\
			\normalsize Outra maneira de representar esse resultado é exibindo o valor percentual dos homens brasileiros que dizem cuidar de sua beleza. \\
			Qual é o valor percentual que faz essa representação?
			\begin{enumerate}[a)]
				\item 80\%
				\item 8\%
				\item 0,8\%
				\item 0,08\%
				\item 0,008\% \newpage
			\end{enumerate}
			\item Sandra recebe R\$ 3.200,00 por mês e gasta R\$ 448,00 com impostos. Quantos por cento de seu salário ela gasta com impostos? \\\\\\\\\\\\\\\\
			\item Leia o seguinte texto e responda às questões. \\\\
			\large Segundo pesquisas do IBGE no mês de março, a safra de arroz, milho, soja e feijão de 2019 terá uma alta de 1,6 \%. A produção desses grãos para 2019 foi estimada em 230 milhões de toneladas, o que equivale a 1,6\% acima da safra de 2018. O arroz, o milho e a soja representam 93\% da estimativa da produção total. \\
			Ao analisar uma perspectiva nacional, Mato Grosso é considerado maior produtor de grãos, seguido de Paraná e Rio Grande do Sul. Esses três estados juntos representam 58 \% do total nacional. \\
			A primeira safra do feijão teve uma produção anual de 1,6 milhão de toneladas em 2018 e a estimativa do IBGE é que haja uma redução de 8 \%. \\
			Essa redução se deve ao clima e ao fato de os produtores preferirem o plantio de soja na 1ª safra, reduzindo a área cultivada de feijão. \\\\
			\footnotesize Fonte: $<$https://agenciadenoticias.ibge.gov.br/\\media/com\underline{~}mediaibge/arquivos/3260ca45d84508\\bbdb1e32bec3bd960c.pdf$>$. \normalsize
			\begin{enumerate}[a)]
				\item Qual é a estimativa da produção de grãos para 2019? \\\\\\\\\\
				\item Quantas toneladas de feijão o IBGE estima que serão produzidas em 2019? \\\\\\\\\\
				\item Qual é a estimativa de produção anual da 1ª safra do feijão para 2019? \\\\\\\\\\
			\end{enumerate}
			\item Em uma população com 150 000 pessoas, 72\% são motoristas e, destes, 46\% tem carro preto. Determine o número de motoristas que não tem carro preto. \\\\\\\\\\
		\end{enumerate}
	\end{multicols}
\end{document}