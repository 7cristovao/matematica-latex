\documentclass[a4paper,14pt]{article}
\usepackage{extsizes}
\usepackage{amsmath}
\usepackage{amssymb}
\everymath{\displaystyle}
\usepackage{geometry}
\usepackage{fancyhdr}
\usepackage{multicol}
\usepackage{graphicx}
\usepackage[brazil]{babel}
\usepackage[shortlabels]{enumitem}
\usepackage{cancel}
\columnsep=2cm
\hoffset=0cm
\textwidth=8cm
\setlength{\columnseprule}{.1pt}
\setlength{\columnsep}{2cm}
\renewcommand{\headrulewidth}{0pt}
\geometry{top=1in, bottom=1in, left=0.7in, right=0.5in}

\pagestyle{fancy}
\fancyhf{}
\fancyfoot[C]{\thepage}

\begin{document}
	
	\noindent\textbf{EF08MA08-A~-~Matemática} 
	
	\begin{center}
		\textbf{Revisão: problemas que contam histórias ou apresentam situações (Versão professor)}
	\end{center}
	
	
	\noindent\textbf{Nome:} \underline{\hspace{10cm}}
    \noindent\textbf{Data:} \underline{\hspace{4cm}}
	
	%\section*{Questões de Matemática}
	
	\begin{multicols}{2}
	\begin{enumerate}	
		\item Duas pessoas farão conjuntamente a pintura de um muro, cada uma trabalhando a partir de uma das extremidades. Se uma delas pintar 2/5 do muro e a outra os 15 m restantes, a extensão deste muro é de: 
		\begin{enumerate}[a)]
			\item 25 m
			\item 35 m
			\item 42 m
			\item 45 m
			\item 20 m
	    \end{enumerate}
    
        Resposta: alternativa a) \\
        
        Solução: O muro terá $\frac{5}{5}$ de comprimento, uma pessoa vai pintar $\frac{2}{5}$ e a outra 15m, que será os $\frac{3}{5}$ restantes. \\
        Então: se $\frac{3}{5}$ = 15m, \\$\frac{1}{5}$ será de 5m $\times$ 5 = 25m.
        
        \item A bilheteria de um teatro só trabalha com ingressos "lugares A" e "lugares B" com preços de R\$ 16,00 e R\$ 10, respectivamente. Uma pessoa adquiriu, por R\$ 192,00, 15 ingressos. Quantos ingressos de "lugares A" e quantos de "lugares B" ela adquiriu, respectivamente?
        \begin{enumerate}[a)]
        	\item 3 e 12.
        	\item 6 e 9.
        	\item 7 e 8.
        	\item 5 e 10.
        	\item 1 e 14.
        \end{enumerate}
    
        Resposta: alternativa c)
        
        Se A = 16,00 e B = 10,00 e 15(A, B) = R\$ 192,00 \\
        Solução: Se comprou 15 de ambos, então: A + B = 15 \\
        Se a soma dos gastos foi 192,00, então: 16A + 10B = 192 \\
        Portanto temos 2 equações \\
        $\begin{cases}
        	A + B = 15 \cdot (-10) \\
        	16A + 10B = 192 \\
        \end{cases}$\\
        Pelo método da soma, multiplica-se a 1ª equação por -10. \\
        $\begin{cases}
        	-10A~\cancel{- 10B} = 150 \\
        	~~16A~\cancel{+ 10B} = 192 \\
        \end{cases}$ \\
        $
        \underline{\hspace{5cm}} \\
        $ \\
        $\cdot~~~~~6A + 0B = 42$ \\
        $\cdot~~~~~6A = 42$ \\
        $\cdot~~~~~A = 7$ \\
        \\
        Se $A + B = 15$\\
        $7 + B = 15$ \\
        $B = 15 - 7$ \\
        $B = 8$ \\
        
        \newpage
    
        \item Na criação de uma placa, um funcionário tem espaço de 9 cm de largura para cada letra do título. Se no título houvesse mais dez letras, o espaço seria reduzido para 6 cm. O número de letras que formam esse título é:
        \begin{enumerate}[a)]
        	\item 20
        	\item 25
        	\item 15
        	\item 10
        	\item 30
        \end{enumerate}
    
        Resposta: alternativa a) \\
        
        Vamos chamar o número inicial de letras do título de x.\\
        
        Segundo as informações fornecidas, quando havia x letras, o espaço disponível para cada letra era de 9 cm. Quando o título aumentou para x+10 letras, o espaço disponível para cada letra diminuiu para 6 cm.\\
        
        Podemos criar uma equação com essas informações:\\
        
        Quando havia x letras: 9x cm de espaço total para as letras.\\
        
        Quando há x+10 letras: 6(x+10) cm de espaço total para as letras. \\
        
        Assim:\\
        $9 \cdot x=(x+10) \cdot 6$\\\\
         
        $9x=6x+60$ \\
        $9x-6x=60$ \\
        $3x=60$ \\ \\
        $x = \frac{60}{3}$\\ \\
        $x=20$ \\
        
        \item Descubra um número, de acordo com as informações dadas a seguir:
        \begin{itemize}
        	\item É um número de dois algarismos.
        	\item O algarismo das dezenas é o triplo do algarismo das unidades.
        	\item Trocando os dois algarismos de lugar, obtemos um segundo número. Se subtraio o segundo número do primeiro, o resultado é 54.
        \end{itemize}
    
        Solução: análise combinatória \\
        (x, y) onde x = 3y \\

        	$Se~y = 1 \Rightarrow x = 3~N^\circ = 13 - 31$ \\
        	$Se~y = 2 \Rightarrow x = 6~N^\circ = 26 - 62$ \\
        	$Se~y = 3 \Rightarrow x = 9~N^\circ = 39 - 93$ \\
            \\ \\
        Invertendo os algarismos teremos: \\
        31 - 13 = 18 \\
        62 - 26 = 36 \\
        93 - 39 = 54 \\ \\
        Portanto os números são 93 e 39.
    
    \end{enumerate}        
    \end{multicols}    

\end{document}