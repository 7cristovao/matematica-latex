\documentclass[a4paper,14pt]{article}
\usepackage{extsizes}
\usepackage{amsmath}
\usepackage{amssymb}
\everymath{\displaystyle}
\usepackage{geometry}
\usepackage{fancyhdr}
\usepackage{multicol}
\usepackage{graphicx}
\usepackage[brazil]{babel}
\usepackage[shortlabels]{enumitem}
\usepackage{cancel}
\columnsep=2cm
\hoffset=0cm
\textwidth=8cm
\setlength{\columnseprule}{.1pt}
\setlength{\columnsep}{2cm}
\renewcommand{\headrulewidth}{0pt}
\geometry{top=1in, bottom=1in, left=0.7in, right=0.5in}

\pagestyle{fancy}
\fancyhf{}
\fancyfoot[C]{\thepage}

\begin{document}
	
	\noindent\textbf{EF08MA08-A~-~Matemática} 
	
	\begin{center}
		\textbf{Revisão: problemas que contam histórias ou apresentam situações - Exercícios Adicionais(Versão estudante)}
	\end{center}
	
	
	\noindent\textbf{Nome:} \underline{\hspace{10cm}}
    \noindent\textbf{Data:} \underline{\hspace{4cm}}
	
	%\section*{Questões de Matemática}
	
	\begin{multicols}{2}
	\begin{enumerate}
		\item Ramon tem hoje o quádruplo da idade de seu irmão Lucca. Quantos anos tinha Ramon quando Lucca nasceu, sabendo que juntos, atualmente, eles somam 30 anos?
		\vspace{18cm}
		\item Para uma excursão ao Aquário de São Paulo, uma escola alugou 4 ônibus de 50 lugares cada, que foram todos lotados. O triplo do número de meninas mais 50. Quantos meninos e quantas meninas foram ao passeio?
		\vspace{18cm}
		\item Numa estante, Mariana empilhou seus 15 livros, sendo que alguns têm 4cm de espessura e outros têm 7 cm de espessura. Quantos livros empilhados há de cada espessura, sabendo que a pilha de livros mede 81 cm? 
		\vspace{20cm}
		\item Tenho X reais para comprar y cadeiras idênticas para uma festa. Se eu comprar a quantidade necessária na loja Alis, cujo preço unitário é de R\$ 30,00, sobrarão R\$ 150,00; porém, caso eu queira comprar a quantidade necessária na loja Betis, cujo preço unitário é R\$ 40,00, faltarão R\$ 100,00. Logo, X vale:
		\begin{enumerate}[a)]
			\item R\$ 775,00
			\item R\$ 900,00
			\item R\$ 835,00
			\item R\$ 1.100,00
			\item R\$ 1.050,00
		\end{enumerate}
	    $~$ \\ $~$ \\ $~$ \\ $~$ \\ $~$ \\ $~$ \\ $~$ \\ $~$ \\ $~$ \\ $~$ \\ $~$ \\ $~$ \\ $~$ \\ $~$ \\ $~$ \\ $~$ \\ $~$ \\ $~$ \\ $~$ \\ $~$ \\ $~$ \\ $~$ \\ $~$ \\ $~$
	    %\item Escreva, quando possível, na forma $\sqrt[n]{a}$, para $n \in \mathbb{N^*}$ e $a \in \mathbb{R}$:
	    %\begin{enumerate}[a)]
	    %	\item $5^\frac{1}{8}$
	    %	\item
        %\end{enumerate}
    \end{enumerate}        
    \end{multicols}    
\end{document}