\documentclass[a4paper,14pt]{article}
\usepackage{extsizes}
\usepackage{amsmath}
\usepackage{amssymb}
\everymath{\displaystyle}
\usepackage{geometry}
\usepackage{fancyhdr}
\usepackage{multicol}
\usepackage{graphicx}
\usepackage[brazil]{babel}
\usepackage[shortlabels]{enumitem}
\usepackage{cancel}
\columnsep=2cm
\hoffset=0cm
\textwidth=8cm
\setlength{\columnseprule}{.1pt}
\setlength{\columnsep}{2cm}
\renewcommand{\headrulewidth}{0pt}
\geometry{top=1in, bottom=1in, left=0.7in, right=0.5in}

\pagestyle{fancy}
\fancyhf{}
\fancyfoot[C]{\thepage}

\begin{document}
	
	\noindent\textbf{7FMA150~-~Matemática} 
	
	\begin{center}
		\textbf{Potências - Propriedades (II) (Versão estudante)}
	\end{center}
	
	
	\noindent\textbf{Nome:} \underline{\hspace{10cm}}
    \noindent\textbf{Data:} \underline{\hspace{4cm}}
	
	%\section*{Questões de Matemática}
	
	\begin{multicols}{2}
		Admitindo definidas as potências e raízes valem as seguintes propriedades: \\
		P4. $a^m \cdot b^m = (a \cdot b)^m$ e $\sqrt[m]{a} \cdot \sqrt[m]{a \cdot b}$\\
		P5. $\frac{a^m}{b^m} = \bigg(\frac{a}{b}\bigg)^m$ e $\frac{\sqrt[m]{a}}{\sqrt[m]{b}} = \sqrt[m]{\frac{a}{b}}$ 
	\begin{enumerate}
		\item Calcule, obtendo o resultado na forma $a^\frac{m}{n}$:
		\begin{enumerate}[a)]
			\item $(11^4)^\frac{1}{4}$\\\\\\
			\item $(2^{-7})^{-\frac{3}{7}}$\\\\\\
			\item $(7^{-\frac{4}{9}})^\frac{9}{8}$\\\\\\
			\item $((5^{-3})^\frac{4}{9})^{-\frac{3}{4}}$\\\\\\
		\end{enumerate}
        \item Calcule:
        \begin{enumerate}[a)]
        	\item $\sqrt{\sqrt{3}}~$\\\\\\
        	\item $\sqrt{\sqrt{\sqrt{7}}}$\\\\\\
        	\item $\sqrt[5]{\sqrt[4]{\sqrt{5}}}$\\\\\\
        	\item $\sqrt[3]{\sqrt[-7]{\sqrt{2}}}$\\\\\\
        \end{enumerate}
        \item Obtenha o valor da expressão a seguir na forma de raiz:
        \begin{enumerate}[a)]
        	\item $\frac{\sqrt[4]{5} \cdot \sqrt[5]{\sqrt{5}} \cdot \sqrt[-4]{5}}{\sqrt[12]{5^9} \cdot \sqrt[6]{5^3}}$\\\\\\\\\\
        \end{enumerate}
        \item Simplifique as expressões a seguir:
        \begin{enumerate}[a)]
        	\item $4^\frac{2}{7} \cdot 3^\frac{2}{7}$\\\\\\
        	\item $5^\frac{1}{3} \cdot 3^\frac{1}{3} \cdot \bigg(\frac{1}{45}\bigg)^\frac{1}{3}$\\\\\\\\
        	\item $\sqrt[6]{2} \cdot \sqrt[6]{7} \cdot \sqrt[6]{5}$\\\\\\
        \end{enumerate}
        \item Aplique P5, calculando até onde for possível fazer a conta:
        \begin{enumerate}[a)]
        	\item $\frac{3^\frac{3}{5}}{7^\frac{3}{5}}$\\\\\\
        	\item $\frac{2^{-\frac{2}{9}}}{5^{-\frac{2}{9}}}$\\\\\\
        	\item $\frac{\sqrt[3]{5}}{\sqrt[3]{7}}$\\\\\\
        	\item $\frac{\sqrt{11}}{\sqrt{2}}$\\\\\\
        \end{enumerate}
        \item Calcule:
        \begin{enumerate}[a)]
        	\item $27^\frac{2}{3}$\\\\\\
        	\item $\sqrt{\sqrt{81}}$\\\\\\
        	\item $(\sqrt{11})^4$\\\\\\
        	\item $(\sqrt[6]{7})^3$\\\\\\
        \end{enumerate}
        \item Escreva na forma $\sqrt[n]{a^m}$, com \\$a \in \mathbb{R}$, $m$, $n \in \mathbb{N}$ e $n \neq 0$.
        \begin{enumerate}[a)]
        	\item $5^\frac{1}{3} \cdot 2^{1}{3} \cdot 10^{-\frac{1}{3}}$\\\\\\
        	\item $\sqrt[10]{9} \cdot \sqrt[5]{5}$\\\\\\
        	\item $\bigg( \frac{8^\frac{1}{6}}{\sqrt[12]{49}} \bigg)^{-4}$\\\\\\
        	\item $\frac{\sqrt[7]{8} \cdot \sqrt[7]{22} \cdot \sqrt[7]{9}}{\sqrt[7]{11} \cdot \sqrt[7]{16}}$\\\\\\
        \end{enumerate}
        \item Sendo $a$, $b$ e $c$ números reais positivos, a expressão\\ $\sqrt[3]{a^5 \cdot b^7 \cdot c} \cdot \sqrt[3]{a^3 \cdot b^2 \cdot c^4}$ é equivalente a:
        \begin{enumerate}[a)]
            \item $b^2 \cdot c \cdot \sqrt[3]{a \cdot c^2}$
            \item $a^4 \cdot b^2 \cdot c^3 \cdot \sqrt[3]{b \cdot c}$
            \item $a \cdot b \cdot c^2 \cdot \sqrt[3]{a^2 \cdot b}$
            \item $a^2 \cdot c \cdot \sqrt[3]{a \cdot b^2 \cdot c}$
            \item $a^2 \cdot b^3 \cdot c \cdot \sqrt[3]{a^2 \cdot c^2}$\\\\\\\\\\\\\\\\\\\\
        \end{enumerate}
        \item Sendo $a$, $b$ e $c$ números reais positivos e considerando \\\\$y = \frac{3x^2}{2z^2} \bigg(\frac{27x^3}{\sqrt[4]{64z^6}}\bigg)^{-\frac{2}{3}}$, logo $y$ equivale a:
        \begin{enumerate}[a)]
        	\item $\frac{1}{3z}$
        	\item $2x^2$
        	\item $\frac{3z^2}{\sqrt[4]{x^3}}$
        	\item $x$
        	\item $\frac{\sqrt[4]{x^2}}{3z \cdot x^6}$\\\\\\\\\\\\\\\\\\\\\\\\\\\\\\\\\\\\\
        \end{enumerate}
        \item Considerando $k^3 = 8^2$, $m^2 = 8^7$ e $n^5 = 8^6$, então $(k \cdot m \cdot n)^{30}$ é equivalente a:
        \begin{enumerate}[a)]
        	\item $8^{74}$
        	\item $8^{135}$
        	\item $8^{57}$
        	\item $8^{161}$
        	\item $8^{92}$
        \end{enumerate}
    $~$ \\ $~$ \\ $~$ \\ $~$ \\ $~$ \\ $~$ \\ $~$ \\ $~$ \\ $~$ \\ $~$ \\ $~$ \\ $~$ \\ $~$ \\ $~$ \\ $~$ \\ $~$ \\ $~$ \\ $~$ \\ $~$ \\ $~$ \\ $~$ \\ $~$ \\ $~$ \\ $~$ \\ $~$ \\ $~$ \\ $~$ \\ $~$ \\ $~$ \\ $~$
    \end{enumerate}        
    \end{multicols}
\end{document}