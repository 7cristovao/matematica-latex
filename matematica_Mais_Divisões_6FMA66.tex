\documentclass[a4paper,14pt]{article}
\usepackage{float}
\usepackage{extsizes}
\usepackage{amsmath}
\usepackage{amssymb}
\everymath{\displaystyle}
\usepackage{geometry}
\usepackage{fancyhdr}
\usepackage{multicol}
\usepackage{graphicx}
\usepackage[brazil]{babel}
\usepackage[shortlabels]{enumitem}
\usepackage{cancel}
\columnsep=2cm
\hoffset=0cm
\textwidth=8cm
\setlength{\columnseprule}{.1pt}
\setlength{\columnsep}{2cm}
\renewcommand{\headrulewidth}{0pt}
\geometry{top=1in, bottom=1in, left=0.7in, right=0.5in}

\pagestyle{fancy}
\fancyhf{}
\fancyfoot[C]{\thepage}

\begin{document}
	
	\noindent\textbf{6FMA66~Matemática} 
	
	\begin{center}Mais divisões (Versão estudante)
	\end{center}
	
	\noindent\textbf{Nome:} \underline{\hspace{10cm}}
	\noindent\textbf{Data:} \underline{\hspace{4cm}}
	
	%\section*{Questões de Matemática}
	
	
    \begin{multicols}{2}
    	\begin{enumerate}
    		\item Complete as divisões a seguir:
    		\begin{enumerate}[a)]
    			\item $105 \div 4$ \\\\\\\\
    			\item $523 \div 5$ \\\\\\\\
    			\item $421 \div 12$ \\\\\\\\
    			\item $953 \div 11$ \\\\\\\\
    		\end{enumerate}
    		\item Na loteria federal, o prêmio de R\$ 700.000,00 foi dividido entre 146 acertadores. Desprezando os centavos, determine a parte de cada ganhador.\\\\\\\\\\\\\\
    		\item Determine o quociente e o resto da divisão euclidiana de 11627 por 607. \\\\\\\\\\\\
    		\item Um caminhão deve transportar 19872 bombons em caixas que acomodam 16 bombons. Quantas caixas serão necessárias?\\\\\\\\\\\\\\\\
    		\textbf{Desafio olímpico}\\\\
    		Ao multiplicar um número inteiro por 506, Flávio esqueceu o zero e fez a multiplicação por 56, obtendo corretamente 1008. Se ele fizesse a conta com 506, qual seria o resultado obtido?
    		\begin{enumerate}[a)]
    			\item 9062
    			\item 9160
    			\item 9108
    			\item 9506
    			\item 9560
    		\end{enumerate}
    		\item Efetue as divisões a seguir:
    		\begin{enumerate}[a)]
    			\item $171 \div 24$ \\\\\\\\
    			\item $334 \div 31$ \\\\\\\\
    			\item $521 \div 42$ \\\\\\\\
    			\item $511 \div 32$ \\\\\\\\
    			\item $350 \div 43$ \\\\\\\\
    			\item $852 \div 52$ \\\\\\\\
    			\item $693 \div 63$ \\\\\\\\
    			\item $983 \div 88$ \\
    		\end{enumerate}
    	    \item Efetue as divisões a seguir:
    	    \begin{enumerate}[a)]
    	    	\item $254 \div 35$ \\\\\\\\
    	    	\item $443 \div 36$ \\\\\\\\
    	    	\item $390 \div 27$ \\\\\\\\
    	    	\item $342 \div 53$ \\\\\\\\
    	    	\item $631 \div 55$ \\\\\\\\
    	    	\item $567 \div 27$ \\\\\\\\
    	    	\item $682 \div 52$ \\\\\\\\
    	    	\item $500 \div 34$ \\
    	    \end{enumerate}    
            \item O número 33534 é divisível por 54? Justifique. \\\\\\\\\\\\\\
            \item André ganhou R\$ 503,00 em um jogo e resolveu dividir igualmente entre seus 6 filhos. Como 503 não é divisível por 6, ele determinou o quociente e o resto da divisão, cabendo a cada um o maior quociente possível. O resto do dinheiro ele usou para comprar balas ao preço de 9 por 1 real, dividindo igualmente o máximo de balas entre ele e seus filhos. As balas que sobraram, André deu para seu neto. Quantas balas cada um recebeu? \\\\\\\\\\\\\\\\\\\\\\\\\\\\\\\\\\
            \item Um número dividido por 47 resulta em 12. Qual é o número? \\\\\\\\\\\\\\
            \item Marcos decidiu colocar um aquário em sua sala e após montá-lo, foi a um $pet shop$ com uma certa quantia em dinheiro para comprar alguns peixes. Depois de pesquisar, decidiu levar 16 peixes pequenos que custavam R\$ 12,00 cada. Se Marcos recebeu 8 reais de troco, qual era a quantia levada por ele?
        \end{enumerate}
    $~$ \\ $~$ \\ $~$ \\ $~$ \\ $~$ \\ $~$ \\ $~$ \\ $~$ \\ $~$ \\ $~$ \\ $~$ \\ $~$ \\ $~$ \\ $~$ \\ $~$ \\ $~$ \\ $~$ \\ $~$ \\ $~$ \\ $~$ \\ $~$ 
    \end{multicols}
\end{document}