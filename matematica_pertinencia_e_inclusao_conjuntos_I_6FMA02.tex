\documentclass[a4paper,14pt]{article}

\usepackage{comment} % Para comentar várias linhas ao mesmo tempo

%matemática
\usepackage{amsmath}
\usepackage{amssymb}

%diagramação
\usepackage{extsizes}
\everymath{\displaystyle}
\usepackage{geometry}
\usepackage{fancyhdr}
\usepackage{multicol}
\usepackage{graphicx}
\usepackage[brazil]{babel}
\usepackage[shortlabels]{enumitem}
\usepackage{cancel}
\usepackage{textcomp}
\usepackage{tcolorbox}

%tabelas
\usepackage{array} % Para melhor formatação de tabelas
\usepackage{longtable}
\usepackage{booktabs}  % Para linhas horizontais mais bonitas
\usepackage{float}   % Para usar o modificador [H]
\usepackage{caption} % Para usar legendas em tabelas
\usepackage{wrapfig} % Para usar tabelas e figuras flutuantes


%tikzpicture
\usepackage{tikz}
\usepackage{scalerel}
\usepackage{pict2e}
\usepackage{tkz-euclide}
\usetikzlibrary{calc}
\usetikzlibrary{patterns,arrows.meta}
\usetikzlibrary{shadows}
\usetikzlibrary{external}

%pgfplots
\usepackage{pgfplots}
\pgfplotsset{compat=newest}
\usepgfplotslibrary{statistics}
\usepgfplotslibrary{fillbetween}

%colours
\usepackage{xcolor}



\columnsep=2cm
\hoffset=0cm
\textwidth=8cm
\setlength{\columnseprule}{.1pt}
\setlength{\columnsep}{2cm}
\renewcommand{\headrulewidth}{0pt}
\geometry{top=1in, bottom=1in, left=0.7in, right=0.5in}

\pagestyle{fancy}
\fancyhf{}
\fancyfoot[C]{\thepage}

\begin{document}
	
	\noindent\textbf{6FMA02 - Matemática} 
	
	\begin{center}Pertinência e inclusão (I) (Versão estudante)
	\end{center}
	
	\noindent\textbf{Nome:} \underline{\hspace{10cm}}
	\noindent\textbf{Data:} \underline{\hspace{4cm}}
	
	%\section*{Questões de Matemática}
	
	\begin{multicols}{2}
		\noindent Dados dois conjuntos $A$ e $B$, dizemos que $A \subset B$ se, e somente se, todo elemento de $A$ é elemento de $B$. Notamos $A \subset B$ ou $B \supset A$. \\
		A negação da sentença "Todo elemento de $A$ é elemento de $B$ é "Existe um elemento de $A$ que não é elemento de $B$". Logo, $A \not\subset B$ se, e somente se, existe um elemento que está em $A$, mas não está em $B$.
		\noindent\textsubscript{-----------------------------------------------------------------------}
		\begin{enumerate}
			\item Apresente as maneira de se ler $A \subset B$. \\\\
			\underline{~~~~~~~~~~~~~~~~~~~~~~~~~~~~~~~~~~~~~~~~~~~~~~} \\\\
			ou \\\\
			\underline{~~~~~~~~~~~~~~~~~~~~~~~~~~~~~~~~~~~~~~~~~~~~~~} \\\\
			ou \\\\
			\underline{~~~~~~~~~~~~~~~~~~~~~~~~~~~~~~~~~~~~~~~~~~~~~~} \\\\
			ou \\\\
			\underline{~~~~~~~~~~~~~~~~~~~~~~~~~~~~~~~~~~~~~~~~~~~~~~} \\\\
			ou \\\\
			\underline{~~~~~~~~~~~~~~~~~~~~~~~~~~~~~~~~~~~~~~~~~~~~~~}
			
			\item Complete. \\
			$A \subset B$ se, e somente se, \underline{~~~~~~~~~~~~~~~~~~~~~~~~~~~~~~~~~~~~~~~~~~~~~~}.
			\item Reescreva em linguagem simbólica.
			\begin{enumerate}[a)]
				\item 1 é elemento do conjunto $S$. \\\\\\\\
				\item 3 não é elemento do conjunto $M$. \\\\\\\\
				\item 2 não pertence ao conjunto $E$ \\\\\\\\
				\item $A$ não é subconjunto de $B$. \\\\\\\\
				\item $B$ contém $A$. \\\\\\\\
				\item $S$ é subconjunto de $T$. \\\\\\\\
				\item $D$ é parte de $E$. \\\\\\\\
				\item $\{2\}$ não está contido em $\{3\}.$ \\\\\\\\
				\item $\{1, 2\}$ está contido em $K$. \\\\\\\\
				\item $\{0, 1, 2\}$ é superconjunto de \{1, 2\}. \\\\\\\\
			\end{enumerate}
			\item Escreva a negação das sentenças abaixo.
			\begin{enumerate}[a)]
				\item Todos os carros são pretos. \\\\\\
				\item Todo elemento de $A$ é elemento de $B$. \\\\\\
				\item Todas as aves sabem voar. \\\\\\\\
			\end{enumerate}
			\item Complete. \\
			$A \not\subset B$ se, e somente se, \\\\ \underline{~~~~~~~~~~~~~~~~~~~~~~~~~~~~~~~~~~~~~~~~~~~~~~}.
			%5 e 6
			\item Reescreva em linguagem simbólica.
			\begin{enumerate}[a)]
				\item 2 é elemento do conjunto $A$. \\\\\\\\
				\item 3 não é elemento de $B$. \\\\\\\\
				\item 5 não pertence ao conjunto $C$. \\\\\\\\
				\item $D$ é subconjunto de $E$. \\\\\\\\
				\item $N$ contém $M$. \\\\\\\\
				\item $R$ é parte de $S$. \\\\\\\\
				\item \{1, 2\} está contido em $F$. \\\\\\\\
				\item \{7\} não está contido em $G$. \\\\\\\\
				\item O conjunto vazio está contido em $T$. \\\\\\\\
				\item O conjunto cujo elemento é o conjunto vazio é subconjunto de $L$. \\\\\\\\
			\end{enumerate}
			\item Escreva a negação das sentenças abaixo.
			\begin{enumerate}[a)]
				\item Todos os patos são amarelos. \\\\\\\\
				\item Todas as crianças gostam de correr. \\\\\\\\
				\item Todo elemento de $B$ é elemento de $A$. \\\\\\\\
			\end{enumerate}
		\end{enumerate}
		$~$ \\ $~$ \\ $~$ \\ $~$ \\ $~$ \\ $~$ \\ $~$ \\ $~$ \\ $~$ \\ $~$ \\ $~$ \\ $~$ \\ $~$ \\ $~$ \\ $~$ \\ $~$ \\ $~$ \\ $~$ \\ $~$ \\ $~$ \\ $~$ \\ $~$ \\ $~$ \\ $~$ \\ $~$ \\ $~$ \\ $~$ \\ $~$ \\ $~$ \\ $~$ \\ $~$ \\ $~$ \\ $~$ \\ $~$ \\ $~$ \\ $~$ \\ 
	\end{multicols}
\end{document}