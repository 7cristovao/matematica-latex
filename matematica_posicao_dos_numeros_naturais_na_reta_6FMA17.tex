\documentclass[a4paper,14pt]{article}

\usepackage{comment} % Para comentar várias linhas ao mesmo tempo

%matemática
\usepackage{amsmath}
\usepackage{amssymb}

%diagramação
\usepackage{extsizes}
\everymath{\displaystyle}
\usepackage{geometry}
\usepackage{fancyhdr}
\usepackage{multicol}
\usepackage{graphicx}
\usepackage[brazil]{babel}
\usepackage[shortlabels]{enumitem}
\usepackage{cancel}
\usepackage{textcomp}
\usepackage{tcolorbox}

%tabelas
\usepackage{array} % Para melhor formatação de tabelas
\usepackage{longtable}
\usepackage{booktabs}  % Para linhas horizontais mais bonitas
\usepackage{float}   % Para usar o modificador [H]
\usepackage{caption} % Para usar legendas em tabelas
\usepackage{wrapfig} % Para usar tabelas e figuras flutuantes


%tikzpicture
\begin{comment}
	\usepackage{tikz}
	\usepackage{scalerel}
	\usepackage{pict2e}
	\usepackage{tkz-euclide}
	\usetikzlibrary{calc}
	\usetikzlibrary{patterns,arrows.meta}
	\usetikzlibrary{shadows}
	\usetikzlibrary{external}
\end{comment}


%pgfplots
\usepackage{pgfplots}
\pgfplotsset{compat=newest}
\usepgfplotslibrary{statistics}
\usepgfplotslibrary{fillbetween}

%colours
\usepackage{xcolor}



\columnsep=2cm
\hoffset=0cm
\textwidth=8cm
\setlength{\columnseprule}{.1pt}
\setlength{\columnsep}{2cm}
\renewcommand{\headrulewidth}{0pt}
\geometry{top=1in, bottom=1in, left=0.7in, right=0.5in}

\pagestyle{fancy}
\fancyhf{}
\fancyfoot[C]{\thepage}

\begin{document}
	
	\noindent\textbf{6FMA17 - Matemática} 
	
	\begin{center}Posição dos números naturais na reta (Versão estudante)
	\end{center}
	
	\noindent\textbf{Nome:} \underline{\hspace{10cm}}
	\noindent\textbf{Data:} \underline{\hspace{4cm}}
	
	%\section*{Questões de Matemática}
	
	\begin{multicols}{2}
		\noindent Podemos representar os números naturais em uma reta. Quando somamos 1 a um número natural, estamos encontrando seu sucessor, se diminuirmos 1, encontramos seu antecessor. O 0 (zero) é o único número natural que não tem antecessor natural, ou seja, na reta, não há números naturais à esquerda de zero. \\
		Para somar o número $b$ ao número $a$, partimos de $a$ e caminhamos $b$ unidades para a direita; para subtrair $b$ do número $a$, partimos de $a$ e caminhamos $b$ unidades para a esquerda.
		\noindent\textsubscript{-----------------------------------------------------------------------}
		\begin{enumerate} 
			\item Represente, na reta abaixo, os números naturais 2, 3, 7 e 12. \\\\\\
			\includegraphics[width=1\linewidth]{6FMA17_imagens/imagem1}
			\item Quando você soma 1 a qualquer número, seu valor aumenta ou diminui? \\\\\\\\
			\item Qual é o sucessor de 0? E seu antecessor? \\\\\\\\
			\item O número 5 é quantas unidades menor do que 14? \\\\\\
			\item Coloque o sinal $<$ ou $>$ nos espaços indicados, de forma que a afirmação seja verdadeira.
			\begin{enumerate}[a)]
				\item 9 $\underline{~~~~~~~~~~~~~~~}$ 7
				\item 12 $\underline{~~~~~~~~~~~~~~~}$ 15
				\item 2 $\underline{~~~~~~~~~~~~~~~}$ 3
				\item 1 001 $\underline{~~~~~~~~~~~~~~~}$ 999
			\end{enumerate}
			%60 a 63
			\item Complete a figura abaixo, colocando os números que estão faltando. \\\\
			\includegraphics[width=1\linewidth]{6FMA17_imagens/imagem2}
			\item \begin{enumerate}[a)]
				\item O número 9 é quantas unidades menor que 17? \\\\\\\\
				\item Represente uma reta com os números 3, 7, 8, e 14. \\\\\\\\\\\\
				\item Complete a reta abaixo, colocando os números que faltam. \\\\
				\includegraphics[width=1\linewidth]{6FMA17_imagens/imagem3}
				\item Sejam $a$ e $b$ dois números naturais quaisquer tais que $b > a$. Represente na reta os pontos associados a esses dois números.
			\end{enumerate}
		\end{enumerate}
		$~$ \\ $~$ \\ $~$ \\ $~$ \\ $~$ \\ $~$ \\ $~$ \\ $~$ \\ $~$ \\ $~$ \\ $~$ \\ $~$ \\ $~$ \\ $~$ \\ $~$ \\ $~$ \\ $~$ \\ $~$ \\ $~$ \\ $~$ \\ $~$ \\ $~$ \\ $~$ \\ $~$ \\ $~$ \\ $~$ \\ $~$ \\ $~$ \\ $~$ \\ $~$ \\ $~$ \\ $~$ \\ $~$ \\ $~$ \\ $~$ \\ $~$ \\ $~$ \\ $~$ \\ $~$ \\ $~$ \\ $~$ \\ $~$ \\ $~$ \\ $~$ \\ $~$ \\ $~$ \\ $~$ \\ $~$ \\ $~$ \\ $~$ \\ $~$ \\ $~$ \\ $~$ \\ $~$ \\ $~$ \\ $~$ \\ $~$ \\ $~$ \\ $~$ \\ $~$ \\ $~$ \\ $~$ \\ $~$ \\ $~$ \\ $~$ \\ $~$ \\ $~$ \\ $~$ \\ $~$ \\ $~$ \\ $~$ \\ $~$ \\ $~$ \\ $~$
	\end{multicols}
\end{document}