\documentclass[a4paper,14pt]{article}
\usepackage{float}
\usepackage{extsizes}
\usepackage{amsmath}
\usepackage{amssymb}
\everymath{\displaystyle}
\usepackage{geometry}
\usepackage{fancyhdr}
\usepackage{multicol}
\usepackage{graphicx}
\usepackage[brazil]{babel}
\usepackage[shortlabels]{enumitem}
\usepackage{cancel}
\usepackage{textcomp}
\usepackage{array} % Para melhor formatação de tabelas
\usepackage{longtable}
\usepackage{booktabs}  % Para linhas horizontais mais bonitas
\usepackage{float}   % Para usar o modificador [H]
\usepackage{caption} % Para usar legendas em tabelas
\usepackage{tcolorbox}
\usepackage{wrapfig} % Para usar tabelas e figuras flutuantes

\columnsep=2cm
\hoffset=0cm
\textwidth=8cm
\setlength{\columnseprule}{.1pt}
\setlength{\columnsep}{2cm}
\renewcommand{\headrulewidth}{0pt}
\geometry{top=1in, bottom=1in, left=0.7in, right=0.5in}

\pagestyle{fancy}
\fancyhf{}
\fancyfoot[C]{\thepage}

\begin{document}
	
	\noindent\textbf{6FMA64 - Matemática} 
	
	\begin{center}Divisão (Versão estudante)
	\end{center}
	
	\noindent\textbf{Nome:} \underline{\hspace{10cm}}
	\noindent\textbf{Data:} \underline{\hspace{4cm}}
	
	%\section*{Questões de Matemática}
	\begin{multicols}{2}
    		\noindent Se $a$ dividido por $b$ é igual a $c$ (que escrevemos como $a : b = c$), então $a = b \cdot c$. \\
    		Exemplo: 40 : 8 = 5, já que $8 \cdot 5 = 40$. \\
    		Zero dividido por qualquer número natural diferente de zero é igual a zero. \\
    		Exemplos: 0 : 5 = 0 e 0 : 8 = 0. \\
    		\textbf{Não é possível dividir um número por zero.} \\
    		\textsubscript{---------------------------------------------------------------------}
    		\begin{enumerate}
    			\item O Sr. Fernando tem quatro filhos e quer repartir igualmente entre eles a quantia de 36 reais. Quanto deve receber cada filho? \\\\\\\\\\
    			\item Complete as igualdades a seguir:
    			\begin{enumerate}[a)]
    				\item 21 : 3 = \\\\\\\\
    				\item $18 \div 6$ = \\\\\\\\
    				\item $\frac{44}{4} = $ \\
    				\item $48 \div 3 = $ \\\\\\\\
    				\item $\frac{42}{7} = $ \\\\\\\\
    				\item 54 : 6 = \\\\\\\\
    			\end{enumerate}
    			\item Armando quer distribuir suas 52 bolinhas de gude para todos os seus 4 irmãos. Para evitar brigas, cada um deles deve receber o mesmo número de bolinhas. Quantas? \\\\\\\\\\\\
    			\item Quanto dá um número dividido por 1? \\\\\\\\
    			\item Calcule $\frac{0}{100}$. \\\\\\\\\\\\
    			\item Existe algum número igual a $\frac{0}{0}$~? \\\\\\\\\\\\
    			\item Você conhece algum número representado por $\frac{8}{0}$? \\\\\\\\\\\\
    			\item Um caminhão deve transportar 9128 garrafas em caixas que acomodam 28 garrafas. Se cada caixa deve acomodar o máximo possível de garrafas, quantas caixas serão necessárias para esse transporte? \\\\\\\\\\\\
    			\item Complete as sequências a seguir:
    			\begin{enumerate}[a)]
    				\item 1, 2, 8, 48, 384, ..... .
    				\item 4, 12, 48, 144, 576, ..... .
    			\end{enumerate}
    			\item As sequências a seguir têm pelo menos uma regra de formação. Descubra o elemento que falta em cada uma delas, representado por $x$, e explique o que você pensou.
    			\begin{enumerate}[a)]
    				\item 270, 90, 30, $x$.
    				\item 5, 20, 80, 320, $x$.
    				\item 28 561, 2 197, 169, $x$.
    				\item 2, 14, 98, 686, $x$.
    				\item 31, 25, 19, $x$.
    				\item 18 750, 3 750, 750, $x$, 30, 6.
    			\end{enumerate}
    			\item Renato pensou em um número e duvidou que Patrícia o descobrisse. Ela, então, pediu que ele multiplicasse o número por 4, somasse 17 e dividisse o resultado por 3. Renato fez as contas e disse que o seu resultado foi 67. Após alguns cálculos, Patrícia obteve o número pensado por Renato. Qual era esse número?
        	\end{enumerate}
        	$~$ \\ $~$ \\ $~$ \\ $~$ \\ $~$ \\ $~$ \\ $~$ \\ $~$ \\ $~$ \\ $~$ \\ $~$ \\ $~$ \\ $~$ 
	\end{multicols}
\end{document}