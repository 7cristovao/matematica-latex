\documentclass[a4paper,14pt]{article}
\usepackage{float}
\usepackage{extsizes}
\usepackage{amsmath}
\usepackage{amssymb}
\everymath{\displaystyle}
\usepackage{geometry}
\usepackage{fancyhdr}
\usepackage{multicol}
\usepackage{graphicx}
\usepackage[brazil]{babel}
\usepackage[shortlabels]{enumitem}
\usepackage{cancel}
\usepackage{textcomp}
\usepackage{array} % Para melhor formatação de tabelas
\usepackage{longtable}
\usepackage{booktabs}  % Para linhas horizontais mais bonitas
\usepackage{float}   % Para usar o modificador [H]
\usepackage{caption} % Para usar legendas em tabelas
\usepackage{tcolorbox}
\usepackage{wrapfig} % Para usar tabelas e figuras flutuantes

\columnsep=2cm
\hoffset=0cm
\textwidth=8cm
\setlength{\columnseprule}{.1pt}
\setlength{\columnsep}{2cm}
\renewcommand{\headrulewidth}{0pt}
\geometry{top=1in, bottom=1in, left=0.7in, right=0.5in}

\pagestyle{fancy}
\fancyhf{}
\fancyfoot[C]{\thepage}

\begin{document}
	
	\noindent\textbf{6FMA62 - Matemática} 
	
	\begin{center}Exercícios de adição e subtração de números inteiros (Versão estudante)
	\end{center}
	
	\noindent\textbf{Nome:} \underline{\hspace{10cm}}
	\noindent\textbf{Data:} \underline{\hspace{4cm}}
	
	%\section*{Questões de Matemática}
	\begin{multicols}{2}
    		\begin{enumerate}
    			\item Calcule:
    			\begin{enumerate}[a)]
    				\item -28 + 5 = \\\\\\\\
    				\item (-15)+(-46) =  \\\\\\\\
    				\item (-57)+(-408) = \\\\\\\\
    				\item 2 +(-9) = \\\\\\\\
    				\item 164 +(-623) = \\\\\\\\
    				\item (-724) + (-217) =  \\\\\\\\
    				\item (-856) + 412 =  \\\\\\\\
    				\item 49 - 79 =  \\\\\\\\
    				\item -858 - 276 = \\\\\\\\
    				\item -68 + 37 = \\\\\\\\
    				\item 52-(-31) = \\\\\\\\
    				\item -93 -(-237) = \\\\\\\\
    				\item 64 - 529 = \\\\\\\\
    				\item -76 -(+45) = \\\\\\\\
    				\item 219 - 82 = \\\\\\\\
    				\item -5 + 55 = \\\\\\\\
    				\item -4 + 4 = \\\\\\\\
    				\item 0 - 97 = \\\\\\\\
    				\item 7 - 0 = \\\\\\\\
    				\item 8 - 8 = \\\\\\\\
    				\item 896 - 578 = \\\\\\\\
    				\item 63 - 874 = \\\\\\\\
    				\item -204 -(+66) = \\\\\\\\
    				\item 343 - 204 = \\\\\\\\
    				\item -751 - 18 = \\\\\\\\
    				\item 13 - 129 = \\\\\\\\
    			\end{enumerate}
    			\textbf{Desafio olímpico} \\\\
    			\noindent (OBMEP) Lúcia e Antônio disputaram várias partidas de um jogo no qual cada um começa com 5 pontos. Em cada partida, o vencedor ganha 2 pontos e o derrotado perde 1 ponto, não havendo empates. Ao final, Lúcia ficou com dez pontos e Antônio ganhou exatamente três partidas. Quantas partidas eles disputaram ao todo?
    			\begin{enumerate}[a)]
    				\item 6
    				\item 7
    				\item 8
    				\item 9
    				\item 10
    			\end{enumerate}
    			\item Efetuar.
    			\begin{enumerate}[a)]
    				\item -46 - 73 + 53 \\\\\\\\
    				\item -78 - 15 - 62 + 84 \\\\\\\\
    				\item 46 + 87 - 521 + 38 \\\\\\\\
    				\item 63 - 108 - 47 + 89 \\\\\\\\
    				\item 5 + 37 - 81 + 34 - 23 \\\\\\\\
    				\item -4 - 7 + 91 - 66 \\\\
    				\item -11 + 44 - 75 - 2 + 123 \\\\\\\\
    				\item -86 + 31 + 66 - 48 + 50 \\\\\\\\
    				\item -2 + 3 - 6 + 8 - 7 + 15 \\\\\\\\
    				\item -6 + 9 + 12 - 5 + 9 - 21 \\\\\\\\
    				\item 3 - 4 + 5 - 6 + 10 - 25 + 15 + 18 \\\\\\\\
    				\item 5 - 5 + 5 - 5 + 5 - 5 + 5 - 5 - 5 \\\\\\\\
    			\end{enumerate}
    			\item Calcular.
    			\begin{enumerate}[a)]
    				\item 48 - 200 + 96 - 25 - 9 + 11 + 61 \\\\\\\\
    				\item -33 - 7 + 92 + 52 - 55 - 70 \\\\\\\\
    				\item -317 - 879 + 364 - 214 - 8 \\\\\\\\
    				\item 39 + 451 - 205 - 41 - 1 + 16 + 217 - 51 - 7 + 91 + 24 - 512 - 73 \\\\\\\\
    				\item 31 + 246 - 492 - 12 + 880 - 763 \\\\\\\\
    				\item -46 - 397 - 44 - 184 - 3 + 41 + 1863 - 422 - 9 + 149 + 216 - 90 - 9 \\\\\\\\
    				\item 2 - 4 + 6 - 8 + 10 - 12 + 14 - 16 + 18 \\\\\\\\
    				\item -2 + 4 - 6 + 8 - 10 + 12 - 14 + 16 - 18 + 20 \\
    				\item 555 - 333 + 777 - 999 \\\\\\\\
    				\item 234 - 345 + 456 - 567 + 678 \\\\\\\\
    				\item 81 - 253 - 252 + 648 - 214 + 93 \\\\\\\\
    				\item 673 - 2888 - 404 + 163 - 74 + 12 \\\\\\\\
    			\end{enumerate}
    			\item Efetuar.
    			\begin{enumerate}[a)]
    				\item 12 - 36 - 27 - 321 + 86 \\\\\\\\
    				\item -620 + 753 + 61 \\\\\\\\
    				\item -281 + 678 - 83 \\\\\\\\
    				\item 3 - 22 + 56 - 203 - 57 \\\\\\\\
    				\item 52 + 13 - 39 - 40 \\\\\\\\
    				\item -11 + 17 - 6 - 1 \\\\\\\\
    				\item 18 + 1 + 6 - 4 - 6 - 15 \\\\\\\\
    				\item 61 - 25 - 10 - 7 + 28 \\\\\\\\
    				\item 128 + 31 - 208 - 144 \\\\\\\\
    				\item 484 + 621 + 373 - 2000 \\\\\\\\
    				\item 625 - 746 + 879 + 108 \\\\
    				\item -37 + 213 - 60 + 58 - 71 \\\\\\\\
    			\end{enumerate}
    			\item Calcular.
    			\begin{enumerate}[a)]
    				\item 679 - 421 - 27 + 36 - 28 \\\\\\\\
    				\item 243 - 108 + 215 - 121 + 15 \\\\\\\\
    				\item 766 + 253 - 382 - 12 + 111 \\\\\\\\
    				\item -513 + 444 - 326 + 18 \\\\\\\\
    				\item 2 - 6 + 13 - 4 + 17 - 8 + 12 - 15 \\\\\\\\
    				\item 205 - 81 + 77 - 42 + 63 - 26 \\\\\\\\
    				\item 2317 - 618 - 197 + 423 - 670 \\\\\\\\
    				\item 216 - 19 + 36 - 101 - 15 \\\\\\\\
    				\item 323 - 215 + 182 - 11 - 15 - 108 \\\\\\\\
    				\item 38 - 14 + 209 - 123 - 6 \\\\\\\\
    				\item 2000 - 825 - 217 + 621 - 17 - 204 - 389 \\\\\\\\
    				\item 825 - 417 + 16 + 71 - 68 \\\\\\\\
    			\end{enumerate}
    			%43 a 46
        	\end{enumerate}
        	$~$ \\ $~$ \\ $~$ \\ $~$ \\ $~$ \\ $~$ \\ $~$ \\ $~$ \\ $~$ \\ $~$ \\ $~$ \\ $~$ \\ $~$ \\ $~$ \\ $~$ \\ $~$ \\ $~$ \\ $~$ \\ $~$ \\ $~$ \\ $~$ \\ $~$ \\ $~$ \\ $~$ \\ $~$ \\ $~$ \\ $~$ \\ $~$ \\ $~$ \\ $~$ \\ $~$ \\ $~$ \\ $~$ \\ $~$ \\ $~$ \\ $~$ \\ $~$ \\ $~$ \\ $~$ \\ $~$ \\ $~$ \\ $~$ \\ $~$ \\ $~$ \\ $~$ \\ $~$ \\ $~$ \\ $~$ \\ $~$
	\end{multicols}
\end{document}