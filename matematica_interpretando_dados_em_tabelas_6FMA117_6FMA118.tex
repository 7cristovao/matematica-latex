\documentclass[a4paper,14pt]{article}

\usepackage{comment} % Para comentar várias linhas ao mesmo tempo

%matemática
\usepackage{amsmath}
\usepackage{amssymb}

%diagramação
\usepackage{extsizes}
\everymath{\displaystyle}
\usepackage{geometry}
\usepackage{fancyhdr}
\usepackage{multicol}
\usepackage{graphicx}
\usepackage[brazil]{babel}
\usepackage[shortlabels]{enumitem}
\usepackage{cancel}
\usepackage{textcomp}
\usepackage{tcolorbox}

%tabelas
\usepackage{array} % Para melhor formatação de tabelas
\usepackage{longtable}
\usepackage{booktabs}  % Para linhas horizontais mais bonitas
\usepackage{float}   % Para usar o modificador [H]
\usepackage{caption} % Para usar legendas em tabelas
\usepackage{wrapfig} % Para usar tabelas e figuras flutuantes
\usepackage{xcolor} % Para cores do fundo de tabelas
\usepackage{colortbl} % Para cores do fundo de tabelas

%tikzpicture
\begin{comment}
	\usepackage{tikz}
	\usepackage{scalerel}
	\usepackage{pict2e}
	\usepackage{tkz-euclide}
	\usetikzlibrary{calc}
	\usetikzlibrary{patterns,arrows.meta}
	\usetikzlibrary{shadows}
	\usetikzlibrary{external}
\end{comment}


%pgfplots
\usepackage{pgfplots}
\pgfplotsset{compat=newest}
\usepgfplotslibrary{statistics}
\usepgfplotslibrary{fillbetween}

%colours
\usepackage{xcolor}



\columnsep=2cm
\hoffset=0cm
\textwidth=8cm
\setlength{\columnseprule}{.1pt}
\setlength{\columnsep}{2cm}
\renewcommand{\headrulewidth}{0pt}
\geometry{top=1in, bottom=1in, left=0.7in, right=0.5in}

\pagestyle{fancy}
\fancyhf{}
\fancyfoot[C]{\thepage}

\begin{document}
	
	\noindent\textbf{6FMA117, 6FMA118 - Matemática} 
	
	\begin{center}Interpretando dados em tabelas (Versão estudante)
	\end{center}
	
	\noindent\textbf{Nome:} \underline{\hspace{10cm}}
	\noindent\textbf{Data:} \underline{\hspace{4cm}}
	
	%\section*{Questões de Matemática}
	
	\begin{multicols}{2}
		% \noindent \\
		% \noindent\textsubscript{--------------------------------------------------------------------------}
		\begin{enumerate} 
			\item Muitas usinas hidrelétricas estão situadas em barragens. As características de algumas das grandes represas e usinas brasileiras estão apresentadas no quadro a seguir.
			\begin{table}[H]
					\begin{tabular}{|p{2.4cm}|p{2.4cm}|p{2.4cm}|}
						\hline
						\textbf{Usina} & \textbf{Área alagada (km²)} & \textbf{Potência} \\ \hline
						$A$   & 3 298   & 2 040   \\ \hline
						$B$   & 4 720   & 3 508   \\ \hline
						$C$   & 5 100   & 3 467   \\ \hline
						$D$   & 6 403   & 1 345   \\ \hline
						$E$   & 2 176   & 1 803   \\ \hline
					\end{tabular}
			\end{table}
			A razão entre a área da região alagada por uma represa e a potência produzida pela usina nela instalada é uma das formas de estimar a relação entre o dano e o benefício trazidos por um projeto hidrelétrico. A partir dos dados apresentados no quadro, qual é o projeto que mais onerou o ambiente em termos de área alagada por potência? \\\\\\\\\\\\\\\\\\
			\item Visando adotar um sistema de reutilização de água, uma indústria testou cinco sistemas com diferentes fluxos de entrada de água suja e fluxos de saída de água purificada.
			\begin{table}[H]
				\begin{tabular}{|p{2.4cm}|p{2.4cm}|p{2.4cm}|}
					\hline
					\textbf{~} & \textbf{Fluxo de entrada (água suja)} & \textbf{Fluxo de saída (água purificada)} \\ \hline
					Sistema I   & 60 L/h   & 10 L/h   \\ \hline
					Sistema II   & 15 L/h   & 6 L/h   \\ \hline
					Sistema III   & 30 L/h   & 10 L/h   \\ \hline
					Sistema IV   & 25 L/h   & 5 L/h   \\ \hline
					Sistema V   & 36 L/h   & 14 L/h   \\ \hline
				\end{tabular}
			\end{table}
			Supondo que o custo por litro de água purificada seja o mesmo para todos os sistemas, qual é o sistema pelo qual obtém-se maior eficiência na purificação? \newpage
			\item O índice de massa corpórea (IMC) é uma medida que permite aos médicos fazer uma avaliação preliminar das condições físicas e dos risco de uma pessoa desenvolver certas doenças, conforme mostra a tabela a seguir.
			\begin{table}[H]
				\begin{tabular}{|p{1.8cm}|p{3cm}|p{2.4cm}|}
					\hline
					\textbf{IMC} & \textbf{Classificação} & \textbf{Risco de doenças} \\ \hline
					menos de 18,5   & magreza   & elevado   \\ \hline
					entre 18,5 e 24,9   & normalidade   & baixo   \\ \hline
					entre 25 e 29,9   & sobrepeso   & elevado   \\ \hline
					entre 30 e 39,9   & obesidade   & muito elevado   \\ \hline
					40 ou mais   & obesidade grave   & muitíssimo elevado   \\ \hline
				\end{tabular}
				\caption*{Fonte: $<$www.somatematica.com.br$>$.}
			\end{table}
			Considere as seguintes informações a respeito de Andreia, Carlos, Daniel, Eduarda e Flávia:
			\begin{table}[H]
				\begin{tabular}{|p{1.6cm}|p{1.6cm}|p{1.6cm}|p{1.6cm}|}
					\hline
					\textbf{Nome} & \textbf{Peso (kg)} & \textbf{Altura (m)} & \textbf{IMC} \\ \hline
					Andreia   & 64,8   & 1,80 & 20  \\ \hline
					Carlos   & 49,5   & 1,50 & 22  \\ \hline
					Daniel   & 70,4   & 1,60 & 27,5 \\ \hline
					Eduarda   & 86,7   & 1,70 & 30   \\ \hline
					Flávia   & 56,6  & 1,60 & 22,5   \\ \hline
				\end{tabular}
			\end{table}
			Podemos afirmar que:
			\begin{enumerate}[a)]
				\item Carlos está com sobrepeso, mas não corre risco de desenvolver doenças?
				\item Eduarda está dentro dos padrões de normalidade?
				\item Daniel está com sobrepeso e seu risco de desenvolver doenças é elevado?
				\item Para se determinar o IMC de uma pessoa, utilizamos a seguinte fórmula: \\\\ IMC = $\frac{\text{massa(kg)}}{\text{[altura]²(m)}}$. \\\\ Construa um fluxograma que determine o IMC de um indivíduo e, conforme a primeira tabela, indique a qual classificação ele pertence. \newpage
			\end{enumerate}
			\item Para responder à questão, considere a tabela a seguir que apresenta dados sobre a concentração de cinco substâncias no sangue e na urina de um indivíduo.
			\begin{table}[H]
				\begin{tabular}{|p{2.8cm}|p{2.2cm}|p{2.2cm}|}
					\hline
					\textbf{Substâncias} & \textbf{Concen- tração no sangue (mg/100 mL)} & \textbf{Concen- tração na urina (mg/100 mL)} \\ \hline
					ureia & 30 & 2 000 \\ \hline
					ácido úrico & 4 & 50 \\ \hline
					sais & 720 & 1 500 \\ \hline
					proteínas & 7 000 & 35 \\ \hline
					glicose & 100 & 4 \\ \hline
				\end{tabular}
				\caption*{Fonte: Construindo~consciência~2008.}
			\end{table}
			Observando os dados da tabela, pode-se concluir que em 1 litro de sangue, o total, em mg, de ureia e ácido úrico, é, aproximadamente:
			\begin{enumerate}[a)]
				\item 240
				\item 260
				\item 300
				\item 320
				\item 340
			\end{enumerate}
			\textbf{Agora é a sua vez}
			\\
			\item Vamos fazer uma coleta de informações na sua sala de aula:
			\begin{enumerate}[a)]
				\item \textbf{Passo 1: coleta} \\
				Pergunte aos seus colegas em que mês cada um deles nasceu e anote. \newpage
				\item \textbf{Passo 2: organização} \\
				Coloque os dados na tabela a seguir. Caso você tenha alguma dúvida, fale com o seu professor.
					\begin{table}[H]
					\begin{tabular}{|p{3.5cm}|p{3.5cm}|}
						\hline
						\textbf{Mês do nascimento} & \textbf{Número de alunos}  \\ \hline
						Janeiro & ~~  \\ \hline
						Fevereiro & ~~ \\ \hline
						Março & ~~ \\ \hline
						Abril & ~~ \\ \hline
						Maio & ~~  \\ \hline
						Junho & ~~  \\ \hline
						Julho & ~~  \\ \hline
						Agosto & ~~  \\ \hline
						Setembro & ~~  \\ \hline
						Outubro & ~~  \\ \hline
						Novembro & ~~  \\ \hline
						Dezembro & ~~  \\ \hline
						\textbf{Total} & ~~  \\ \hline
					\end{tabular}
				\end{table}
				Observe que o total de alunos deve ser o número de alunos que estavam na sua sala no dia da coleta. Confira!
			\item \textbf{Passo 3: interpretação}
			\begin{itemize}
				\item Quantos alunos nasceram em fevereiro? \\\\\\\\\\
				\item Quantos alunos nasceram nas férias (janeiro, julho ou dezembro)? \\\\\\
				\item Quantos alunos nasceram em um mês cujo nome começa com a letra A? \\\\\\
				\item Existe um mês do ano em que nasceram mais de 10 colegas? Qual? \\\\\\
				(O espaço abaixo é reservado para a sua criatividade. Faça perguntas aos seus colegas sobre a tabela e anote suas respostas.) \\\\\\
				\item \underline{~~~~~~~~~~~~~~~~~~~~~~~~~~~~~~~~~~~~~} \\\\\\
				\item \underline{~~~~~~~~~~~~~~~~~~~~~~~~~~~~~~~~~~~~~} \\\\\\
			\end{itemize}
			\item \textbf{Passo 4: conclusão} \\
			Anote as conclusões que você achar interessantes. Peça ajuda aos seus colegas e ao professor.
			\end{enumerate}
			%29 a 35
			\item Imagine uma eleição envolvendo 3 candidatos $A, B, C$ e 36 eleitores (votantes). Cada eleitor vota fazendo uma ordenação dos três candidatos. Os resultados são os seguintes:
			\begin{table}[H]
				\begin{tabular}{|p{3.5cm}|p{3.5cm}|}
					\hline
					\textbf{Ordenação} & \textbf{Nº de eleitores}  \\ \hline
					$A B C$ & 11  \\ \hline
					$A C B$ & 5 \\ \hline
					$B A C$ & 7 \\ \hline
					$B C A$ & 3 \\ \hline
					$C A B$ & 6  \\ \hline
					$C B A$ & 4  \\ \hline
					\textbf{Total} & 36  \\ \hline
				\end{tabular}
			\end{table}
			A primeira linha do quadro descreve que 11 eleitores escolheram $A$ em 1º lugar, $B$ em 2º lugar e $C$ em 3º lugar e assim por diante. \\
			Considere o sistema de eleição no qual cada candidato ganha 3 pontos quando é escolhido em 1º lugar, 2 pontos quando é escolhido em 2º lugar e 1 ponto quando é escolhido em 3º lugar. O candidato que acumular mais pontos é eleito. Nesse caso:
			\\
			\begin{enumerate}[a)]
				\item $C$ é eleito com 71 pontos.
				\item $A$ é eleito com 64 pontos.
				\item $B$ é eleito com 72 pontos.
				\item $C$ é eleito com 64 pontos.
				\item $A$ é eleito com 81 pontos.
			\end{enumerate}
			\item Na cidade de Lucas e Renata haverá shows no teatro principal. Pensando em todos, foram feitos pacotes para que as pessoas escolhessem o que seria melhor para si.
			\begin{table}[H]
				\begin{tabular}{|p{2cm}|p{2.6cm}|p{2.6cm}|}
					\hline
					\textbf{Pacote} & \textbf{Taxa fixa} & \textbf{Taxa por show} \\ \hline
					1 & - & R\$ 80,00 \\ \hline
					2 & R\$ 40,00 & R\$ 60,00 \\ \hline
					3 & R\$ 60,00 & R\$ 70,00 \\ \hline
				\end{tabular}
			\end{table}
			Lucas assistirá 6 shows e Renata, 4. As melhores opções para Lucas e Renata são, respectivamente, os pacotes:
			\begin{enumerate}[a)]
				\item 1 e 2.
				\item 2 e 3.
				\item 3 e 3.
				\item 2 e 1.
				\item 2 e 2.
			\end{enumerate}
			\item No quadro seguinte são informados os turnos em que foram eleitos os prefeitos das capitais de todos os estados brasileiros em 2004.
			\begin{table}[H]
				\begin{tabular}{|p{0.4cm}|p{4.6cm}|p{2cm}|}
					\hline
				 & \textbf{Cidade} & \textbf{Turno} \\ \hline
					1 & Aracaju (SE) & 1º \\ \hline
					2 & Belém (PA) & 2º \\ \hline
					3 & Belo Horizonte (MG) & 1º \\ \hline
					4 & Boa Vista (RR) & 1º \\ \hline
					5 & Campo Grande (MS) & 1º \\ \hline
					6 & Cuiabá (MT) & 2º \\ \hline
					7 & Curitiba (PR) & 2º \\ \hline
					8 & Florianópolis (SC) & 2º \\ \hline
					9 & Fortaleza (CE) & 2º \\ \hline
					10 & Goiânia (GO) & 2º \\ \hline
					11 & João Pessoa (PB) & 1º \\ \hline
					12 & Macapá (AP) & 1º \\ \hline
					13 & Maceió (AL) & 2º \\ \hline
					14 & Manaus (AM) & 2º \\ \hline
					15 & Natal (RN) & 2º \\ \hline
					16 & Palmas (TO) & 1º \\ \hline
					17 & Porto Alegre (RS) & 2º \\ \hline
					18 & Porto Velho (RO) & 2º \\ \hline
					19 & Recife (PE) & 1º \\ \hline
					20 & Rio Branco (AC) & 1º \\ \hline
					21 & Rio de Janeiro (RJ) & 1º \\ \hline
					22 & Salvador (BA) & 2º \\ \hline
					23 & São Luís (MA) & 1º \\ \hline
					24 & São Paulo (SP) & 2º \\ \hline
					25 & Teresina (PI) & 2º \\ \hline
					26 & Vitória (ES) & 2º \\ \hline
				\end{tabular}
				\caption*{Fonte: TSE. Almanaque Abril - Brasil 2005. São Paulo: Editora Abril, 2005 }
			\end{table}
			Na região Norte, a porcentagem de prefeitos eleitos no 2º turno foi de, aproximadamente:
			\begin{enumerate}[a)]
				\item 42,86\%
				\item 44,44\%
				\item 50,00\%
				\item 57,14\%
				\item 57,69\%
			\end{enumerate}
			\item Os níveis de irradiância ultravioleta efetiva (IUV) indicam o risco de exposição ao sol para pessoas de pele do tipo II - pele de pigmentação clara. O tempo de exposição segura (TES) corresponde ao tempo de exposição ao raios solares sem que ocorram queimaduras de pele. A tabela a seguir mostra a correlação entre riscos de exposição, IUV e TES.
			\begin{table}[H]
				\begin{tabular}{|p{2.2cm}|p{2.2cm}|p{2.2cm}|}
					\hline
					\textbf{Risco de exposição} & \textbf{IUV} & \textbf{TES (em minutos)} \\ \hline
					baixo & 0 a 2 & máximo 60 \\ \hline
					médio & 3 a 5 & 30 a 60 \\ \hline
					alto & 6 a 8 & 20 a 30 \\ \hline
					extremo & acima de 8 & máximo 20 \\ \hline
				\end{tabular}
			\end{table}
			Uma das maneiras de se proteger contra queimaduras provocadas pela radiação ultravioleta é o uso dos protetores solares, cujo Fator de Proteção Solar (FPS) é calculado da seguinte maneira:
			\begin{equation*}
				\text{FPS} = \frac{\text{TPP}}{\text{TPD}}
			\end{equation*}
			TPP = tempo de exposição mínima para produção de vermelhidão na pele protegida (em minutos). \\
			TPD = tempo de exposição mínima para produção de vermelhidão na pele desprotegida (em minutos). \\
			O FPS mínimo que uma pessoa de pele tipo II necessita para evitar queimaduras ao se expor ao sol, considerando TPP o intercalo das 12h às 14h, num dia em que a irradiância efetiva é maior que 8, de acordo com os dados fornecidos, é: 
			\begin{enumerate}[a)]
				\item 5
				\item 6
				\item 8
				\item 10
				\item 20
			\end{enumerate}
			\item A tabela a seguir contém as informações nutricionais de um bolo de banana com 360 g. Observe e responda ao que se pede. \\\\
			
			\begin{table}[H]
				\begin{tabular}{|p{2.5cm}|p{2.9cm}|p{1.6cm}|}
					\hline
					\multicolumn{3}{|p{7cm}|}{\textbf{Informação Nutricional (porção de 60 g - fatia média)}} \\
					\hline
					\multicolumn{2}{|p{5.5cm}|}{Quantidade por porção} & \%VD* \\
					\hline
					valor energético & 175 kcal = 735 kJ & 9 \\ \hline
					carboidratos & 3 a 5 & 30 a 60 \\ \hline
					proteínas & 6 a 8 & 20 a 30 \\ \hline
					gorduras totais & acima de 8 & máximo 20 \\ \hline
					gorduras saturadas & acima de 8 & máximo 20 \\ \hline
					gorduras $trans$ & acima de 8 & máximo 20 \\ \hline
					fibra alimentar & acima de 8 & máximo 20 \\ \hline
					sódio & acima de 8 & máximo 20 \\ \hline
				\end{tabular}
				\caption*{(*) Valores diários de referência com base em uma dieta de 2000 kcal ou 8400 kj. Seus valores diários podem ser maiores ou menores dependendo de suas necessidades energéticas.}
				\label{tab:exemplo}
			\end{table}
			\begin{enumerate}[a)]
				\item Qual o valor energético em kcal, do bolo inteiro? \\\\\\\\
				\item A que fração do bolo corresponde 1 fatia? \\\\\\\\
				\item Qual a quantidade de gorduras não saturadas em 1 fatia? \newpage
				\item Qual a quantidade de gorduras não saturadas no bolo inteiro? \\\\\\\\
			\end{enumerate}
			\item Observe as tabelas a seguir e dê o que se pede:
			\begin{table}[H]
				\begin{tabular}{|p{2.6cm}|p{2.4cm}|p{2.4cm}|}
					\hline
					\multicolumn{3}{|p{6.8cm}|}{\textbf{Renda (em milhões de R\$)}} \\
					\hline
					\textbf{Equipe} & \textbf{2007} & \textbf{2008} \\ \hline
					Flamengo & 13.157.515 & 25.830.093 \\ \hline
					Fluminense & * & 20.762.439 \\ \hline
					Corinthians & 7.214.211 & 16.776.826 \\ \hline
					Internacional & 8.347.811 & 11.807.576 \\ \hline
					Cruzeiro & 6.481.044 & 11.170.594 \\ \hline
					Grêmio & 13.237.706 & 13.197.601 \\ \hline
					Botafogo & 7.074.740 & 10.962.300 \\ \hline
					São Paulo & 13.382.214 & 16.506.080 \\ \hline
					Vasco & * & 8.552.142 \\ \hline
					Atlético-MG & 6.813.046 & * \\ \hline
					Palmeiras & 7.613.607 & 16.213.724 \\ \hline
					Bahia & 7.171.109 & * \\ \hline
				\end{tabular}
			\end{table}
			\begin{table}[H]
				\begin{tabular}{|p{2.6cm}|p{2.4cm}|p{2.4cm}|}
					\hline
					\multicolumn{3}{|p{6.8cm}|}{\textbf{Maiores públicos)}} \\
					\hline
					\textbf{Equipe} & \textbf{2007} & \textbf{2008} \\ \hline
					Flamengo & 877 826 & 1 453 477 \\ \hline
					Fluminense & * & 1 054 664 \\ \hline
					Corinthians & 601 447 & 1 054 664 \\ \hline
					Internacional & 584 917 & 791 394 \\ \hline
					Cruzeiro & 617 179 & 788 195 \\ \hline
					Grêmio & 768 533 & 746 342 \\ \hline
					Botafogo & 800 278 & 730 211 \\ \hline
					São Paulo & 837 761 & 720 533 \\ \hline
					Sport & 699 504 & 686 025 \\ \hline
					Atlético-MG & 661 013 & 613 019 \\ \hline
					Palmeiras & * & 567 153 \\ \hline
					Bahia & 882 147 & * \\ \hline
				\end{tabular}
			\end{table}
			(*) Dados ausentes indicam que o clube não figurou nas listas de 10 maiores públicos ou rendas do ano.
			$<$http://colunistas.ig.com.br/ \\ sergiopatrick/2008/12/23/campeoes-de-bilheteria/$>$
			\begin{enumerate}[a)]
				\item Qual foi a equipe brasileira de maior renda em 2007? E em 2008?
				\item Qual foi a equipe brasileira de maior público em 2007? E em 2008? \newpage
			\end{enumerate}
			\item A tabela a seguir mostra o desempenho dos 6 últimos colocados no campeonato interdistrital de futebol amador antes do último jogo.
			\begin{table}[H]
				\begin{tabular}{|p{2.8cm}|p{1.2cm}|p{1.2cm}|p{1.2cm}|}
					\hline
					\textbf{Equipe} & \textbf{$J$} & \textbf{$P$} & \textbf{$V$} \\ \hline
					Unidos da Série B   & 18   & 15 & 4  \\ \hline
					Fanfarrões   & 18   & 14 & 3  \\ \hline
					Bola Furada   & 18   & 13 & 4 \\ \hline
					Perdedores   & 18   & 12 & 2   \\ \hline
					Lanterninha   & 18  & 9 & 2   \\ \hline
					Rebaixados   & 18  & 8 & 2   \\ \hline
				\end{tabular}
			\end{table}
			Legenda: \\ \textbf{$J$} = jogos disputados; \\ \textbf{$P$} = pontos ganhos; \\ \textbf{$V$} = número de vitórias; 
			\begin{itemize}
				\item \textbf{Critério de desempate:} número de vitórias.
				\item \textbf{Regulamento:} uma vitória vale 3 pontos, um empate vale 1 ponto e uma derrota vale 0 ponto. Sabendo que na última rodada os 6 últimos não vão jogar entre si e que os4 últimos da tabela do campeonato serão rebaixados para a divisão inferior, podemos afirmar que, ao final do campeonato:
				\begin{enumerate}[a)]
					\item o Unidos da Série B com certeza estará livre do rebaixamento.
					\item se o Fanfarrões perder e o Bola Furada empatar, ambos podem ser rebaixados.
					\item o Lanterninha ainda pode se livrar do rebaixamento.
					\item se o Perdedores vencer e o Fanfarrões empatar, ambos se livram do rebaixamento.
					\item o Lanterninha, em hipótese nenhuma, terminará o campeonato no último lugar.
				\end{enumerate}
			\end{itemize}
		\end{enumerate}
		$~$ \\ $~$ \\ $~$ \\ $~$ \\ $~$ \\ $~$ \\ $~$ \\ $~$ \\ $~$ \\ $~$ \\ $~$ \\ $~$ \\ $~$ \\ $~$ \\ $~$ \\ $~$ \\ $~$ \\ $~$ \\ $~$ \\ $~$ \\ $~$ \\ $~$ \\ $~$ \\ $~$ \\ $~$ \\ $~$ \\ $~$ \\ $~$ \\ $~$ \\ $~$ \\ $~$ \\ $~$
	\end{multicols}
\end{document}