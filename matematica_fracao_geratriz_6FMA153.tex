\documentclass[a4paper,14pt]{article}

\usepackage{comment} % Para comentar várias linhas ao mesmo tempo

%matemática
\usepackage{amsmath}
\usepackage{amssymb}

%diagramação
\usepackage{extsizes}
\everymath{\displaystyle}
\usepackage{geometry}
\usepackage{fancyhdr}
\usepackage{multicol}
\usepackage{graphicx}
\usepackage[brazil]{babel}
\usepackage[shortlabels]{enumitem}
\usepackage{cancel}
\usepackage{textcomp}
\usepackage{tcolorbox}

%tabelas
\usepackage{array} % Para melhor formatação de tabelas
\usepackage{longtable}
\usepackage{booktabs}  % Para linhas horizontais mais bonitas
\usepackage{float}   % Para usar o modificador [H]
\usepackage{caption} % Para usar legendas em tabelas
\usepackage{wrapfig} % Para usar tabelas e figuras flutuantes
\usepackage{xcolor} % Para cores do fundo de tabelas
\usepackage{colortbl} % Para cores do fundo de tabelas
\usepackage{upgreek} % Para inserir caracteres gregos

%tikzpicture
\begin{comment}
	\usepackage{tikz}
	\usepackage{scalerel}
	\usepackage{pict2e}
	\usepackage{tkz-euclide}
	\usetikzlibrary{calc}
	\usetikzlibrary{patterns,arrows.meta}
	\usetikzlibrary{shadows}
	\usetikzlibrary{external}
\end{comment}


%pgfplots
\usepackage{pgfplots}
\pgfplotsset{compat=newest}
\usepgfplotslibrary{statistics}
\usepgfplotslibrary{fillbetween}

%colours
\usepackage{xcolor}



\columnsep=2cm
\hoffset=0cm
\textwidth=8cm
\setlength{\columnseprule}{.1pt}
\setlength{\columnsep}{2cm}
\renewcommand{\headrulewidth}{0pt}
\geometry{top=1in, bottom=1in, left=0.7in, right=0.5in}

\pagestyle{fancy}
\fancyhf{}
\fancyfoot[C]{\thepage}

\begin{document}
	
	\noindent\textbf{6FMA153 - Matemática} 
	
	\begin{center}Fração geratriz (Versão estudante)
	\end{center}
	
	\noindent\textbf{Nome:} \underline{\hspace{10cm}}
	\noindent\textbf{Data:} \underline{\hspace{4cm}}
	
	%\section*{Questões de Matemática}
	
	\begin{multicols}{2}
	    \noindent Transformando dízimas periódicas em frações: \\
	    \begin{itemize}
	    	\item $x = 0,\overline{3}$
	    	\begin{align*}
	    		10x &= 3{,}333\ldots \\
	    		-x &= 0{,}333\ldots \\
	    		\cline{1-2}
	    		9x &= 3 \\
	    		x &= \frac{3}{9} = \frac{1}{3}
	    	\end{align*}
	    	
	    	\item $x = 0,1\overline{6}$
	    	\begin{align*}
	    		100x &= 16{,}666\ldots \\
	    		-10x &= \;\;1{,}666\ldots \\
	    		\cline{1-2}
	    		90x &= 15 \\
	    		x &= \frac{15}{90} = \frac{1}{6}
	    	\end{align*}
	    \end{itemize}
		\noindent\textsubscript{--------------------------------------------------------------------------}
		\begin{enumerate} 
			\item Diga quais frações a seguir originam dízimas periódicas:
			\begin{enumerate}[a)]
				\item $\frac{1}{3}$ \\\\\\\\
				\item $\frac{1}{8}$ \\\\\\\\
				\item $\frac{1}{5000}$ \\\\\\\\
				\item $\frac{1}{18}$ \\\\\\\\
				\item $\frac{1}{22}$ \\\\\\\\
			\end{enumerate}
			\item O número 0,999...9... é maior do que 1, menor do que 1 ou igual a 1? \\\\\\\\\\
			\item Determine as geratrizes das dízimas periódicas a seguir:
			\begin{enumerate}[a)]
				\item 0,444...4... \\\\\\\\
				\item 2,888...8... \\\\\\\\
				\item $7,\dot{2}$ \\\\\\\\
				\item $0,323232...$ \\\\\\\\
				\item $0,1\overline{26}$ \\\\\\\\
				\item $4,1\dot{7}$ \\\\\\\\
			\end{enumerate}
			%30 a 34
			\item Obtenha a geratriz de cada uma das dízimas periódicas a seguir:
			\begin{enumerate}[a)]
				\item 0,444... \\\\\\\\
				\item 2,888... \\\\\\\\
				\item 0,3444... \\\\\\\\
				\item 1,7333... \\\\\\\\
				\item $3,\dot{2}$ \\\\\\\\
				\item $0,\overline{84}$ \\\\\\\\
				\item $2,0\dot{6}$ \\\\\\\\
				\item $4,6\overline{93}$ \\\\\\\\
			\end{enumerate}
			\item Assinale \textbf{V} (verdadeiro) ou \textbf{F} (falso) e justifique:
			\begin{enumerate}[a)]
				\item (~~) $0,999... < 1$
				\item (~~) $\frac{13}{9} = 1,444...$
				\item (~~) $2,1\overline{76} = \frac{70}{33}$
				\item (~~) $\frac{3206}{999} = 3,\overline{209}$
			\end{enumerate}
			\item Qual é a geratriz da dízima periódica 0,37222...? \\\\\\\\\\
			\item A fração geratriz da dízima periódica composta $0,7\dot{1}$ é:
			\begin{enumerate}[a)]
				\item $\frac{23}{45}$
				\item $\frac{32}{45}$
				\item $\frac{61}{90}$
				\item $\frac{67}{90}$
				\item $\frac{71}{99}$
			\end{enumerate}
			\item Usando lápis e papel ou sua calculadora, quando possível, transforme em numerais decimais as frações: \\\\
			$\frac{1}{2}, \frac{1}{3}, \frac{1}{4}, \frac{1}{5}, \frac{1}{6}, \frac{1}{7}, \frac{1}{8}, \frac{1}{9}$ e $\frac{1}{10}$.
		\end{enumerate}
		$~$ \\ $~$ \\ $~$ \\ $~$ \\ $~$ \\ $~$ \\ $~$ \\ $~$ \\ $~$ \\ $~$ \\ $~$ \\ $~$ \\ $~$ \\ $~$ \\ $~$ \\ $~$ \\ $~$ \\ $~$ \\ $~$ \\ $~$ \\ $~$ \\ $~$ \\ $~$ \\ $~$ \\ $~$ \\ $~$ \\ $~$ \\ $~$ \\ $~$ \\ $~$ \\ $~$ \\ $~$ \\ $~$ \\ $~$ \\ $~$ \\ $~$ \\ $~$ \\ $~$ \\ $~$ \\ $~$ \\ $~$ \\ $~$ \\ $~$ \\ $~$ \\ $~$ \\ $~$ \\ $~$ \\ $~$ \\ $~$ \\ $~$ \\ $~$ \\ $~$ \\ $~$ \\ $~$ \\ $~$ \\
	\end{multicols}
\end{document}