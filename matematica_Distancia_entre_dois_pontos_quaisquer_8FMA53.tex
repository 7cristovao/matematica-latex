\documentclass[a4paper,14pt]{article}
\usepackage{float}
\usepackage{extsizes}
\usepackage{amsmath}
\usepackage{amssymb}
\everymath{\displaystyle}
\usepackage{geometry}
\usepackage{fancyhdr}
\usepackage{multicol}
\usepackage{graphicx}
\usepackage[brazil]{babel}
\usepackage[shortlabels]{enumitem}
\usepackage{cancel}
\columnsep=2cm
\hoffset=0cm
\textwidth=8cm
\setlength{\columnseprule}{.1pt}
\setlength{\columnsep}{2cm}
\renewcommand{\headrulewidth}{0pt}
\geometry{top=1in, bottom=1in, left=0.7in, right=0.5in}

\pagestyle{fancy}
\fancyhf{}
\fancyfoot[C]{\thepage}

\begin{document}
	
	\noindent\textbf{8FMA53~Matemática} 
	
	\begin{center}Distância entre dois pontos quaisquer (Versão estudante)
	\end{center}
	
	\noindent\textbf{Nome:} \underline{\hspace{10cm}}
	\noindent\textbf{Data:} \underline{\hspace{4cm}}
	
	%\section*{Questões de Matemática}
	
	
    \begin{multicols}{2}
    	Sendo $A = (a; b)$ e $B = (c; d)$, $d(A, B) = \sqrt{(a - c)^2 + (b - d)^2}$.\\
    	\textsubscript{---------------------------------------------------------------------}
    	\begin{enumerate}
    		\item Nos itens a seguir, são dados dois pontos. Apresentar a distância entre eles usando a fórmula dada anteriormente.
    		\begin{enumerate}[a)]
    			\item $A = (4; 7)$ e $B = (4; -3)$.\\\\\\\\\\\\
    			\item $A = (-8; 2)$ e $B = (2; 0)$.\\\\\\\\\\\\
    			\item $A = (0; 6)$ e $B = (0; -3)$.\\\\\\\\\\\\
    			\item $A = (2; 3)$ e $B = (3; 3)$.\\\\\\\\\\
            \end{enumerate}
            \item Encontre o perímetro do quadrilátero $ABCD$, $A = (8; 2)$, $B = (6, 5)$, $C = (2, 3)$ e $D = (5, 1)$.\\\\\\\\\\\\\\\\\\\\\\\\\\\\
            \item Encontre os pontos $A$ e $B$ de forma que os triângulos $ACD$ e $BCD$, com $C = (0; 1)$ e $D = (2; 3)$, sejam equiláteros.\\\\\\\\\\\\\\\\\\\\\\\\
            \item Se o ponto $A = (m; -1)$ é equidistante dos pontos $B = (-6; -4)$ e $C = (2; 2)$, então $D = (m + 2;~m - 2)$ é um ponto pertencente:
            \begin{enumerate}[a)]
            	\item ao eixo das ordenadas.
            	\item à reta de equação $y = -x$
            	\item ao terceiro quadrante.
            	\item ao primeiro quadrante.
            	\item ao eixo das abscissas.\\\\\\\\\\\\\\\\\\\\\\\\\\\\\\\\\\\\\\\\\\\\\\\\\\\\\\\\
            \end{enumerate}
            \item Os pontos $A = (a; 5)$ e $B = (-1; b)$ pertencem à reta de equação $4x - y + 1 = 0$. A distância entre $A$ e $B$ é:
            \begin{enumerate}[a)]
            	\item $3\sqrt{5}$
            	\item $2\sqrt{17}$
            	\item $4\sqrt{7}$
            	\item $2\sqrt{13}$
            	\item $3\sqrt{11}$
            \end{enumerate}
        \end{enumerate}
    $~$ \\ $~$ \\ $~$ \\ $~$ \\ $~$ \\ $~$ \\ $~$ \\ $~$ \\ $~$ \\ $~$ \\ $~$ \\ $~$ \\ $~$ \\ $~$ \\ $~$ \\ $~$ \\ $~$ \\ $~$ \\ $~$ \\ $~$ \\ $~$ \\ $~$ \\ $~$ \\ $~$ \\ $~$ \\ $~$ \\ $~$ \\ $~$ \\ $~$     
    \end{multicols}
\end{document}