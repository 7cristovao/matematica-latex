\documentclass[a4paper,14pt]{article}

\usepackage{comment} % Para comentar várias linhas ao mesmo tempo

%matemática
\usepackage{amsmath}
\usepackage{amssymb}

%diagramação
\usepackage{extsizes}
\everymath{\displaystyle}
\usepackage{geometry}
\usepackage{fancyhdr}
\usepackage{multicol}
\usepackage{graphicx}
\usepackage[brazil]{babel}
\usepackage[shortlabels]{enumitem}
\usepackage{cancel}
\usepackage{textcomp}
\usepackage{tcolorbox}

%tabelas
\usepackage{array} % Para melhor formatação de tabelas
\usepackage{longtable}
\usepackage{booktabs}  % Para linhas horizontais mais bonitas
\usepackage{float}   % Para usar o modificador [H]
\usepackage{caption} % Para usar legendas em tabelas
\usepackage{wrapfig} % Para usar tabelas e figuras flutuantes
\usepackage{xcolor} % Para cores do fundo de tabelas
\usepackage{colortbl} % Para cores do fundo de tabelas

%tikzpicture
\begin{comment}
	\usepackage{tikz}
	\usepackage{scalerel}
	\usepackage{pict2e}
	\usepackage{tkz-euclide}
	\usetikzlibrary{calc}
	\usetikzlibrary{patterns,arrows.meta}
	\usetikzlibrary{shadows}
	\usetikzlibrary{external}
\end{comment}


%pgfplots
\usepackage{pgfplots}
\pgfplotsset{compat=newest}
\usepgfplotslibrary{statistics}
\usepgfplotslibrary{fillbetween}

%colours
\usepackage{xcolor}



\columnsep=2cm
\hoffset=0cm
\textwidth=8cm
\setlength{\columnseprule}{.1pt}
\setlength{\columnsep}{2cm}
\renewcommand{\headrulewidth}{0pt}
\geometry{top=1in, bottom=1in, left=0.7in, right=0.5in}

\pagestyle{fancy}
\fancyhf{}
\fancyfoot[C]{\thepage}

\begin{document}
	
	\noindent\textbf{6FMA24 - Matemática} 
	
	\begin{center}Módulo ou valor absoluto de um número inteiro (Versão estudante)
	\end{center}
	
	\noindent\textbf{Nome:} \underline{\hspace{10cm}}
	\noindent\textbf{Data:} \underline{\hspace{4cm}}
	
	%\section*{Questões de Matemática}
	
	\begin{multicols}{2}
		\noindent O módulo ou valor absoluto de um número inteiro $x$ é a distância de zero até o número $x$. Indicamos o módulo de $x$ por $|x|$. Assim, $|x| = d(0, x)$. \\
		Para todo inteiro $x$, temos: $|x| \geq 0$; $|x| = 0$ se, e somente se, $x = 0$; e $|-x| = |x|$. \\
		Existem outras definições equivalentes. A mais usada é: \\
		$|x| = \begin{cases}
			x, \text{se~} x \geq 0 \\
			-x, \text{se~} x < 0
		\end{cases}$ \\
		\noindent\textsubscript{-----------------------------------------------------------------------}
		\begin{enumerate} 
			\item Complete.
			\begin{enumerate}[a)]
				\item $|-7| = \underline{~~~~~~~~~~~~~~~~~~~~}$
				\item $|2| = \underline{~~~~~~~~~~~~~~~~~~~~}$
				\item $|0| = \underline{~~~~~~~~~~~~~~~~~~~~}$
				\item $|8 - 7| = \underline{~~~~~~~~~~~~~~~~~~~~}$
				\item $|5 - 2| = \underline{~~~~~~~~~~~~~~~~~~~~}$
				\item $-|-3| = \underline{~~~~~~~~~~~~~~~~~~~~}$
				\item $-|9| = \underline{~~~~~~~~~~~~~~~~~~~~}$
				\item $-|-|6|| = \underline{~~~~~~~~~~~~~~~~~~~~}$
				\item $|-|-1|| = \underline{~~~~~~~~~~~~~~~~~~~~}$
			\end{enumerate}
			\item Assinale \textbf{V} (verdadeiro) ou \textbf{F} (falso). Se a sentença for falsa, dê o contraexemplo.
			\begin{enumerate}[a)]
				\item (~~) Para todo $x$ inteiro positivo, $|x| = x$. \\\\\\\\
				\item (~~) Para todo $x$ inteiro negativo, $|x| = x$. \\\\\\\\
				\item (~~) Para todo $x$ inteiro negativo, $|x| = -x$.
				\item (~~) Para todo $x$ inteiro, $|x| = x$.
			\end{enumerate}
			\item Assinale \textbf{V} (verdadeiro) ou \textbf{F} (falso).
			\begin{enumerate}[a)]
				\item (~~) Para todo inteiro $x, |x| \geq 0$. \\\\\\\\\\
				\item (~~) O único inteiro que tem valor absoluto igual a ele mesmo é o zero. \\\\\\\\\\
				\item (~~) Existe algum inteiro $x$ tal que $|x| < 0$.
				 \\\\\\
				\item (~~) Existe algum inteiro $x$ tal que $|x| \leq 0$. \\\\\\\\
				\item (~~) Se $x$ é negativo, então $|x| = -x$. \\\\\\\\
				\item (~~) Se $x$ é positivo, então $|x| = x$. \\\\\\\\
				\item (~~) O único número inteiro que tem valor absoluto zero é o zero. \\\\\\\\
			\end{enumerate}
			\textbf{Desafio olímpico} \\\\
			(OBMEP) Marina, ao comprar uma blusa de R\$ 17,00, enganou-se e deu ao vendedor uma nota de R\$ 10,00 e outra de R\$ 50,00. O vendedor, distraído, deu o troco como se Marina lhe tivesse dado duas notas de R\$ 10,00. Qual foi o prejuízo de Marina?
			\begin{enumerate}[a)]
				\item R\$ 13,00
				\item R\$ 37,00
				\item R\$ 40,00
				\item R\$ 47,00
				\item R\$ 50,00
			\end{enumerate}
			%85 a 89
			\item Defina $|x|$. \\\\\\\\\\\\\\\\\\\\
			\item Complete.
			\begin{enumerate}[a)]
				\item $|-2| = .....$
				\item $|3| = .....$
				\item $|0| = .....$
				\item $-|-1| = .....$
				\item $|8 - 3| = .....$
				\item $|9 - 7| = .....$
				\item $-|7 - 6| = .....$
				\item $-|-|4|| = .....$
				\item $|-|-2|| = .....$
			\end{enumerate}
			\item Assinale \textbf{V} (verdadeiro) ou \textbf{F} (falso).
			\begin{enumerate}[a)]
				\item (~~) -(-3) = -3.
				\item (~~) O oposto de -(-7) é -7.
				\item (~~) O simétrico do simétrico de 15 é -15.
				\item (~~) Se $x$ é inteiro, então $-x$ é negativo.
				\item (~~) Se $x$ é inteiro e $x > 0$, então $-x$ é negativo.
				.
				\item (~~) Se $x \in \mathbb{Z}$ e $-x$ é positivo, então $x$ é positivo.
				\item (~~) Para todo $x \in \mathbb{Z}$, -(-x) = x.
				\item (~~) Para todo inteiro $x$, $|x| > 0$.
				\item (~~) Existe algum inteiro $x$ tal que $|x| \leq 0$.
			\end{enumerate}
			\item Assinale \textbf{V} (verdadeiro) ou \textbf{F} (falso).
			\begin{enumerate}[a)]
				\item (~~) Se $-x$ é positivo, então $|-x| = x$.
				\item (~~) Se $|x| = 0$, então $x = 0$.
				\item (~~) Se $x$ é negativo, então $|-x| = -x$.
				\item (~~) Para todo inteiro $x, |-x| < |x|$.
				\item (~~) Para todo inteiro $x, -|x| \leq x \leq |x|$.
			\end{enumerate}
			\item Resolva.
			\begin{enumerate}[a)]
				\item Calcule o oposto do módulo de -3. \\\\\\\\\\\\\\\\\\\\\\\\\\\\\\\\
				\item Calcule o módulo do oposto de -3. \\\\\\\\\\\\\\\\\\\\
				\item Existe algum número inteiro tal que o oposto do seu módulo seja igual ao módulo de seu oposto? \\\\\\\\\\\\\\\\\\\\
			\end{enumerate}
		\end{enumerate}
		$~$ \\ $~$ \\ $~$ \\ $~$ \\ $~$ \\ $~$ \\ $~$ \\ $~$ \\ $~$ \\ $~$ \\ $~$ \\ $~$ \\ $~$ \\ $~$ \\ $~$
	\end{multicols}
\end{document}