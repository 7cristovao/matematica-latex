\documentclass[a4paper,14pt]{article}
\usepackage{float}
\usepackage{extsizes}
\usepackage{amsmath}
\usepackage{amssymb}
\everymath{\displaystyle}
\usepackage{geometry}
\usepackage{fancyhdr}
\usepackage{multicol}
\usepackage{graphicx}
\usepackage[brazil]{babel}
\usepackage[shortlabels]{enumitem}
\usepackage{cancel}
\usepackage{textcomp}
\usepackage{array} % Para melhor formatação de tabelas
\usepackage{longtable}
\usepackage{booktabs}  % Para linhas horizontais mais bonitas
\usepackage{float}   % Para usar o modificador [H]
\usepackage{caption} % Para usar legendas em tabelas
\usepackage{tcolorbox}

\columnsep=2cm
\hoffset=0cm
\textwidth=8cm
\setlength{\columnseprule}{.1pt}
\setlength{\columnsep}{2cm}
\renewcommand{\headrulewidth}{0pt}
\geometry{top=1in, bottom=1in, left=0.7in, right=0.5in}

\pagestyle{fancy}
\fancyhf{}
\fancyfoot[C]{\thepage}

\begin{document}
	
	\noindent\textbf{6FMA100 - Matemática} 
	
	\begin{center}Frações: multiplicação e divisão (Versão estudante)
	\end{center}
	
	\noindent\textbf{Nome:} \underline{\hspace{10cm}}
	\noindent\textbf{Data:} \underline{\hspace{4cm}}
	
	%\section*{Questões de Matemática}
	~ \\ ~
	\begin{multicols}{2}
		\noindent Sendo $a, b, c, d \in \mathbb{Z}$; $b \neq 0$, $c \neq 0$ e $d \neq 0$, temos:
		\begin{itemize}
			\item $a \cdot \frac{b}{c} = \frac{a \cdot b}{c}$ 
			\item $\frac{a}{b} \cdot \frac{c}{d} = \frac{a \cdot c}{b \cdot d}$
			\item $\frac{a}{b} : \frac{c}{d} = \frac{a}{b} \cdot \frac{d}{c} = \frac{a \cdot d}{b \cdot c}$
		\end{itemize}
		O inverso de um número racional não nulo $\frac{a}{b}$ é $\frac{b}{a}$.
		\textsubscript{---------------------------------------------------------------------}
    	\begin{enumerate}
    		\item Calcule:
    		\begin{enumerate}[a)]
    			\item $4 \cdot \frac{7}{11}$ \\\\\\\\\\\\\\\\
    			\item $5 \cdot \frac{6}{5}$ \\\\\\\\\\\\\\\\
    			\item $-9 \cdot \frac{-3}{2}$ \\\\\\\\\\\\\\\\
    			\item $-8 \cdot \bigg(-\frac{2}{32} \bigg)$ \\\\\\\\\\\\\\\\
    		\end{enumerate}
    		\item Calcule:
    		\begin{enumerate}[a)]
    			\item $\frac{4}{9}$ de 63 \\\\\\\\\\\\\\\\\\\\\\
    			\item $\frac{-5}{7}$ de 42 \\\\\\\\\\\\\\\\
    			\item $\frac{3}{13}$ de 117 \\\\\\\\\\\\\\\\
    		\end{enumerate}
    		\item Determine o valor da expressão: \\\\
    		$8 - \frac{4}{9} \cdot 18 + 7 \cdot \bigg(-\frac{12}{4} \bigg)$ \\\\\\\\\\\\\\\\
    		\item Faça as multiplicações indicadas a seguir:
    		\begin{enumerate}[a)]
    			\item $\frac{3}{4} \cdot \frac{4}{7}$ \\\\\\\\\\\\
    			\item $\frac{19}{6} \cdot \frac{42}{38}$ \\\\\\\\\\\\\\\\
    			\item $\frac{-15}{6} \cdot \bigg(-\frac{36}{3}\bigg)$ \\\\\\\\\\\\\\\\
    		\end{enumerate}
    		\item Calcule: 
    		\begin{enumerate}[a)]
    			\item $\frac{1}{5} : 4$ \\\\\\\\\\\\\\\\
    			\item $-\frac{1}{3} : 6$ \\\\\\\\\\\\\\\\\\ 
    			\item $\frac{3}{4} : 9$ \\\\\\\\\\\\\\\\
    		\end{enumerate}
    		\item Calcule o valor da expressão: \\\\
    		$\frac{3}{-5} : 4 - 2 \cdot \bigg(\frac{-3}{8} \bigg) + 4 : 5$ \\\\\\\\\\\\\\\\\\\\
    		\item Apresente o inverso de:
    		\begin{enumerate}[a)]
    			\item $\frac{4}{9}$ \\\\\\\\\\\\\\\\
    			\item $-\frac{5}{13}$ \\\\\\\\\\
    			\item $4 \frac{7}{2}$ \\\\\\\\\\\\\\\\
    		\end{enumerate}
    		\item Calcule (simplifique sempre que possível):
    		\begin{enumerate}[a)]
    			\item $\frac{1}{2} : \frac{1}{7}$ \\\\\\\\\\\\\\\\
    			\item $\frac{1}{14} : \frac{(-42)}{9}$ \\\\\\\\\\\\\\\\
    			\item $\frac{32}{(-7)} : \frac{(-4)}{9}$ \\\\\\\\
    			\item $\frac{-6}{5} : \frac{1}{3}$ \\\\
    		\end{enumerate}
    		\item Dê o inverso de:
    		\begin{enumerate}[a)]
    			\item $8$ \\\\\\
    			\item $-12$ \\\\\\
    			\item $-5$ \\\\\\
    			\item $\frac{2}{7}$ \\\\\\
    			\item $\frac{9}{5}$ \\\\\\
    			\item $-\frac{13}{2}$ \\\\\\
    			\item $\frac{3}{101}$ \\
    			\item $-\frac{12}{13}$ \\\\\\
    		\end{enumerate}
    		\item Efetue:
    		\begin{enumerate}[a)]
    			\item $\frac{5}{6} \cdot \frac{6}{15}$ \\\\\\
    			\item $\frac{4}{9} \cdot \frac{1}{3}$ \\\\\\\\
    			\item $\frac{8}{3} \cdot \frac{4}{5} \cdot \frac{9}{8}$ \\\\\\\\
    			\item $5 : \frac{1}{4}$ \\\\\\\\
    			\item $6 : \bigg(\frac{3}{4}\bigg)$ \\\\\\\\
    			\item $7 : \bigg(-\frac{1}{4}\bigg)$ \\\\\\\\\\
    			\item $(-6) : \bigg(-\frac{3}{7}\bigg)$ \\\\\\\\
    			\item $\frac{1}{3} : \frac{1}{7}$ \\\\
    			\item $-\frac{4}{5} : \frac{3}{5}$ \\\\
    			\item $\frac{5}{6} : \bigg(\frac{-10}{3}\bigg)$ \\\\\\
    			\item $\frac{2}{3} : \frac{4}{15}$ \\\\\\
    			\item $\dfrac{\dfrac{~1}{~4}}{\dfrac{~1}{~3}}$ \\\\\\
    			\item $\dfrac{\dfrac{~2}{~5}}{\dfrac{~3}{~5}}$ \\\\\\
    			\item $\dfrac{-\dfrac{5}{6}}{-\dfrac{7}{12}}$ \\\\\\
    		\end{enumerate}
    		\item Efetue as operações indicadas a seguir:
    		\begin{enumerate}[a)]
    			\item $4 : \frac{1}{2} - 2 : \frac{1}{3} + \frac{5}{7} : \frac{6}{7} - 9 : \frac{3}{4}$ \\\\\\\\\\\\\\\\\\
    			\item $\frac{1}{6} \cdot 5 + 3 : \bigg(-\frac{2}{3}\bigg) -\frac{1}{2} \cdot \bigg(-\frac{1}{6}\bigg)$ \\\\\\\\\\\\\\\\\\\\\\\\\\\\\\\\\\\\\\
    		\end{enumerate}
    		\item É correto afirmar que $a \cdot \frac{b}{c} = \frac{b}{c} \cdot a$? Justifique sua resposta. \\\\\\\\\\\\\\\\\\\\\\\\\\\\\\\\\\
    		\item (PUC-SP) Se $a$ e $b$ são números inteiros, $1 \leq a < b \leq 9$, o menor valor que $\frac{a + b}{ab}$ pode assumir é:
    		\begin{enumerate}[a)]
    			\item $1$
    			\item $\frac{15}{56}$
    			\item $\frac{2}{9}$
    			\item $\frac{9}{20}$
    			\item $\frac{17}{72}$
    		\end{enumerate}
    		\item A expressão $\dfrac{\dfrac{2}{9} + \dfrac{1}{9} + \dfrac{2}{9}}{\dfrac{2}{9}}$ é igual a:
    		\begin{enumerate}[a)]
    			\item $\frac{7}{9}$
    			\item $\frac{4}{5}$
    			\item $\frac{8}{81}$
    			\item $\frac{2}{9}$
    			\item $\frac{5}{2}$ \\\\\\\\\\\\\\\\\\\\\\
    		\end{enumerate}
    		\item O valor de $\frac{1}{x}$, quando \\\\ $x~$= $\dfrac{1 - \dfrac{1}{1 + \dfrac{1}{3}}}{1 + \dfrac{1}{1 + \dfrac{1}{2}}}$, é um número:
    		\begin{enumerate}[a)]
    			\item inteiro.
    			\item maior que 2 e menor que 5.
    			\item maior que 6 e menor que 7.
    			\item menor que -2.
    			\item maior que 5 e menor que 6.
    		\end{enumerate}
    	\end{enumerate}
    $~$ \\ $~$ \\ $~$ \\ $~$ \\ $~$ \\ $~$ \\ $~$ \\ $~$ \\ $~$ \\ $~$ \\ $~$ \\ $~$ \\ $~$ \\ $~$ \\ $~$ \\ $~$ \\ $~$ \\ $~$ \\ $~$ \\ $~$ \\ $~$ \\ $~$ \\ $~$ \\ $~$ \\ $~$ \\ $~$
	\end{multicols}
\end{document}