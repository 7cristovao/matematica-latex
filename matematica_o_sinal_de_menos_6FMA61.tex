\documentclass[a4paper,14pt]{article}
\usepackage{float}
\usepackage{extsizes}
\usepackage{amsmath}
\usepackage{amssymb}
\everymath{\displaystyle}
\usepackage{geometry}
\usepackage{fancyhdr}
\usepackage{multicol}
\usepackage{graphicx}
\usepackage[brazil]{babel}
\usepackage[shortlabels]{enumitem}
\usepackage{cancel}
\usepackage{textcomp}
\usepackage{array} % Para melhor formatação de tabelas
\usepackage{longtable}
\usepackage{booktabs}  % Para linhas horizontais mais bonitas
\usepackage{float}   % Para usar o modificador [H]
\usepackage{caption} % Para usar legendas em tabelas
\usepackage{tcolorbox}
\usepackage{wrapfig} % Para usar tabelas e figuras flutuantes

\columnsep=2cm
\hoffset=0cm
\textwidth=8cm
\setlength{\columnseprule}{.1pt}
\setlength{\columnsep}{2cm}
\renewcommand{\headrulewidth}{0pt}
\geometry{top=1in, bottom=1in, left=0.7in, right=0.5in}

\pagestyle{fancy}
\fancyhf{}
\fancyfoot[C]{\thepage}

\begin{document}
	
	\noindent\textbf{6FMA61 - Matemática} 
	
	\begin{center}O sinal de "menos" (Versão estudante)
	\end{center}
	
	\noindent\textbf{Nome:} \underline{\hspace{10cm}}
	\noindent\textbf{Data:} \underline{\hspace{4cm}}
	
	%\section*{Questões de Matemática}
	\begin{multicols}{2}
    		\noindent Subtrair significa somar com seu oposto, isto é, x - y = x +(-y). \\
    		Exemplo: 
    		\begin{equation*}5 - 2 = 5 + (-2) = 3\end{equation*} \\\\
    		\textbf{Nomes dos termos da subtração} \\
    		Se $a - b = c$, então:
    		\begin{itemize}
    			\item $a$ é chamado \textbf{minuendo};
    			\item $b$ é chamado \textbf{subtraendo};
    			\item $c$ é chamado \textbf{diferença}.
    		\end{itemize}
    		\textsubscript{---------------------------------------------------------------------}
    		\begin{enumerate}
    			\item As maiores variações de temperatura do mundo encontram-se na Sibéria. A cidade de Yakutsk, por exemplo, chegou a registrar temperaturas que variam de -64°C a 32°C. Qual a diferença entre a maior e a menor temperatura nessa localidade? \\\\\\\\\\\\\\\\\\
    			\item Se $x$ e $y$ são números inteiros, representando-os em uma reta, dizer qual a distância entre $x$ e $y$. Indicamos distância como $d(x, y)$. \\\\\\\\
    			\item Calcular $d(x, y)$, sendo:
    			\begin{enumerate}[I.]
    				\item $x = -3, y = 0$ \\\\\\
    				\item $x = 2, y = 9$ \\\\\\
    				\item $x = 0, y = -5$ \\\\\\
    				\item $x = -6, y = -8$ \\\\\\
    				\item $x = -4, y = 4$ \\\\\\
    				\item $x = 13, y = -7$ \newpage
    			\end{enumerate}
    			\item Paulo ganhou de seu pai uma certa quantia num domingo de manhã. Sua tia deu-lhe também, nessa mesma manhã, R\$ 20,00. À tarde, foi ao $shopping$ e gastou alguns reais. À noite, percebeu que tinha R\$ 8,00 a menos do que seu pai lhe dera. Quanto Paulo gastou nesse domingo? (Fazer esse exercício de duas maneiras. Explicar.) \\\\\\\\\\\\\\
    			\item Determine quantos inteiros há entre os seguintes inteiros (não incluir as extremidades).
    			\begin{enumerate}[a)]
    				\item 3 e 8. \\\\\\
    				\item 40 e 90. \\\\\\
    				\item -50 e 35. \\\\\\
    				\item -43 e 21. \\\\\\
    			\end{enumerate}
    			% 39 a 42
    			\item 
    			\begin{enumerate}[a)]
    				\item Durante uma sexta-feira em uma pequena cidade, a temperatura máxima chegou a 35°C e a mínima, 16°C. Qual foi a variação de temperatura nesse dia? \\\\\\\\\\\\\\
    				\item No mês seguinte, a variação de temperatura durante um sábado foi 52°C. Se a temperatura máxima nesse dia foi 48°C, qual foi a mínima? \\\\\\\\\\\\\\
    				\item Neste item, complete o enunciado com algum número do quadro abaixo e, em seguida, resolva-o: \\
    				Durante uma semana, a variação de temperatura nessa cidade foi de 41°C. Se a temperatura mínima dessa semana foi ...... °C, qual foi a máxima?
    				\begin{center} \begin{tcolorbox}[colback=white, colframe=black, boxrule=0.5mm, width=5cm]
    						$-6 ~~~~~~~ 21 ~~~~~ -1 \\\\
    						13 ~~~~ -15 ~~~~~~~~ 4$
    					\end{tcolorbox} \newpage
    				\end{center}
        		\end{enumerate}
        		\item 
        		\begin{enumerate}[a)]
        			\item Em Montana, nos Estados Unidos, a temperatura variou de 7°C a -49°C durante um único dia em 1916. Qual foi a diferença entre a maior e a menor temperatura nesse dia? \\\\\\\\\\\\\\
        			\item Neste item, complete as lacunas $A$ e $B$ com algum número de sua respectiva coluna da tabela e, em seguida, resolva o enunciado: \\
        			Na cidade de Alice, a temperatura variou de $A$ a $B$ durante um outono. Qual a diferença entre a maior e a menor temperatura nesse outono?
					\begin{table}[H]
						\centering
						\begin{tabular}{|c|c|}
							\hline
							\textbf{A} & \textbf{B} \\ 
							\hline
							-3°C & 21°C \\ 
							\hline
							5°C & 34°C \\ 
							\hline
							-6°C & 19°C \\ 
							\hline
							8°C & 28°C \\
							\hline
							2°C & 30°C \\
							\hline
						\end{tabular}
						\vspace{8cm}
					\end{table}
        		\end{enumerate}
        		\item Em uma fase de um jogo de batalha, a nave de Manoel está com -25 pontos de energia quando ele pede para Ana assumir o controle. Como já conhece essa fase, ela assume o controle da nave e, depois de atingir 140 pontos de energia, Manoel pede de volta o controle da nave. Quantos pontos de energia Ana obteve até Manoel pedir de volta o controle? \newpage
        		\item Para cada caso (I e II), determine quantos inteiros há entre os seguintes inteiros: \\
        		I. Incluindo as extremidades.
        		\begin{enumerate}[a)]
        			\item 4 e 12. \\\\\\
        			\item 21 e 53. \\\\\\
        			\item $a$ e $b$ com $a < b$. \\\\\\
        			\noindent II. Não incluindo as extremidades.
        			\item 5 e 14. \\\\\\
        			\item 87 e 123. \\\\\\
        			\item $c$ e $d$ com $c < d$. \\\\\\
        		\end{enumerate}
        	\end{enumerate}
        	$~$ \\ $~$ \\ $~$ \\ $~$ \\ $~$ \\ $~$ \\ $~$ \\ $~$ \\ $~$ \\ $~$ \\ $~$ \\ $~$ \\ $~$ \\ $~$ \\ $~$ \\ $~$ \\ $~$ \\ $~$ \\ $~$ \\ $~$ \\ $~$ \\ $~$ \\ $~$ \\ $~$ \\ $~$ \\ $~$ \\ $~$ \\ $~$ \\ $~$ \\ $~$ \\ $~$ \\ $~$ \\ $~$ \\ $~$ \\ $~$ \\ $~$ \\ $~$ \\ $~$ \\ $~$ \\ $~$ \\ $~$ \\ $~$ \\ $~$ \\ $~$ \\ $~$ \\ $~$ \\ $~$ \\ $~$ \\ $~$ \\ $~$ \\ $~$
	\end{multicols}
\end{document}