\documentclass[a4paper,14pt]{article}
\usepackage{extsizes}
\usepackage{amsmath}
\usepackage{amssymb}
\everymath{\displaystyle}
\usepackage{geometry}
\usepackage{fancyhdr}
\usepackage{multicol}
\usepackage{graphicx}
\usepackage[brazil]{babel}
\usepackage[shortlabels]{enumitem}
\usepackage{cancel}
\columnsep=2cm
\hoffset=0cm
\textwidth=8cm
\setlength{\columnseprule}{.1pt}
\setlength{\columnsep}{2cm}
\renewcommand{\headrulewidth}{0pt}
\geometry{top=1in, bottom=1in, left=0.7in, right=0.5in}

\pagestyle{fancy}
\fancyhf{}
\fancyfoot[C]{\thepage}

\begin{document}
	
	\noindent\textbf{EF07MA08-A~-~Matemática} 
	
	\begin{center}
		\textbf{Potências de expoentes negativos (Versão professor)}
	\end{center}
	
	
	\noindent\textbf{Nome:} \underline{\hspace{10cm}}
    \noindent\textbf{Data:} \underline{\hspace{4cm}}
	
	%\section*{Questões de Matemática}
	
	\begin{multicols}{2}
		%
	\begin{enumerate}	
		\item Escrever usando radical:
		\begin{enumerate}[a)]
			\item $12^{-\frac{1}{4}} = \frac{1}{12^\frac{1}{4}} = \frac{1}{\sqrt[4]{12}}$ \\\\
			\item $5^{-\frac{5}{6}} = \frac{1}{5^{\frac{5}{6}}} = \frac{1}{\sqrt[6]{5^5}}$ \\\\
			\item $(-7)^{-\frac{8}{3}} = \frac{1}{(-7)^{\frac{8}{3}}} = \frac{1}{\sqrt[3]{(-7)^8}}$ \\\\
	    \end{enumerate}
        \item Escrever usando expoente racional:
        \begin{enumerate}[a)]
        	\item $\sqrt[-5]{9} = 9^{-\frac{1}{5}}$ \\\\
        	\item $\sqrt[9]{(-4)^{-5}} = (-4)^{-\frac{5}{9}} = \frac{1}{(-4)^\frac{5}{9}}$ \\\\
        	\item $\sqrt[-5]{(-6)^{10}} = (-6)^{-\frac{10}{5}} = \frac{1}{(-6)^\frac{10}{5}}$ \\\\
        	\item $\frac{1}{\sqrt{4^5}} = \frac{1}{4^\frac{5}{2}} = 4^{-\frac{5}{2}}$ \\\\\\\\
        \end{enumerate}
        \item Calcule o que for possível com o que sabemos até agora:
        \begin{enumerate}[a)] 
        	\item $64^\frac{1}{2} = \sqrt[2]{64^1} = \sqrt{64} = 8$ \\\\\\
        	\item $(-8)^\frac{1}{3} = \sqrt[3]{(-8)^1} = -2$ \\\\\\
        	\item $0^\frac{31}{72} = 0$ \\\\\\
        	\item $1^{-\frac{5}{7}} = 1$ \\\\\\
        \end{enumerate}
        \item Calcule:
        \begin{enumerate}[a)]
        	\item $(5^{-1} + 3^{-1})^{-1} = \bigg(\frac{1}{5} + \frac{1}{3}\bigg)^{-1} = \frac{1}{\frac{1}{5}+\frac{1}{3}} = \frac{1}{\frac{3}{15}+\frac{5}{15}} = \frac{1}{\frac{8}{15}} = \frac{15}{8}$ \\
        	\item $\bigg(\bigg(\frac{1}{5}\bigg)^{-1} + \bigg(\frac{1}{3}\bigg)^{-1}\bigg)^{-1} = \frac{1}{\frac{1}{\frac{1}{5}}+\frac{1}{\frac{1}{3}}} = \frac{1}{8}$ \\
        	\item $\big(5^{-\frac{1}{2}} + 3^{-\frac{1}{2}}\big)^{-2} = \frac{1}{(5^{{-\frac{1}{2}}}+3^{-\frac{1}{2}})^2} = \frac{1}{\bigg({\frac{1}{\sqrt{3}}} + {\frac{1}{\sqrt{5}}} \bigg)^2}$
        \end{enumerate}
    \end{enumerate}        
    \end{multicols}    

\end{document}