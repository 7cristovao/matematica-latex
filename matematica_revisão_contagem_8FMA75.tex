\documentclass[a4paper,14pt]{article}
\usepackage{float}
\usepackage{extsizes}
\usepackage{amsmath}
\usepackage{amssymb}
\everymath{\displaystyle}
\usepackage{geometry}
\usepackage{fancyhdr}
\usepackage{multicol}
\usepackage{graphicx}
\usepackage[brazil]{babel}
\usepackage[shortlabels]{enumitem}
\usepackage{cancel}
\usepackage{textcomp}
\columnsep=2cm
\hoffset=0cm
\textwidth=8cm
\setlength{\columnseprule}{.1pt}
\setlength{\columnsep}{2cm}
\renewcommand{\headrulewidth}{0pt}
\geometry{top=1in, bottom=1in, left=0.7in, right=0.5in}

\pagestyle{fancy}
\fancyhf{}
\fancyfoot[C]{\thepage}

\begin{document}
	
	\noindent\textbf{8FMA75 - Matemática} 
	
	\begin{center}Revisão: contagem (Versão estudante)
	\end{center}
	
	\noindent\textbf{Nome:} \underline{\hspace{10cm}}
	\noindent\textbf{Data:} \underline{\hspace{4cm}}
	
	%\section*{Questões de Matemática}
	
	
    \begin{multicols}{2}
		\begin{enumerate}
			\item Quantos múltiplos de 8 têm exatamente três algarismos? \\\\\\\\\\\\\\\\\\\\\\\\\\\\
			\item Se jogarmos dois dados, de quantas maneiras podemos obter soma dos pontos igual a 9? \\\\\\\\\\\\\\\\\\\\\\\\\\\\\\
			\item André tem uma $playlist$ com gêneros musicais variados: 17 são de rock, 12 são de jazz, 21 são de pop e 8 são trilhas sonoras. De quantas maneiras André pode escolher uma música de sua $playlist$? \\\\\\\\\\\\\\\\\\\\\\\\\\\\
			\item Sejam $x$ e $y$ dois números naturais. Quantos pares de números satisfazem a desigualdade $x + y < 10$? \\\\\\\\\\\\\\\\\\
			\item Eduardo e Fábio jogam o seguinte jogo: um deles lança uma moeda 7 vezes. Se algum momento ocorrem duas caras seguidas, Eduardo vence; se ocorrer uma coroa, depois cara, Fábio vence; caso contrário, ocorre empate.
			\begin{enumerate}[a)]
				\item De quantas formas pode se desenrolar o jogo? \\\\\\\\\\\\\\\\\\\\\\\\\\\\
				\item Em quantas delas Eduardo vence? \\\\\\\\\\\\\\\\\\\\\\\\\\\\
				\item Em quantas delas Fábio vence? \\\\\\\\\\\\\\\\\\\\\\\\\\\\
			\end{enumerate}
		    \item Ao cume de uma montanha conduzem 6 caminhos. De quantas maneiras um turista pode subir e depois descer a montanha:
		    \begin{enumerate}[a)]
		    	\item sem restrições? \\\\\\\\\\\\\\\\
		    	\item de forma que a descida seja por um caminho diferente da subida? \\\\\\\\\\\\\\
		    \end{enumerate}
	        \item Quantas vezes escrevemos o algarismo 4 ao escrevermos todos os números inteiros de 1 até 999? \\\\\\\\\\\\\\\\\\\\\\\\\\\\
	        \item Daniela, junto com sua mãe, foi a uma padaria. Sua mãe lhe disse que ela poderia escolher quantos brigadeiros e beijinhos quisesse, desde que não escolhesse mais que nove doces. Na padaria só havia oito brigadeiros e sete beijinhos. De quantas formas Daniela pôde escolher os doces? \\\\\\\\\\\\\\\\\\\\\\\\\\\\
	        \item Quantos números inteiros positivos menores que 1000 são pares, múltiplos de 3 e não são múltiplos de 5? \\\\\\\\\\\\\\\\\\\\\\\\\\\\
	        \item Saulo joga uma moeda sucessivamente. Se em algum momento ele obtiver duas caras consecutivas, vence. Se obtiver uma coroa seguida de uma cara, perde. Se não acontecer nenhuma dessas sequências em cinco lançamentos, Saulo vence. De quantas maneiras poderá se desenrolar o jogo? Em quantas delas Saulo vence? \\\\\\\\\\\\\\\\\\\\\\
	        \item As cidades $A$ e $B$ são ligadas por 4 rodovias e 3 estradas de terra. De quantas maneiras podemos ir de $A$ para $B$ e voltar de $B$ para $A$, considerando que:
	        \begin{enumerate}[a)]
	        	\item podemos utilizar qualquer caminho tanto na ida como na volta? \\\\\\\\\\\\\\\\\\\\\\\\\\\\
	        	\item devemos utilizar uma rodovia na ida e uma estrada de terra na volta? \\\\\\\\\\\\\\\\\\\\\\\\\\\\
	        	\item podemos utilizar qualquer caminho tanto na ida como na volta, mas não podemos utilizar a mesma rodovia ou estrada de terra da ida na volta? \\\\\\\\\\\\\\\\\\\\\\\\\\
	        \end{enumerate}
            \item \begin{enumerate}[a)]
            	\item Quantos são os anagramas da palavra
	            \begin{center}
	            	APARTAMENTO
	            \end{center}
	            que começam por APART? \\\\\\\\\\\\\\\\\\\\\\\\\\\\
	         	\item Agora é a sua vez! Apresente um enunciado como o do item anterior e depois resolva-o.
            \end{enumerate}
        \end{enumerate}
    \end{multicols}
\end{document}