\documentclass[a4paper,14pt]{article}

\usepackage{comment} % Para comentar várias linhas ao mesmo tempo

%matemática
\usepackage{amsmath}
\usepackage{amssymb}

%diagramação
\usepackage{extsizes}
\everymath{\displaystyle}
\usepackage{geometry}
\usepackage{fancyhdr}
\usepackage{multicol}
\usepackage{graphicx}
\usepackage[brazil]{babel}
\usepackage[shortlabels]{enumitem}
\usepackage{cancel}
\usepackage{textcomp}
\usepackage{tcolorbox}

%tabelas
\usepackage{array} % Para melhor formatação de tabelas
\usepackage{longtable}
\usepackage{booktabs}  % Para linhas horizontais mais bonitas
\usepackage{float}   % Para usar o modificador [H]
\usepackage{caption} % Para usar legendas em tabelas
\usepackage{wrapfig} % Para usar tabelas e figuras flutuantes
\usepackage{xcolor} % Para cores do fundo de tabelas
\usepackage{colortbl} % Para cores do fundo de tabelas

%tikzpicture
\begin{comment}
	\usepackage{tikz}
	\usepackage{scalerel}
	\usepackage{pict2e}
	\usepackage{tkz-euclide}
	\usetikzlibrary{calc}
	\usetikzlibrary{patterns,arrows.meta}
	\usetikzlibrary{shadows}
	\usetikzlibrary{external}
\end{comment}


%pgfplots
\usepackage{pgfplots}
\pgfplotsset{compat=newest}
\usepgfplotslibrary{statistics}
\usepgfplotslibrary{fillbetween}

%colours
\usepackage{xcolor}



\columnsep=2cm
\hoffset=0cm
\textwidth=8cm
\setlength{\columnseprule}{.1pt}
\setlength{\columnsep}{2cm}
\renewcommand{\headrulewidth}{0pt}
\geometry{top=1in, bottom=1in, left=0.7in, right=0.5in}

\pagestyle{fancy}
\fancyhf{}
\fancyfoot[C]{\thepage}

\begin{document}
	
	\noindent\textbf{6FMA85 - Matemática} 
	
	\begin{center}Mínimo múltiplo comum (Versão estudante)
	\end{center}
	
	\noindent\textbf{Nome:} \underline{\hspace{10cm}}
	\noindent\textbf{Data:} \underline{\hspace{4cm}}
	
	%\section*{Questões de Matemática}
	
	\begin{multicols}{2}
		\noindent Para somar frações com denominadores diferentes, é necessário calcular o mínimo múltiplo comum (mmc) de ambos. Encontramos o mínimo múltiplo comum utilizando um algoritmo abaixo exemplificado. \\
		Vamos determinar o mínimo múltiplo comum de 12 e 15. \\
			
		\begin{tabular}[H]{cc|c}
			12, & 15 & 2 \\
			~6, & 15 & 2 \\
			~3, & 15 & 3 \\
			~1, & ~5 & 5 \\
			~1, & ~1 & ~ \\
		\end{tabular}
		\\\\
		O número procurado é $2 \cdot 2 \cdot 3 \cdot 5 = 60$, ou seja, mmc (12, 15) = 60. \\
		\noindent\textsubscript{-----------------------------------------------------------------------}
		\begin{enumerate} 
			\item Determine o mmc (12, 32).  \\\\\\\\\\\\\\\\
			\item Transforme as frações $\frac{1}{12}$ e $\frac{1}{32}$ em outras equivalentes de mesmo denominador. \\\\\\\\\\
			\item Qual é o menor número natural não nulo que pode ser dividido exatamente pelos números 12, 15 e 16? \\\\\\\\\\\\\\\\\\\\\\\\\\\\
			\item Calcule o menor múltiplo comum dos números 12, 18 e 42. \\\\\\\\\\\\\\\\\\\\\\\\\\\\
			%43 a 46
			\item Calcule o mínimo múltiplo comum de:
			\begin{enumerate}[a)]
				\item 6 e 16. \\\\\\\\\\\\\\\\\\\\\\\\
				\item 14 e 18. \\\\\\\\\\\\\\\\\\\\\\\\
				\item 9, 15 e 24.  \\\\\\\\\\\\\\\\\\\\\\
				\item 12, 25 e 28. \\\\\\\\\\\\\\\\\\\\
			\end{enumerate}
			\item Considere a fração $\frac{a}{b}$, com $a, b \in \mathbb{Z}, b \neq 0$. Diga se as frações a seguir são ou não equivalentes a ela, justificando sua resposta.
			\begin{enumerate}[a)]
				\item $\frac{8a}{8b}$ \\\\\\\\\\\\\\
				\item $\frac{a + a + a}{b + b + b}$ \\\\\\\\\\\\\\
				\item $\frac{a - 2}{b - 2}$ \\\\\\\\\\\\
				\item $\frac{a \cdot a}{b \cdot b}$ \\\\\\\\\\\\\\\\\\
			\end{enumerate}
			\item O mínimo múltiplo comum de dois números, tais que um é o sucessor do outro, vale 462. Quais são esses números? \\\\\\\\\\\\\\\\\\\\\\\\\\\\\\\\\\\\\\\\\\\\\\\\\\\\
			\item É correto afirmar que o mmc $(a, b, c) = a \cdot b \cdot c$ para todos os valores naturais de $a, b$ e $c$? Justifique. \\\\\\\\\\\\\\\\\\
		\end{enumerate}
		$~$ \\ $~$ \\ $~$ \\ $~$ \\ $~$ \\ $~$ \\ $~$ \\ $~$ \\ $~$ \\ $~$ \\ $~$ \\ $~$ \\ $~$ \\ $~$ \\ $~$ \\ $~$ \\ $~$ \\ $~$ \\ $~$ \\ $~$ \\ $~$ \\ $~$ \\ $~$ \\ $~$ \\ $~$ \\ $~$ \\ $~$ \\ $~$
	\end{multicols}
\end{document}