\documentclass[a4paper,14pt]{article}
\usepackage{extsizes}
\usepackage{amsmath}
\usepackage{amssymb}
\everymath{\displaystyle}
\usepackage{geometry}
\usepackage{fancyhdr}
\usepackage{multicol}
\usepackage{graphicx}
\usepackage[brazil]{babel}
\usepackage[shortlabels]{enumitem}
\usepackage{cancel}
\columnsep=2cm
\hoffset=0cm
\textwidth=8cm
\setlength{\columnseprule}{.1pt}
\setlength{\columnsep}{2cm}
\renewcommand{\headrulewidth}{0pt}
\geometry{top=1in, bottom=1in, left=0.7in, right=0.5in}

\pagestyle{fancy}
\fancyhf{}
\fancyfoot[C]{\thepage}

\begin{document}
	
	\noindent\textbf{7FMA149~-~Matemática} 
	
	\begin{center}
		\textbf{Potências - Propriedades (I) (Versão professor)}
	\end{center}
	
	
	\noindent\textbf{Nome:} \underline{\hspace{10cm}}
    \noindent\textbf{Data:} \underline{\hspace{4cm}}
	
	%\section*{Questões de Matemática}
	
	\begin{multicols}{2}
		Para $a > 0$, valem as seguintes propriedades: \\
		P1. $a^m \cdot a^n = a^{m + n}$ (vale também para mais de dois fatores) \\
		P2. $\frac{a^m}{a^n} = a^{m - n}$ \\
		P3. $(a^m)^n = a^{m \cdot n}$ e $\sqrt[n]{\sqrt[m]{a}} = \sqrt[m \cdot n]{a}$ \\
		Se $a < 0$, verificar sinal primeiro.
	\begin{enumerate}
        \item Efetue os produtos a seguir, aplicando a propriedade P1 e escrevendo o resultado na forma de expoente racional:
        \begin{enumerate}[a)]
        	\item $2^\frac{1}{2} \cdot 2^\frac{1}{5}$ = $2^{\frac{1}{2} + \frac{1}{5}} = 2^{\frac{5}{10}+\frac{2}{10}} = 2^\frac{7}{10}$\\
        	\item $(-5)^\frac{1}{4} \cdot (-5)^\frac{1}{3}$ = $(-5)^{\frac{1}{4} + \frac{1}{3}} = (-5)^{\frac{3}{12} + \frac{4}{12}} = (-5)^\frac{7}{12}$ \\
        	\item $\pi^\frac{2}{5} \cdot \pi^\frac{1}{6}$ = $\pi^{\frac{2}{5} + \frac{1}{6}} = \pi^{\frac{12}{30} + \frac{5}{30}} = \pi^{\frac{17}{30}}$\\
        	\item $\bigg(\frac{1}{3}\bigg)^{\frac{7}{11}} \cdot \bigg(\frac{1}{3}\bigg)^{\frac{1}{11}} \cdot  \bigg(\frac{1}{3}\bigg)^{\frac{3}{11}}$ \\= $\bigg(\frac{1}{3}\bigg)^{\frac{7}{11} + \frac{1}{11} + \frac{3}{11}} = \bigg(\frac{1}{3}\bigg)^\frac{11}{11} \\= \bigg(\frac{1}{3}\bigg)^1 = \frac{1}{3}$\\
        \end{enumerate}
        \item Calcule, obtendo o resultado na forma de $\sqrt{~~}$ e simplificando o que for possível:
        \begin{enumerate}[a)]
        	\item $\sqrt[5]{3} \cdot \sqrt[7]{3}$ = $3^\frac{1}{5} \cdot 3^\frac{1}{7} = 3^{\frac{1}{5} + \frac{1}{7}} = 3^{\frac{7}{35} + \frac{5}{35}} = 3^\frac{12}{35} = \sqrt[35]{3^{12}}$\\
        	\item $\sqrt[3]{-7} \cdot \sqrt[11]{-7}$ = $(-7)^\frac{1}{3} \cdot (-7)^\frac{1}{11} \\= (-7)^{\frac{1}{3} + \frac{1}{11}} = (-7)^{\frac{11}{33} + \frac{3}{33}} \\= (-7)^\frac{14}{33} = \sqrt[33]{-7^{14}}$ \\
        	\item $\sqrt[9]{-8} \cdot \sqrt[7]{-8} \cdot \sqrt[5]{-8}$ \\= $(-8)^\frac{1}{9} \cdot (-8)^\frac{1}{7} \cdot (-8)^\frac{1}{5} \\= (-8)^{\frac{1}{9} + \frac{1}{7} + \frac{1}{5}} \\= (-8)^{\frac{35}{315} + \frac{45}{315} + \frac{63}{315}} \\= (-8)^\frac{143}{315} = \sqrt[315]{-8^{143}}$ \\
        	\item $\sqrt[4]{6^5} \cdot \sqrt[4]{6^2} = 6^\frac{5}{4} \cdot 6^\frac{2}{4} = 6^{\frac{5}{4} + \frac{2}{4}} \\= 6^\frac{7}{4} = 6^\frac{4}{4} \cdot 6^\frac{3}{4} = 6^1 \cdot 6^\frac{3}{4} = 6 \cdot \sqrt[4]{6^3}$
        \end{enumerate}
        \item Obtenha os quocientes na forma $a^\frac{m}{n}$.
        \begin{enumerate}[a)]
        	\item $\frac{5^\frac{5}{6}}{5^\frac{11}{12}} = 5^{\frac{5}{6} - \frac{11}{12}} = 5^\frac{10 - 11}{12} = 5^{-\frac{1}{12}}$\\
        	\item $\frac{(-3)^\frac{7}{5}}{(-3)^\frac{2}{3}}$ = $-3^{\frac{7}{5} - \frac{2}{3}} = -3^{\frac{21-10}{15}} \\= -3^\frac{11}{15}$\\
        	\item $\frac{\pi^{-\frac{1}{3}}}{\pi^{-\frac{5}{7}}}$ = $\pi^{-\frac{1}{3} - (-\frac{5}{7})} = \pi^{-\frac{1}{3} + \frac{5}{7}} \\= \pi^\frac{8}{21}$\\
        \end{enumerate}
        \item Calcule o valor da expressão a seguir, obtendo o resultado na forma de potência com expoente racional:
        \begin{enumerate}[a)]
        	\item $\frac{2^\frac{2}{3} \cdot 2^\frac{4}{5}}{2^\frac{7}{3} \cdot 2^{-\frac{3}{5}}} \cdot 2^\frac{8}{8}$ = $\frac{2^{\frac{2}{3} + \frac{4}{5} + 1}}{2^{\frac{7}{3} - \frac{3}{5}}} \\\\ = \frac{2^{\frac{10}{15} + \frac{12}{15} + \frac{15}{15}}}{2^{\frac{35}{15} - \frac{9}{15}}} = \frac{2^\frac{37}{15}}{2^\frac{26}{15}} = 2^{\frac{37}{15} - \frac{26}{15}} \\\\= 2^\frac{11}{15}$
        \end{enumerate}
        \item Calcule, obtendo o resultado na forma de raiz:
        \begin{enumerate}[a)]
        	\item $\frac{\sqrt[4]{7}}{\sqrt[5]{7}}$ = $\frac{7^\frac{1}{4}}{7^\frac{1}{5}} = 7^{\frac{1}{4} - \frac{1}{5}} = 7^\frac{5 - 4}{20} \\= 7^\frac{1}{20} = \sqrt[20]{7^1} = \sqrt[20]{7}$ \\
        	\item $\frac{\sqrt[8]{3}}{\sqrt[9]{3}}$ = $\frac{3^\frac{1}{8}}{3^\frac{1}{9}} = 3^\frac{9-8}{72} = 3^\frac{1}{72} \\= \sqrt[72]{3^1} = \sqrt[72]{3}$\\
        \end{enumerate}
        \item Calcule:
        \begin{enumerate}[a)]
            \item $\frac{\sqrt[3]{3} \cdot \sqrt[4]{3}}{\sqrt[5]{3^2} \cdot \sqrt[4]{3^3}}$ \\\\\\= $\frac{3^\frac{1}{3} \cdot 3^\frac{1}{4}}{3^\frac{2}{5} \cdot 3^\frac{3}{4}} = \frac{3^{\frac{1}{3} + \frac{1}{4}}}{3^{\frac{2}{5} + \frac{3}{4}}} = \frac{3^{\frac{4}{12} + \frac{3}{12}}}{3^{\frac{8}{20} + \frac{15}{20}}} \\\\\\= \frac{3^\frac{7}{12}}{3^\frac{23}{20}} = 3^{\frac{7}{12} - \frac{23}{20}} = 3^{\frac{35}{60} + \frac{(-69)}{60}} \\\\= 3^{-\frac{34}{60}} = 3^{-\frac{17}{30}}$ \\
            \item Agora é a sua vez. Apresente uma expressão como a do item anterior e depois calcule-a. \\\\\\\\\\
        \end{enumerate}
        \item Calcule:
        \begin{enumerate}[a)]
        	\item $\frac{\sqrt[5]{0^{11}} + \sqrt[9]{(-1)^4} + \sqrt{2^4}}{2 \cdot \sqrt[3]{27}- \sqrt[12]{1}}$ \\= $\frac{0 + 1 + \sqrt{16}}{2 \cdot \sqrt[3]{3^3} - 1} = \frac{0 + 1 + 4}{2 \cdot 3 - 1} \\= \frac{5}{6 - 1} = \frac{5}{5} = 1$\\
        	\item $\frac{1^\frac{11}{2} - 0^\frac{3}{4} + 16^\frac{1}{2} - 27^\frac{1}{3}}{8^\frac{1}{3} \bigg(125^\frac{1}{3} - 16^\frac{1}{4} \bigg)}$ 
        	\\= $\frac{1 - 0 + \sqrt[2]{16} - \sqrt[3]{27}}{\sqrt[3]{8} \cdot (\sqrt[3]{125} - \sqrt[4]{16})} \\\\\\= \frac{1+4-3}{2 \cdot (5-2)} = \frac{2}{6} = \frac{1}{3}$ \\
        	\item $27^{0,333...} - 4^{0,5}$ \\
        	= $27^\frac{1}{3} - 4^\frac{1}{2} \\\\
        	= \sqrt[3]{27} - \sqrt[2]{4} \\\\
        	= 3 - 2 = 1$ 
       \end{enumerate}
       \item Escreva o resultado na forma de potência de base real e expoente racional:
       \begin{enumerate}[a)]
           \item $8^\frac{1}{2} \cdot 8^\frac{4}{3}$ \\ 
           = $8^{\frac{1}{2} + \frac{4}{3}} = 8^{\frac{3}{6} + \frac{8}{6}} = 8^\frac{11}{6}$\\
           \item $9^\frac{5}{4} \cdot 3^\frac{7}{3} \cdot 27^\frac{1}{5}$ \\
           = $(3^2)^\frac{5}{4} \cdot 3^\frac{7}{3} \cdot (3^3)^\frac{1}{5} \\
           = 3^{\frac{10}{4} + \frac{7}{3} + \frac{3}{5}} \\
           = 3^{\frac{150 + 140 + 36}{60}} = 3^\frac{326}{60} = 3 ^\frac{163}{30}$\\
           \item $\sqrt{2} \cdot \sqrt[7]{2} \cdot \sqrt[3]{2}$ \\
           = $2^\frac{1}{2} \cdot 2^\frac{1}{7} \cdot 2^\frac{1}{3} = 2^{\frac{1}{2} + \frac{1}{7} + \frac{1}{3}} \\
           = 2^{\frac{21}{42} + \frac{6}{42} + \frac{14}{42}} = 2^\frac{41}{42}$ \\
           \item $\frac{7^\frac{2}{9}}{7^\frac{7}{9}} = 7^{\frac{2}{9} - \frac{7}{9}} = 7^{-\frac{5}{9}}$ \\
           \item $\frac{\sqrt{10}}{\sqrt[3]{100}} = \frac{10^\frac{1}{2}}{10^\frac{2}{3}} \\\\= 10^{\frac{1}{2} - \frac{2}{3}} = 10^\frac{3-4}{6} = 10^{-\frac{1}{6}}$ \\
           \item $\bigg((-1)^\frac{4}{11}\bigg)^\frac{22}{7}$ \\
           = $ (-1)^{\frac{4}{\cancel{11}} \cdot \frac{\cancel{22}}{7}} \\
           = (-1)^{\frac{4}{1} \cdot \frac{2}{7}} \\ 
           = (-1)^\frac{8}{7}$ \\\\
           \item $\sqrt[4]{\sqrt[~]{\sqrt[10]{\sqrt[3]{5}}}}$ \\
           = $ \sqrt[4 \cdot 2 \cdot 10 \cdot 3]{5} = \sqrt[240]{5} = 5^\frac{1}{240}$\\
       \end{enumerate}
       \item Calcule:
       \begin{enumerate}[a)]
           \item $\frac{4^\frac{5}{7} \cdot 4^\frac{3}{4} \cdot \bigg(4^\frac{1}{4} \bigg)^\frac{8}{7}}{\sqrt[4]{4} \cdot \sqrt[7]{4}}$ \\\\\\ 
           = $\frac{4^{\frac{5}{7} + \frac{3}{4}} \cdot 4^{\frac{1}{\cancel{4}} \cdot \frac{\cancel{8}}{7}}}{4^\frac{1}{4} \cdot 4^\frac{1}{7}} \\\\
           = \frac{4^{\frac{20}{28} + \frac{21}{28}} \cdot 4^{\frac{1}{1} \cdot \frac{2}{7}}}{4^{\frac{1}{4} + \frac{1}{7}}} \\\\
           = \frac{4^{\frac{41}{28} + \frac{8}{28}}}{4^\frac{11}{28}} \\\\
           = \frac{4^\frac{49}{28}}{4^\frac{11}{28}} = 4^{\frac{49}{28}-\frac{11}{28}} = 4^\frac{38}{28} \\\\
           = 4^\frac{28}{28} \cdot 4^\frac{10}{28} = 4^1 \cdot 4^\frac{10}{28} = 4 \cdot 4^\frac{10}{28} \\\\
           = 4 \cdot 4^\frac{5}{14} = 4 \cdot \sqrt[14]{4^5}$\\
       \end{enumerate}
       \item Calcule sem passar para expoente fracionário:
       \begin{enumerate}[a)]
       	   \item $\sqrt[3]{7} \cdot \sqrt[5]{7} = \sqrt[15]{7^8}$\\
       	   \item $\sqrt[3]{5^2} \cdot \sqrt[7]{5^6} = \sqrt[21]{5^{14 + 18}} = \sqrt[21]{5^{32}}$ \\
       	   \item $\frac{\sqrt[6]{3^{11}}}{\sqrt[4]{3^7}} = \sqrt[24]{3^{(44 - 42)}} = \sqrt[24]{3^2} \\\\
           = \sqrt[12]{3}$ \\
       	   \item $\sqrt[5]{6} \cdot \sqrt[3]{6} \cdot \sqrt{6}
       	   = \sqrt[30]{6^{(6 + 10 + 15)}} \\
       	   = \sqrt[30]{6^{31}}$
       \end{enumerate}
       \item A expressão $\left\{ \left[ y^\frac{7}{2} \cdot \bigg( \frac{\sqrt[7]{y^3}}{\sqrt[3]{y}} \bigg)^7 \right]^\frac{1}{2} \right\}^\frac{3}{5}$ \\ é igual a:
       \begin{enumerate}[a)]
       	\item $y^5$ \item $y^\frac{37}{10}$ \item $y^\frac{5}{4}$ \item $y^\frac{23}{20}$ \item $y^3$
       \end{enumerate}
       Resposta: \\
       $\left\{ \left[ y^\frac{7}{2} \cdot \bigg( \frac{\sqrt[7]{y^3}}{\sqrt[3]{y}} \bigg)^7 \right]^\frac{1}{2} \right\}^\frac{3}{5}$ \\
       = $\left\{ \left[ y^\frac{7}{2} \cdot \bigg( \frac{y^\frac{3}{7}}{y^\frac{1}{3}}\bigg)^7 \right]^\frac{1}{2} \right\}^\frac{3}{5} \\\\
       = \left\{ \left[ y^\frac{7}{2} \cdot \bigg( {y^{\frac{3}{7} - \frac{1}{3}}}\bigg)^7 \right]^\frac{1}{2} \right\}^\frac{3}{5} \\\\ 
       = \left\{ \left[ y^\frac{7}{2} \cdot \bigg( {y^{\frac{9}{21} - \frac{7}{21}}}\bigg)^7 \right]^\frac{1}{2} \right\}^\frac{3}{5} \\\\
       = \left\{ \left[ y^\frac{7}{2} \cdot \bigg( {y^\frac{2}{21}}\bigg)^7 \right]^\frac{1}{2} \right\}^\frac{3}{5} \\\\
       = \left\{ \left[ y^\frac{7}{2} \cdot {y^{7 \cdot \frac{2}{21}}} \right]^\frac{1}{2} \right\}^\frac{3}{5} \\\\
       = \left\{ \left[ y^\frac{7}{2} \cdot {y^\frac{\cancel{14}}{\cancel{21}}} \right]^\frac{1}{2} \right\}^\frac{3}{5} \\\\
       = \left\{ \left[ y^\frac{7}{2} \cdot {y^\frac{2}{3}} \right]^\frac{1}{2} \right\}^\frac{3}{5} \\\\
       = \left\{ \left[ y^{\frac{7}{2} + \frac{2}{3}} \right]^\frac{1}{2} \right\}^\frac{3}{5} \\\\
       = \left\{ \left[ y^\frac{21+4}{6} \right]^\frac{1}{2} \right\}^\frac{3}{5} \\\\
       = \left\{ \left[ y^\frac{25}{6} \right]^\frac{1}{2} \right\}^\frac{3}{5} \\\\
       = \left\{ y^{\frac{25}{6} \cdot \frac{1}{2}} \right\}^\frac{3}{5} \\\\
       = \left\{ y^\frac{25}{12} \right\}^\frac{3}{5} \\\\
       = y^{\frac{25}{12} \cdot \frac{3}{5}}  \\\\
       = y^\frac{75}{60}
       = y^\frac{25}{20} \\\\
       = y^\frac{5}{4} 
       $ \\\\
       Portanto alternativa c
    \end{enumerate}
    $~$ \\ $~$ \\ $~$ \\ $~$ \\ $~$ \\ $~$ \\ $~$ \\ $~$ \\ $~$ \\ $~$ \\ $~$ \\ $~$ \\ $~$ \\ $~$ \\ $~$ \\ $~$ \\ $~$ \\ $~$ \\ $~$ \\ $~$ \\ $~$ \\ $~$ \\ $~$ \\ $~$ \\ $~$ \\ $~$ \\ $~$ \\ $~$ \\ $~$ \\ $~$ \\ $~$ \\ $~$ \\ $~$ \\ $~$ \\ $~$ \\ $~$ \\ $~$ \\ $~$ \\ $~$ \\ $~$ \\ $~$ \\ $~$ \\ $~$ \\ $~$ \\ $~$ \\ $~$ \\ $~$ \\ $~$ \\ $~$ \\ $~$ \\ $~$ \\ $~$ \\ $~$      
    \end{multicols}
\end{document}