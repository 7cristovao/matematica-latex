\documentclass[a4paper,14pt]{article}
\usepackage{extsizes}
\usepackage{amsmath}
\usepackage{geometry}
\usepackage{fancyhdr}
\usepackage{multicol}
\usepackage{graphicx}
\usepackage[brazil]{babel}
\usepackage{enumitem}
\columnsep=2cm
\hoffset=0cm
\textwidth=8cm
\setlength{\columnseprule}{.1pt}
\setlength{\columnsep}{2cm}
\renewcommand{\headrulewidth}{0pt}
\geometry{top=1in, bottom=1in, left=1in, right=1in}

\pagestyle{fancy}
\fancyhf{}
\fancyfoot[C]{\thepage}

\begin{document}
	
	\noindent\textbf{EF09MA13-A, EF05MA12, EF05MA17, EF08MA23, EF09MA09, EF07MA26, EF09MA02}
	
	\begin{center}
		\textbf{Exercícios de Matemática - Caderno de Exercícios de Vestibulinho - Versão p/ estudante}
	\end{center}
	
	\bigskip
	
	\noindent\textbf{Nome:} \underline{\hspace{15cm}}
	
	\bigskip
	\section*{Questões de Matemática - SENAI 2023}
	
	\begin{enumerate}
		\item Em uma viagem entre as cidades de São Paulo e Araçatuba-SP, distantes 440 km uma da outra, Fabiana,
		partindo do km 0, percorreu na 1ª hora do trajeto, 20 km. Na 2ª hora ela estava no km 42,5, na 3ª hora,
		no km 65, e assim sucessivamente. 
		\newline
		Ao completar a 12ª hora do percurso, a que distância Fabiana estava de Araçatuba-SP?
		\begin{enumerate}
			\item 47,5 km.
			\item 172,5 km.
			\item 267,5 km.
			\item 372,5 km.
			\item 392,5 km.
		\end{enumerate}
		\vspace{1cm}
	    
	    \item Uma determinada receita de bolo indica que será necessário \(\frac{3}{4}\)
	    de uma xícara de leite. Sabendo que se
	    refere a uma xícara com capacidade total de 240 mL, quantos mililitros de leite serão necessários nessa
	    receita?
	    \begin{enumerate}
	    	\item 60.
	    	\item 80.
	    	\item 160.
	    	\item 180.
	    	\item 320.
	    \end{enumerate}
	    \vspace{4cm}
	    
	    \item Em um polígono convexo, quanto maior for o número de lados, maior será a soma dos ângulos internos.
	    \newline Veja alguns exemplos na tabela a seguir.
	    \newline
	    \newline
	    % Início da tabela
	    \begin{tabular}{|c|c|} % Define duas colunas centradas com linhas verticais
	    	\hline % Linha horizontal no topo da tabela
	    	Polígono convexo & Soma dos ângulos internos \\ % Conteúdo da primeira linha da tabela
	    	\hline % Linha horizontal
	    	Triângulo & 180°. \\ % Conteúdo da segunda linha da tabela
	    	\hline % Linha horizontal
	    	Quadrilátero & 360° \\ % Conteúdo da terceira linha da tabela
	    	\hline % Linha horizontal no final da tabela
	    	Pentágono & 540° \\ % Conteúdo da terceira linha da tabela
	    	\hline % Linha horizontal no final da tabela
	    \end{tabular}
        \newline
        \newline
        Considerando o padrão apresentado, qual o valor da soma dos ângulos internos em um decágono
        convexo?
        \begin{enumerate}
        	\item 1440°.
        	\item 1620°.
        	\item 1260°.
        	\item 1800°.
        	\item 1080°.
        \end{enumerate}
        \vspace{1cm}
        
        \item Joana tem um apartamento no 13º andar de um edifício de 15 andares. A distância entre o piso do andar,
        onde ela tem o apartamento, e o piso térreo é 39,0 m. Seu marido Augusto, cuja altura é de 1,8 m, está
        parado do lado desse edifício e projeta uma sombra de 30,0 cm.
        Nesse mesmo instante, a sombra projetada pelo edifício onde se encontra o apartamento de Joana é igual a 
        \begin{enumerate}
        	\item 6,5m.
        	\item 6,6m.
        	\item 7,0m.
        	\item 7,5m.
        	\item 8,0m.
        \end{enumerate}
        \vspace{4cm}
        
        \item Na tabela a seguir, estão registradas as velocidades (em km/h) de todos os carros que passaram por um
        radar, ao longo de 1 hora.
        
        % Início da tabela
        \begin{tabular}{|c|c|c|c|c|} % Define cinco colunas centradas com linhas verticais
        	\hline % Linha horizontal no topo da tabela
        	32 & 34 & 48 & 52 & 50 \\ % Conteúdo da primeira linha da tabela
        	\hline % Linha horizontal
        	42 & 54 & 56 & 49 & 25 \\ % Conteúdo da segunda linha da tabela
        	\hline % Linha horizontal
        	35 & 45 & 46 & 74 & 68 \\ % Conteúdo da terceira linha da tabela
        	\hline % Linha horizontal no final da tabela
        	37 & 50 & 48 & 40 & 41 \\ % Conteúdo da terceira linha da tabela
        	\hline % Linha horizontal no final da tabela
        \end{tabular}
        \newline
        \newline
        Se a velocidade máxima permitida era de 50 km/h, sem tolerância, qual foi a frequência relativa (em %)
        de carros multados nesse período?
        \newline
        \begin{enumerate}
        	\item 5\%.
        	\item 10\%.
        	\item 25\%.
        	\item 30\%.
        	\item 35\%.
        \end{enumerate}
        \vspace{4cm}
        
        \item O gerente de uma loja de cosméticos verificou que sobraram 5 produtos distintos em seu estoque e
        decidiu montar kits com 3 desses produtos, para sortear entre os clientes.
        Nessas condições, quantos kits diferentes podem ser montados?
        \begin{enumerate}
        	\item 6.
        	\item 10.
        	\item 20.
        	\item 30.
        	\item 60.
        \end{enumerate}
        \vspace{4cm}
        
        \item Rubens está comprando uma bicicleta por R\$ 314,40 de entrada e mais 4 parcelas que se encontram
        em progressão geométrica. As duas primeiras parcelas são de R\$ 300,00 e R\$ 240,00, respectivamente.
        Quando Rubens terminar de pagar todas as parcelas, ele terá pago pela bicicleta o valor total de
        \begin{enumerate}
        	\item R\$ ~~854,40.
        	\item R\$ ~~885,60.
        	\item R\$ 1.154,40.
        	\item R\$ 1.200,00.
        	\item R\$ 1.529,40.
        \end{enumerate}
        \vspace{0cm}
        
        \item Uma lombo-faixa foi construída com uma rampa com 15° de ambos os lados. A base do triângulo
        retângulo das rampas mede 1,94 m de cada lado e a superfície plana mede 3,0 m, conforme mostra a
        figura. As medidas não estão na proporção correta.
        \begin{figure}[h] % Ambiente figure para a imagem
        	\centering
        	\includegraphics[width=1\textwidth]{senai01.png} % Substitua "nome_do_arquivo" pelo nome do arquivo da sua imagem
        \end{figure}
        \newline
        Dados: sen 15° = 0,25;
        cos 15° = 0,97;
        tg 15° = 0,27.
        \newline
        \newline
        Depois de atravessar completamente a rampa, o carro terá se deslocado
        \begin{enumerate}
        	\item 7 m.
        	\item 18,52 m.
        	\item 17,36 m.
        	\item 6,88 m.
        	\item 4,94 m.
        \end{enumerate}
    	\vspace{2cm}
    	
    	\item Um arquiteto projetou um galpão. A figura fora de escala a seguir, representa um dos cortes desse
    	projeto, que mostra detalhes, principalmente, do telhado. Para informações técnicas de instalação do
    	telhado é necessário determinar as medidas dos segmentos BC e CD. 
    	\begin{figure}[h] % Ambiente figure para a imagem
    		\centering
    		\includegraphics[width=0.5\textwidth]{senai02.png} % Substitua "nome_do_arquivo" pelo nome do arquivo da sua imagem
    	\end{figure}
    	\newline
    	Dados: sen 30º = 0,50; sen 60º = 0,86.
    	\newline
    	\newline
    	Sabendo que AC= 4 metros, qual é a medida de (BC + CD), em metros?
    	\begin{enumerate}
    		\item 3,35 m.
    		\item 4,65 m.
    		\item 5,44 m.
    		\item 8,00 m.
    		\item 12,65 m.
    	\end{enumerate}
    	\vspace{0cm}
    	
    	\item Um investidor está acompanhando a variação do valor de compra de uma ação na bolsa de valores. Ele
    	registrou dois valores: o primeiro, às 00h00, e o segundo, às 08h00, e fez o esboço de um gráfico, que
    	revelou uma tendência linear de queda, conforme mostrado na figura fora de escala a seguir.
    	\newline
   		\begin{figure}[h] % Ambiente figure para a imagem
    		\centering
    		\includegraphics[width=1\textwidth]{senai03.png} % Substitua "nome_do_arquivo" pelo nome do arquivo da sua imagem
    	\end{figure}
    	\newline
    	\newline
    	\newline
    	\newline
    	\newline
    	\newline
    	Supondo que essa tendência se mantenha, qual será o valor de compra dessa ação às 10h00?
    	\begin{enumerate}
    		\item R\$ ~2,50.
    		\item R\$ ~8,00.
    		\item R\$ ~9,50.
    		\item R\$ ~9,60.
    		\item R\$ 10,00.
    	\end{enumerate}
    	\vspace{0cm}
    	
    	\item A figura a seguir representa a visão lateral de uma casa na qual será construída uma rampa reta, $\overline{AC}$
    	especialmente para a acessibilidade de um dos moradores que é cadeirante. A distância entre A e B é
    	de 6 m. Entre os pontos B e C, a distância equivale a 10 m e o ângulo ABC é de 120°.
    	
    	\begin{figure}[h] % Ambiente figure para a imagem
    		\centering
    		\includegraphics[width=1\textwidth]{senai04.png} % Substitua "nome_do_arquivo" pelo nome do arquivo da sua imagem
    	\end{figure}
    	Dadas essas informações, qual deve ser o comprimento da rampa, em metros?
    	\newline
    	\newline
    	\newline
	   	\begin{enumerate}
    		\item 4.
    		\item 14.
    		\item 16.
    		\item 76.
    		\item 196.
    	\end{enumerate}
        
        \item Considere que o lucro de uma empresa é modelado pela função
        L(p) = $-10p^2 + 70p - 60$
        onde p é o número de unidades vendidas e L é o lucro obtido (em milhares de reais).
        Sendo assim, qual é o lucro máximo que essa empresa poderá obter?
        \begin{enumerate}
        	\item 1.
        	\item 6.
        	\item 3,5.
        	\item 62,5.
        	\item 625.
        \end{enumerate}
    
        \item Um sistema de controle de temperatura foi instalado em determinado ambiente de estudo científico. Esse
        aparelho liga automaticamente quando a temperatura está a zero graus Celsius, ativando a variação de
        temperatura, segundo a função $y = 0,2x^2
        - 2,4x$, em que y é a temperatura do ambiente, em graus Celsius,
        e x é o tempo decorrido, em horas, após o início do ciclo. Sabe-se, ainda, que esse aparelho é desligado
        automaticamente, quando a temperatura retorna a zero graus Celsius e começa um novo ciclo.
        Segundo essas informações, qual será, respectivamente, a temperatura mínima, em graus Celsius,
        atingida nesse ambiente e o tempo, em horas, decorrido em cada ciclo?
        \begin{enumerate}
        	\item -7,2 e 12.
        	\item -7,2 e 6.
        	\item 5,7 e 12.
        	\item -4,8 e 6.
        	\item 4,8 e 24.
        \end{enumerate}
        
        \item Em um salão de festas serão colocadas mesas e cadeiras para um evento beneficente. O espaço,
        ocupado por uma mesa e quatro cadeiras, considerando quatro pessoas sentadas confortavelmente,
        é de $4m^2$
        .
        Tendo em vista que o salão é retangular e de medidas 20 m de comprimento por 18 m de largura, quantos
        conjuntos de uma mesa e quatro cadeiras são possíveis formar em todo o salão?
        \begin{enumerate}
			\item 23.
			\item 38.
			\item 72.
			\item 76.
			\item 90.
		\end{enumerate}
	
	    \item Os triângulos semelhantes ABC e DEF representam prateleiras fabricadas em uma marcenaria. Um
	    cliente encomendou uma prateleira com as medidas do triângulo DEF, conforme figura fora de escala a
	    seguir, pedindo, entretanto, para que fosse colocada uma fita especial preenchendo totalmente os lados
	    DE e DF que vão encostar na parede.
	    \newline
	    \newline
	    \newline
	    \newline
	    \newline
	    \newline
	    \newline
	    \newline
	    \newline
	    \newline
	    \newline
	    \newline
	    \newline
        \begin{figure}[h] % Ambiente figure para a imagem
        	\centering
        	\includegraphics[width=1\textwidth]{senai05.png} % Substitua "nome_do_arquivo" pelo nome do arquivo da sua imagem
        \end{figure}
        Segundo essas informações, para produzir a prateleira, qual será o comprimento mínimo necessário,
        em cm, dessa fita especial?
        \begin{enumerate}
        	\item 15.
        	\item 25.
        	\item 35.
        	\item 40.
        	\item 60.
        \end{enumerate}
    
        \item Em um edifício de 2 andares, com 2 apartamentos por andar, as vagas de garagem 01 e 02 são presas
        como mostra a figura, sendo necessário realizar um rodízio semestral com os moradores.
        
        \begin{figure}[h] % Ambiente figure para a imagem
        	\centering
        	\includegraphics[width=1\textwidth]{senai06.png} % Substitua "nome_do_arquivo" pelo nome do arquivo da sua imagem
        \end{figure}
    
	    Considerando as informações da página anterior, qual o número máximo de combinações diferentes que podem ser feitas para que todos passem por todas as vagas?
	    \begin{enumerate}
        	\item 256.
        	\item 24.
        	\item 16.
        	\item 8.
        	\item 4.
        \end{enumerate}
        
        \item Durante um jogo oficial de vôlei, um dos jogadores dá uma “cortada” na bola, no exato momento em que
        ela se encontra a uma altura de 3,1 m do solo e perfeitamente alinhada à rede, conforme ilustrado na
        figura fora de escala a seguir.
        \newline
        \newline 
        \begin{figure}[h] % Ambiente figure para a imagem
        	\centering
        	\includegraphics[width=1\textwidth]{senai07.png} % Substitua "nome_do_arquivo" pelo nome do arquivo da sua imagem
        \end{figure}
        Dados: sen 37º = 0,60; cos 37º = 0,80; tg 37º = 0,75.
        \newline
        \newline
        Se a bola realiza uma trajetória retilínea que forma um ângulo de 37° com o solo, a que distância
        (aproximada, em metros) do pé do poste que sustenta a rede, a bola toca o solo?
        \begin{enumerate}
        	\item 2,21.
        	\item 2,32.
        	\item 3,88.
        	\item 4,13.
        	\item 5,17.
        \end{enumerate}
        \item O preço de uma viagem de táxi é definido por um valor fixo conhecido como bandeirada e um valor
        variável que depende do número de quilômetros rodados. Um taxista realiza viagens cobrando
        R\$ 10,00 a bandeirada e R\$ 4,00 por quilômetro rodado.
        Sabendo que “y” é o valor, em reais, pago pela viagem e “x” é o número de quilômetros rodados pelo
        taxista durante o trajeto, afirma-se que esta função é corretamente representada por
        
        \begin{enumerate}[label=\Roman*.]
        	\item y = 4x +10..
        	\item y = 10x + 4.
        	\item x = 4y +10.
        	\item f(x) = 4x + 10.
        \end{enumerate}
        
        Dessas informações, conclui-se que está correto o modo de representação indicado
        
        \begin{enumerate}
        	\item somente em I.
        	\item somente em II. 
        	\item somente em III.
        	\item somente em IV.
        \end{enumerate}
    
        \item As fórmulas matemáticas a seguir, referem-se a duas importantes leis trigonométricas.
        \newline
        \newline
        Lei dos senos: $\frac{a}{sen~A} = \frac{b}{sen~B} = \frac{c}{sen~C}$
        \newline
        \newline
        Lei dos cossenos: $a^2 = b^2 + c^2~–~2 \cdot b \cdot c \cdot cos~A$
        \newline
        Considere as seguintes afirmações sobre essas duas leis. 
        
        \begin{enumerate}[label=\Roman*.]
        	\item A lei dos senos permite relacionar lados e ângulos apenas em triângulos retângulos.
        	\item É possível aplicar a lei dos cossenos para relacionar lados e ângulos de triângulos que não possuem
        	um ângulo reto.
        	\item É possível aplicar a lei dos senos a um triângulo retângulo, para se obter o seno de um de seus
        	ângulos agudos.
        	\item Num triângulo de medidas 5 cm, 7 cm e 8 cm, o valor do cosseno do menor ângulo interno é  $\frac{\sqrt{3}}{2}$.
        \end{enumerate}
    	São verdadeiras as afirmações feitas apenas em
    
        \begin{enumerate}
        	\item II e III.
        	\item I e II.
        	\item II e IV.
        	\item III e IV.
        	\item I e IV.
        \end{enumerate}
        
        \item A respeito dos conjuntos numéricos, considere as seguintes afirmativas.
        \begin{enumerate}[label=\Roman*.]
        	\item Todo número Racional também é Inteiro.
        	\item Se um número é Natural, então ele também é Inteiro e Racional.
        	\item Todo número que é Irracional, também é Real.
        	\item Se um número é Inteiro, então ele também é Racional e Real.
        \end{enumerate}
        
        Está correto o que se afirmar em
        
        \begin{enumerate}
        	\item I, II, III e IV.
        	\item III e IV, apenas. 
        	\item II, III e IV, apenas.
        	\item II e III, apenas.
        	\item I e II, apenas.
        \end{enumerate}
        
	\end{enumerate}
    

	
\end{document}