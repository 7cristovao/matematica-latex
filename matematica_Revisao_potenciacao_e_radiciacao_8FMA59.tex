\documentclass[a4paper,14pt]{article}
\usepackage{float}
\usepackage{extsizes}
\usepackage{amsmath}
\usepackage{amssymb}
\everymath{\displaystyle}
\usepackage{geometry}
\usepackage{fancyhdr}
\usepackage{multicol}
\usepackage{graphicx}
\usepackage[brazil]{babel}
\usepackage[shortlabels]{enumitem}
\usepackage{cancel}
\columnsep=2cm
\hoffset=0cm
\textwidth=8cm
\setlength{\columnseprule}{.1pt}
\setlength{\columnsep}{2cm}
\renewcommand{\headrulewidth}{0pt}
\geometry{top=1in, bottom=1in, left=0.7in, right=0.5in}

\pagestyle{fancy}
\fancyhf{}
\fancyfoot[C]{\thepage}

\begin{document}
	
	\noindent\textbf{8FMA59~Matemática} 
	
	\begin{center}Revisão: potenciação e radiciação (Versão estudante)
	\end{center}
	
	\noindent\textbf{Nome:} \underline{\hspace{10cm}}
	\noindent\textbf{Data:} \underline{\hspace{4cm}}
	
	%\section*{Questões de Matemática}
	
	
    \begin{multicols}{2}
    	Se $a^\frac{m}{n} \in \mathbb{R}$ e $\sqrt[n]{a^m} \in \mathbb{R}$, então $a^\frac{m}{n} = \sqrt[n]{a^m}$, para $a \geq 0$. Para $a < 0$, você deve verificar o sinal de $a^\frac{m}{n}$ e $\sqrt[n]{a^m}$ \\
    	\textsubscript{---------------------------------------------------------------------}
    	\begin{enumerate}
    		\item Escrever, quando possível, na forma $\sqrt[n]{a}$, com $a \in \mathbb{R}$ e $n \in \mathbb{N^*}$:
    		\begin{enumerate}[a)]
    			\item $5^\frac{1}{2}$ \\\\\\\\
    			\item $12^\frac{1}{5}$ \\\\\\\\
    			\item $(-3)^\frac{7}{4}$ \\\\\\\\
    			\item $-17^\frac{1}{9}$ \\\\\\\\
    			\item $7^\frac{3}{4}$ \\\\\\\\
    			\item $3^\frac{3}{2}$ \\\\\\\\
    	    \end{enumerate}
            \item Escrever, quando possível, na forma $a^\frac{m}{n}$, com $a \in \mathbb{R}$ e $m$, $n \in \mathbb{N^*}$:
            \begin{enumerate}[a)]
            	\item $\sqrt[4]{9}$ \\\\\\\\
            	\item $\sqrt[3]{2^4}$ \\\\\\\\
            	\item $\sqrt[12]{9^4}$ \\\\\\\\
            	\item $\sqrt[5]{(-7)^3}$ \\\\\\\\
            	\item $\sqrt[9]{(-3)^3}$ \\\\\\\\
            	\item $\sqrt[7]{(-6)^4}$ \\\\\\\\
            \end{enumerate}
            \item Complete os itens abaixo, utilizando a forma $\sqrt[n]{a}$ (com $a \in \mathbb{R}$ e $n \in \mathbb{N^*}$) ou a forma $a^\frac{m}{n}$ (com $a \in \mathbb{R}$ e $m, n \in \mathbb{N^*}$), quando possível:
            \begin{enumerate}[a)]
            	\item $15^\frac{2}{7}$ \\\\\\\\
            	\item $\sqrt[4]{5^3}$ \\\\\\\\
            	\item $\sqrt[5]{(-13)^2}$ \\\\\\\\
            	\item $-21^\frac{1}{3}$ \\\\\\\\
            \end{enumerate}
            \item Assinale $\textbf{V}$ ou $\textbf{F}$:
            \begin{enumerate}[a)]
            	\item (~~) $\sqrt[5]{3^2} = 3^\frac{2}{5}$
            	\item (~~) $\sqrt[4]{16} = 4^2$
            	\item (~~) $\sqrt[9]{7^3} = 7^\frac{9}{3}$
            	\item (~~) $3^\frac{1}{7} = \sqrt{3^7}$
            	\item (~~) $(-2)^\frac{12}{15} = \sqrt[15]{(-2)^{12}}$
            	\item (~~) $(-11)^\frac{7}{5} = \sqrt[5]{(-11)^7}$ 
            \end{enumerate}
        \end{enumerate}
    $~$ \\ $~$ \\ $~$ \\ $~$ \\ $~$ \\ $~$ \\ $~$ \\ $~$ \\ $~$ \\ $~$ \\ $~$ \\ $~$ \\ $~$ \\ $~$ \\ $~$ \\ $~$ \\ $~$ \\ $~$ \\ $~$ \\ $~$ \\ $~$ \\ $~$ \\ $~$ \\ $~$ \\ $~$ \\ $~$ \\ $~$ \\ $~$ \\ $~$ \\ $~$ \\ $~$ \\ $~$ \\ $~$ \\ $~$ \\ $~$ \\ $~$ \\ $~$ \\   
    \end{multicols}
\end{document}