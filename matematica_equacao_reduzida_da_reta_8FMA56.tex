\documentclass[a4paper,14pt]{article}
\usepackage{float}
\usepackage{extsizes}
\usepackage{amsmath}
\usepackage{amssymb}
\everymath{\displaystyle}
\usepackage{geometry}
\usepackage{fancyhdr}
\usepackage{multicol}
\usepackage{graphicx}
\usepackage[brazil]{babel}
\usepackage[shortlabels]{enumitem}
\usepackage{cancel}
\usepackage{textcomp}
\columnsep=2cm
\hoffset=0cm
\textwidth=8cm
\setlength{\columnseprule}{.1pt}
\setlength{\columnsep}{2cm}
\renewcommand{\headrulewidth}{0pt}
\geometry{top=1in, bottom=1in, left=0.7in, right=0.5in}

\pagestyle{fancy}
\fancyhf{}
\fancyfoot[C]{\thepage}

\begin{document}
	
	\noindent\textbf{8FMA56~Matemática} 
	
	\begin{center}Equação reduzida da reta (Versão estudante)
	\end{center}
	
	\noindent\textbf{Nome:} \underline{\hspace{10cm}}
	\noindent\textbf{Data:} \underline{\hspace{4cm}}
	
	%\section*{Questões de Matemática}
	
	
    \begin{multicols}{2}
    	\noindent Toda reta não perpendicular ao eixo $x$ tem uma equação reduzida $y = ax +b, a, b \in \mathbb{R}$
    	
    	\noindent\textsubscript{~---------------------------------------------------------------------------}
    	\begin{enumerate}
    		\item Encontre a equação reduzida, quando existir, de cada uma das retas cujas equações gerais são:
    		\begin{enumerate}[a)]
    		    \item $3x - y + 4 = 0$ \\\\\\\\\\\\\\\\\\
    		    \item $-5x + 3y + 8 = 0$ \\\\\\\\\\\\\\\\\\
    		    \item $3y - 9 = 0$ \\\\\\\\\\\\
    		    \item $2x - 18 = 0$ \\\\\\\\\\\\\\\\\\
    		\end{enumerate}
    	    \item Determine, caso existam, os coeficientes angular e linear de cada uma das retas do item anterior.
    	    \begin{enumerate}[a)]
    	        \item ~\\\\\\\\\\\\\\\\\\\\\\\\\\\\\\\\\\\\\\\\
    	        \item ~\\\\\\\\\\\\\\\\\\\\\\\\\\\\\\\\\\\\\\
    	        \item ~\\\\\\\\\\\\\\\\\\\\\\\\\\\\\\\\\\\\\\\\
    	        \item ~\\\\\\\\\\\\\\\\\\
    	    \end{enumerate}
            \item Apresente uma equação para a reta $\stackrel{\leftarrow\rightarrow}{AB}$ nos seguintes casos:
        	\begin{enumerate}[a)]
        		\item $A = (2; 8)$ e $B = (2; -4)$ \\\\\\\\\\\\\\\\\\\\
        		\item $A = (0; \sqrt{3})$ e $B = (0; -7)$ \\\\\\\\\\\\\\\\\\\\\\\\\\\\
        		\item $A = (-5; 2)$ e $B = \bigg(-5; \frac{3}{2}\bigg)$.\\\\\\\\\\\\\\\\\\\\\\\\
        	\end{enumerate}
            \item Faça o gráfico da reta $r$, de equação $y = 5$. Qual a posição de $r$ em relação à reta $s$ de equação $x = 5$? O que podemos dizer sobre retas cujas equações são $y = m$ e $x = n, m. n \in \mathbb{R}$? \\\\\\\\\\\\\\\\\\\\
            \item Em cada item, dada a equação, desenhar o gráfico da relação em $\mathbb{R^2}$. Verifique seu esboço com o professor utilizando o GeoGebra.
            \begin{enumerate}[a)]
            	\item $x - 5y = 0$
            	\item $3x - 4y - 7 = 0$
            	\item $x - 6 = 0$
            	\item $x + y = 4$
            	\item $x - y = 4$
            	\item $y = 3$
            	\item $x = 3$
            	\item $y = -5$
            	\item $4x - y = 4$
            	\item $-x + 2y = 3$
            	\item $x + 6y = 9$
            	\item $x - 2y = 0$
            	\item $x = 3y$
            	\item $x = -2y$
            	\item $x - y = 0$
            \end{enumerate}
            \item O lugar geométrico dos pontos ($x$; $y$), que equidistam os pontos $A = (-3; 2)$ e $B = (5; 0)$, é:
            \begin{enumerate}[a)]
            	\item $4x - y - 3 = 0$
            	\item $x + 4y - 5 = 0$
            	\item $x - y + 3 = 0$
            	\item $-x + 3y + 1 = 0$
            	\item $x - y + 5 = 0$
            \end{enumerate}
            \item Se o ponto $A = (x; y)$ é tal que $y - 3x = -2$, em que quadrantes $A$ pode estar situado?
            \begin{enumerate}[a)]
            	\item 1\textordmasculine{~}e 3\textordmasculine{~}
            	\item 2\textordmasculine{~}e 4\textordmasculine{~}
            	\item 1\textordmasculine{}, 3\textordmasculine{~}e 4\textordmasculine{~}
            	\item 2\textordmasculine{}, 3\textordmasculine{~}e 4\textordmasculine{~}
            	\item 1\textordmasculine{}, 2\textordmasculine{~}e 3\textordmasculine{~}
            \end{enumerate}
            	
        \end{enumerate}
    $~$ \\ $~$ \\ $~$ \\ $~$ \\ $~$ \\
    \end{multicols}
\end{document}