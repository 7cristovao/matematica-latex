\documentclass[a4paper,14pt]{article}

\usepackage{comment} % Para comentar várias linhas ao mesmo tempo

%matemática
\usepackage{amsmath}
\usepackage{amssymb}

%diagramação
\usepackage{extsizes}
\everymath{\displaystyle}
\usepackage{geometry}
\usepackage{fancyhdr}
\usepackage{multicol}
\usepackage{graphicx}
\usepackage[brazil]{babel}
\usepackage[shortlabels]{enumitem}
\usepackage{cancel}
\usepackage{textcomp}
\usepackage{tcolorbox}

%tabelas
\usepackage{array} % Para melhor formatação de tabelas
\usepackage{longtable}
\usepackage{booktabs}  % Para linhas horizontais mais bonitas
\usepackage{float}   % Para usar o modificador [H]
\usepackage{caption} % Para usar legendas em tabelas
\usepackage{wrapfig} % Para usar tabelas e figuras flutuantes
\usepackage{xcolor} % Para cores do fundo de tabelas
\usepackage{colortbl} % Para cores do fundo de tabelas

%tikzpicture
\begin{comment}
	\usepackage{tikz}
	\usepackage{scalerel}
	\usepackage{pict2e}
	\usepackage{tkz-euclide}
	\usetikzlibrary{calc}
	\usetikzlibrary{patterns,arrows.meta}
	\usetikzlibrary{shadows}
	\usetikzlibrary{external}
\end{comment}


%pgfplots
\usepackage{pgfplots}
\pgfplotsset{compat=newest}
\usepgfplotslibrary{statistics}
\usepgfplotslibrary{fillbetween}

%colours
\usepackage{xcolor}



\columnsep=2cm
\hoffset=0cm
\textwidth=8cm
\setlength{\columnseprule}{.1pt}
\setlength{\columnsep}{2cm}
\renewcommand{\headrulewidth}{0pt}
\geometry{top=1in, bottom=1in, left=0.7in, right=0.5in}

\pagestyle{fancy}
\fancyhf{}
\fancyfoot[C]{\thepage}

\begin{document}
	
	\noindent\textbf{6FMA111 - Matemática} 
	
	\begin{center}Introdução à contagem: princípio aditivo (Versão estudante)
	\end{center}
	
	\noindent\textbf{Nome:} \underline{\hspace{10cm}}
	\noindent\textbf{Data:} \underline{\hspace{4cm}}
	
	%\section*{Questões de Matemática}
	
	\begin{multicols}{2}
		\noindent \begin{itemize} 
		\item Quando espalhamos objetos, complicamos sua contagem. Porém, quando guardamos esses objetos, organizamos sua contagem.
		\item \textbf{Princípio aditivo:} sempre que tivermos possibilidades de escolhas independentes, ou seja, sempre que uma possibilidade automaticamente exclua a outra, iremos \textbf{somar} a quantidade total de possibilidades. Esse processo é caracterizado pela conjunção \textbf{ou}.
		\end{itemize}
		\noindent\textsubscript{-----------------------------------------------------------------------}
		\begin{enumerate} 
			\item Paula vai colorir as bandeiras da festa junina de seu colégio. Ela dispõe de somente 5 cores de tinta: azul, vermelho, amarelo, rosa e verde. Sabendo que cada bandeira será de uma só cor, de quantas maneiras ela pode pintar uma bandeira? \\\\\\\\\\\\\\\\\\\\\\\\
			\item Para ir trabalhar, Bruno pode ir à pé, pode ir de bicicleta, pode ir de patinete ou pode ir de ônibus. De quantas maneiras diferentes ele pode ir trabalhar? \\\\\\\\\\\\\\\\\\\\\\\\
			\item Júlia é muito popular em sua cidade. Certo sábado à tarde, ela foi convidada para ir ao cinema com Marcela, para ir à festa de aniversário de Fábio, para ir ao $shopping$ com Natália e para ir ao parque com Roberto. Se Júlia só puder aceitar um dos convites, de quantas maneiras diferentes ela poderia escolher seu programa para sábado? \\\\\\\\\\\\
			\item Durante uma aula de Educação Física, Flávia poderia escolher entre jogar vôlei, basquete, handebol ou futebol. De quantas maneiras diferentes ela pode escolher o que praticar nessa aula? \\\\\\\\\\\\\\\\\\\\\\\\\\\\
			% 1 a 4
			\item Ligando as cidades $A$ e $B$, existem seis linhas de ônibus e cinco de trem. De quantos modos diferentes é possível ir da cidade $A$ até a cidade $B$ se a escolha pode ser de ônibus ou de trem? \\\\\\\\\\\\\\\\\\\\\\\\\\\\
			\item Amanda entrou em uma loja e decidiu comprar uma blusa. A vendedora mostrou as diferentes cores desse modelo de blusa: azul, branca, preta, vermelha, amarela e rosa. \\
			Sabendo que ela vai escolher apenas uma, quantas opções ela tem? \\\\\\\\\\\\\\\\\\\\\\\\
			\item De quantas maneiras podemos colocar uma peça em um tabuleiro de xadrez?
		\end{enumerate}
		$~$ \\ 	$~$ \\ 	$~$ \\ 	$~$ \\ 	$~$ \\ 	$~$ \\ 	$~$ \\ 	$~$ \\ 	$~$ \\ 	$~$ \\ 	$~$ \\ 	$~$ \\ 	$~$ \\ 	$~$ \\ 	$~$ \\ 
	\end{multicols}
\end{document}