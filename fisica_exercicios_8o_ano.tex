\documentclass[a4paper,14pt]{article}
\usepackage{float}
\usepackage{extsizes}
\usepackage{amsmath}
\usepackage{amssymb}
\everymath{\displaystyle}
\usepackage{geometry}
\usepackage{fancyhdr}
\usepackage{multicol}
\usepackage{graphicx}
\usepackage[brazil]{babel}
\usepackage[shortlabels]{enumitem}
\usepackage{cancel}
\usepackage{textcomp}
\usepackage{array} % Para melhor formatação de tabelas
\usepackage{longtable}
\usepackage{booktabs}  % Para linhas horizontais mais bonitas
\usepackage{float}   % Para usar o modificador [H]
\usepackage{caption} % Para usar legendas em tabelas
\usepackage{tcolorbox}
\usepackage{comment}

\columnsep=2cm
\hoffset=0cm
\textwidth=8cm
\setlength{\columnseprule}{.1pt}
\setlength{\columnsep}{2cm}
\renewcommand{\headrulewidth}{0pt}
\geometry{top=1in, bottom=1in, left=0.7in, right=0.5in}

\pagestyle{fancy}
\fancyhf{}
\fancyfoot[C]{\thepage}

\begin{document}
	
	\noindent\textbf{Física} 
	
	\begin{center}Resposta dos exercícios - 8º ano
	\end{center}
	
	%\section*{Questões de Física}
	\noindent Q1. As principais fases da lua são nova, minguante, crescente e cheia. \\\\
	Q2. O movimento da Lua afeta nas marés da terra da seguinte maneira, quando a lua está cheia ou seja mais próximo da Terra as marés se elevam e quando a lua está nova, ou seja mais distante, a água do mar fica em um nível mais baixo. \\\\
	Q3. Os tipos de eclipse são: \\\\
	Eclipse anular: quando a Lua está alinhada com a Terra e com o Sol, que projeta a sua luz na Terra e aparece a sombra da Terra na Lua. O eclipse é visível durante a noite ou de madrugada e a cor da Lua fica vermelha. \\\\
	Eclipse parcial solar: observado durante o dia com instrumentos de segurança para não ficar cego, quando a Terra fica alinhada com a Lua e com o Sol, e quando se observa só, a Lua fica na frente do Sol e tapa parcialmente os raios solares que chegam na Terra. \\\\
	Eclipse total solar: também observado durante o dia com instrumentos de segurança para não ficar cego, quando a terra fica alinhada com a Lua e com Sol, e a Lua tapa completamente os raios solares formando uma espécie de coroa solar e escurecendo o céu, o que o senso comum comenta é que o dia vira noite por alguns minutos. São belos espetáculos da natureza. \\\\
	Q4. Ela descreve a força gravitacional que atua entre dois objetos com massa. \\\\
	Princípios da Lei: \\\\
	Atração Universal: Toda partícula no universo atrai todas as outras partículas com uma força proporcional ao produto das suas massas e inversamente proporcional ao quadrado da distância entre elas. \\
	Força Gravitacional: A força é sempre atraente, nunca repulsiva. \\\\
	Fórmula da Força Gravitacional: \\\\
	A força gravitacional F entre dois objetos com massas $m_1$ e $m_2$, separados por uma distância 
	$d$, é dada pela fórmula: \\\\
	\begin{itemize}
	\item $F = G \cdot \frac{m_1 \cdot m_2}{d^2}$ \\\\
	onde: $F$ é a força gravitacional (em Newtons, N)
	\item $G$ é a constante gravitacional universal - aproximadamente $6,674 \cdot 10^{-11} Nm^2 / kg^2$
	\item $m_1$ e $m_2$ são as massas dos objetos (em quilogramas, kg)
	\item $d$ é a distância entre os centros dos objetos (em metros, m)
	\end{itemize}
	Essa fórmula permite calcular a força gravitacional entre qualquer dois corpos celestes ou objetos com massa.\\\\

	%A) \\\\ $F = 6,67 \times 10^{-11} \cdot \frac{6 \times 10^{24} \cdot 2 \times 10^{30}}{(1,5 \times 10^{11})^2}$ \\\\
	%$F = 6,67 \times 10^{-11} \cdot \frac{12 \times 10^{24} \cdot 10^{30}}{1,5 \times 10^{11} \cdot 1,5 \times 10^{11}}$ \\\\
	%$F = 6,67 \times 10^{-11} \cdot \frac{12 \times 10^{24} \cdot 10^{30}}{2,25 \times 10^{22}}$ \\\\
	%$F = 6,67 \times 10^{-11} \cdot \frac{12 \times 10^{54}}{2,25 \times 10^{22}}$ \\\\
	%$F = 6,67 \times 10^{-11} \cdot \frac{1,2 \times 10^{55}}{2,25 \times 10^{22}}$ \\\\
	%$F = 6,67 \times 10^{-11} \cdot 0,53 \times 10^{55} : 10^{22}$ \\\\
	%$F = 6,67 \times 10^{-11} \cdot 0,53 \times 10^{33}$ \\\\
	%$F = 6,67 \times 10^{-11} \cdot 5,3 \times 10^{32}$ \\\\
	%$F = 35,351 \times 10^{11} \cdot 10^{32}$ \\\\
	%$F = 35,351 \times 10^{21}$ \\\\
	%$F = 3,5351 \times 10^{22}$ \\\\
	%$F = 3,54 \times 10^{22}$ N \\\\\\\\
	%B) \\\\
	%$F = 6,67 \times 10^{-11} $\\
	\noindent Q6. De acordo com a Física, massa e peso são conceitos distintos e não são as mesmas coisas \\\\
	A massa é uma grandeza escalar que mede a quantidade de matéria presente em um corpo. Ela é invariável, significando que a massa de um objeto permanece constante independentemente do local onde ele se encontra no universo. \\\\
	A massa é medida em unidades como quilogramas (kg), gramas (g), ou toneladas (t) no Sistema Internacional de Unidades \\\\
	O peso é uma grandeza vetorial que representa a força gravitacional exercida sobre um corpo. Ele depende da massa do corpo e da aceleração da gravidade local. \\\\
	A fórmula para calcular o peso é $Peso = massa \times g$ onde $g$ é a aceleração da gravidade. O peso é medido em Newtons (N) no Sistema Internacional de Unidades. \\\\
	Q8. \\\\
	O tempo atmosférico se refere às condições meteorológicas de um determinado momento ou curto período, como horas ou dias, em um local específico. \\\\
	Pode mudar rapidamente, variando de uma hora para outra ou de um dia para o outro. Fatores como temperatura, umidade, vento, e precipitação influenciam essas mudanças. \\\\
	Inclui temperatura, chuva, vento, umidade, pressão atmosférica, e outros fenômenos meteorológicos. \\\\
	O clima é o padrão duradouro das condições meteorológicas em uma região, observadas ao longo de um período extenso, geralmente de pelo menos 30 anos. \\\\
	Não muda de um momento para o outro, mas pode variar ao longo dos anos. Representa um comportamento estatístico de variáveis meteorológicas \\\\
	Inclui latitude, altitude, relevo, maritimidade, continentalidade, massas de ar, correntes marítimas, e outros fatores geográficos que influenciam os elementos climáticos como temperatura, umidade, e precipitação.\\\\
	
	\noindent Q9. \\\\
	Pressão, volume e temperatura. \\\\
	
	\noindent Q10. \\\\
	Transformação Isotérmica: \\
	
	Nesta transformação, a temperatura do gás é mantida constante. A pressão e o volume variam, enquanto a temperatura permanece a mesma. \\
	A Lei de Boyle, que estados que, em uma transformação isotérmica, a pressão do gás é inversamente proporcional ao seu volume. Matematicamente, isso é expresso como: \\\\
	$P_0 \cdot V_0 = P_1 \cdot V_1$ \\\\
	
	\noindent Isobárica:  \\
	
	\noindent Nesta transformação, a pressão do gás é mantida constante. O volume e a temperatura variam, enquanto a pressão permanece a mesma. \\
	A Lei de Charles-Gay-Lussac, que afirma que, com a pressão constante, o volume do gás é diretamente proporcional à sua temperatura absoluta. Matematicamente, isso é expresso como: \\\\
	$\frac{V_0}{T_0} = \frac{V_1}{T_1}$ \\\\
	
	
	\noindent Isovolumétrica:
	
	\noindent Nesta transformação, o volume do gás é mantido constante. A pressão e a temperatura variam, enquanto o volume permanece o mesmo. \\
	 A segunda lei de Charles-Gay-Lussac, que afirma que, com o volume constante, a pressão do gás é diretamente proporcional à sua temperatura absoluta. Matematicamente, isso é expresso como: \\\\
	 $\frac{P_0}{T_0} = {P_1}{T_1}$ \\\\
\end{document}