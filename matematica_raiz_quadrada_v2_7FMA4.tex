\documentclass[a4paper,14pt]{article}
\usepackage{extsizes}
\usepackage{amsmath}
\usepackage{amssymb}
\everymath{\displaystyle}
\usepackage{geometry}
\usepackage{fancyhdr}
\usepackage{multicol}
\usepackage{graphicx}
\usepackage[brazil]{babel}
\usepackage[shortlabels]{enumitem}
\columnsep=2cm
\hoffset=0cm
\textwidth=8cm
\setlength{\columnseprule}{.1pt}
\setlength{\columnsep}{2cm}
\renewcommand{\headrulewidth}{0pt}
\geometry{top=1in, bottom=1in, left=1in, right=1in}

\pagestyle{fancy}
\fancyhf{}
\fancyfoot[C]{\thepage}

\begin{document}
	
	\noindent\textbf{7FMA4~-~Matemática} 
	
	\begin{center}
		\textbf{Raiz Quadrada (Versão estudante)}
	\end{center}
	
	\bigskip
	
	\noindent\textbf{Nome:} \underline{\hspace{10cm}}
    \noindent\textbf{Data:} \underline{\hspace{4cm}}
	
	\bigskip
	%\section*{Questões de Matemática}
	
	\begin{multicols}{2}
	\begin{itemize}
		\item Se $a \geq 0$ e $b \geq 0$, temos:
		\begin{center}
			$a^2 = b \Leftrightarrow \sqrt{b} = a$
		\end{center}
	    \item Se $a \geq 0$, $\sqrt{-a}$ não é um número racional.
	    \item Se $a \geq 0$, $b \geq 0$ e $\sqrt{\frac{a}{b}} \in \mathbb{Q_+}$, temos:
	    \begin{center}
	    	$\sqrt{\frac{a}{b}} = \frac{\sqrt{a}}{\sqrt{b}}$
	    \end{center}  
	\end{itemize}
	\begin{enumerate}
		
		\item Com base na informação fornecida, completar:
		\begin{enumerate}[a)]
			\item $2^2 = 4 \Leftrightarrow \sqrt{4} = $
			\item $3^2 = 9 \Leftrightarrow \sqrt{9} = $
			\item $4^2 = 16 \Leftrightarrow \sqrt{16} = $ 
			\item $5^2 = 25 \Leftrightarrow \sqrt{25} = $
			\item $7^2 = 49 \Leftrightarrow \sqrt{49} = $
			\item $8^2 = 64 \Leftrightarrow \sqrt{64} = $
			\item $10^2 = 100 \Leftrightarrow \sqrt{100} = $
			\item $11^2 = 121 \Leftrightarrow \sqrt{121} = $
			\item $15^2 = 225 \Leftrightarrow \sqrt{225} = $
			\item $1^2 = 1 \Leftrightarrow \sqrt{1} = $
			\item $0^2 = 0 \Leftrightarrow \sqrt{0} = $
			\item $18^2 = 324 \Leftrightarrow \sqrt{~~~~} = 18$
			\item $17^2 = 289 \Leftrightarrow \sqrt{~~~~} = 17$
			\item $16^2 = 256 \Leftrightarrow \sqrt{~~~~} = 16$
			\item $14^2 = 196 \Leftrightarrow \sqrt{~~~~} = 14$
			\item $13^2 = 169 \Leftrightarrow \sqrt{~~~~} = 13$
			\item $12^2 = 144 \Leftrightarrow \sqrt{~~~~} = 12$
			\item $9^2 = 81 \Leftrightarrow \sqrt{~~~~} = 9$
			\item $6^2 = 36 \Leftrightarrow \sqrt{~~~~} = 6$
	    \end{enumerate}
    	\item Completar:
        \begin{enumerate}[a)]
        	\item $\sqrt{4} = $
        	\item $\sqrt{0} = $
        	\item $\sqrt{9} = $
        	\item $\sqrt{25} = $
        	\item $\sqrt{64} = $
        	\item $\sqrt{144} = $
        	\item $\sqrt{400} = $
        	\item $-\sqrt{1} = $
        	\item $-\sqrt{36} = $
        	\item $-\sqrt{225} = $
        	\item $-\sqrt{121} = $
        	\item $-\sqrt{16} = $
        	\item $\sqrt{~~~~~~~~~~~~~~} = 30$
        	\item $\sqrt{~~~~~~~~~~~~~~} = 19$
        	\item $\sqrt{~~~~~~~~~~~~~~} = 14$
        	\item $\sqrt{~~~~~~~~~~~~~~} = 23$
        	\item $\sqrt{~~~~~~~~~~~~~~} = 1$
        	\item $-\sqrt{~~~~~~~~~~~~~~} = -7$
        	\item $-\sqrt{~~~~~~~~~~~~~~} = -9$
        	\item $-\sqrt{~~~~~~~~~~~~~~} = -21$
        \end{enumerate}
    	\item Completar:
    	\begin{enumerate}[a)]
    		\item $\sqrt{\frac{1}{25}} = $
    		\item $\sqrt{\frac{9}{49}} = $
    		\item $\sqrt{\phantom{6cm}~~~~~~~~~~} = \frac{2}{3}$
    		\\\\
    		\item $\sqrt{\frac{64}{169}} = $
    		\item $\sqrt{\frac{225}{16}} = $
    		\item $\sqrt{\phantom{6cm}~~~~~~~~~~} = \frac{16}{5}$
    		\\\\
    		\item $\sqrt{\phantom{6cm}~~~~~~~~~~} = \frac{1}{20}$
    		\\\\
    		\item $\sqrt{\phantom{6cm}~~~~~~~~~~} = \frac{11}{2}$
    		\\\\
    		\item $\sqrt{\phantom{6cm}~~~~~~~~~~} = \frac{21}{9}$
    		\\\\
    	\end{enumerate}
		
		
    \end{enumerate}        
    \end{multicols}    

\end{document}