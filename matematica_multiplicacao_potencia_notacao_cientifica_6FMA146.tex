\documentclass[a4paper,14pt]{article}

\usepackage{comment} % Para comentar várias linhas ao mesmo tempo

%matemática
\usepackage{amsmath}
\usepackage{amssymb}

%diagramação
\usepackage{extsizes}
\everymath{\displaystyle}
\usepackage{geometry}
\usepackage{fancyhdr}
\usepackage{multicol}
\usepackage{graphicx}
\usepackage[brazil]{babel}
\usepackage[shortlabels]{enumitem}
\usepackage{cancel}
\usepackage{textcomp}
\usepackage{tcolorbox}

%tabelas
\usepackage{array} % Para melhor formatação de tabelas
\usepackage{longtable}
\usepackage{booktabs}  % Para linhas horizontais mais bonitas
\usepackage{float}   % Para usar o modificador [H]
\usepackage{caption} % Para usar legendas em tabelas
\usepackage{wrapfig} % Para usar tabelas e figuras flutuantes
\usepackage{xcolor} % Para cores do fundo de tabelas
\usepackage{colortbl} % Para cores do fundo de tabelas
\usepackage{upgreek} % Para inserir caracteres gregos

%tikzpicture
\begin{comment}
	\usepackage{tikz}
	\usepackage{scalerel}
	\usepackage{pict2e}
	\usepackage{tkz-euclide}
	\usetikzlibrary{calc}
	\usetikzlibrary{patterns,arrows.meta}
	\usetikzlibrary{shadows}
	\usetikzlibrary{external}
\end{comment}


%pgfplots
\usepackage{pgfplots}
\pgfplotsset{compat=newest}
\usepgfplotslibrary{statistics}
\usepgfplotslibrary{fillbetween}

%colours
\usepackage{xcolor}



\columnsep=2cm
\hoffset=0cm
\textwidth=8cm
\setlength{\columnseprule}{.1pt}
\setlength{\columnsep}{2cm}
\renewcommand{\headrulewidth}{0pt}
\geometry{top=1in, bottom=1in, left=0.7in, right=0.5in}

\pagestyle{fancy}
\fancyhf{}
\fancyfoot[C]{\thepage}

\begin{document}
	
	\noindent\textbf{6FMA146 - Matemática} 
	
	\begin{center}Multiplicação (Versão estudante)
	\end{center}
	
	\noindent\textbf{Nome:} \underline{\hspace{10cm}}
	\noindent\textbf{Data:} \underline{\hspace{4cm}}
	
	%\section*{Questões de Matemática}
	
	\begin{multicols}{2}
	    \noindent A multiplicação de potências com base dez, para $m_1, m_2, ..., m_i \in \mathbb{N}, i \in \mathbb{N}^*$, vale a seguinte propriedade: \\
	    \begin{center} $10^{m_1} \cdot 10^{m_2} \cdot ... \cdot 10^{m_i} = 10^{m_1 + m_2 + ... + m_i}$ 
	    \end{center}
	    O mesmo ocorre quando há um número $k$ multiplicando uma potência de dez. Para $a, b \in \mathbb{R}$, temos: \\
	    \begin{center} $(a \cdot 10^m) \cdot (b \cdot 10^n) = a \cdot b \cdot 10^{m + n}$ 
	    \end{center}
	    A \textbf{notação científica} representa um número não nulo da forma $k \cdot 10^n$, onde $n \in \mathbb{Z}$ e $1 \leq k < 10$. \\
		\noindent\textsubscript{--------------------------------------------------------------------------}
		\begin{enumerate} 
			\item Expresse com potência de dez:
			\begin{enumerate}[a)]
				\item 1 000 \\\\\\
				\item 10 000 \\\\\\
				\item 100 \\\\\\
				\item 0,1 \\\\\\
				\item 0,001 \\\\\\
				\item 0,00001 \\\\\\
			\end{enumerate}
			\item Escreva na forma mais simples:
			\begin{enumerate}[a)]
				\item $10^4 \cdot 10^{21}$ \\\\\\
				\item $10^3 \cdot 10^{10}$ \\\\\\
				\item $10^{62} \cdot 10^{14} \cdot 10^2$ \\\\\\
				\item $10^{32} \cdot 10^4 \cdot 10^7$ \\\\\\
				\item $10^2 \cdot 10^5 \cdot 10^8 \cdot 10^{16}$ \\\\\\
				\item $10^3 \cdot 10^7 \cdot 10^4 \cdot 10^9$ \\\\\\
			\end{enumerate}
			\item Para $A = 230$, $B = 5 \cdot 10^7$ e $C = 700 000$, calcule:
			\begin{enumerate}[a)]
				\item $A \cdot B$ \\\\\\
				\item $B \cdot C$ \\\\\\
				\item $A \cdot B \cdot C$ \\\\\\
			\end{enumerate}
			\item Passe para a notação científica:
			\begin{enumerate}[a)]
				\item 2 900 \\\\\\
				\item 0,00057 \\\\\\
				\item 427 000 000 \\\\\\
				\item 0,00000186 \\\\\\
			\end{enumerate}
			\item Calcule e dê sua resposta utilizando a notação científica:
			\begin{enumerate}[a)]
				\item $8 \cdot 10^{12} \cdot 41 \cdot 10^3$ \\\\\\
				\item $1,7 \cdot 10^8 \cdot 3,4 \cdot 10^{11}$ \\\\\\
				\item $7,2 \cdot 10^3 \cdot 40 \cdot 10^2$ \\\\\\
				\item $300 \cdot 4,8 \cdot 10^4 \cdot 0,007$ \\\\\\
			\end{enumerate}
			\item Expresse com potência de dez:
			\begin{enumerate}[a)]
				\item 0,01
				\item 1 000
				\item 0,0000001
				\item 0,0001
				\item 1 000 000 000
				\item 100 000 000
			\end{enumerate}
			\item Simplifique
			\begin{enumerate}[a)]
				\item $10^{12} \cdot 10^{16}$
				\item $10^3 \cdot 10^7$
				\item $10^3 \cdot 10^7 \cdot 10^9$
				\item $10^4 \cdot 10^{-3} \cdot 10^6$ 
				\item $10^2 \cdot 10^4 \cdot 10^6 \cdot 10^7$
				\item $10^8 \cdot 10^6 \cdot 10^9 \cdot 10^5$
			\end{enumerate}
			\item Passe para a notação científica:
			\begin{enumerate}[a)]
				\item 790
				\item 0,0034
				\item 1 340 000
				\item 0,00006123
				\item 97 400 000
				\item 0,001
			\end{enumerate}
			\item Calcule utilizando a notação científica:
			\begin{enumerate}[a)]
				\item $21 \cdot 10^3 \cdot 2 \cdot 10^4$
				\item $0,45 \cdot 10^7 \cdot 21,0 \cdot 10^6$
				\item $4 \cdot 10^7 \cdot 160,00 \cdot 10^8$
				\item $4 100 \cdot 0,087 \cdot 10^2 \cdot 0,0007$ 
				\item $0,0046 \cdot 10^4 \cdot 200 000 \cdot 60 \cdot 10^9$
				\item $4,3 \cdot 10^8 \cdot 2,6 \cdot 10^2 \cdot 3,9 \cdot 10^6$
			\end{enumerate}
		\end{enumerate}
			 $~$ \\ $~$ \\ $~$ \\ $~$ \\ $~$ \\ $~$ \\ $~$ \\ $~$ \\ $~$ \\ $~$ \\ $~$ \\ $~$ \\ $~$ \\ $~$ \\ $~$ \\ $~$ \\ $~$ \\ $~$ \\ $~$ \\ $~$ \\ $~$ \\ $~$ \\ $~$ \\ $~$ \\ $~$ \\ $~$ \\ $~$ \\ $~$ \\ $~$ \\ $~$ \\ $~$ \\ $~$ \\ $~$ \\ $~$ \\ $~$ \\ $~$ \\ $~$ \\ $~$ \\ $~$ \\ $~$ \\ $~$ \\ $~$ \\ $~$ \\ $~$ \\ $~$ \\ $~$ \\ $~$ \\ $~$ \\ $~$ \\ $~$ \\ $~$ \\ $~$ \\ $~$ \\ $~$ \\ $~$ \\ $~$ \\ $~$ \\ $~$ \\ $~$ \\ $~$ \\ $~$ \\ $~$ \\ $~$ \\ $~$ \\ $~$ \\ $~$ \\ $~$ \\ $~$ \\ $~$ \\ $~$ \\ $~$ 
	\end{multicols}
\end{document}