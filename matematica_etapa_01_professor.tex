\documentclass[a4paper,14pt]{article}
\usepackage{extsizes}
\usepackage{amsmath}
\usepackage{amssymb}
\everymath{\displaystyle}
\usepackage{geometry}
\usepackage{fancyhdr}
\usepackage{multicol}
\usepackage{graphicx}
\usepackage[brazil]{babel}
\usepackage[shortlabels]{enumitem}
\usepackage{cancel}
\columnsep=2cm
\hoffset=0cm
\textwidth=8cm
\setlength{\columnseprule}{.1pt}
\setlength{\columnsep}{2cm}
\renewcommand{\headrulewidth}{0pt}
\geometry{top=1in, bottom=1in, left=0.7in, right=0.5in}

\pagestyle{fancy}
\fancyhf{}
\fancyfoot[C]{\thepage}

\begin{document}
	
	\noindent\textbf{EF08MA08-A~-~Matemática} 
	
	\begin{center}
		\textbf{Revisão: problemas que contam histórias ou apresentam situações (Versão professor)}
	\end{center}
	
	
	\noindent\textbf{Nome:} \underline{\hspace{10cm}}
    \noindent\textbf{Data:} \underline{\hspace{4cm}}
	
	%\section*{Questões de Matemática}
	
	\begin{multicols}{2}
	\begin{enumerate}	
		\item Duas pessoas farão conjuntamente a pintura de um muro, cada uma trabalhando a partir de uma das extremidades. Se uma delas pintar 2/5 do muro e a outra os 15 m restantes, a extensão deste muro é de: 
		\begin{enumerate}[a)]
			\item 25 m
			\item 35 m
			\item 42 m
			\item 45 m
			\item 20 m
	    \end{enumerate}
    
        Resposta: alternativa a) \\
        
        Vamos chamar a extensão do muro de x metros.\\
        
        Se uma pessoa pinta $\frac{2}{5}$ do muro e a outra pessoa pinta os 15 metros restantes, a quantidade pintada por cada uma somada deve ser igual ao comprimento total do muro.\\
        
        A pessoa que pintou $\frac{2}{5}$ do muro cobriu uma fração desse muro, enquanto a outra pessoa cobriu 
        15 metros, então temos a seguinte equação:\\
        
        $\frac{2}{5}x+15=x$\\
        
        Vamos resolver isso para encontrar x:\\
        
        Primeiro, multiplique ambos os lados da equação por 5 para eliminar o denominador:\\
        
        2x+75=5x\\
        
        Agora, traga os termos com x para um lado da equação:\\
        
        5x-2x=75\\
        
        3x=75\\
        
        Agora, isole x dividindo ambos os lados por 3:\\
        $x = \frac{75}{3}$\\
        x=25\\
        
        Portanto, a extensão do muro é de 25 metros.\\
        
        \newpage
        
        \item A bilheteria de um teatro só trabalha com ingressos "lugares A" e "lugares B" com preços de R\$ 16,00 e R\$ 10, respectivamente. Uma pessoa adquiriu, por R\$ 192,00, 15 ingressos. Quantos ingressos de "lugares A" e quantos de "lugares B" ela adquiriu, respectivamente?
        \begin{enumerate}[a)]
        	\item 3 e 12.
        	\item 6 e 9.
        	\item 7 e 8.
        	\item 5 e 10.
        	\item 1 e 14.
        \end{enumerate}
    
        Resposta: alternativa c)
        
        Vamos chamar o número de ingressos de "lugares A" de
        x e o número de ingressos de "lugares B" de y. \\
        
        Sabemos que o total de ingressos adquiridos foi de 15 e que o preço de um ingresso de "lugares A" é R\$ 16,00, enquanto o de "lugares B" é R\$ 10,00. \\
        
        Podemos, então, montar um sistema de equações com essas informações:\\
        
        A soma dos ingressos de "lugares A" e "lugares B" é igual a 15:
        $x+y=15$\\
        
        O valor total gasto foi R\$ 192,00:\\
        O valor total gasto com os ingressos de "lugares A" (x ingressos) mais o valor total gasto com os ingressos de "lugares B" (y ingressos) é igual a R\$ 192,00:\\
        
        $16x+10y=192$\\
        
        Agora, podemos resolver esse sistema de equações para encontrar x e y. Vamos usar o método de substituição.\\
        
        A partir da equação 
        x+y=15, podemos isolar y em termos de x:\\
        $y=15-x$ \\ \\
        Substituímos esse valor de y na segunda equação:\\\\
        16x+10y=192 \\
        16x+10(15-x)=192\\
        16x+150-10x=192\\
        6x+150=192\\
        6x=42\\ \\
        x = $\frac{42}{6}$ \\
        x=7 \\
        
        Agora que temos o valor de x=7, podemos encontrar y: \\
        
        y=15-x \\
        y=15-7 \\
        y=8 \\
        
        Portanto, a pessoa adquiriu 7 ingressos de "lugares A" e 8 ingressos de "lugares B", respectivamente. 
    
        \item Na criação de uma placa, um funcionário tem espaço de 9 cm de largura para cada letra do título. Se no título houvesse mais dez letras, o espaço seria reduzido para 6 cm. O número de letras que formam esse título é:
        \begin{enumerate}[a)]
        	\item 20
        	\item 25
        	\item 15
        	\item 10
        	\item 30
        \end{enumerate}
    
        Resposta: alternativa a) \\
        
        Vamos chamar o número inicial de letras do título de x.\\
        
        Segundo as informações fornecidas, quando havia x letras, o espaço disponível para cada letra era de 9 cm. Quando o título aumentou para x+10 letras, o espaço disponível para cada letra diminuiu para 6 cm.\\
        
        Podemos criar uma equação com essas informações:\\
        
        Quando havia x letras: 9x cm de espaço total para as letras.\\
        
        Quando há x+10 letras: 6(x+10) cm de espaço total para as letras. \\
        
        Então, temos a equação:\\
         9x=6(x+10)\\
        
        Vamos resolver essa equação para encontrar o valor de x:\\
         
        9x=6x+60 \\
        9x-6x=60 \\
        3x=60 \\ \\
        $x = \frac{60}{3}$\\
        x=20 \\
        
        Portanto, o número inicial de letras do título é x=20.
        \newpage
        
        \item Descubra um número, de acordo com as informações dadas a seguir:
        \begin{itemize}
        	\item É um número de dois algarismos.
        	\item O algarismo das dezenas é o triplo do algarismo das unidades.
        	\item Trocando os dois algarismos de lugar, obtemos um segundo número. Se subtraio o segundo número do primeiro, o resultado é 54.
        \end{itemize}
    
        Resposta: Vamos chamar o algarismo das dezenas de x e o algarismo das unidades de y. \\
        
        Sabemos que o número é de dois algarismos, então ele pode ser representado como 10x+y.\\
        
        Pelo enunciado, o algarismo das dezenas é o triplo do algarismo das unidades, então x=3y.\\
        
        Quando trocamos os algarismos de lugar, o número se torna 10y+x.\\
        
        A diferença entre esses dois números 54, então a equação que podemos montar é: \\
        (10x+y)-(10y+x)=54\\
        
        Substituímos x por 3y na equação: \\
        (10(3y)+y)-(10y+3y)=54 \\
        30y+y-10y-3y=54 \\
        31y-13y=54 \\
        18y=54 \\\\
        y = $\frac{54}{18}$ \\
        y=3 \\ \\
        Agora que encontramos y=3, podemos encontrar x, já que x=3y: \\
        
        x=3 $\times$ 3 \\
        
        x=9 \\
        
        Portanto, o número original é 10x+y=10×9+3=90+3=93. \\
        
        Quando trocamos os algarismos de lugar, o número se torna 10y+x=10×3+9=30+9=39. \\
        
        E de fato, a diferença entre esses dois números é 93-39=54.
        
        Logo, o número é 93.
    
    \end{enumerate}        
    \end{multicols}    

\end{document}