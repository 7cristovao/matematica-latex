\documentclass[a4paper,14pt]{article}

%matemática
\usepackage{amsmath}
\usepackage{amssymb}

%diagramação
\usepackage{extsizes}
\everymath{\displaystyle}
\usepackage{geometry}
\usepackage{fancyhdr}
\usepackage{multicol}
\usepackage{graphicx}
\usepackage[brazil]{babel}
\usepackage[shortlabels]{enumitem}
\usepackage{cancel}
\usepackage{textcomp}
\usepackage{tcolorbox}

%tabelas
\usepackage{array} % Para melhor formatação de tabelas
\usepackage{longtable}
\usepackage{booktabs}  % Para linhas horizontais mais bonitas
\usepackage{float}   % Para usar o modificador [H]
\usepackage{caption} % Para usar legendas em tabelas
\usepackage{wrapfig} % Para usar tabelas e figuras flutuantes

%tikzpicture
\usepackage{tikz}
\usepackage{scalerel}
\usepackage{pict2e}
\usepackage{tkz-euclide}
\usetikzlibrary{calc}
\usetikzlibrary{patterns,arrows.meta}
\usetikzlibrary{shadows}
\usetikzlibrary{external}

%pgfplots
\usepackage{pgfplots}
\pgfplotsset{compat=newest}
\usepgfplotslibrary{statistics}
\usepgfplotslibrary{fillbetween}

%colours
\usepackage{xcolor}



\columnsep=2cm
\hoffset=0cm
\textwidth=8cm
\setlength{\columnseprule}{.1pt}
\setlength{\columnsep}{2cm}
\renewcommand{\headrulewidth}{0pt}
\geometry{top=1in, bottom=1in, left=0.7in, right=0.5in}

\pagestyle{fancy}
\fancyhf{}
\fancyfoot[C]{\thepage}

\begin{document}
	
	\noindent\textbf{6FMA67 - Matemática} 
	
	\begin{center}Ângulos planos e o compasso (Versão estudante)
	\end{center}
	
	\noindent\textbf{Nome:} \underline{\hspace{10cm}}
	\noindent\textbf{Data:} \underline{\hspace{4cm}}
	
	%\section*{Questões de Matemática}
	\begin{multicols}{2}
		\noindent \textbf{Ângulos planos}
		\begin{center}
			\begin{tikzpicture}
				\draw[{Circle[scale=1.5]}-{Circle[scale=1.5]}] (0,0) -- (2,0)
				node[pos=0, below]{A}  node[pos=1, below]{B}; 
				\draw[-{Stealth[scale=2]}] (2,0) -- (4,0);
				\draw[{Circle[scale=1.5]}-{Circle[scale=1.5]}] (5,0) -- (8,0)
				node[pos=0, above]{A} node[pos=1, above]{B};
			\end{tikzpicture}
		\end{center}
		\noindent Indicamos por $\overrightarrow{AB}$ a semirreta que tem origem $A$ e passa pelo ponto $B$. Já o segmento de reta que une os pontos $A$ e $B$ é indicado por $\overline{AB}$. \\
		Um ângulo é a união de duas semirretas de mesma origem.
		\begin{center}
			\begin{tikzpicture}
				\draw[{Circle[scale=1.5]}-{Circle[scale=1.5]}] (1,3) -- (4.5,6)
				node[pos=0, left]{A} node[pos=1, above]{B};
				\draw (4.5,6) -- (7,8);
				\draw[-{Circle[scale=1.5]}]  (1,3) -- (4,2)
				node[pos=1, below]{C};
				\draw (4,2) -- (7,1);
			\end{tikzpicture}
		\end{center}
		As semirretas $\overrightarrow{AB}$ e $\overrightarrow{AC}$ têm a mesma origem $A$. \\
		As semirretas $\overrightarrow{AB}$ e $\overrightarrow{AC}$ são chamadas de lados do ângulo. Se os lados de um ângulo são semirretas opostas aos lados de outro ângulo, dizemos que são ângulos opostos pelo vértice. \\
		O ângulo indicado no triângulo a seguir pode ser representado por $A\hat{B}C$ ou simplesmente $\hat{B}$.
		\begin{center}
			\begin{tikzpicture}
				\coordinate[label=left:B] (B) at (1,1);
				\coordinate[label=right:C] (C) at (7,1);
				\coordinate[label=above:A] (A) at (4,6);
				
				\draw (B) -- (A) -- (C) -- cycle;
				
				\tkzMarkAngle(C,B,A)
			\end{tikzpicture}
		\end{center}
		\noindent \textbf{Compasso} \\\\
		Para desenhar uma circunferência precisamos saber seu centro e seu raio. Posicionamos a ponta seca no centro e a distância entre a ponta-seca e a ponta da grafite é o raio. \\
		O diâmetro de uma circunferência é igual ao dobro do seu raio.
	\end{multicols}		
    	\noindent\textsubscript{-----------------------------------------------------------------------------------------------------------------------------------------------------------}
    \begin{multicols}{2}
    	\begin{enumerate}
   			\item Dados os pontos $A$ e $B$ a seguir, desenhe a semirreta $\overrightarrow{AB}$ com caneta azul e a semirreta $\overrightarrow{BA}$ com caneta vermelha. Em seguida, responda.
   			\begin{enumerate}[a)]
   				\item As semirretas são iguais ou diferentes? \\\\\\
   				\item O que as duas semirretas têm em comum? \\\\\\\\\\\\
   				\item Podemos dizer que elas são duas semirretas opostas? Explique.  \\\\\\\\\\\\
   				\item Desenhe duas semirretas opostas. \\\\\\\\\\\\
   			\end{enumerate}
   			\item Desenhe a reta $\overleftrightarrow{AB}$, o segmento $\overline{AC}$, a semirreta $\overrightarrow{AD}$ e a circunferência de centro $C$ passando pelo ponto $B$ a seguir. Use régua e compasso. \\\\\\\\\\\\\\\\\\\\
   			\item \begin{enumerate}[a)]
   				\item Desenhe o ângulo $M\hat{P}N$, dados $M, N, P$ a seguir. \\\\\\\\\\\\\\\\\\\\\\\\\\\\\\\\\\
   				\item Desenhe o ângulo $M\hat{P}N$. Qual é a sua conclusão a respeito dos dois ângulos desenhados? \\\\\\\\\\\\\\\\\\\\\\\\\\\\\\\\\\\\
   			\end{enumerate}
   			\item Dê o nome de todos os ângulos internos do quadrilátero de vértices $A, B, C$ e $D$, representado a seguir.
   			\begin{center}
   				\begin{tikzpicture}
   					\coordinate[label=left:A] (A) at (1,7.5);
   					\coordinate[label=right:B] (B) at (7,8);
   					\coordinate[label=right:C] (C) at (6.5,5);
   					\coordinate[label=below:D] (D) at (2.5,2);
   					
   					\draw (A) -- (B) -- (C) -- (D) -- cycle;
   				\end{tikzpicture}
   			\end{center}
   			\vspace{14cm}
   			\item Desenhe a circunferência de centro $O$ e raio $r$, fornecidos a seguir: 
   			\begin{center}
   				\begin{tikzpicture}
   					\coordinate (R0) at (1.5,12);
   					\coordinate (R1) at (5.5,12);
   					\coordinate[label=below:O] (O) at (3.5,4);
   					\draw (R0) -- (R1);
   					\tkzLabelSegment[label=above:r](R0,R1){~}
   					\filldraw[black] (3.5,4) circle (2pt);
   				\end{tikzpicture}
   			\end{center}
   			\vspace{6cm}
   			\item Uma circunferência tem um diâmetro medindo 18 cm. Quantos centímetros mede o raio dessa circunferência? \newpage
   			\item Desenhe, usando a régua, as semirretas:
   			\begin{center}
	   			$\overrightarrow{AB}$, $\overrightarrow{AC}$, $\overrightarrow{CB}$, $\overrightarrow{BD}$ e $\overrightarrow{DA}$
	   			\begin{tikzpicture}
	   				\coordinate[label=left:C] (C) at (1,7);
	   				\filldraw[black] (1,7) circle (2pt);
	   				\coordinate[label=left:A] (A) at (3,1);
	   				\filldraw[black] (3,1) circle (2pt);
	   				\coordinate[label=above:D] (D) at (4,4);
	   				\filldraw[black] (4,4) circle (2pt);
	   				\coordinate[label=above:B] (B) at (7,6);
	   				\filldraw[black] (7,6) circle (2pt);
	   			\end{tikzpicture}
	   		\end{center}
	   		\item Identifique os pares de ângulos opostos pelo vértice.
	   		\begin{center}
	   			\begin{tikzpicture}
	   				%ponto BB
	   				\coordinate (BB) at (1,8);
	   				%ponto A
	   				\coordinate[label=above:A] (A) at (3,7);
	   				\filldraw[black] (3,7) circle (2pt);
	   				%ponto B
	   				\coordinate[label=above:B] (B) at (2,7.5);
	   				\filldraw[black] (2,7.5) circle (2pt);
	   				%ponto C
	   				\coordinate[label=above:C] (C) at (4,7.5);
	   				\filldraw[black] (4,7.5) circle (2pt);
	   				%ponto CC
	   				\coordinate (CC) at (5,8);
	   				%\filldraw[black] (5,8) circle (2pt);
	   				%ponto H
	   				\coordinate[label=left:H] (H) at (0,5.5);
	   				\filldraw[black] (0,5.5) circle (2pt);
	   				%ṕonto D
	   				\coordinate[label=right:D] (D) at (6, 5.5);
	   				\filldraw[black] (6,5.5) circle (2pt);
	   				%ponto F
	   				\coordinate[label=above:F] (F) at (3,4);
	   				\filldraw[black] (3,4) circle (2pt);
	   				%ponto G
	   				\coordinate[label=left:G] (G) at (3,2);
	   				\filldraw[black] (3,2) circle (2pt);
	   				%ponto E
	   				\coordinate[label=above:E] (E) at (5,3);
	   				\filldraw[black] (5,3) circle (2pt);
	   				%ponto EE
	   				\coordinate (EE) at (6,2.5);
	   				%ponto GG
	   				\coordinate (GG) at (3,1);
	   				\draw (BB) -- (B) -- (A) -- (C) -- (CC);
	   				\draw (H) -- (A) -- (D) -- (F) -- cycle;
	   				\draw (F) -- (E) -- (EE);
	   				\draw (F) -- (G) -- (GG);
	   			\end{tikzpicture}
	   		\end{center}
	   		\vspace{6cm}
	   		\item Desenhe os ângulos $M\hat{Q}P$, $P\hat{M}N$, $M\hat{Q}R$ e represente-os por $x$, $y$ e $z$, respectivamente.
	   		\begin{center}
	   			\begin{tikzpicture}`
	   				\coordinate[label=below:R] (R) at (0,0);
	   				\filldraw[black] (0,0) circle (2pt);
	   				\coordinate[label=below:Q] (Q) at (2,3);
	   				\filldraw[black] (2,3) circle (2pt);
	   				\coordinate[label=above:M] (M) at (2,7);
	   				\filldraw[black] (2,7) circle (2pt);
	   				\coordinate[label=below:P] (P) at (6,2.5);
	   				\filldraw[black] (6,2.5) circle (2pt);
	   				\coordinate[label=above:N] (N) at (7,5);
	   				\filldraw[black] (7,5) circle (2pt);
	   			\end{tikzpicture}
	   		\end{center}
	   		\begin{enumerate}[a)]
	   			\item \textbf{V} ou \textbf{F}:$x > y > z$.
	   			\item Coloque em ordem decrescente as medidas $x, y, z$. \newpage
	   		\end{enumerate}
	   		\item Desenhe, usando régua, a reta $\overleftrightarrow{BD}$, o segmento $\overline{CD}$ e as semirretas $\overrightarrow{BA}, \overrightarrow{DE}$ e $\overrightarrow{BC}$:
	   		\begin{center}
	   			\begin{tikzpicture}
	   				\coordinate[label=below:C] (C) at (7,7);
	   				\filldraw[black] (7,7) circle (2pt);
	   				\coordinate[label=above:B] (B) at (3,5);
	   				\filldraw[black] (3,5) circle (2pt);
	   				\coordinate[label=below:E] (E) at (2.5,0);
	   				\filldraw[black] (2.5,0) circle (2pt);
	   				\coordinate[label=left:D] (D) at (0.5,2);
	   				\filldraw[black] (0.5,2) circle (2pt);
	   				\coordinate[label=above:A] (A) at (0,4);
	   				\filldraw[black] (0,4) circle (2pt);
	   			\end{tikzpicture}
	   		\end{center}
	   		\begin{enumerate}[a)]
	   			\item Há algum par de semirretas opostas? Se sim, quais? \\\\\\\\\\\\\\\\\\
	   			\item Coloque em ordem crescente as medias $A\hat{B}D, C\hat{B}D, B\hat{C}D,$ e $C\hat{D}E$. \\\\\\
	   		\end{enumerate}
	   		\item Dadas as medidas $r$ e $s$, copie para seu caderno os pontos $P$ e $Q$ da figura a seguir, mantendo a mesma distância entre $P$ e $Q$, em seguida desenhe as circunferências de centro $P$ e raio $r$ e centro $Q$ e raio $s$. Você percebe alguma coisa de particular nesse desenho?
	   		\begin{center}
	   			\begin{tikzpicture}
		   			\coordinate (R0) at (1,4);
		   			\coordinate (R1) at (4,4);
		   			\coordinate (S0) at (1,3);
		   			\coordinate (S1) at (5,3);
		   			\coordinate[label=above:P] (P) at (0,2);
		   			\coordinate[label=above:Q] (Q) at (7,1);
		   			\filldraw[black] (0,2) circle (2pt);
		   			\filldraw[black] (7,1) circle (2pt);
		   			\draw (R0) -- (R1);
		   			\draw (S0) -- (S1);
		   			\tkzLabelSegment[label=above:r](R0,R1){~}
		   			\tkzLabelSegment[label=above:s](S0,S1){~}
	   			\end{tikzpicture}
	   		\end{center}
	   		\vspace{3cm}
	   		\item Na figura abaixo, trace duas circunferências: uma com centro em $A$ e que passa por $B$ e outra com centro em B e raio igual a $r$, cuja medida é dada a seguir:
	   		\begin{center}
		   		\begin{tikzpicture}
		   			\coordinate (R0) at (4,6);
		   			\coordinate (R1) at (6,6);
		   			\coordinate[label=right:B] (B) at (3.5, 3);
		   			\coordinate[label=left:A] (A) at (0,0);
		   			\filldraw[black] (3.5,3) circle (2pt);
		   			\filldraw[black] (0,0) circle (2pt);
		   			\draw (R0) -- (R1);
		   			\tkzLabelSegment[label=above:r](R0,R1){~};
		   		\end{tikzpicture}
		   	\end{center}
		   	A circunferência com centro em $B$ passa pelo ponto $A$? Justifique.
		   	\item Desenhe em seu caderno uma circunferência que 10 cm de diâmetro e diga quantos centímetros ela tem de raio.
	    \end{enumerate} 
        $~$ \\ $~$ \\ $~$ \\ $~$ \\ $~$ \\ $~$ \\ $~$ \\ $~$ \\ $~$ \\ $~$ \\ $~$ \\ $~$ \\ $~$  \\ $~$ \\ $~$ \\ $~$ \\ $~$ \\ $~$ \\ $~$ \\ $~$ \\ $~$ \\ $~$ \\ $~$ \\ $~$ \\ $~$ \\ $~$ \\ $~$ \\ $~$ \\ $~$ \\ $~$ \\ $~$ \\ $~$ \\ $~$ \\ $~$ \\ $~$ \\ $~$ \\ $~$ \\ $~$ \\ $~$ \\ $~$ \\ $~$ \\ $~$ \\ $~$ \\ $~$ \\ $~$ \\ $~$ \\ $~$ \\ $~$ \\ $~$ \\ $~$ \\ $~$ \\ $~$ \\ $~$ \\ $~$ \\ $~$ \\ $~$ \\ $~$ \\ $~$ \\ $~$ \\ $~$ \\ $~$ \\ $~$ \\ $~$ \\ $~$ \\ $~$ \\ $~$ \\ $~$ \\ $~$ \\ $~$ \\ $~$ \\ $~$ \\ $~$ \\ $~$ \\ $~$ \\ $~$ \\ $~$ \\ $~$ \\ $~$ \\ $~$
	\end{multicols}
\end{document}