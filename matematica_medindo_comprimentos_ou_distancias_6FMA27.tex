\documentclass[a4paper,14pt]{article}

\usepackage{comment} % Para comentar várias linhas ao mesmo tempo

%matemática
\usepackage{amsmath}
\usepackage{amssymb}

%diagramação
\usepackage{extsizes}
\everymath{\displaystyle}
\usepackage{geometry}
\usepackage{fancyhdr}
\usepackage{multicol}
\usepackage{graphicx}
\usepackage[brazil]{babel}
\usepackage[shortlabels]{enumitem}
\usepackage{cancel}
\usepackage{textcomp}
\usepackage{tcolorbox}

%tabelas
\usepackage{array} % Para melhor formatação de tabelas
\usepackage{longtable}
\usepackage{booktabs}  % Para linhas horizontais mais bonitas
\usepackage{float}   % Para usar o modificador [H]
\usepackage{caption} % Para usar legendas em tabelas
\usepackage{wrapfig} % Para usar tabelas e figuras flutuantes


%tikzpicture
\usepackage{tikz}
\usepackage{scalerel}
\usepackage{pict2e}
\usepackage{tkz-euclide}
\usetikzlibrary{calc}
\usetikzlibrary{patterns,arrows.meta}
\usetikzlibrary{shadows}
\usetikzlibrary{external}


%pgfplots
\usepackage{pgfplots}
\pgfplotsset{compat=newest}
\usepgfplotslibrary{statistics}
\usepgfplotslibrary{fillbetween}

%colours
\usepackage{xcolor}



\columnsep=2cm
\hoffset=0cm
\textwidth=8cm
\setlength{\columnseprule}{.1pt}
\setlength{\columnsep}{2cm}
\renewcommand{\headrulewidth}{0pt}
\geometry{top=1in, bottom=1in, left=0.7in, right=0.5in}

\pagestyle{fancy}
\fancyhf{}
\fancyfoot[C]{\thepage}

\begin{document}
	
	\noindent\textbf{6FMA27 - Matemática} 
	
	\begin{center}Medindo comprimentos ou distâncias (Versão estudante)
	\end{center}
	
	\noindent\textbf{Nome:} \underline{\hspace{10cm}}
	\noindent\textbf{Data:} \underline{\hspace{4cm}}
	
	%\section*{Questões de Matemática}
	
	\begin{multicols}{2}
		\noindent Para medirmos um comprimento qualquer, por exemplo, um segmento de reta, utilizamos a régua.
		\\
		\begin{center}
			\begin{tabular}[H]{|c|c|}
				\hline
				\multicolumn{2}{|c|}{\textbf{Unidades de medida}} \\ \hline
				quilômetro & km \\ \hline
				metro & m \\ \hline
				centímetro & cm \\ \hline
				milímetro & mm \\ \hline
			\end{tabular}
		\end{center}
		\begin{itemize}
			\item 1 km = 1 000 m.
			\item 1 m = 100 cm.
			\item 1 cm = 10 mm.
		\end{itemize}
		\noindent\textsubscript{-----------------------------------------------------------------------}
		\begin{enumerate} 
			\item Um segmento mede 13 m e 28 cm. Qual é o seu comprimento expresso em milímetros? \\\\\\\\\\\\\\\\\\\\\\\\\\\\
			\item 72 000 mm valem quantos metros? \\\\\\\\\\\\\\\\\\\\
			\item Assinale \textbf{V} (verdadeiro) ou \textbf{F} (falso).
			\begin{enumerate}[a)]
				\item (~~) 1 m = 100 cm
				\item (~~) 1 m = 1 000 mm
				\item (~~) 3 cm = 300 mm
				\item (~~) 7 km = 700 m
				\item (~~) 8 m = 800 cm
				\item (~~) 400 mm = 4 cm
				\item (~~) 6 000 mm = 6 m
				\item (~~) 9 km = 900 m \newpage
			\end{enumerate}
			\item Qual é o comprimento e a largura do retângulo abaixo, em milímetros?
			\begin{center}
				\begin{tikzpicture}
					%\draw[lightgray] (0,0) grid (7,7);
					\coordinate[label=above:~] (A) at (1,7);
					\coordinate[label=above:~] (B) at (6,7);
					\coordinate[label=above:~] (C) at (1,0);
					\coordinate[label=above:~] (D) at (6,0);
					\draw (1,7) -- (6,7) -- (6,0) -- (1,0) -- (1,7);
				\end{tikzpicture}
			\end{center}
			\item Dê as medidas dos lados do triângulo $ABC$ abaixo.
			\begin{center}
				\begin{tikzpicture}
					%\draw[lightgray] (0,0) grid (7,7);
					\coordinate[label=left:A] (A) at (1,7);
					\coordinate[label=right:B] (B) at (6,7);
					\coordinate[label=below:C] (C) at (2,0);
					\draw (1,7) -- (6,7) -- (2,0) -- (1,7);
				\end{tikzpicture}
			\end{center}
			%94 a 98
			\columnbreak\item O fio de um aparelho eletrônico mede 95 cm e 8 mm. Determine o seu comprimento em mm. \\\\\\\\\\\\\\\\\\\\
			\item Assinale \textbf{V} (verdadeiro) ou \textbf{F} (falso).
			\begin{enumerate}[a)]
				\item (~~) 100 cm = 1 m
				\item (~~) 82 m = 820 000 mm
				\item (~~) 130 mm = 13 m
				\item (~~) 6 km = 600 000 cm
				\item (~~) 18 cm = 180 mm
				\item (~~) 35 000 m = 35 km
				\item (~~) 7 m = 70 cm
			\end{enumerate}
			\item Dê o valor, em centímetros, de:
			\begin{enumerate}[a)]
				\item 6 m
				\item 270 mm
				\item 18 m
				\item 135 km
				\item 4 000 mm \newpage
			\end{enumerate}
			\item Calcule, em milímetros:
			\begin{enumerate}[a)]
				\item 1 m + 40 cm - 70 cm \\\\\\\\\\\\
				\item 7 m + 2 $\cdot$ 290 cm - 4 $\cdot$ 70 mm  \\\\\\\\\\\\
				\item 6 $\cdot$ 8 m - 9 $\cdot$ 2 cm + 3 $\cdot$ 19 mm  \\\\\\\\\\\\\\\\\\\\\\\\\\\\\\\\\\\\\\\\\\\\\\
			\end{enumerate}
			\item Alexandre comprou, em uma loja de materiais de construção, três tipos de tubo plástico: 15 tubos do tipo $A$, 18 tubos do tipo $B$ e 32 tubos do tipo $C$. \\
			As medidas dos tubos $A$,$B$ e $C$ são, respectivamente, 1 200 cm, 5 m e 3 000 mm a unidade. O metro de qualquer tubo, independentemente do tipo escolhido, custa 3 reais. Qual o valor pago por Alexandre?
		\end{enumerate}
		$~$ \\ $~$ \\ $~$ \\ $~$ \\ $~$ \\ $~$ \\ $~$ \\ $~$ \\ $~$ \\ $~$ \\ $~$ \\ $~$ \\ $~$ \\ $~$ \\ $~$ \\ $~$ \\ $~$ \\ $~$ \\ $~$ \\ $~$ \\ $~$ \\ $~$ \\ $~$ \\ $~$ \\ $~$ \\ $~$ \\ $~$ \\ $~$ \\ $~$
	\end{multicols}
\end{document}