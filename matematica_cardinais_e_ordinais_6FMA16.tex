\documentclass[a4paper,14pt]{article}

\usepackage{comment} % Para comentar várias linhas ao mesmo tempo

%matemática
\usepackage{amsmath}
\usepackage{amssymb}

%diagramação
\usepackage{extsizes}
\everymath{\displaystyle}
\usepackage{geometry}
\usepackage{fancyhdr}
\usepackage{multicol}
\usepackage{graphicx}
\usepackage[brazil]{babel}
\usepackage[shortlabels]{enumitem}
\usepackage{cancel}
\usepackage{textcomp}
\usepackage{tcolorbox}

%tabelas
\usepackage{array} % Para melhor formatação de tabelas
\usepackage{longtable}
\usepackage{booktabs}  % Para linhas horizontais mais bonitas
\usepackage{float}   % Para usar o modificador [H]
\usepackage{caption} % Para usar legendas em tabelas
\usepackage{wrapfig} % Para usar tabelas e figuras flutuantes


%tikzpicture
\begin{comment}
	\usepackage{tikz}
	\usepackage{scalerel}
	\usepackage{pict2e}
	\usepackage{tkz-euclide}
	\usetikzlibrary{calc}
	\usetikzlibrary{patterns,arrows.meta}
	\usetikzlibrary{shadows}
	\usetikzlibrary{external}
\end{comment}


%pgfplots
\usepackage{pgfplots}
\pgfplotsset{compat=newest}
\usepgfplotslibrary{statistics}
\usepgfplotslibrary{fillbetween}

%colours
\usepackage{xcolor}



\columnsep=2cm
\hoffset=0cm
\textwidth=8cm
\setlength{\columnseprule}{.1pt}
\setlength{\columnsep}{2cm}
\renewcommand{\headrulewidth}{0pt}
\geometry{top=1in, bottom=1in, left=0.7in, right=0.5in}

\pagestyle{fancy}
\fancyhf{}
\fancyfoot[C]{\thepage}

\begin{document}
	
	\noindent\textbf{6FMA16 - Matemática} 
	
	\begin{center}Cardinais e ordinais (Versão estudante)
	\end{center}
	
	\noindent\textbf{Nome:} \underline{\hspace{10cm}}
	\noindent\textbf{Data:} \underline{\hspace{4cm}}
	
	%\section*{Questões de Matemática}
	
	\begin{multicols}{2}
		\noindent Quando usamos números para contar, dizemos que eles são cardinais, e quando os usamos para ordenar, dizemos que são ordinais. \\
		Em uma sequência, cada coisa ou objeto é chamado elemento. \\
		O conjunto {0, 1, 2, 3, 4, 5, ...} é chamado conjunto dos números naturais e é representado por $\mathbb{N}$.
		\noindent\textsubscript{-----------------------------------------------------------------------}
		\begin{enumerate} 
			\item Escreva, da maneira como são falados, os seguintes ordinais:
			\begin{enumerate}[a)]
				\item 9º \\\\
				\item 34º \\\\
				\item 41ª \\\\
				\item 87ª \\\\
				\item 126º \\\\
				\item 342ª \\\\
			\end{enumerate}
			\item Determine o segundo elemento da sequência 2, $\underline{~~~~~~~}$, 8, 11, 14, 17.
			\item Qual é o ducentésimo octogésimo sétimo elemento da sequência 1, 0, 1, 0, 1, 0, 1, 0, ...? \\\\\\
			\item Apresente o conjunto dos números naturais maiores do que 87 e menores do que 90. \\\\\\\\\\
			\item Escreva o conjunto formado pelos números naturais menores do que 27. \\\\\\\\\\
			\item Coloque um \textbf{V} (verdadeiro) ou um \textbf{F} (falso) no espaço entre parênteses, conforme a afirmação for verdadeira ou falsa:
			\begin{enumerate}[a)]
				\item (~~) $8 > 20$
				\item (~~) $1 000 000 > 999 999$
				\item (~~) $487 > 219$
				\item (~~) $85 < 63$
			\end{enumerate}
			%56 a 59
			\item Qual é o octogésimo sétimo elemento da sequência 1, 2, 3, 1, 2, 3, 1, 2, 3, ...? \\\\\\\\\\
			\item Complete a sequência abaixo até o décimo termo. \\
			1, 1, 3, 5, 9, 15, ... \\
			Explique a regra que você usou. \\\\\\\\\\
			\item Apresente o conjunto de todos os naturais:
			\begin{enumerate}[a)]
				\item menores que 10. \\\\\\\\\\
				\item entre 7 e 18, inclusive. \\\\\\\\\\
				\item ímpares menores que 18. \\\\\\\\\\\\
				\item pares de um único algarismo. \\\\\\\\\\
				\item maiores que zero e menores que 15. \\\\\\\\\\
			\end{enumerate}
			\item Observe a sequência: \\
			\begin{center}
				01234567891011..
			\end{center}
			Qual deverá ser o 31º algarismo a ser escrito? \\\\\\\\\\
		\end{enumerate}
		$~$ \\ $~$ \\ $~$ \\ $~$ \\ $~$ \\ $~$ \\ $~$ \\ $~$ \\ $~$ \\ $~$ \\ $~$ \\ $~$ \\ $~$ \\ $~$ \\ $~$ \\
	\end{multicols}
\end{document}