\documentclass[a4paper,14pt]{article}
\usepackage{float}
\usepackage{extsizes}
\usepackage{amsmath}
\usepackage{amssymb}
\everymath{\displaystyle}
\usepackage{geometry}
\usepackage{fancyhdr}
\usepackage{multicol}
\usepackage{graphicx}
\usepackage[brazil]{babel}
\usepackage[shortlabels]{enumitem}
\usepackage{cancel}
\usepackage{textcomp}
\usepackage{array} % Para melhor formatação de tabelas
\usepackage{longtable}
\usepackage{booktabs}  % Para linhas horizontais mais bonitas
\usepackage{float}   % Para usar o modificador [H]
\usepackage{caption} % Para usar legendas em tabelas
\usepackage{tcolorbox}

\columnsep=2cm
\hoffset=0cm
\textwidth=8cm
\setlength{\columnseprule}{.1pt}
\setlength{\columnsep}{2cm}
\renewcommand{\headrulewidth}{0pt}
\geometry{top=1in, bottom=1in, left=0.7in, right=0.5in}

\pagestyle{fancy}
\fancyhf{}
\fancyfoot[C]{\thepage}

\begin{document}
	
	\noindent\textbf{6FMA45 - Matemática} 
	
	\begin{center}O algoritmo da multiplicação (Versão estudante)
	\end{center}
	
	\noindent\textbf{Nome:} \underline{\hspace{10cm}}
	\noindent\textbf{Data:} \underline{\hspace{4cm}}
	
	%\section*{Questões de Matemática}
	\begin{multicols}{2}
		\noindent Nesta aula, praticamos o algoritmo da multiplicação. Para entender seu funcionamento vamos checar, por exemplo, a multiplicação $16 \times 7$. \\
		$16 \times 7 = 7 \times 16 = 7 \times (10 + 6) = 7 \times 10 + 7 \times 6 = 70 + 42 = 70 + 40 + 2 = 110 + 2 = 100 + 10 + 2$
		\begin{itemize}
			\item 16 unidades é o mesmo que 1 dezena e 6 unidades.
			\item 1 dezena multiplicada por 7 é igual a 7 dezenas e 6 unidades multiplicadas por 7 resulta em 42 unidades, ou seja, 4 dezenas e 2 unidades.
			\item Temos agora 7 + 4 = 11 dezenas e 2 unidades, que é igual a 110 + 2 = 112.\\
			Na prática:
		\end{itemize}
	\textsubscript{---------------------------------------------------------------------}
    	\begin{enumerate}
    		\item Complete as contas a seguir.
    		\begin{enumerate}[a)]
    		\item \[
    		\begin{array}{ccc}
    			~ & 1 & 7 \\
    			\times & ~ & 8 \\
    			\hline
    		\end{array}
    		\] \\\\\\
    		\item \[
    		\begin{array}{ccc}
    			~ & 4 & 5 \\
    			\times & ~ & 6 \\
    			\hline
    		\end{array}
    		\] \\
    		\item \[
    		\begin{array}{ccc}
    			~ & 7 & 9 \\
    			\times & ~ & 8 \\
    			\hline
    		\end{array}
    		\] \\\\\\
    		\item \[
    		\begin{array}{cccc}
    			~ & 4 & 2 & 1 \\
    			\times & ~ & ~ & 7 \\
    			\hline
    		\end{array}
    		\] \\\\\\
    		\item \[
    		\begin{array}{ccc}
    			~ & 5 & 2 \\
    			\times & 2 & 1 \\
    			\hline
    		\end{array}
    		\] \\\\\\
    		\item \[
    		\begin{array}{ccc}
    			~ & 7 & 6 \\
    			\times & 2 & 8 \\
    			\hline
    		\end{array}
    		\] \\\\\\
    		\item \[
    		\begin{array}{cccc}
    			~ & 3 & 2 & 1 \\
    			\times & ~ & 4 & 3 \\
    			\hline
    		\end{array}
    		\] \\\\\\
    		\item \[
    		\begin{array}{ccc}
    			~ & 8 & 6 \\
    			\times & 7 & 9 \\
    			\hline
    		\end{array}
    		\] \\\\\\
    		\item \[
    		\begin{array}{cccc}
    			~ & 2 & 1 & 5 \\
    			\times & ~ & 5 & 3 \\
    			\hline
    		\end{array}
    		\] \\\\\\
    		\end{enumerate}
    		\item Quanto vale 456 multiplicado por 22? \\\\\\\\\\\\\\\\\\\\
    		\item Qual é o valor do produto $66 \cdot 141?$ \\\\\\\\\\\\\\\\\\\\\\\\
    		\item Calcule $34 \times 34 \times 34$. \\\\\\\\\\\\\\\\\\\\\\\\\\\\\\
    		\item Um caminhão transporta 228 caixas com 36 garrafas de refrigerante cada uma. Quantas garrafas são transportadas pelo caminhão? \\\\\\\\\\\\\\
    		\item Durante uma mudança, Júlio carregou 5 caixas de 25 kg cada. Quantos quilogramas ele carregou nessa mudança? \newpage
    		\item Um caminhão carrega 1426 caixas de bombons com 24 bombons cada uma. Quantos bombons são carregados pelo caminhão: \\\\\\\\\\\\\\
    		\item Complete as contas abaixo:
    		\begin{enumerate}[a)]
    			\item \[
    			\begin{array}{ccc}
    				~ & 1 & 4 \\
    				\times & ~ & 7 \\
    				\hline
    			\end{array}
    			\] \\\\\\
    			\item \[
    			\begin{array}{ccc}
    				~ & 4 & 8 \\
    				\times & ~ & 6 \\
    				\hline
    			\end{array}
    			\] \\\\\\
    			\item \[
    			\begin{array}{ccc}
    				~ & 7 & 7 \\
    				\times & ~ & 9 \\
    				\hline
    			\end{array}
    			\] \\\\\\
    			\item \[
    			\begin{array}{cccc}
    				~ & 1 & 5 & 3 \\
    				\times & ~ & ~ & 8 \\
    				\hline
    			\end{array}
    			\] \\\\\\
    			\item \[
    			\begin{array}{ccc}
    				~ & 3 & 7 \\
    				\times & 1 & 5 \\
    				\hline
    			\end{array}
    			\] \\\\\\
    			\item \[
    			\begin{array}{ccc}
    				~ & 8 & 6 \\
    				\times & 2 & 7 \\
    				\hline
    			\end{array}
    			\] \\\\\\
    			\item \[
    			\begin{array}{cccc}
    				~ & 1 & 3 & 4 \\
    				\times & ~ & 6 & 3 \\
    				\hline
    			\end{array}
    			\] \\\\\\
    			\item \[
    			\begin{array}{ccc}
    				~ & 9 & 6 \\
    				\times & 3 & 8 \\
    				\hline
    			\end{array}
    			\] \\\\\\
    			\item \[
    			\begin{array}{cccc}
    				~ & 1 & 8 & 4 \\
    				\times & ~ & 5 & 7 \\
    				\hline
    			\end{array}
    			\] \newpage
    		\end{enumerate}
    		\item Calcule:
    		\begin{enumerate}[a)]
    			\item $128 \cdot 31$ \\\\\\\\\\\\\\\\
    			\item $109 \cdot 36$ \\\\\\\\\\\\\\\\
    			\item $13 \cdot 13 \cdot 13$ \\\\\\\\\\\\\\\\\\\\\\
    			\item $136 \cdot 37 + 46 \cdot 42$ \\\\\\\\\\\\\\
    		\end{enumerate}
    		\item Ana comprou em uma papelaria 5 canetas de 2 reais cada e, em outra, 3 canetas de 4 reais cada.
    		\begin{enumerate}[a)]
    			\item Represente, usando adição e multiplicação, o valor gasto por Ana. \\\\\\\\\\\\\\\\\\\\
    			\item Quanto Ana gastou no total? \\\\\\\\\\\\\\\\\\\\
    		\end{enumerate}
    		\item Gabriela comprou 4 quilogramas de carne que custava 7 reais o quilograma. Depois foi a uma loja e comprou 9 metros de tecido que custava 6 reais o metro e ainda lhe sobraram 38 reais. Quanto Gabriela tinha inicialmente? 
    	\end{enumerate}
    	$~$ \\ $~$
	\end{multicols}
\end{document}