\documentclass[a4paper,14pt]{article}

\usepackage{comment} % Para comentar várias linhas ao mesmo tempo

%matemática
\usepackage{amsmath}
\usepackage{amssymb}

%diagramação
\usepackage{extsizes}
\everymath{\displaystyle}
\usepackage{geometry}
\usepackage{fancyhdr}
\usepackage{multicol}
\usepackage{graphicx}
\usepackage[brazil]{babel}
\usepackage[shortlabels]{enumitem}
\usepackage{cancel}
\usepackage{textcomp}
\usepackage{tcolorbox}

%tabelas
\usepackage{array} % Para melhor formatação de tabelas
\usepackage{longtable}
\usepackage{booktabs}  % Para linhas horizontais mais bonitas
\usepackage{float}   % Para usar o modificador [H]
\usepackage{caption} % Para usar legendas em tabelas
\usepackage{wrapfig} % Para usar tabelas e figuras flutuantes
\usepackage{xcolor} % Para cores do fundo de tabelas
\usepackage{colortbl} % Para cores do fundo de tabelas

%tikzpicture
\begin{comment}
	\usepackage{tikz}
	\usepackage{scalerel}
	\usepackage{pict2e}
	\usepackage{tkz-euclide}
	\usetikzlibrary{calc}
	\usetikzlibrary{patterns,arrows.meta}
	\usetikzlibrary{shadows}
	\usetikzlibrary{external}
\end{comment}


%pgfplots
\usepackage{pgfplots}
\pgfplotsset{compat=newest}
\usepgfplotslibrary{statistics}
\usepgfplotslibrary{fillbetween}

%colours
\usepackage{xcolor}



\columnsep=2cm
\hoffset=0cm
\textwidth=8cm
\setlength{\columnseprule}{.1pt}
\setlength{\columnsep}{2cm}
\renewcommand{\headrulewidth}{0pt}
\geometry{top=1in, bottom=1in, left=0.7in, right=0.5in}

\pagestyle{fancy}
\fancyhf{}
\fancyfoot[C]{\thepage}

\begin{document}
	
	\noindent\textbf{6FMA136 - Matemática} 
	
	\begin{center}Propriedades da divisibilidade (Versão estudante)
	\end{center}
	
	\noindent\textbf{Nome:} \underline{\hspace{10cm}}
	\noindent\textbf{Data:} \underline{\hspace{4cm}}
	
	%\section*{Questões de Matemática}
	
	\begin{multicols}{2}
	    \noindent Para $a, b, c$ e $d$ inteiros, temos: \\
	    \textbf{D1.} $ a~|~a$ \\
	    \textbf{D2.} $ a~|~b$ e $b~|~c \Rightarrow a~|~c$ \\
	    \textbf{D3.} $ a~|~b$ e $a~|~c \Rightarrow a~|~b + c$ \\
	    \textbf{D4.} $ a~|~b$ e $a~|~c \Rightarrow a~|~b - c$ \\
	    \textbf{D5.} $ a~|~b$ e $c~|~d \Rightarrow ac~|~bd$ \\
	    \textbf{D6.} $ a~|~b \Rightarrow a~|~b \cdot c$ \\
	    \textbf{D7.} $ a~|~b \Rightarrow ac~|~bc$ \\
	    \textbf{D8.} $ a~|~b$ e $b~|~a \Rightarrow a = b$~ou~$a = -b$ \\
		\noindent\textsubscript{--------------------------------------------------------------------------}
		\begin{enumerate} 
			\item Dê um exemplo para as propriedades de D1 a D7. \\\\\\\\\\\\\\\\\\\\\\\\
			\item Verifique se é verdade que: \\
			$a~|~c$ e $c~|~b \Rightarrow a~|~b + c$ \\\\\\\\\\\\\\\\
			\item Demonstrar D3. \\\\\\\\\\\\\\\\\\\\\\\\\\\\\\\\
			\item Demonstrar D4. \newpage
			%49 a 53
			\item Demonstre D5. \\\\\\\\\\\\\\\\\\\\\\\\\\
			\item Demonstre D6. \\\\\\\\\\\\\\\\\\\\\\\\\\
			\item Demonstre D7. \\\\\\\\\\\\\\\\\\\\\\
			\item Demonstre D8. \\\\\\\\\\\\\\\\\\\\\\\\
			\item Verifique se é verdade que: \\
			$a~|~b$ e $c~|~d$ e $b~|~c \Rightarrow a^2~|~bd$ \\\\\\\\\\\\\\\\\\\\\\\\
		\end{enumerate}
		$~$ \\ $~$ \\ $~$ \\ $~$ \\ $~$ \\ $~$ \\ $~$ \\ $~$ \\ $~$ \\ $~$ \\ $~$ \\ $~$ \\ $~$ \\ $~$
	\end{multicols}
\end{document}