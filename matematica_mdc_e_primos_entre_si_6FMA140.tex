\documentclass[a4paper,14pt]{article}

\usepackage{comment} % Para comentar várias linhas ao mesmo tempo

%matemática
\usepackage{amsmath}
\usepackage{amssymb}

%diagramação
\usepackage{extsizes}
\everymath{\displaystyle}
\usepackage{geometry}
\usepackage{fancyhdr}
\usepackage{multicol}
\usepackage{graphicx}
\usepackage[brazil]{babel}
\usepackage[shortlabels]{enumitem}
\usepackage{cancel}
\usepackage{textcomp}
\usepackage{tcolorbox}

%tabelas
\usepackage{array} % Para melhor formatação de tabelas
\usepackage{longtable}
\usepackage{booktabs}  % Para linhas horizontais mais bonitas
\usepackage{float}   % Para usar o modificador [H]
\usepackage{caption} % Para usar legendas em tabelas
\usepackage{wrapfig} % Para usar tabelas e figuras flutuantes
\usepackage{xcolor} % Para cores do fundo de tabelas
\usepackage{colortbl} % Para cores do fundo de tabelas

%tikzpicture
\begin{comment}
	\usepackage{tikz}
	\usepackage{scalerel}
	\usepackage{pict2e}
	\usepackage{tkz-euclide}
	\usetikzlibrary{calc}
	\usetikzlibrary{patterns,arrows.meta}
	\usetikzlibrary{shadows}
	\usetikzlibrary{external}
\end{comment}


%pgfplots
\usepackage{pgfplots}
\pgfplotsset{compat=newest}
\usepgfplotslibrary{statistics}
\usepgfplotslibrary{fillbetween}

%colours
\usepackage{xcolor}



\columnsep=2cm
\hoffset=0cm
\textwidth=8cm
\setlength{\columnseprule}{.1pt}
\setlength{\columnsep}{2cm}
\renewcommand{\headrulewidth}{0pt}
\geometry{top=1in, bottom=1in, left=0.7in, right=0.5in}

\pagestyle{fancy}
\fancyhf{}
\fancyfoot[C]{\thepage}

\begin{document}
	
	\noindent\textbf{6FMA140 - Matemática} 
	
	\begin{center}mdc e primos entre si (Versão estudante)
	\end{center}
	
	\noindent\textbf{Nome:} \underline{\hspace{10cm}}
	\noindent\textbf{Data:} \underline{\hspace{4cm}}
	
	%\section*{Questões de Matemática}
	
	\begin{multicols}{2}
	    \noindent
	    \begin{itemize}
	    	\item Para $a, b, c \in \mathbb{Z}, a \neq 0$ ou $b \neq 0$: \\
	    	mdc (a, b, c) = mdc (mdc $(a, b), c$)
	    	\item Dois números inteiros são primos entre si (ou relativamente primos) se, e somente se, seu mdc é 1.
	    \end{itemize}
		\noindent\textsubscript{--------------------------------------------------------------------------}
		\begin{enumerate} 
			\item Calcular, usando a definição, o mdc de:
			\begin{enumerate}[a)]
				\item 3, 4 e 7. \\\\\\\\\\\\\\\\\\\\
				\item 12, -18 e 24. \\\\\\\\\\\\\\\\\\\\
				\item 12, -18, 24 e -8. \\\\\\\\\\\\\\\\\\\\
				\item 6, 6 e 6. \\\\\\\\\\\\\\\\\\\\
			\end{enumerate}
			\item Determine 3 números cujo mdc seja 7. \\\\\\\\\\\\\\\\\\\\
			\item Assinale \textbf{V} (verdadeiro) ou \textbf{F} (falso).
			\begin{enumerate}[a)]
				\item (~~) 2 e 7 são primos entre si.
				\item (~~) -2, 9 e 18 são relativamente primos.
				\item (~~) 13, 52 e -13 são primos entre si.
				\item (~~) -3, -1, 0, 1, 2 e 5 são relativamente primos.
				\item (~~) 5, 9 e 45 são primos entre si.
			\end{enumerate}
			\item Determine:
			\begin{enumerate}[a)]
				\item 3 números cujo mdc é 9. \\\\\\\\\\\\\\\\\\\\
				\item 2 números consecutivos primos entre si. \\\\\\\\\\\\\\\\\\\\\\\\
				\item 2 números compostos consecutivos primos entre si. \\\\\\\\\\\\\\\\\\\\
				\item 3 números primos entre si.
			\end{enumerate}
			%68 e 69
			\item Assinale os itens que apresentam números primos entre si:
			\begin{enumerate}[a)]
				\item 6 e 9. \\\\\\\\\\
				\item -5 e 12. \\\\\\\\\\
				\item 2, 6 e 8. \\\\\\\\\\
				\item 2, 7 e -13. \\\\\\\\\\
				\item 5, -12 e 18. \\\\\\\\\\
				\item -6, 4 e 21. \\\\\\\\\\
			\end{enumerate}
			\item Explique por que todos os pares de números consecutivos são também primos entre si. \\\\\\\\\\\\\\\\\\\\
		\end{enumerate}
		$~$ \\ $~$ \\ $~$ \\ $~$ \\ $~$ \\ $~$ \\ $~$ \\ $~$ \\ $~$ \\ $~$ \\ $~$ \\ $~$ \\ $~$ \\ $~$ \\ $~$ \\ $~$ \\ $~$ \\ $~$ \\ $~$ \\ $~$ \\ $~$ \\ $~$ \\ $~$ \\ $~$ \\ $~$ \\ $~$ \\ $~$ \\ $~$ \\ $~$ \\ $~$ \\ $~$ \\ $~$ \\ $~$ \\ $~$ \\ $~$ \\ $~$ \\ $~$ \\ $~$ \\ $~$ \\ $~$ \\ $~$ \\ $~$ \\ $~$ \\ $~$ \\ $~$ \\ $~$ \\ $~$ \\ $~$ \\ $~$ \\ $~$ \\ $~$ \\ $~$ \\ $~$ \\ $~$ \\ $~$ \\ $~$ \\ $~$
	\end{multicols}
\end{document}