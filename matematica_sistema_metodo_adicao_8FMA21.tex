\documentclass[a4paper,14pt]{article}
\usepackage{extsizes}
\usepackage{amsmath}
\usepackage{amssymb}
\everymath{\displaystyle}
\usepackage{geometry}
\usepackage{fancyhdr}
\usepackage{multicol}
\usepackage{graphicx}
\usepackage[brazil]{babel}
\usepackage[shortlabels]{enumitem}
\usepackage{cancel}
\columnsep=2cm
\hoffset=0cm
\textwidth=8cm
\setlength{\columnseprule}{.1pt}
\setlength{\columnsep}{2cm}
\renewcommand{\headrulewidth}{0pt}
\geometry{top=1in, bottom=1in, left=0.7in, right=0.5in}

\pagestyle{fancy}
\fancyhf{}
\fancyfoot[C]{\thepage}

\begin{document}
	
	\noindent\textbf{8FMA21~-~Matemática} 
	
	\begin{center}Sistema de equações: Método da adição (Versão estudante)
	\end{center}
	
	
	\noindent\textbf{Nome:} \underline{\hspace{10cm}}
	\noindent\textbf{Data:} \underline{\hspace{4cm}}
	
	%\section*{Questões de Matemática}
	
	\begin{multicols}{2}
	    \begin{enumerate}
	    	\item Sendo $U = \mathbb{R}^2$, resolva os sistemas a seguir pelo método da adição.
	    	\begin{enumerate}[a)]
	    		\item $\begin{cases}
		    		$x + y = 6$ \\
		    		$x - y = 4$	
		    	\end{cases}$ \\\\\\\\\\\\\\\\\\\\
	    		\item $\begin{cases}
	    			$x - y = -2$ \\
	    			$3x + 3y = 8$	
	    		\end{cases}$ \\\\\\\\\\\\\\\\\\\\
	    		\item $\begin{cases}
	    			$3x + 4y = 3$ \\
	    			$7x + 5y = -8$	
	    		\end{cases}$ \\\\\\\\\\\\\\\\\\\\
    			\item $\begin{cases}
    				x + y^2 = 1 \\
    				6x - 3y = 6	
    			\end{cases}$ \\\\\\\\\\\\\\\\\\\\
    			\item $\begin{cases}
    				\frac{1}{x} + \frac{1}{y} = -6 \\\\
    				\frac{1}{x} - \frac{1}{y} = 5	
    			\end{cases}$ \\\\\\\\\\\\\\\\\\\\
			\end{enumerate}
			Desafio olímpico \\
			Um grupo de 14 amigos comprou 8 pizzas. Eles comeram todas as pizzas, sem sobrar nada. Se cada menino comeu uma pizza inteira e cada menina comeu meia pizza, quantas meninas havia no grupo?
			\begin{enumerate}[a)]
				\item 4
				\item 6
				\item 8
				\item 10
				\item 12 \\\\\\\\\\\\\\\\
			\end{enumerate}
			\item Resolva os sistemas a seguir pelo método da adição.
			\begin{enumerate}[a)]
				\item $\begin{cases}
					x + y = 4 \\
					x - y = 10	
				\end{cases}$ \\\\\\\\\\\\\\\\\\\\
				\item $\begin{cases}
					4x + 7y = 5 \\
					x - 3y = 1	
				\end{cases}$ \\\\\\\\\\\\\\\\\\\\
				\item $\begin{cases}
					x^2 + y = 2 \\
					-x + 3y = 8	
				\end{cases}$ \\\\\\\\\\\\\\\\\\\\
				\item $\begin{cases}
					\frac{1}{x} + \frac{1}{y} = 6 \\\\
					\frac{3}{x} - \frac{4}{y} = -1
				\end{cases}$ \\\\\\\\\\\\\\\\\\\\
			\end{enumerate}
			\item Os números reais $x$ e $y$ são tais que $\begin{cases}
				x + y = 3 \\
				x - 6y = -2
			\end{cases}$, então $\frac{x}{y}$ é:
			\begin{enumerate}[a)]
				\item 1
				\item $\frac{16}{5}$
				\item 5
				\item $-\frac{7}{15}$
				\item 12
			\end{enumerate}	
   	    \end{enumerate}
    \end{multicols}

\end{document}