\documentclass[a4paper,14pt]{article}
\usepackage{extsizes}
\usepackage{amsmath}
\usepackage{amssymb}
\everymath{\displaystyle}
\usepackage{geometry}
\usepackage{fancyhdr}
\usepackage{multicol}
\usepackage{graphicx}
\usepackage[brazil]{babel}
\usepackage[shortlabels]{enumitem}
\usepackage{cancel}
\columnsep=2cm
\hoffset=0cm
\textwidth=8cm
\setlength{\columnseprule}{.1pt}
\setlength{\columnsep}{2cm}
\renewcommand{\headrulewidth}{0pt}
\geometry{top=1in, bottom=1in, left=0.7in, right=0.5in}

\pagestyle{fancy}
\fancyhf{}
\fancyfoot[C]{\thepage}

\begin{document}
	
	\noindent\textbf{8FMA16~-~Matemática} 
	
	\begin{center}Sistemas envolvendo equações do tipo ax²+by+c (Versão estudante)
	\end{center}
	
	
	\noindent\textbf{Nome:} \underline{\hspace{10cm}}
	\noindent\textbf{Data:} \underline{\hspace{4cm}}
	
	%\section*{Questões de Matemática}
	\begin{multicols}{2}
		\begin{enumerate}
			\item Resolver os seguintes sistemas no universo $U = \mathbb{R}^2$, usando o método da substituição.
			\begin{enumerate}[a)]
				\item $\begin{cases}
					x^2 + 2y = 4 \\
					x - 3y = -2
				\end{cases}$ \\\\\\\\\\\\\\\\\\\\
			    \item $\begin{cases}
			    	6a + 3b = 12 \\
			    	a^2 + 4b = 4
			    \end{cases}$ \\\\\\\\\\\\\\\\\\\\
		        \item $\begin{cases}
		        	5u^2 + 8v = 10 \\
		        	u - 4v = -6
		        \end{cases}$ \\\\\\\\\\\\
	            \item $\begin{cases}
	            	2y^2 + 5y = 17 \\
	            	x - 3y = -8
	            \end{cases}$ \\\\\\\\\\\\\\\\\\\\\\\\\\\\
                \item $\begin{cases}
                	5y^2 + 6y = 6 \\
                	8x - 6y = -1
                \end{cases}$ \\\\\\\\\\\\\\\\\\\\
                \item $\begin{cases}
                	u + v = 1 \\
                	u^2 + 2v = 5
                \end{cases}$ \\\\\\\\\\\\
			\end{enumerate}
		    \item Resolva os seguintes sistemas ($U = \mathbb{R}^2$):
		    \begin{enumerate}[a)]
		    	\item $\begin{cases}
		    		2x^2 - y = 0 \\
		    		2x + 4y = 2
		    	\end{cases}$ \\\\\\\\\\\\\\\\\\\\
	    	    \item $\begin{cases}
	    	    	-4x + 3y = 0 \\
	    	    	x - 2y^2 = 8
	    	    \end{cases}$ \\\\\\\\\\\\\\\\\\\\
    	        \item $\begin{cases}
    	        	2u^2 - 7v = 1 \\
    	        	u + 4v = 2
    	        \end{cases}$ \\\\\\\\\\\\\\\\\\\\\\\\
                \item $\begin{cases}
                	4x - 3y^2 = 0 \\
                	3x + 2y^2 = 9
                \end{cases}$ \\\\\\\\\\\\\\\\\\\\
                \item $\begin{cases}
                	-x^2 + y = 2 \\
                	3x - 2y = -4
                \end{cases}$ \\\\\\\\\\\\\\\\\\\\
                \item $\begin{cases}
                	-7u + 4v = 1 \\
                	2u - 3v^2 = -1
                \end{cases}$ \\\\\\\\\\\\\\\\\\\\\\\\\\\\\\
	    	\end{enumerate}
    	    \item Se ($x; y \in \mathbb{R}^* \times \mathbb{R}^*$ é uma solução do sistema)\\
    	    $\begin{cases}
    	    	(x - 2)^2 + (y - 3) = 4 \\
    	    	3(x - 2)^2 + 5(y - 3) = 12
    	    \end{cases}$ \\
            então $x + y$ é igual a: \\
            \begin{enumerate}[a)]
            	\item 1
            	\item 3
            	\item 4
            	\item 6
            	\item 7
            \end{enumerate}
		\end{enumerate}
	$~$ \\ $~$ \\ $~$ \\ $~$ \\ $~$ \\ $~$ \\ $~$ \\ $~$ \\ $~$ \\ $~$ \\ $~$ \\ $~$ \\ $~$ \\ $~$ \\ $~$ \\ $~$ \\ $~$ \\ $~$ \\ $~$ \\ $~$ \\ $~$ \\ $~$ \\ $~$ \\ $~$ \\ $~$ \\ $~$ \\ $~$ \\ $~$ \\ $~$ \\ $~$ \\ $~$ \\ $~$ \\ $~$ \\ $~$ \\ $~$ \\ $~$ \\ $~$ \\ $~$ \\ $~$ \\ $~$ \\ $~$ \\ $~$ \\ $~$ \\ $~$ \\ $~$ \\ $~$ \\ $~$ \\ $~$ \\ $~$ \\ $~$ \\ $~$ \\ $~$ \\ $~$ \\ $~$ \\ $~$ \\ $~$ \\ $~$ \\ $~$ \\ $~$ \\ $~$ \\ $~$ \\ $~$ \\ $~$ \\ $~$ \\ $~$ \\ $~$ \\ $~$ \\ $~$ \\ 
    \end{multicols}

\end{document}