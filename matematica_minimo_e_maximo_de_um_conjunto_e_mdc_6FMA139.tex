\documentclass[a4paper,14pt]{article}

\usepackage{comment} % Para comentar várias linhas ao mesmo tempo

%matemática
\usepackage{amsmath}
\usepackage{amssymb}

%diagramação
\usepackage{extsizes}
\everymath{\displaystyle}
\usepackage{geometry}
\usepackage{fancyhdr}
\usepackage{multicol}
\usepackage{graphicx}
\usepackage[brazil]{babel}
\usepackage[shortlabels]{enumitem}
\usepackage{cancel}
\usepackage{textcomp}
\usepackage{tcolorbox}

%tabelas
\usepackage{array} % Para melhor formatação de tabelas
\usepackage{longtable}
\usepackage{booktabs}  % Para linhas horizontais mais bonitas
\usepackage{float}   % Para usar o modificador [H]
\usepackage{caption} % Para usar legendas em tabelas
\usepackage{wrapfig} % Para usar tabelas e figuras flutuantes
\usepackage{xcolor} % Para cores do fundo de tabelas
\usepackage{colortbl} % Para cores do fundo de tabelas

%tikzpicture
\begin{comment}
	\usepackage{tikz}
	\usepackage{scalerel}
	\usepackage{pict2e}
	\usepackage{tkz-euclide}
	\usetikzlibrary{calc}
	\usetikzlibrary{patterns,arrows.meta}
	\usetikzlibrary{shadows}
	\usetikzlibrary{external}
\end{comment}


%pgfplots
\usepackage{pgfplots}
\pgfplotsset{compat=newest}
\usepgfplotslibrary{statistics}
\usepgfplotslibrary{fillbetween}

%colours
\usepackage{xcolor}



\columnsep=2cm
\hoffset=0cm
\textwidth=8cm
\setlength{\columnseprule}{.1pt}
\setlength{\columnsep}{2cm}
\renewcommand{\headrulewidth}{0pt}
\geometry{top=1in, bottom=1in, left=0.7in, right=0.5in}

\pagestyle{fancy}
\fancyhf{}
\fancyfoot[C]{\thepage}

\begin{document}
	
	\noindent\textbf{6FMA139 - Matemática} 
	
	\begin{center}Mínimo e máximo de um conjunto e mdc (Versão estudante)
	\end{center}
	
	\noindent\textbf{Nome:} \underline{\hspace{10cm}}
	\noindent\textbf{Data:} \underline{\hspace{4cm}}
	
	%\section*{Questões de Matemática}
	
	\begin{multicols}{2}
	    \noindent Dado um conjunto $A$, temos:
	    \begin{itemize}
	    	\item mínimo de $A: x \in A$ tal que $x \leq y$, para todo $y \in A$.
	    	\item máximo de $A: x \in A$ tal que $x \geq y$, para todo $y \in A$.
	    \end{itemize}
	    \noindent Para $a, b \in Z, a \neq 0$ ou $b \neq 0$:
	    \begin{itemize}
	    	\item mdc $(a, b)$ = máx. $(D_+(a) \cap D_+(b))$
	    	\item mdc $(a, b)$ = mdc $(b, a)$ = mdc $(|a|,|b|)$
	    \end{itemize}
		\noindent\textsubscript{--------------------------------------------------------------------------}
		\begin{enumerate} 
			\item Determine o que se pede:
			\begin{enumerate}[a)]
				\item min. \{1, 3, 4, 6\} \\\\\\\\\\
				\item máx. \{1, 3, 4, 6\} \\\\\\\\\\
				\item mín. \{-1\} \\\\\\\\\\
				\item máx. \{-1\} \\\\\\\\\\
				\item mín. $\mathbb{N}$ \\\\\\\\\\
				\item máx. $\mathbb{N}$ \\\\\\\\\\
				\item mín. $\mathbb{Z}^*$ \\\\\\\\\\
				\item máx. $\mathbb{Z}^*$ \\\\\\\\\\
			\end{enumerate}
			\item Calcular, usando a definição, o mdc de:
			\begin{enumerate}[a)]
				\item 2 e 5. \\\\\\\\\\\\\\
				\item 2 e 6. \\\\\\\\\\\\\\
				\item -3 e 10. \\\\\\\\\\\\\\
				\item -6 e 3. \\\\\\\\\\\\\\
				\item 16 e 36. \\\\\\\\\\\\\\
			\end{enumerate}
			\item Refaça os itens de $a$ e $e$ do exercício anterior usando o diagrama de Venn. \\\\\\\\\\\\\\\\\\\\\\\\
			\item Escreva o que se pede:
			\begin{enumerate}[a)]
				\item Dois conjuntos com o mesmo máximo. \\\\\\\\\\\\\\
				\item Dois conjuntos com o mesmo mínimo. \newpage
				\item Dois conjuntos cujo máximo de um seja o mínimo do outro. \\\\\\\\\\\\\\
				\item Dois números cujo mdc seja 3. \\\\\\\\\\\\\\
				\item Dois números cujo mdc seja 15. \\\\\\\\\\\\\\
			\end{enumerate}
			%62 a 67
			\item Encontre o mdc dos números a seguir utilizando o diagrama de Venn:
			\begin{enumerate}[a)]
				\item 12 e 18. \\\\\\\\\\\\\\
				\item -4 e 10. \\\\\\\\\\\\\\
				\item 10 e 11. \\\\\\\\\\\\\\
				\item 13 e 39. \\\\\\\\\\\\\\
				\item -3 e -12. \\\\\\\\\\\\\\
			\end{enumerate}
			\item Determine os máximos e os mínimos, se existirem, dos conjuntos a seguir:
			\begin{enumerate}[a)]
				\item \{-3, -2, 0, 2, 3\} \\\\\\
				\item \{0\} \\\\\\
				\item \{..., -9, -7, -5\} \\\\\\
				\item \{3, 4, 5, ...\} \\\\\\
				\item $\mathbb{N}$ \\\\\\
				\item $\mathbb{Z}_+^*$ \\\\\\
				\item $\mathbb{Z}_-$ \\\\\\
				\item $D_+(15)$ \\\\\\
			\end{enumerate}
			\item Calcular, usando a definição, o mdc de:
			\begin{enumerate}[a)]
				\item 6 e 8. \\\\\\\\\\\\
				\item 12 e 28. \\\\\\\\\\\\
				\item -16 e 72. \\\\\\\\\\\\
				\item 6, 9 e 21. \\\\\\\\\\\\
				\item -12, 32 e 40. \\\\\\\\\\\\
			\end{enumerate}
			\item Determine:
			\begin{enumerate}[a)]
				\item dois conjuntos com o mesmo máximo. \newpage
				\item dois conjuntos com o mesmo mínimo. \\\\\\\\\\\\
				\item dois números cujo mdc é 7. \\\\\\\\\\\\
			\end{enumerate}
			\item Determine o mdc de 4, -10 e 22 utilizando o diagrama de Venn.
		\end{enumerate}
		$~$ \\ $~$ \\ $~$ \\ $~$ \\ $~$ \\ $~$ \\ $~$ \\ $~$ \\ $~$ \\ $~$ \\ $~$ \\ $~$ \\ $~$ \\ $~$ \\ $~$ \\ $~$ \\ $~$ \\ $~$ \\ $~$ \\ $~$ \\ $~$ \\ $~$ \\ $~$ \\ $~$ \\ $~$ \\ $~$ \\ $~$ \\ $~$ \\ $~$ \\ $~$ \\ $~$ \\ $~$ \\ $~$ \\ $~$ \\ $~$ \\ $~$ \\ $~$ \\ $~$ \\ $~$ \\ $~$ \\ $~$ \\ $~$ \\ $~$ \\ $~$ \\ $~$ \\ $~$ \\ $~$ \\ $~$ \\ $~$ \\ $~$ \\ $~$ \\ $~$ \\ $~$ \\ $~$ \\ $~$ \\ $~$ \\ $~$ \\ $~$ \\ $~$ \\ $~$ \\ $~$ \\ $~$ \\ $~$ \\ $~$ \\ $~$
	\end{multicols}
\end{document}