\documentclass[a4paper,14pt]{article}
\usepackage{extsizes}
\usepackage{amsmath}
\usepackage{amssymb}
\everymath{\displaystyle}
\usepackage{geometry}
\usepackage{fancyhdr}
\usepackage{multicol}
\usepackage{graphicx}
\usepackage[brazil]{babel}
\usepackage[shortlabels]{enumitem}
\usepackage{cancel}
\columnsep=2cm
\hoffset=0cm
\textwidth=8cm
\setlength{\columnseprule}{.1pt}
\setlength{\columnsep}{2cm}
\renewcommand{\headrulewidth}{0pt}
\geometry{top=1in, bottom=1in, left=0.7in, right=0.5in}

\pagestyle{fancy}
\fancyhf{}
\fancyfoot[C]{\thepage}

\begin{document}
	
	\noindent\textbf{EF09MA03-A~-~Matemática} 
	
	\begin{center}
		\textbf{Revisão: Potências de expoentes fracionários (Versão estudante)}
	\end{center}
	
	
	\noindent\textbf{Nome:} \underline{\hspace{10cm}}
    \noindent\textbf{Data:} \underline{\hspace{4cm}}
	
	%\section*{Questões de Matemática}
	
	\begin{multicols}{2}
		Se $a^\frac{m}{n} \in \mathbb{R}$, então $a^\frac{m}{n} = \sqrt[n]{a^m}$, para $a \geq 0$. Para $a < 0$, você deve verificar o sinal de $a^\frac{n}{m}$ e $\sqrt[n]{a^m}$
	\begin{enumerate}	
		\item Escreva na forma de raiz quando possível:
		\begin{enumerate}[a)]
			\item $4^\frac{1}{3}$ \\
			\item $8^\frac{1}{5}$ \\
			\item $(-7)^\frac{1}{6}$ \\
			\item $11^\frac{1}{4}$ \\
	    \end{enumerate}
        \item Escreva na forma de potência com expoente racional, quando possível:
        \begin{enumerate}[a)]
        	\item $\sqrt[3]{7}$ \\
        	\item $\sqrt{13}$ \\
        	\item $\sqrt[3]{125}$ \\
        	\item $\sqrt[4]{81}$ \\
        	\item $\sqrt[7]{9}$ \\
        \end{enumerate}
        \item Escrever, quando possível, na forma $\sqrt{~}$:
        \begin{enumerate}[a)] 
        	\item $3^\frac{2}{4}$ \\
        	\item $13^\frac{3}{5}$ \\
        	\item $(-6)^\frac{4}{7}$ \\
        	\item $(-9)^\frac{3}{8}$ \\
        \end{enumerate}
        \item Escrever se possível, com expoente racional:
        \begin{enumerate}[a)]
        	\item $\sqrt[7]{5^6}$ \\
        	\item $\sqrt[9]{4^{12}}$ \\
        	\item $\sqrt[5]{(-11)^3}$ \\
        	\item $\sqrt[11]{(-2)^6}$ \\
        \end{enumerate}
        \item Assinale V(verdadeiro) ou F(falso):
        \begin{enumerate}[a)]
        	\item (~~)~$(-1)^\frac{6}{6} = \sqrt[6]{(-1)^6}$ \\
        	\item (~~)~$\sqrt[5]{(-1)^5} = (-1)^\frac{5}{5}$ \\
        	\item (~~)~$(-1)^\frac{4}{4} = (-1)$ \\
        	\item (~~)~$(-1)^\frac{16}{12} = (-1)^\frac{4}{3}$ \\
        	\item (~~)~$\sqrt{x^2} = |x|$~para todo $x \in \mathbb{R}$ \\
        	\item (~~)~$\sqrt[7]{x^7} = x$~para todo $x \in \mathbb{R}$ \\
        \end{enumerate}
    \end{enumerate}        
    \end{multicols}    

\end{document}