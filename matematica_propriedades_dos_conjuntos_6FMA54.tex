\documentclass[a4paper,14pt]{article}
\usepackage{float}
\usepackage{extsizes}
\usepackage{amsmath}
\usepackage{amssymb}
\everymath{\displaystyle}
\usepackage{geometry}
\usepackage{fancyhdr}
\usepackage{multicol}
\usepackage{graphicx}
\usepackage[brazil]{babel}
\usepackage[shortlabels]{enumitem}
\usepackage{cancel}
\usepackage{textcomp}
\usepackage{array} % Para melhor formatação de tabelas
\usepackage{longtable}
\usepackage{booktabs}  % Para linhas horizontais mais bonitas
\usepackage{float}   % Para usar o modificador [H]
\usepackage{caption} % Para usar legendas em tabelas
\usepackage{tcolorbox}

\columnsep=2cm
\hoffset=0cm
\textwidth=8cm
\setlength{\columnseprule}{.1pt}
\setlength{\columnsep}{2cm}
\renewcommand{\headrulewidth}{0pt}
\geometry{top=1in, bottom=1in, left=0.7in, right=0.5in}

\pagestyle{fancy}
\fancyhf{}
\fancyfoot[C]{\thepage}

\begin{document}
	
	\noindent\textbf{6FMA54 - Matemática} 
	
	\begin{center}Propriedades dos conjuntos (Versão estudante)
	\end{center}
	
	\noindent\textbf{Nome:} \underline{\hspace{10cm}}
	\noindent\textbf{Data:} \underline{\hspace{4cm}}
	
	%\section*{Questões de Matemática}
	\begin{multicols}{2}
    		\noindent Os diagramas de Venn são utilizados também para visualizar algumas propriedades de conjuntos. Por exemplo, a propriedade: \\
    		\begin{equation*} A \cap (B \cup C) = (A \cap B) \cup (A \cap C) \\ \end{equation*}
    		\begin{enumerate}[I.]
    			\item Construir $A\cap (B \cup C)$
    			\includegraphics[width=1.1\linewidth]{"6FMA54_imagens/imagem1"}
    			\columnbreak
    			\item Construir $(A \cap B) \cup (A \cap C)$
    			\includegraphics[width=1.1\linewidth]{"6FMA54_imagens/imagem2"}
    			Obtemos a mesma região hachurada para \\
    			$A \cap (B \cup C)$ e para \\ $(A \cap B) \cup (A \cap C)$. \\ Logo, $A \cap (B \cup C) = \\ (A \cap B) \cup (A \cap C)$.
    		\end{enumerate}
    \end{multicols}
    \noindent\textsubscript{-----------------------------------------------------------------------------------------------------------------------------------------------------------}
    \begin{multicols}{2}
    		\begin{enumerate}
    			\item Utilizando diagramas de Venn e o exemplo apresentado, verifique as seguintes igualdades:
    			\begin{enumerate}[a)]
    				\item $(A \cup B) \cup C = A \cup (B \cup C)$
    				\begin{enumerate}[I.]
    					\item $(A \cup B) \cup C$ \\
    					\noindent\includegraphics[width=1.1\linewidth]{"6FMA54_imagens/imagem3"} \\
    					\item $A \cup (B \cup C)$ \\
    					\noindent\includegraphics[width=1.1\linewidth]{"6FMA54_imagens/imagem4"} \newpage
    				\end{enumerate}
    				\item $(B \cap C) \cap (A \cap C) = (A \cap B) \cap C$
    				\begin{enumerate}[I.]
    					\item $(B \cap C) \cap (A \cap C)$ \\
    					\noindent\includegraphics[width=1.1\linewidth]{"6FMA54_imagens/imagem5"} \\
    					\item $(A \cap B) \cap C$ \\
    					\noindent\includegraphics[width=1.1\linewidth]{"6FMA54_imagens/imagem6"}
    				\end{enumerate}
    				\item $A \cup (B \cap C) = (A \cup B) \cap (A \cup C)$
    				\begin{enumerate}[I.]
    					\item $A \cup (B \cap C)$ \\
    					\includegraphics[width=1.1\linewidth]{"6FMA54_imagens/imagem7"}
    					\item $(A \cup B) \cap (A \cup C)$ \\
    					\noindent\includegraphics[width=1.1\linewidth]{"6FMA54_imagens/imagem8"}
    				\end{enumerate}
    			\end{enumerate}
    			\item Em cada item, determine em que condições os conjuntos $A$ e $B$ satisfazem a igualdade.
    			\begin{enumerate}[a)]
    				\item $A \cap B = B$ \\\\\\\\
    				\item $A \cup B = A$ \\\\\\\\
    				\item $A \cap B = \varnothing$ \\\\\\\\
    				\item $A \cup B = A \cap B$ \\\\\\\\
    			\end{enumerate}
    			\textbf{Desafio olímpico} \\\\
    			\noindent Na figura ao lado há três circunferências que formam sete regiões. Em cada uma delas há um número obtido a partir da soma dos números escritos nas regiões vizinhas. Sabendo-se que duas regiões são vizinhas quando seus limites têm mais de um ponto comum, o valor de $x$ indicado na figura é: \\\\
    			\textbf{a)} -3 ~ \textbf{b)} 4 ~ \textbf{c)} -2 ~ \textbf{d)} 3 ~ \textbf{e)} -1 \newpage
    			\item Utilizando diagramas de Venn, verifique se as igualdades são verdadeiras ou falsas.
    			\begin{enumerate}[a)]
    				\item $A \cup (A \cap B) = A$
    				\item $(A \cup B) \cap C = A \cup (B \cap C)$
    				\item $B \cap (A \cup B) = B$
    				\item $A \cap (B \cap C) = (A \cap B) \cap C$
    				\item $(A \cup B) \cup (B \cap C) = \\ (A \cap B) \cap (B \cup C)$
    				\item $(A \cup B) \cup (B \cup C) = A \cup (B \cup C)$
    			\end{enumerate}
    			\item Para quaisquer conjuntos $A, B$ e $C$ é verdade que $(A \cap B) \cup C = (A \cap C) \cup (B \cap C)$? \\\\\\\\\\\\\\\\\\\\\\\\\\\\\\\\\\\\\\\\\\\\\\\\\\
    			\item 
    			\begin{enumerate}[a)]
    				\item Sabendo que os conjuntos $A$ e $B$ são distintos e que $A \cap B = A$, determine $A \cup B$. \\\\\\\\\\\\\\\\\\\\\\\\\\\\\\\\\\
    				\item Para os conjuntos $A, B$ e $C$ temos $B \subset A$ e $A \cap C = \varnothing$. Determine $A \cap B$, $A \cup B$ e $B \cap C$.
    			\end{enumerate}
    		\end{enumerate}
    		$~$ \\ $~$ \\ $~$ \\ $~$ \\ $~$ \\ $~$ \\ $~$ \\ $~$ \\ $~$ \\ $~$ \\ $~$ \\ $~$ \\ $~$ \\ $~$ \\ $~$ \\ $~$ \\
	\end{multicols}
\end{document}