\documentclass[a4paper,14pt]{article}
\usepackage{float}
\usepackage{extsizes}
\usepackage{amsmath}
\usepackage{amssymb}
\everymath{\displaystyle}
\usepackage{geometry}
\usepackage{fancyhdr}
\usepackage{multicol}
\usepackage{graphicx}
\usepackage[brazil]{babel}
\usepackage[shortlabels]{enumitem}
\usepackage{cancel}
\usepackage{textcomp}
\usepackage{array} % Para melhor formatação de tabelas
\usepackage{longtable}
\usepackage{booktabs}  % Para linhas horizontais mais bonitas
\usepackage{float}   % Para usar o modificador [H]
\usepackage{caption} % Para usar legendas em tabelas
\usepackage{tcolorbox}

\columnsep=2cm
\hoffset=0cm
\textwidth=8cm
\setlength{\columnseprule}{.1pt}
\setlength{\columnsep}{2cm}
\renewcommand{\headrulewidth}{0pt}
\geometry{top=1in, bottom=1in, left=0.7in, right=0.5in}

\pagestyle{fancy}
\fancyhf{}
\fancyfoot[C]{\thepage}

\begin{document}
	
	\noindent\textbf{6FMA102 - Matemática} 
	
	\begin{center}Propriedades da multiplicação e da adição (Versão estudante)
	\end{center}
	
	\noindent\textbf{Nome:} \underline{\hspace{10cm}}
	\noindent\textbf{Data:} \underline{\hspace{4cm}}
	
	%\section*{Questões de Matemática}
	~ \\ ~
	\begin{multicols}{2}
		\noindent Propriedade distributiva:
		\begin{equation*}
			a(b + c) = ab + ac
		\end{equation*}
		\noindent Para colocar uma letra em evidência:
		\begin{equation*}
			ab + ac = a(b + c)
		\end{equation*}
		\textsubscript{---------------------------------------------------------------------}
    	\begin{enumerate}
    		\item Nas expressões a seguir, as letras representam números inteiros. Aplicar a propriedade distributiva.
    		\begin{enumerate}[a)]
    			\item $a(2 + 5)$ \\\\\\\\
    			\item $b(a - 3)$ \\\\\\\\
    			\item $x(2 - x)$ \\\\\\\\
    			\item $y(a + b)$ \\\\\\\\
    			\item $y^2(6 - y)$ \\\\\\\\
    			\item $a(8 - 3 + 1)$ \\\\\\\\
    			\item $5(-a + b - c)$ \\\\\\\\
    			\item $3(2 - x + 2x^2)$ \\\\\\\\
    			\item $-4(2x + y - 3z)$ \\\\\\\\
    			\item $-1(6a - 3b - 7c)$ \\\\\\\\
    			\item $6(2 - x)$ \\\\\\\\
    			\item $8(x - 6y)$ \\\\\\\\
    		\end{enumerate}
    		\item Nas expressões a seguir, as letras representam números inteiros. Colocar $x$ em evidência.
    		\begin{enumerate}[a)]
    			\item $3x - 4x$ \\\\\\\\
    			\item $8x + bx$ \\\\\\\\
    			\item $xy - x$ \\\\\\\\
    			\item $x^2 - ax$ \\\\\\\\
    			\item $xy^4 + x^3$ \\\\\\\\
    			\item $7x^3 - 4x + 2x^2$ \\\\\\\\
    			\item $x^3 - 4xy + xyz$ \\\\\\\\
    			\item $-x^2 + 3xy - x$ \\\\\\\\
    			\item $-2ax^3 - 3bx$ \\\\\\\\
    			\item $8ax^2 - x^3 + 5x$ \\\\\\\\
    			\item $-6x^4 - 3x^2 + x$ \\\\\\\\
    			\item $3a^2x + 2bx^2 - x$ \newpage
    		\end{enumerate}
    		\item Nas expressões a seguir, as letras representam números inteiros. Aplicar a propriedade distributiva.
    		\begin{enumerate}[a)]
    			\item $x(a + 3)$ \\\\\\\\
    			\item $a(b - 1)$ \\\\\\\\
    			\item $x(x + 4)$ \\\\\\\\
    			\item $y(2 + a - b)$ \\\\\\\\
    			\item $y(y^2 - 6y + 4)$ \\\\\\\\
    			\item $a(x - 3y + 7z)$ \\\\\\\\
    			\item $-8(a - 1)$ \\\\\\\\
    			\item $-6(x + 5)$ \\\\\\\\
    			\item $3(8 - y)$ \\\\\\\\
    			\item $-1(3x - x^2)$ \\\\\\\\
    			\item $-6(x - 5x)$ \\\\\\\\
    			\item $2(3x^2 - 5x + 4)$ \\\\\\\\
    		\end{enumerate}
    		\item Nas expressões a seguir, as letras representam números inteiros. Colocar $x$ em evidência.
    		\begin{enumerate}[a)]
    			\item $2x + ax$ \\\\\\\\
    			\item $ax - 3x$ \\\\\\\\
    			\item $ax^2 + bx$ \\\\\\\\
    			\item $xy^2 - x^4$ \\\\\\\\
    			\item $x^3 - ax$ \\\\\\\\
    			\item $9x + ax - bx$ \\\\\\\\
    			\item $5ax - 8x^2 - x^3$ \\\\\\\\
    			\item $-4x^2 - 9xy + 7x$ \\\\\\\\
    			\item $2ax - bx^3 + 3cx^4$ \\\\\\\\
    			\item $9abx^2 - 5cx - 8 dx^3$ \\\\
    			\item $6x^2y - 5y^2x - 2x^3$ \\\\\\\\
    			\item $7a^2b^4cx^3 - 2a^2bx^2 + 5c^2x$ \\\\\\\\
    		\end{enumerate}
    		\item Usando as propriedades da adição e da multiplicação de inteiros, mostre que $8x - x = 7x$ \\\\\\\\\\\\\\\\\\\\
    		\item Mostre que $(a + b)(c + d) = ac + ad + bc + bd$ \\\\\\\\\\\\\\
    	\end{enumerate}
    $~$ \\ $~$ \\ $~$ \\ $~$ \\ $~$ \\ $~$ \\ $~$ \\ $~$ \\ $~$ \\ $~$
	\end{multicols}
\end{document}