\documentclass[a4paper,14pt]{article}
\usepackage{float}
\usepackage{extsizes}
\usepackage{amsmath}
\usepackage{amssymb}
\everymath{\displaystyle}
\usepackage{geometry}
\usepackage{fancyhdr}
\usepackage{multicol}
\usepackage{graphicx}
\usepackage[brazil]{babel}
\usepackage[shortlabels]{enumitem}
\usepackage{cancel}
\usepackage{textcomp}
\usepackage{array} % Para melhor formatação de tabelas
\usepackage{longtable}
\usepackage{booktabs}  % Para linhas horizontais mais bonitas
\usepackage{float}   % Para usar o modificador [H]
\usepackage{caption} % Para usar legendas em tabelas
\usepackage{tcolorbox}
\usepackage{wrapfig} % Para usar tabelas e figuras flutuantes

\columnsep=2cm
\hoffset=0cm
\textwidth=8cm
\setlength{\columnseprule}{.1pt}
\setlength{\columnsep}{2cm}
\renewcommand{\headrulewidth}{0pt}
\geometry{top=1in, bottom=1in, left=0.7in, right=0.5in}

\pagestyle{fancy}
\fancyhf{}
\fancyfoot[C]{\thepage}

\begin{document}
	
	\noindent\textbf{6FMA60 - Matemática} 
	
	\begin{center}Propriedades dos inteiros (Versão estudante)
	\end{center}
	
	\noindent\textbf{Nome:} \underline{\hspace{10cm}}
	\noindent\textbf{Data:} \underline{\hspace{4cm}}
	
	%\section*{Questões de Matemática}
	\begin{multicols}{2}
    		\noindent Sendo $a$ e $b$ inteiros, também vale $(-a) + (-b) = -(a + b)$. \\
    		Exemplo: \\
    		(-3) + (-5) = -3( + 5) = -8
    		Se $x, y$ e $z$ são inteiros e $x = y$, então $x + z = y + z$.	\\
    		Exemplo: \\
    		2 = 2, então 2 + 3 = 2 + 3. \\
    		Para todo $x, y$ e $z$ inteiros, se $x > y$, então $x + z > y + z$. \\
    		Exemplo: \\
    		$3 > 2$, então $\underbrace{3 + 5}_\text{8} > \underbrace{2 + 5}_\text{7}$.  \\\\
    		\textsubscript{---------------------------------------------------------------------}
    		\begin{enumerate}
    			\item Calcular $a + b$, sendo:
    			\begin{enumerate}[a)]
    				\item $a = 4$ e $b = -9$. \\\\\\\\
    				\item $a = -3$ e $b = 12$. \\\\\\\\
    				\item $a = -53$ e $b = -8$. \\\\\\\\
    				\item $a = 0$ e $b = -17$. \\\\
    				\item $a = 6$ e $b = 2$. \\\\\\\\
    			\end{enumerate}
    			\item Calcule $(-a) + b$ e $a + (-b)$, sendo:
    			\begin{enumerate}[a)]
    				\item $a = 2$ e $b = 8$. \\\\\\\\
    				\item $a = -3$ e $b = 6$. \\\\\\\\
    				\item $a = 1$ e $b = 0$. \\\\\\\\
    				\item $a = -4$ e $b = -1$ \\\\\\\\
    				\item $a = 5$ e $b = -9$ \\\\\\\\    				
    			\end{enumerate}
    			Você percebe alguma relação entre $(-a) + b$ e $a +(-b)$? \\
    			Descreva usando letras. \\\\\\\\
    			\item Dê três exemplos numéricos para cada uma das propriedades abaixo.
    			\begin{enumerate}[a)]
    				\item "Para tudo $x, y$ e $z$ inteiros, se $x = y$, então $x + z = y + z$." \\\\\\\\
    				\item "Para todo $x, y$ e $z$ inteiros, se $x > y$, então $x + z > y + z$." \\\\\\\\   				
    			\end{enumerate}
    			\item Você viu que para $x, y$ e $z$ inteiros, se $x = y$, então $x + z = y + z$. \\
    			Demonstre agora que se $x + z = y + z$, então $x = y$. \\\\\\\\
    			\item Suponha que $a = 9$, $b = 8$, $c = 6$ e $d = -5$. Insira parênteses na expressão
    			\begin{equation*}a - b - c - d\end{equation*}
    			para obter o maior e o menor valor possível. Quais são esses valores? \\
    			\item Calcule $-(a + b)$ e $(-a) +(-b)$, sendo:
    			\begin{enumerate}[a)]
    				\item $a = 12; b = 3$ \\\\\\\\
    				\item $a = -9; b = -2$ \\\\\\\\
    				\item $a = -6; b = 18$ \\\\\\\\
    				\item $a = 32; b = -54$ \\\\\\\\		
    			\end{enumerate}
    			Há alguma relação entre $-(a + b)$ e $(-a)+(-b)$? Descreva usando letras. \\\\\\\\\\\\
    			\item Sejam $x = 8, y = 5, \ = -4$ e $w = -2$. Insira parênteses na expressão
    			\begin{equation*}x - y - z + w\end{equation*}
    			para obter o maior e o menor valor possível e determine quais são esses valores.
    		\end{enumerate}
	\end{multicols}
\end{document}