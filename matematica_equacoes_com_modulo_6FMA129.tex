\documentclass[a4paper,14pt]{article}

\usepackage{comment} % Para comentar várias linhas ao mesmo tempo

%matemática
\usepackage{amsmath}
\usepackage{amssymb}

%diagramação
\usepackage{extsizes}
\everymath{\displaystyle}
\usepackage{geometry}
\usepackage{fancyhdr}
\usepackage{multicol}
\usepackage{graphicx}
\usepackage[brazil]{babel}
\usepackage[shortlabels]{enumitem}
\usepackage{cancel}
\usepackage{textcomp}
\usepackage{tcolorbox}

%tabelas
\usepackage{array} % Para melhor formatação de tabelas
\usepackage{longtable}
\usepackage{booktabs}  % Para linhas horizontais mais bonitas
\usepackage{float}   % Para usar o modificador [H]
\usepackage{caption} % Para usar legendas em tabelas
\usepackage{wrapfig} % Para usar tabelas e figuras flutuantes
\usepackage{xcolor} % Para cores do fundo de tabelas
\usepackage{colortbl} % Para cores do fundo de tabelas

%tikzpicture
\begin{comment}
	\usepackage{tikz}
	\usepackage{scalerel}
	\usepackage{pict2e}
	\usepackage{tkz-euclide}
	\usetikzlibrary{calc}
	\usetikzlibrary{patterns,arrows.meta}
	\usetikzlibrary{shadows}
	\usetikzlibrary{external}
\end{comment}


%pgfplots
\usepackage{pgfplots}
\pgfplotsset{compat=newest}
\usepgfplotslibrary{statistics}
\usepgfplotslibrary{fillbetween}

%colours
\usepackage{xcolor}



\columnsep=2cm
\hoffset=0cm
\textwidth=8cm
\setlength{\columnseprule}{.1pt}
\setlength{\columnsep}{2cm}
\renewcommand{\headrulewidth}{0pt}
\geometry{top=1in, bottom=1in, left=0.7in, right=0.5in}

\pagestyle{fancy}
\fancyhf{}
\fancyfoot[C]{\thepage}

\begin{document}
	
	\noindent\textbf{6FMA129 - Matemática} 
	
	\begin{center}Equações com módulo (Versão estudante)
	\end{center}
	
	\noindent\textbf{Nome:} \underline{\hspace{10cm}}
	\noindent\textbf{Data:} \underline{\hspace{4cm}}
	
	%\section*{Questões de Matemática}
	
	\begin{multicols}{2}
	    \noindent Para resolver equações com módulo é necessário saber que: \\
	    \begin{itemize}
	    	\item $|x| = \begin{cases}
	    		x, \text{se~} x \geq 0 \\
	    		-x, \text{se~} x \leq 0
	    	\end{cases}$
	    	\item $|x| \geq 0$, para todo $x$ racional.
	    	\item $|x| = a \Leftrightarrow x = a$ ou $x = -a$, para qualquer $a$ racional não negativo.
	    \end{itemize}
		\noindent\textsubscript{--------------------------------------------------------------------------}
		\begin{enumerate} 
			\item Resolver as equações ($U = \mathbb{Q}$).
			\begin{enumerate}[a)]
				\item $|y| = 0$ \\\\\\\\\\\\
				\item $|x| = \frac{2}{3}$ \\\\\\\\\\\\
				\item $|x| = 7$ \\\\\\\\\\\\
				\item $|x| = -8$ \\\\\\\\\\\\
				\item $|x| = \frac{5}{2}$ \\\\\\\\\\\\
				\item $|x| = -1$ \\\\\\\\\\\\
			\end{enumerate}
			\item Resolver as equações ($U = \mathbb{Q}$).
			\begin{enumerate}[a)]
				\item $3|x| + 2 = 9$ \\\\\\\\\\\\
				\item $-2|x| + 11 = 7 + |x|$ \\\\\\\\\\\\
				\item $|6x - 1| = 0$ \\\\\\\\\\\\
				\item $3 \cdot |x + 1| - 5 - 2 \cdot |x + 1| = 0$ \\\\\\\\\\\\
			\end{enumerate}
			%13 a 14
			\item Resolver as equações ($U = \mathbb{Q}$).
			\begin{enumerate}[a)]
				\item $|x| = \frac{4}{3}$ \\\\\\\\\\\\
				\item $|x| = 4$ \\\\\\\\\\\\
				\item $|x| = \frac{-1}{5}$ \\\\\\\\\\\\
				\item $|x| = \frac{3}{8}$ \\\\\\\\\\\\
				\item $|y| = 0$ \\\\\\\\\\\\
			\end{enumerate}
			\item Resolver as equações ($U = \mathbb{Q}$).
			\begin{enumerate}[a)]
				\item $2|y| - 7 = 0$ \\\\\\\\\\\\
				\item $-3|x| + 5 = 1$ \\\\\\\\\\\\
				\item $4|x| + 3 = 8|x|$ \\\\\\\\\\\\
				\item $|3x + 5| = 0$ \\\\\\\\\\\\
				\item $|2y - 7| = 0$ \\\\\\\\\\\\
				\item $|9|x| - 2 = 4|x| + 1$ \\\\\\\\\\\\
				\item $-|4x + 3| + \frac{2}{5} = |4x + 3| - 4$ \\\\\\\\\\\\\\\\\\\\\\
				\item $|x| - 6 - |x| + 4$ \\\\\\\\\\\\
				\item $(3|x - 2| + 5 + |x - 2| - 7)(4|x| - 3) = 0$ \\\\\\\\\\\\
				\item $|y|(|y| - 3) \cdot (2|4y + 3| - 1) = 0$ \\\\\\\\\\\\
				\item $|5x - 2|(|2x + 7| - 6)(2|x - 3| - 5)|x + 8| = 0$ \\\\\\\\\\\\
			\end{enumerate}
		\end{enumerate}
		$~$ \\ $~$ \\ $~$ \\ $~$ \\ $~$ \\ $~$ \\ $~$ \\ $~$ \\ $~$ \\ $~$ \\ $~$ \\ $~$ \\ $~$ \\ $~$ \\ $~$ \\ $~$ \\ $~$ \\ $~$
	\end{multicols}
\end{document}