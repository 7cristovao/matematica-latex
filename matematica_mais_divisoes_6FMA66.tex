\documentclass[a4paper,14pt]{article}
\usepackage{float}
\usepackage{extsizes}
\usepackage{amsmath}
\usepackage{amssymb}
\everymath{\displaystyle}
\usepackage{geometry}
\usepackage{fancyhdr}
\usepackage{multicol}
\usepackage{graphicx}
\usepackage[brazil]{babel}
\usepackage[shortlabels]{enumitem}
\usepackage{cancel}
\usepackage{textcomp}
\usepackage{array} % Para melhor formatação de tabelas
\usepackage{longtable}
\usepackage{booktabs}  % Para linhas horizontais mais bonitas
\usepackage{float}   % Para usar o modificador [H]
\usepackage{caption} % Para usar legendas em tabelas
\usepackage{tcolorbox}
\usepackage{wrapfig} % Para usar tabelas e figuras flutuantes

\columnsep=2cm
\hoffset=0cm
\textwidth=8cm
\setlength{\columnseprule}{.1pt}
\setlength{\columnsep}{2cm}
\renewcommand{\headrulewidth}{0pt}
\geometry{top=1in, bottom=1in, left=0.7in, right=0.5in}

\pagestyle{fancy}
\fancyhf{}
\fancyfoot[C]{\thepage}

\begin{document}
	
	\noindent\textbf{6FMA66 - Matemática} 
	
	\begin{center}Mais divisões (Versão estudante)
	\end{center}
	
	\noindent\textbf{Nome:} \underline{\hspace{10cm}}
	\noindent\textbf{Data:} \underline{\hspace{4cm}}
	
	%\section*{Questões de Matemática}
	\begin{multicols}{2}
		\noindent Para acharmos o quociente e o resto, por exemplo, de 137 por 6, fazemos o seguinte:
		\begin{table}[H]
			\centering
			\begin{tabular}{ll}
				\multicolumn{1}{l|}{137} & 6 \\ \cline{2-2} 
				~                      & ~
			\end{tabular}
		\end{table}
		Como 1 dividido por 6 dá quociente 0, devemos tomar 13, em vez de 1. Já que 13 dividido por 6 tem quociente 2 e o resto 1, escrevemos: \\
		Exemplo: \begin{table}[H]
			\centering
			\begin{tabular}{ll}
				\multicolumn{1}{l|}{137} & 6 \\ \cline{2-2} 
				~1                      & 2
			\end{tabular}
		\end{table} 
		Em seguida, "abaixamos" o 7 e como 17 dividido por 6 tem quociente 2 e resto 5, temos: \\
		\begin{table}[H]
			\centering
			\begin{tabular}{ll}
				\multicolumn{1}{l|}{137} & 6 \\ \cline{2-2} 
				~17                      & 22 \\
				~~5
			\end{tabular}
		\end{table} 
		Logo, o quociente da divisão de 137 por 6 é 22 e o resto é 5.
    	\textsubscript{---------------------------------------------------------------------}
    	\begin{enumerate}
   			\item Complete as divisões a seguir \\
   			\begin{enumerate}[a)]
   				\item ~ \\
    			\begin{tabular}{ll}
    			\multicolumn{1}{l|}{105} & 4 \\ \cline{2-2} 
    			~                      & ~ \\\\
    			\end{tabular} \\\\
    			\item ~ \\
    			\begin{tabular}{ll}
    				\multicolumn{1}{l|}{523} & 5 \\ \cline{2-2} 
    				~                      & ~ \\\\
    			\end{tabular} \\\\\\\\\\
    			\item ~ \\
    			\begin{tabular}{ll}
    				\multicolumn{1}{l|}{421} & 12 \\ \cline{2-2} 
    				~                      & ~ \\\\
    			\end{tabular} \\\\\\\\\\
    			\item ~ \\
    			\begin{tabular}{ll}
    				\multicolumn{1}{l|}{953} & 11 \\ \cline{2-2} 
    				~                      & ~ \\\\
    			\end{tabular} \\\\\\\\\\
   			\end{enumerate}
   			\item Na loteria federal, o prêmio de R\$ 700.000,00 foi dividido entre 146 acertadores. Desprezando os centavos, determine a parte de cada ganhador. \\\\\\
   			\item Determine o quociente e o resto da divisão euclidiana de 11 627 por 607. \\\\\\\\\\\\\\\\\\
   			\item Um caminhão deve transportar 19 872 bombons em caixas que acomodam 16 bombons. Quantas caixas serão necessárias? \\\\\\\\\\\\\\\\\\
   			\textbf{Desafio olímpico} \\\\
   			Ao multiplicar um número inteiro por 506, Flávio esqueceu o zero e fez a multiplicação por 56, obtendo corretamente 1 008. Se ele fizesse a conta com 506, qual seria o resultado obtido?
   			\begin{enumerate}[a)]
   				\item 9 062
   				\item 9 160
   				\item 9 108
   				\item 9 506
   				\item 9 560
   			\end{enumerate}
   			%59 a 64
   			\item Efetue as divisões a seguir:
   			\begin{enumerate}[a)]
   				\item ~ \\
   				\begin{tabular}{ll}
   					\multicolumn{1}{l|}{171} & 24 \\ \cline{2-2} 
   					~                      & ~ \\\\
   				\end{tabular} \\\\\\\\\\\\\\
   				\item ~ \\
   				\begin{tabular}{ll}
   					\multicolumn{1}{l|}{334} & 31 \\ \cline{2-2} 
   					~                      & ~ \\\\
   				\end{tabular} \\\\\\\\\\\\\\
   				\item ~ \\
   				\begin{tabular}{ll}
   					\multicolumn{1}{l|}{521} & 42 \\ \cline{2-2} 
   					~                      & ~ \\\\
   				\end{tabular} \\\\\\\\\\\\\\
   				\item ~ \\
   				\begin{tabular}{ll}
   					\multicolumn{1}{l|}{511} & 32 \\ \cline{2-2} 
   					~                      & ~ \\\\
   				\end{tabular} \\\\\\\\\\\\
   				\item ~ \\
   				\begin{tabular}{ll}
   					\multicolumn{1}{l|}{350} & 43 \\ \cline{2-2} 
   					~                      & ~ \\\\
   				\end{tabular} \\\\\\\\\\\\\\
   				\item ~ \\
   				\begin{tabular}{ll}
   					\multicolumn{1}{l|}{852} & 52 \\ \cline{2-2} 
   					~                      & ~ \\\\
   				\end{tabular} \\\\\\\\\\\\\\
   				\item ~ \\
   				\begin{tabular}{ll}
   					\multicolumn{1}{l|}{693} & 63 \\ \cline{2-2} 
   					~                      & ~ \\\\
   				\end{tabular} \\\\\\\\\\\\\\
   				\item ~ \\
   				\begin{tabular}{ll}
   					\multicolumn{1}{l|}{983} & 88 \\ \cline{2-2} 
   					~                      & ~ \\\\
   				\end{tabular} \\\\\\\\\\
	        \end{enumerate}
	        \item Efetue as divisões a seguir:
	        \begin{enumerate}[a)]
	        	\item ~ \\
	        	\begin{tabular}{ll}
	        		\multicolumn{1}{l|}{254} & 35 \\ \cline{2-2} 
	        		~                      & ~ \\\\
	        	\end{tabular} \\\\\\\\\\\\\\
	        	\item ~ \\
	        	\begin{tabular}{ll}
	        		\multicolumn{1}{l|}{443} & 36 \\ \cline{2-2} 
	        		~                      & ~ \\\\
	        	\end{tabular} \\\\\\\\\\\\\\
	        	\item ~ \\
	        	\begin{tabular}{ll}
	        		\multicolumn{1}{l|}{390} & 27 \\ \cline{2-2} 
	        		~                      & ~ \\\\
	        	\end{tabular} \\\\\\\\\\\\\\
	        	\item ~ \\
	        	\begin{tabular}{ll}
	        		\multicolumn{1}{l|}{342} & 53 \\ \cline{2-2} 
	        		~                      & ~ \\\\
	        	\end{tabular} \\\\\\\\
	        	\item ~ \\
	        	\begin{tabular}{ll}
	        		\multicolumn{1}{l|}{631} & 55 \\ \cline{2-2} 
	        		~                      & ~ \\\\
	        	\end{tabular} \\\\\\\\\\\\\\
	        	\item ~ \\
	        	\begin{tabular}{ll}
	        		\multicolumn{1}{l|}{567} & 27 \\ \cline{2-2} 
	        		~                      & ~ \\\\
	        	\end{tabular} \\\\\\\\\\\\\\
	        	\item ~ \\
	        	\begin{tabular}{ll}
	        		\multicolumn{1}{l|}{682} & 52 \\ \cline{2-2} 
	        		~                      & ~ \\\\
	        	\end{tabular} \\\\\\\\\\\\\\
	        	\item ~ \\
	        	\begin{tabular}{ll}
	        		\multicolumn{1}{l|}{500} & 34 \\ \cline{2-2} 
	        		~                      & ~ \\\\
	        	\end{tabular} \\\\\\\\\\\\
	        \end{enumerate}
	        \item O número 33 534 é divisível por 54? Justifique. \\\\\\\\\\\\\\\\\\
	        \item André ganhou R\$ 503,00 em um jogo e resolveu dividir igualmente entre seus 6 filhos. Como 503 não é divisível por 6, ele determinou o quociente e o resto da divisão, cabendo a cada um o maior quociente possível. O resto do dinheiro ele usou para comprar balas ao preço de 9 por 1 real, dividindo igualmente o máximo de balas entre ele e seus filhos. As balas que sobraram, André deu para seu neto. Quantas balas cada um recebeu? \\\\\\\\\\\\\\\\\\\\\\\\\\\\\\
	        \item Um número dividido por 47 resulta em 12. Qual é o número? \\\\\\\\\\\\\\\\\\
	        \item Marcos decidiu colocar um aquário em sua sala e após montá-lo, foi a um $pet shop$ com uma certa quantia em dinheiro para comprar alguns peixes. Depois de pesquisar, decidiu levar 16 peixes pequenos que custavam R\$ 12,00 cada. Se Marcos recebeu 8 reais de troco, qual era a quantia levada por ele? \\\\\\\\\\\\\\\\\\
	    \end{enumerate} 
        $~$ \\ $~$ \\ $~$ \\ $~$ \\ $~$ \\ $~$ \\ $~$ \\ $~$ \\ $~$ \\ $~$ \\ $~$ \\ $~$ \\ $~$  \\ $~$ \\ $~$ \\ $~$ \\ $~$ \\ $~$ \\ $~$ \\ $~$ \\ $~$ \\ $~$ \\ $~$ \\ $~$ \\ $~$ \\ $~$ \\ $~$ \\ $~$ \\ $~$ \\ $~$ \\ $~$ \\ $~$ \\ $~$ \\ $~$ \\ $~$ \\ $~$ \\ $~$ \\ $~$ \\ $~$ \\ $~$ \\ $~$ \\ $~$ \\ $~$ \\ $~$ \\ $~$ \\ $~$ \\ $~$ \\ $~$ \\ $~$ \\ $~$ \\ $~$ \\
	\end{multicols}
\end{document}