\documentclass[a4paper,14pt]{article}
\usepackage{float}
\usepackage{extsizes}
\usepackage{amsmath}
\usepackage{amssymb}
\everymath{\displaystyle}
\usepackage{geometry}
\usepackage{fancyhdr}
\usepackage{multicol}
\usepackage{graphicx}
\usepackage[brazil]{babel}
\usepackage[shortlabels]{enumitem}
\usepackage{cancel}
\usepackage{textcomp}
\usepackage{array} % Para melhor formatação de tabelas
\usepackage{longtable}
\usepackage{booktabs}  % Para linhas horizontais mais bonitas
\usepackage{float}   % Para usar o modificador [H]
\usepackage{caption} % Para usar legendas em tabelas
\usepackage{tcolorbox}

\columnsep=2cm
\hoffset=0cm
\textwidth=8cm
\setlength{\columnseprule}{.1pt}
\setlength{\columnsep}{2cm}
\renewcommand{\headrulewidth}{0pt}
\geometry{top=1in, bottom=1in, left=0.7in, right=0.5in}

\pagestyle{fancy}
\fancyhf{}
\fancyfoot[C]{\thepage}

\begin{document}
	
	\noindent\textbf{6FMA96 - Matemática} 
	
	\begin{center}Resolvendo equações mais rapidamente (Versão estudante)
	\end{center}
	
	\noindent\textbf{Nome:} \underline{\hspace{10cm}}
	\noindent\textbf{Data:} \underline{\hspace{4cm}}
	
	%\section*{Questões de Matemática}
	~ \\ ~
	\begin{multicols}{2}
		\noindent Podemos utilizar uma notação abreviada e intuitiva, que permite agilidade na resolução de equações. \\
		Nas equações da aula anterior, podemos omitir algumas equivalências:
		\begin{enumerate}[a)]
			\item $2x + 5 = 0 \Leftrightarrow 2x = 0 - 5 \Leftrightarrow \\ 2x = -5 \Leftrightarrow x = -\frac{5}{2} = -\frac{5}{2}$ \\
			Logo $V = \left\{-\frac{5}{2}\right\}$.
			\item $-4x + 3 = 0 \Leftrightarrow -4x = 0 - 3 \Leftrightarrow -4x = -3 \Leftrightarrow x = \frac{-3}{-4} = \frac{3}{4}$ \\
			Logo $V = \left\{\frac{3}{4}\right\}$.
			\item $3x - 7 = 0 \Leftrightarrow 3x = 0 + 7 \Leftrightarrow \\ 3x = 7 \Leftrightarrow x = \frac{7}{3}$ \\
			Logo $V = \left\{\frac{7}{3}\right\}$. \\
			Na primeira equação, por exemplo, é comum dizer que o 5 que está somando do lado esquerdo "passa" para o lado direito fazendo a operação inversa, isto é, subtraindo. Além disso, o 2 que está multiplicando "passa" para o outro lado dividindo (operação inversa). \\
			Vejamos agora como resolver equações da forma $ax + b = 0$, com $a = 0$. \\
			Vamos considerar a equação: \\ $0x - 3 = 0 (U = \mathbb{Q})$, que é equivalente a $0x = 3$. Podemos notar que não é possível continuar o processo como nos exemplos anteriores, pois não é possível dividir por zero. No entanto, é fácil perceber que o valor do primeiro membro é sempre 0 e o segundo membro vale 3. Logo não existe valor racional de $x$ que torne a igualdade verdadeira e assim, $V = \varnothing$. \\
			Observando agora a equação: \\ $0x + 0 = 0 (U = \mathbb{Q})$, que é equivalente a $0x = 0$, percebemos que o valor do primeiro membro é sempre 0, o mesmo acontecendo para o segundo, ou seja, a igualdade é sempre verdadeira. Logo $x$ pode assumir qualquer valor racional e, dessa forma $V = \mathbb{Q}$.
			\end{enumerate}
	\end{multicols}
\noindent\textsubscript{~-----------------------------------------------------------------------------------------------------------------------------------------------------}
	\begin{multicols}{2}
    	\begin{enumerate}
    		\item Resolva as equações ($U = \mathbb{Q}$).
    		\begin{enumerate}[a)]
    			\item $3x - 8 = 0$ \\\\\\\\\\\\\\\\
    			\item $y - 7 = 0$ \\\\\\\\\\\\\\\\
    			\item $-2x + 9 = 0$ \\\\\\\\\\\\\\\\
    			\item $-x - 3 = 0$ \\\\\\\\\\\\\\\\
    			\item $3x + 4 = 0$ \\\\\\\\\\\\\\\\
    			\item $-3t + 3 = 0$ \\\\\\\\\\\\\\\\
    			\item $4k + 12 = 0$ \\\\\\\\\\\\\\\\
    			\item $-2m - 16 = 0$ \\\\\\\\\\\\\\\\
    		\end{enumerate}
    		\item Resolva as equações no universo dos números racionais.
    		\begin{enumerate}[a)]
    			\item $0x + 8 = 0$ \\\\\\\\\\\\\\\\
    			\item $0x = 0$ \\\\\\\\\\\\\\\\
    			\item $0x - 7 = 0$ \\\\\\\\\\\\\\\\
    			\item $0x + \frac{2}{5} = 0$ \\\\\\\\\\\\\\\\
    		\end{enumerate}
    		\item Caminhando pelo corredor de sua escola, André observou na lousa de uma das salas de aula as seguintes equivalências: \\ $(U = \mathbb{Q})\\~5x + \frac{3}{4} + 2x = -\frac{9}{2} + 7x + 10 \Leftrightarrow \\ 5x + 2x - 7x = 10 - \frac{9}{2} - \frac{3}{4} \Leftrightarrow \\ 0 = \frac{19}{4}$ \\\\
    		Ele notou que havia um erro, então aproveitou o intervalo entre as aulas para corrigi-lo. Qual foi o erro visto por André e como ficou a resolução da equação após sua correção? \newpage
    		\item Resolva as equações no universo dos números racionais.
    		\begin{enumerate}[a)]
    			\item $5x - 30 = 0$ \\\\\\\\\\\\\\\\
    			\item $3x - 9 = 0$ \\\\\\\\\\\\\\\\
    			\item $-x - 7 = 0$ \\\\\\\\\\\\\\\\
    			\item $6x - 5 = 0$ \\\\\\\\\\\\\\\\\\
    			\item $-x - 13 = 0$ \\\\\\\\\\\\\\\\
    			\item $2x + 3 = 0$ \\\\\\\\\\\\\\\\
    			\item $-24x + 8 = 0$ \\\\\\\\\\\\\\\\
    			\item $6x + 16 = 0$ \newpage
    		\end{enumerate}
    		\item Resolva as equações ($U = \mathbb{Q}$).
    		\begin{enumerate}[a)]
    			\item $0x + 0 = 0$ \\\\\\\\\\\\\\\\
    			\item $0x = 9$ \\\\\\\\\\\\\\\\
    			\item $0x + 2 = 0$ \\\\\\\\\\\\\\\\
    			\item $0x = 0$ \\\\\\\\\\\\\\\\
    		\end{enumerate}
    	\end{enumerate}
    $~$ \\ $~$ \\ $~$ \\ $~$ \\ $~$ \\ $~$ \\ $~$ \\ $~$ \\ $~$ \\ $~$  \\ $~$  \\ $~$  \\ $~$  \\ $~$  \\ $~$  \\ $~$  \\ $~$  \\ $~$ \\ $~$ \\ $~$ \\ $~$ \\ $~$ \\ $~$ \\ $~$ \\ $~$ \\ $~$ \\ $~$ \\ $~$ \\ $~$ \\ $~$ \\ $~$ \\ $~$ \\ $~$ \\ $~$ \\ $~$ \\ $~$ \\ $~$ \\ $~$ \\ $~$ \\ $~$ \\ $~$ \\ $~$ \\ $~$
	\end{multicols}
\end{document}