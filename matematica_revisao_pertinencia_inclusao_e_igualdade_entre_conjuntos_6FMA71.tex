\documentclass[a4paper,14pt]{article}

\usepackage{comment} % Para comentar várias linhas ao mesmo tempo

%matemática
\usepackage{amsmath}
\usepackage{amssymb}

%diagramação
\usepackage{extsizes}
\everymath{\displaystyle}
\usepackage{geometry}
\usepackage{fancyhdr}
\usepackage{multicol}
\usepackage{graphicx}
\usepackage[brazil]{babel}
\usepackage[shortlabels]{enumitem}
\usepackage{cancel}
\usepackage{textcomp}
\usepackage{tcolorbox}

%tabelas
\usepackage{array} % Para melhor formatação de tabelas
\usepackage{longtable}
\usepackage{booktabs}  % Para linhas horizontais mais bonitas
\usepackage{float}   % Para usar o modificador [H]
\usepackage{caption} % Para usar legendas em tabelas
\usepackage{wrapfig} % Para usar tabelas e figuras flutuantes

\begin{comment}
%tikzpicture
\usepackage{tikz}
\usepackage{scalerel}
\usepackage{pict2e}
\usepackage{tkz-euclide}
\usetikzlibrary{calc}
\usetikzlibrary{patterns,arrows.meta}
\usetikzlibrary{shadows}
\usetikzlibrary{external}
\end{comment}
	
%pgfplots
\usepackage{pgfplots}
\pgfplotsset{compat=newest}
\usepgfplotslibrary{statistics}
\usepgfplotslibrary{fillbetween}

%colours
\usepackage{xcolor}



\columnsep=2cm
\hoffset=0cm
\textwidth=8cm
\setlength{\columnseprule}{.1pt}
\setlength{\columnsep}{2cm}
\renewcommand{\headrulewidth}{0pt}
\geometry{top=1in, bottom=1in, left=0.7in, right=0.5in}

\pagestyle{fancy}
\fancyhf{}
\fancyfoot[C]{\thepage}

\begin{document}
	
	\noindent\textbf{6FMA71 - Matemática} 
	
	\begin{center}Revisão: pertinência, inclusão e igualdade entre conjuntos (Versão estudante)
	\end{center}
	
	\noindent\textbf{Nome:} \underline{\hspace{10cm}}
	\noindent\textbf{Data:} \underline{\hspace{4cm}}
	
	%\section*{Questões de Matemática}
	
	\begin{multicols}{2}
		\noindent 
		\begin{itemize}
			\item Pertinência é uma relação entre um elemento e um conjunto.
			\item Inclusão é uma relação entre dois conjuntos, para todo conjunto $A$, $\varnothing \subset A$ e $A \subset A$.
			\item Igualdade: $A = B$ se, e somente se, $A \subset B$ e $B \subset A$.
		\end{itemize}
		\noindent\textsubscript{--------------------------------------------------------------------------}
    	\begin{enumerate}
   			\item Diga quantos e quais são os elementos dos conjuntos:
   			\begin{enumerate}[a)]
   				\item $A = \{5, 6, 7, 8, 9\}$ \\\\\\\\\\\\\\\\\\\\
   				\item $B = \varnothing$ \\\\\\\\\\\\\\\\\\\\
   				\item $C =\{0, 1, \{2\},\{3, 4\}, 5\}$ \\\\\\\\\\\\\\\\\\\\
   				\item $D = \{0, 1, \{0, 1\}, \varnothing\}$ \\\\\\\\\\\\\\\\\\\\
   			\end{enumerate}
   			\item Escreva em linguagem simbólica.
   			\begin{enumerate}[a)]
   				\item 5 não pertence a $A$. \\\\\\\\\\\\\\\\\\\\
   				\item \{7\} contém $\varnothing$. \\\\\\\\\\\\\\\\
   				\item $\mathbb{N}$ está contido em $\mathbb{Z}$. \\\\\\\\\\\\\\\\
   				\item \{4\} é subconjunto de \{3, 4, 5\}. \\\\\\\\\\\\\\\\
   				\item \{3\} pertence a \{1, 2, 3, 4\}. \\\\\\\\\\\\\\\\
   			\end{enumerate}
   			\item Dado o conjunto \\ $A = \{2, \{1\}, \{2\}\}$, assinale \textbf{V} (verdadeiro) ou \textbf{F} (falso):
   			\begin{enumerate}[a)]
   				\item (~~) $2 \in A$ \\
   				\item (~~) $2 \subset A$ \\
   				\item (~~) $\{2\} \in A$ \\
   				\item (~~) $\{1\} \subset A$ \\
   				\item (~~) $1 \in A$ \\
   				\item (~~) $\{1\} \in A$ \\
   				\item (~~) $\{\{1\}, 2\} \subset A$ \\
   			\end{enumerate}
   			\item Dado o conjunto \\ $A = \{0, 1, 2, 3, \{3\}, \{4\}, 5\}$, assinale \textbf{V} (verdadeiro) ou \textbf{F} (falso).
   			\begin{enumerate}[a)]
   				\item (~~) $0 \in A$
   				\item (~~) $0 \subset A$
   				\item (~~) $A \ni \{3\}$
   				\item (~~) $A \notin \{4\}$
   				\item (~~) $\{\{4\}\} \subset A$
   				\item (~~) $\varnothing \in A$
   				\item (~~) $\varnothing \subset A$
   				\item (~~) $4 \in A$
   				\item (~~) $\{2\} \in A$
   				\item (~~) $\{2\} \subset A$
   				\item (~~) $A \supset \{3, 4, 5\}$
   				\item (~~) $A \not\supset \{0, 1, 2, 3, 4, 5\}$
   			\end{enumerate}
	    \end{enumerate} 
        $~$ \\ $~$ \\ $~$ \\ $~$ \\ $~$ \\ $~$
	\end{multicols}
\end{document}