\documentclass[a4paper,14pt]{article}

\usepackage{comment} % Para comentar várias linhas ao mesmo tempo

%matemática
\usepackage{amsmath}
\usepackage{amssymb}

%diagramação
\usepackage{extsizes}
\everymath{\displaystyle}
\usepackage{geometry}
\usepackage{fancyhdr}
\usepackage{multicol}
\usepackage{graphicx}
\usepackage[brazil]{babel}
\usepackage[shortlabels]{enumitem}
\usepackage{cancel}
\usepackage{textcomp}
\usepackage{tcolorbox}

%tabelas
\usepackage{array} % Para melhor formatação de tabelas
\usepackage{longtable}
\usepackage{booktabs}  % Para linhas horizontais mais bonitas
\usepackage{float}   % Para usar o modificador [H]
\usepackage{caption} % Para usar legendas em tabelas
\usepackage{wrapfig} % Para usar tabelas e figuras flutuantes


%tikzpicture
\usepackage{tikz}
\usepackage{scalerel}
\usepackage{pict2e}
\usepackage{tkz-euclide}
\usetikzlibrary{calc}
\usetikzlibrary{patterns,arrows.meta}
\usetikzlibrary{shadows}
\usetikzlibrary{external}


%pgfplots
\usepackage{pgfplots}
\pgfplotsset{compat=newest}
\usepgfplotslibrary{statistics}
\usepgfplotslibrary{fillbetween}

%colours
\usepackage{xcolor}



\columnsep=2cm
\hoffset=0cm
\textwidth=8cm
\setlength{\columnseprule}{.1pt}
\setlength{\columnsep}{2cm}
\renewcommand{\headrulewidth}{0pt}
\geometry{top=1in, bottom=1in, left=0.7in, right=0.5in}

\pagestyle{fancy}
\fancyhf{}
\fancyfoot[C]{\thepage}

\begin{document}
	
	\noindent\textbf{6FMA26 - Matemática} 
	
	\begin{center}Usando a régua e o esquadro para construir figuras planas (Versão estudante)
	\end{center}
	
	\noindent\textbf{Nome:} \underline{\hspace{10cm}}
	\noindent\textbf{Data:} \underline{\hspace{4cm}}
	
	%\section*{Questões de Matemática}
	
	\begin{multicols}{2}
		\noindent $\overline{AB}$ representa o segmento de reta de extremidades $A$ e $B$. \\
		Os três segmentos de reta que formam um triângulo são chamados lados e os pontos em que estes se unem são os vértices. \\
		Um quadrilátero é uma figura que possui 4 lados. \\
		Um retângulo é um quadrilátero que possui 4 ângulos iguais (retos) e um quadrado é um quadrilátero que possui os 4 lados e os 4 ângulos iguais. \\
		Para desenharmos segmentos perpendiculares, utilizamos o esquadro.
		\\
		\noindent\textsubscript{-----------------------------------------------------------------------}
		\begin{enumerate} 
			\item Desenhe o segmento de reta que tem como extremidades os pontos $M$ e $N$ representados abaixo.
			\begin{center}
				\begin{tikzpicture}
					%\draw[lightgray] (0,0) grid (7,2);
					\filldraw[black] (0,1.5) circle (3pt);
					\filldraw[black] (7,0) circle (3pt);
					\coordinate[label=above:M] (M) at (0,1.5);
					\coordinate[label=above:N] (N) at (7,0);
				\end{tikzpicture}
			\end{center}
			\item Agora, desenhe a semirreta que começa em $M$ e passa pelo ponto $N$.
			\begin{center}
				\begin{tikzpicture}
					%\draw[lightgray] (0,0) grid (7,2);
					\filldraw[black] (0,1.5) circle (3pt);
					\filldraw[black] (7,0) circle (3pt);
					\coordinate[label=above:M] (M) at (0,1.5);
					\coordinate[label=above:N] (N) at (7,0);
				\end{tikzpicture}
			\end{center}
			\item Trace a reta que passa pelos pontos $M$ e $N$.
			\begin{center}
				\begin{tikzpicture}
					%\draw[lightgray] (0,0) grid (7,2);
					\filldraw[black] (0,1.5) circle (3pt);
					\filldraw[black] (7,0) circle (3pt);
					\coordinate[label=above:M] (M) at (0,1.5);
					\coordinate[label=above:N] (N) at (7,0);
				\end{tikzpicture}
			\end{center}
			\item Utilizando régua e lápis, desenhe o triângulo cujos vértices são os pontos $A$, $B$ e $C$ representados abaixo.
			\begin{center}
				\begin{tikzpicture}
					%\draw[lightgray] (0,0) grid (7,2);
					\filldraw[black] (1.5,1.5) circle (3pt);
					\filldraw[black] (7,0) circle (3pt);
					\filldraw[black] (0,-2.5) circle (3pt);
					\coordinate[label=above:A] (A) at (1.5,1.5);
					\coordinate[label=above:C] (C) at (7,0);
					\coordinate[label=above:B] (B) at (0,-2.5);
				\end{tikzpicture}
			\end{center}
			\item \begin{enumerate}[a)]
				\item Desenhe um segmento $BC$ perpendicular ao segmento $AB$ dado abaixo. \\\\\\\\
				\begin{center}
					\begin{tikzpicture}
						%\draw[lightgray] (0,0) grid (7,2);
						\draw (0,0) -- (5,0);
						%\filldraw[black] (0,0) circle (3pt);
						%\filldraw[black] (6,0) circle (3pt);
						\coordinate[label=left:A] (A) at (0,0);
						\coordinate[label=right:B] (B) at (5,0);
					\end{tikzpicture}
				\end{center}
			\end{enumerate}
			\newpage
			\item Qual é a diferença entre um quadrado e um retângulo? \\\\\\\\\\\\\\\\\\\\
			\item A régua de Carlinhos quebrou e ele ficou com o pedaço maior, que permite ler a partir do traço que marca 12 cm. Como ele deve fazer para medir segmentos? \\\\\\\\\\\\\\\\\\\\
			\item \begin{enumerate}[a)]
				\item Desenhe o quadrado cujo lado possua a medida $\ell$ dada abaixo.
				\begin{center}
					\begin{tikzpicture}
						%\draw[lightgray] (0,0) grid (7,2);
						\draw (0,0) -- (5,0);
						\coordinate[label=above:$\ell$] (l) at (2.5,0);
					\end{tikzpicture}
				\end{center}
			\end{enumerate}
		\end{enumerate}
		$~$ \\ $~$ \\ $~$ \\ $~$ \\ $~$ \\ $~$ \\ $~$ \\ $~$ \\ $~$ \\ $~$ \\ $~$ \\ $~$ \\ $~$ \\ $~$ \\ $~$ \\ $~$ \\ $~$ \\ $~$ \\ $~$ \\ $~$ \\ $~$ \\ $~$ \\ $~$ \\ $~$ \\ $~$ \\ $~$ \\ $~$ \\ $~$ \\ $~$ \\ $~$ \\ $~$ \\ $~$ \\ $~$ \\ $~$ \\ $~$ \\ $~$ \\ $~$ \\ $~$ \\ $~$ \\ $~$ \\ $~$ \\ $~$ \\ $~$ \\ $~$ \\ $~$ \\ $~$ \\ $~$ \\ $~$ \\ $~$ \\ $~$ \\ $~$ \\ $~$ \\ $~$  
	\end{multicols}
\end{document}