\documentclass[a4paper,14pt]{article}
\usepackage{float}
\usepackage{extsizes}
\usepackage{amsmath}
\usepackage{amssymb}
\everymath{\displaystyle}
\usepackage{geometry}
\usepackage{fancyhdr}
\usepackage{multicol}
\usepackage{graphicx}
\usepackage[brazil]{babel}
\usepackage[shortlabels]{enumitem}
\usepackage{cancel}
\columnsep=2cm
\hoffset=0cm
\textwidth=8cm
\setlength{\columnseprule}{.1pt}
\setlength{\columnsep}{2cm}
\renewcommand{\headrulewidth}{0pt}
\geometry{top=1in, bottom=1in, left=0.7in, right=0.5in}

\pagestyle{fancy}
\fancyhf{}
\fancyfoot[C]{\thepage}

\begin{document}
	
	\noindent\textbf{8FMA52~Matemática} 
	
	\begin{center}Distância entre dois pontos (Versão estudante)
	\end{center}
	
	\noindent\textbf{Nome:} \underline{\hspace{10cm}}
	\noindent\textbf{Data:} \underline{\hspace{4cm}}
	
	%\section*{Questões de Matemática}
	
	
    \begin{multicols}{2}
    	\begin{enumerate}
    		\item Nos itens a seguir, são dados dois pontos. Apresentar a distância entre eles, fazendo o desenho do triângulo retângulo utilizado.
    		\begin{enumerate}[a)]
    			\item $A = (6; 2) e B = (2; 1)$. \\\\\\\\\\\\\\
    			\item $A = (-2; 1) e B = (2; -3)$.
    			\\\\\\\\\\\\\\
    			\item $A = (-7; 4) e B (-1; 0)$.
    			\\\\\\\\\\\\\\
    			\item $A =(0; 6) e B = (-3; 3)$.
    			\\\\\\\\\\
    			\item $A = (7; 2) e B = (2; 7)$.
    			\\\\\\\\\\\\\\
    			\item $A = (2; -4) e B = (-3; 0)$.
    			\\\\\\\\\\\\			
    	    \end{enumerate}
            \item A distância até o ponto $A = (3; 1)$ até o ponto $B = (b; 2)$ é $\sqrt{5}$. Qual(is) o(s) possível(is) valor(es) de $b$? Que figura é formada quando unimos $A$ ao(s) possível(is) ponto(s) $B$? Classifique essa figura. \\\\\\\\\\\\\\\\\\\\\\\\
            \item \begin{enumerate}[a)]
            	\item Se $A = (7; -2), B = (3, b)$ e a distância entre $A$ e $B$ é 5, então qual(is) o(s) possível(is) valor(es) de $b$? \\\\\\\\\\\\\\\\
            	\item Agora é a sua vez! Crie um enunciado como o do item $a$ e depois resolva-o. \\\\\\\\\\\\\\
            \end{enumerate}
            \item Nos itens a seguir, apresentar a distância entre os dois pontos dados.
	        \begin{enumerate}[a)]
	            \item $A = (10; 4)$ e $B = (4; 1)$ \\\\\\\\\\\\
	            \item $A = (0; 3)$ e $B = (-6; 4)$
	            \\\\\\\\\\\\\\
	            \item $A = (-2; -6)$ e $B = (-2; -9)$
	            \\\\\\\\\\\\\\
	            \item $A = (4; -2)$ e $B = (5; 6)$
	            \\\\\\\\\\\\\\
	            \item $A = (0; -3)$ e $B = (1; 1)$
	            \\\\\\\\\\\\\\
	            \item $A = (4; 0)$ e $B = (-7; -3)$
	            \\\\\\\\\\\\\\
	        \end{enumerate}	
        \end{enumerate}
    $~$ \\ $~$ \\ $~$ \\ $~$ \\ $~$ \\ 
    \end{multicols}
\end{document}