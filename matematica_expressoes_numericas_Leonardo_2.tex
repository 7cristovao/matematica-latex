\documentclass[a4paper,14pt]{article}
\usepackage{float}
\usepackage{extsizes}
\usepackage{amsmath}
\usepackage{amssymb}
\everymath{\displaystyle}
\usepackage{geometry}
\usepackage{fancyhdr}
\usepackage{multicol}
\usepackage{graphicx}
\usepackage[brazil]{babel}
\usepackage[shortlabels]{enumitem}
\usepackage{cancel}
\usepackage{textcomp}
\columnsep=2cm
\hoffset=0cm
\textwidth=8cm
\setlength{\columnseprule}{.1pt}
\setlength{\columnsep}{2cm}
\renewcommand{\headrulewidth}{0pt}
\geometry{top=1in, bottom=1in, left=0.7in, right=0.5in}

\pagestyle{fancy}
\fancyhf{}
\fancyfoot[C]{\thepage}

\begin{document}
	
	\noindent\textbf{Matemática} 
	
	\begin{center}Expressões numéricas \\ (Versão estudante)
	\end{center}
	
	\noindent\textbf{Nome:} \underline{\hspace{10cm}}
	\noindent\textbf{Data:} \underline{\hspace{4cm}}
	
	%\section*{Questões de Matemática}
	
	
    \begin{multicols}{2}
		\begin{enumerate}
			\item Resolva estas expressões numéricas:
			\begin{enumerate}[a)]
				\item $3 + (16 - 4 \times 3) - 6 \div 2 = $ \\\\\\\\\\\\\\\\\\\\\\\\\\\\\\
				\item $78 \div 2 + (9 \times 5) - 33 = $ \\\\\\\\\\\\\\\\\\\\\\\\\\\\
				\item $2 \times (7 - 4) - 12 \div 3 + 25 = $
				 \\\\\\\\\\\\\\\\\\\\\\\\\\\\
				\item $15 + 3 \times 7 - 2 \times 3 + 8 \div 2 =$  \\\\\\\\\\\\\\\\\\\\\\\\\\\\\\\\\\\\
				\item $12 \div 3 \times 2 + 2 = $ \\\\\\\\\\\\\\\\\\\\\\\\\\\\\\\\\\\\\\
				
				\fontsize{10}{\baselineskip} \selectfont
				
				\item $\left\{10 + [5 \times (4 + 2 \times 5 - 8)] \times 2 \right\} - 100 = $ \\\\\\\\\\\\\\\\\\\\\\\\\\\\\\\\\\\\\\\\
				
				\fontsize{14}{\baselineskip} \selectfont
				
				\item $2^4 \div 4 + 3^2 \times 10 + (9 \times 8) = $ \\\\\\\\\\\\\\\\\\\\\\\\\\\\\\\\\\\\\\
				\item $5 \times 9 \div 5 + 20 \div 10 \times 5 = $ \\\\\\\\\\\\\\\\\\\\\\\\\\\\
			\end{enumerate}
        \end{enumerate}
    $~$ \\ $~$ \\ $~$ \\ $~$ \\ $~$ \\ $~$
    \end{multicols}
\end{document}