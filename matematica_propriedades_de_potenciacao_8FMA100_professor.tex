\documentclass[a4paper,14pt]{article}
\usepackage{float}
\usepackage{extsizes}
\usepackage{amsmath}
\usepackage{amssymb}
\everymath{\displaystyle}
\usepackage{geometry}
\usepackage{fancyhdr}
\usepackage{multicol}
\usepackage{graphicx}
\usepackage[brazil]{babel}
\usepackage[shortlabels]{enumitem}
\usepackage{cancel}
\columnsep=2cm
\hoffset=0cm
\textwidth=8cm
\setlength{\columnseprule}{.1pt}
\setlength{\columnsep}{2cm}
\renewcommand{\headrulewidth}{0pt}
\geometry{top=1in, bottom=1in, left=0.7in, right=0.5in}

\pagestyle{fancy}
\fancyhf{}
\fancyfoot[C]{\thepage}

\begin{document}
	
	\noindent\textbf{8FMA100~Matemática} 
	
	\begin{center}Propriedades de potenciação (Versão professor)
	\end{center}
	
	\noindent\textbf{Nome:} \underline{\hspace{10cm}}
	\noindent\textbf{Data:} \underline{\hspace{4cm}}
	
	%\section*{Questões de Matemática}
	
	
    \begin{multicols}{2}
    	\begin{enumerate}[a)]
    		 \item $2^{-1} \cdot 2^3 = \frac{1}{4}$
    		 \\\\\\\\\\
    		 \item $3^5 \cdot 3^{-3} = 9$
    		 \\\\\\\\\\
    		 \item $8^{-9} \cdot 8^{-9} = 8^{-18}$
    		 \\\\\\\\\\
    		 \item $\bigg(\frac{3}{5}\bigg)^2 \cdot \bigg(\frac{3}{5}\bigg) = \frac{27}{125}$
    		 \\\\\\\\\\
    		 \item $\frac{7^2}{7^{~}} = 7$
    		 \\\\\\\\\\\\
    		 \item $\frac{4^4}{4^5} = \frac{1}{4}$
    		 \\\\
    		 \item $\frac{5^3}{5^4} = \frac{1}{5}$
    		 \\\\\\\\\\
    		 \item $(2^3)^4 = 2^{12}$
    		 \\\\\\\\\\
    		 \item $2^{10} \cdot 2^{-10} = 1$
    		 \\\\\\\\\\
    		 \item O valor de $(0,2)^3$ + $(0,16)^2$ é:
    		 Resposta: 0,0336
    		 \\\\\\\\\\\\\\\\\\\\
        \end{enumerate}    
    \end{multicols}
\end{document}