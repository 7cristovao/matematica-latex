\documentclass[a4paper,14pt]{article}
\usepackage{extsizes}
\usepackage{amsmath}
\usepackage{amssymb}
\everymath{\displaystyle}
\usepackage{geometry}
\usepackage{fancyhdr}
\usepackage{multicol}
\usepackage{graphicx}
\usepackage[brazil]{babel}
\usepackage[shortlabels]{enumitem}
\usepackage{cancel}
\columnsep=2cm
\hoffset=0cm
\textwidth=8cm
\setlength{\columnseprule}{.1pt}
\setlength{\columnsep}{2cm}
\renewcommand{\headrulewidth}{0pt}
\geometry{top=1in, bottom=1in, left=0.7in, right=0.5in}

\pagestyle{fancy}
\fancyhf{}
\fancyfoot[C]{\thepage}

\begin{document}
	
	\noindent\textbf{8FMA13~-~Matemática} 
	
	\begin{center}Revisão: racionalização de denominadores (Versão estudante)
	\end{center}
	
	
	\noindent\textbf{Nome:} \underline{\hspace{10cm}}
	\noindent\textbf{Data:} \underline{\hspace{4cm}}
	
	%\section*{Questões de Matemática}
	
	\begin{multicols}{2}
	
		\begin{enumerate}
			\item Racionalizar os denominadores das frações:
			\begin{enumerate}[a)]
				\item $\frac{3}{\sqrt{5}}$\\\\\\\\
				\item $\frac{-7}{\sqrt{7}}$\\\\\\\\
				\item $\frac{2}{\sqrt[3]{4}}$\\\\\\\\
				\item $\frac{3}{\sqrt[4]{6}}$\\\\\\\\
				\item $\frac{1}{\sqrt[6]{3}}$\\\\\\\\
				\item $\frac{1}{\sqrt[5]{3^2}}$\\\\\\\\
				\item $\frac{\sqrt{3}}{\sqrt{5}}$\\\\\\\\
				\item $\frac{3}{\sqrt[7]{2^3}}$\\\\\\\\\\
				\item $\frac{\sqrt{2}}{\sqrt[9]{3^5}}$\\\\\\\\\\
				\item $\frac{\sqrt[7]{4^5}}{\sqrt[8]{6^5}}$\\\\\\\\\\
				\item $\frac{1}{\sqrt[13]{3^9}}$\\\\\\\\\\
				\item $\frac{4}{\sqrt[7]{4^5}}$\\\\\\\\\\
			\end{enumerate}
		    \item Racionalizar os denominadores das frações:
		    \begin{enumerate}[a)]
		    	\item $\frac{1}{\sqrt{5}-1}$ \\\\\\\\\\\\\\\\
		    	\item $\frac{6}{\sqrt{6}+6}$ \\\\\\\\\\\\\\\\
		    	\item $\frac{3}{\sqrt{5}-1}$ \\\\\\\\\\\\\\\\
		    	\item $\frac{5}{\sqrt{13}-\sqrt{8}}$ \\\\\\\\\\\\\\\\
		    \end{enumerate}
	        \item Racionalizar os denominadores das frações:
	        \begin{enumerate}[a)]
	        	\item $\frac{1}{1+\sqrt{5}+\sqrt{6}}$\\\\\\\\\\\\\\\\\\\\\\\\
	        	\item $\frac{1}{\sqrt{5}-\sqrt{12}+\sqrt{7}}$
	        \end{enumerate}
        $~$ \\ $~$ \\ $~$ \\ $~$ \\ $~$ \\ $~$ \\ $~$ \\ $~$ \\ $~$ \\ $~$ \\ $~$ \\ $~$ \\ $~$ \\ $~$ \\ $~$ \\ $~$ \\ $~$ \\ $~$ \\ $~$
            \item Racionalize os denominadores a seguir.
            \begin{enumerate}[a)]
            	\item $\frac{1}{\sqrt{3}}$\\\\\\\\\\
            	\item $\frac{7}{\sqrt{7}}$\\\\\\\\\\
            	\item $-\frac{8}{\sqrt{14}}$\\\\\\\\\\
            	\item $-\frac{1}{\sqrt[3]{4}}$\\\\\\\\\\
            	\item $\frac{4}{\sqrt[3]{5}}$\\\\\\\\\\
            	\item $\frac{1}{\sqrt[7]{2^5}}$\\\\\\\\\\
            	\item $-\frac{6}{\sqrt[12]{5^9}}$\\\\\\\\\\
            	\item $\frac{14}{\sqrt[7]{7^5}}$\\\\\\\\\\
            \end{enumerate}
            \item Racionalize os denominadores a seguir.
            \begin{enumerate}[a)]
                \item $\frac{1}{\sqrt{7}-\sqrt{3}}$\\\\\\\\\\
                \item $\frac{5}{\sqrt{17} + 4}$\\\\\\\\\\
                \item $-\frac{13}{\sqrt{3} + 4}$\\\\\\\\\\
                \item $\frac{\sqrt{3}}{3 - \sqrt{3}}$\\\\\\\\\\
            \end{enumerate}
            \item Sendo $x$, $y \in \mathbb{R^*_+}$, $x \neq y$, tais que $\frac{1}{\sqrt{x}-\sqrt{y}} = \frac{\sqrt{6} + 1}{5}$,\\\\ determine $x$ e $y$. \\\\\\\\\\\\\\\\\\\\\\\\\\\\\\
            \item Racionalize os denominadores a seguir.
            \begin{enumerate}[a)]
            	\item $\frac{1}{\sqrt{12}+\sqrt{5}-\sqrt{3}}$\\\\\\\\\\\\\\\\\\\\\\\\\\\\\\\\\\\\
            	\item $\frac{15}{\sqrt{2}+\sqrt{5}-\sqrt{7}}$\\\\\\\\\\\\\\\\
            \end{enumerate}
        $~$ \\ $~$ \\ $~$ \\ $~$ \\ $~$ \\ $~$ \\ $~$ \\ $~$ \\ $~$ \\ $~$ \\ $~$ \\ $~$ \\ $~$ \\ $~$ \\ $~$ \\ $~$ \\ $~$ \\ $~$ \\ $~$$~$ \\ $~$ \\ $~$ \\ $~$ \\ $~$ \\ $~$ \\ $~$ \\ $~$ \\ $~$ \\ $~$
  	    \end{enumerate}
    \end{multicols}
\end{document}