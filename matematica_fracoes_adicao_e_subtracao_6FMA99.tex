\documentclass[a4paper,14pt]{article}
\usepackage{float}
\usepackage{extsizes}
\usepackage{amsmath}
\usepackage{amssymb}
\everymath{\displaystyle}
\usepackage{geometry}
\usepackage{fancyhdr}
\usepackage{multicol}
\usepackage{graphicx}
\usepackage[brazil]{babel}
\usepackage[shortlabels]{enumitem}
\usepackage{cancel}
\usepackage{textcomp}
\usepackage{array} % Para melhor formatação de tabelas
\usepackage{longtable}
\usepackage{booktabs}  % Para linhas horizontais mais bonitas
\usepackage{float}   % Para usar o modificador [H]
\usepackage{caption} % Para usar legendas em tabelas
\usepackage{tcolorbox}

\columnsep=2cm
\hoffset=0cm
\textwidth=8cm
\setlength{\columnseprule}{.1pt}
\setlength{\columnsep}{2cm}
\renewcommand{\headrulewidth}{0pt}
\geometry{top=1in, bottom=1in, left=0.7in, right=0.5in}

\pagestyle{fancy}
\fancyhf{}
\fancyfoot[C]{\thepage}

\begin{document}
	
	\noindent\textbf{6FMA99 - Matemática} 
	
	\begin{center}Frações: adição e subtração (Versão estudante)
	\end{center}
	
	\noindent\textbf{Nome:} \underline{\hspace{10cm}}
	\noindent\textbf{Data:} \underline{\hspace{4cm}}
	
	%\section*{Questões de Matemática}
	~ \\ ~
	\begin{multicols}{2}
		\noindent Para somar ou subtrair frações de denominadores diferentes, basta transformá-las em frações equivalentes de mesmo denominador e somar ou subtrair os numeradores. \\
		Exemplos:
		\begin{itemize}
			\item $\frac{5}{6} + \frac{3}{8} = \frac{5 \cdot 4 + 3 \cdot 3}{24} = \frac{20 + 9}{24} \\\\ = \frac{29}{24}$
			\item $\frac{1}{20} - \frac{11}{18} = \frac{1 \cdot 9 - 11 \cdot 10}{180} \\\\ = \frac{9 - 110}{180} = - \frac{101}{180}$
		\end{itemize}
		\textsubscript{---------------------------------------------------------------------}
    	\begin{enumerate}
    		\item Calcular.
    		\begin{enumerate}[a)]
    			\item $\frac{1}{3} + \frac{3}{5} + \frac{5}{6} + \frac{6}{7}$ \\\\\\\\\\\\\\\\
    			\item $\frac{2}{3} + \frac{1}{5} + \frac{5}{9} + \frac{1}{15}$ \\\\\\\\
    			\item $7 + \frac{1}{4}$ \\\\\\\\\\\\\\\\
    			\item $\frac{7}{8} - \frac{6}{7} - \frac{5}{6}$ \\\\\\\\\\\\\\\\
    			\item $\frac{1}{2} - \frac{1}{5} - \frac{1}{10}$ \\\\\\\\\\\\\\\\
    			\item $9 - \frac{15}{7}$ \newpage
    		\end{enumerate}
    		\item \begin{enumerate}[label=\Roman*.] % Define a numeração em algarismos romanos
    			\item Determinar o valor das expressões a seguir.
    			\begin{enumerate}[a)]
    				\item $\frac{3}{4} + \frac{2}{3} - \bigg(\frac{5}{6} - \frac{1}{5}\bigg) + \frac{7}{10} - \frac{2}{15}$ \\\\\\\\\\\\\\\\\\\\\\\\\\\\\\\\\\\\
    				\item $\frac{1}{5} - \bigg(\frac{1}{2} + \frac{1}{4} - \frac{1}{3}\bigg) + \frac{1}{6} -\frac{1}{7}$ \\\\\\\\\\\\\\\\\\\\\\\\\\\\\\\\\\\\
    			\end{enumerate}
    			\item Agora, com base nos itens anteriores, apresente duas expressões diferentes com quatro números racionais e resolva-as. \newpage
    		\end{enumerate}
    		\item Calcule:
    		\begin{enumerate}[a)]
    			\item $\frac{1}{3} - \frac{4}{5} + \frac{13}{15} - \frac{2}{3}$ \\\\\\\\\\\\\\\\
    			\item $\frac{13}{18} + \frac{1}{2} + \frac{4}{9}$ \\\\\\\\\\\\\\\\
    			\item $\frac{2}{9} - \frac{3}{8}$ \\\\\\\\\\\\\\\\
    			\item $\frac{7}{15} - \frac{2}{3} - \frac{1}{5}$ \\\\\\\\\\\\\\\\
    			\item $\frac{7}{8} - \frac{1}{2} + \frac{3}{4}$ \\\\\\\\\\\\\\\\
    			\item $-\frac{2}{9} + \frac{3}{4} - \frac{5}{36}$ \\\\\\\\\\\\\\\\
    			\item $-\frac{5}{6} + \frac{7}{8} -\frac{1}{16}$ \\\\\\\\\\\\\\\\
    			\item $\frac{1}{2} - \frac{2}{3} + \frac{3}{4} - \frac{4}{5}$ \\\\\\\\\\\\\\\\
    		\end{enumerate}
    	\end{enumerate}
    $~$ \\ $~$ \\ $~$ \\ $~$ \\ 
	\end{multicols}
\end{document}