\documentclass[a4paper,14pt]{article}
\usepackage{float}
\usepackage{extsizes}
\usepackage{amsmath}
\usepackage{amssymb}
\everymath{\displaystyle}
\usepackage{geometry}
\usepackage{fancyhdr}
\usepackage{multicol}
\usepackage{graphicx}
\usepackage[brazil]{babel}
\usepackage[shortlabels]{enumitem}
\usepackage{cancel}
\usepackage{textcomp}
\usepackage{array} % Para melhor formatação de tabelas
\usepackage{longtable}
\usepackage{booktabs}  % Para linhas horizontais mais bonitas
\usepackage{float}   % Para usar o modificador [H]
\usepackage{caption} % Para usar legendas em tabelas

\columnsep=2cm
\hoffset=0cm
\textwidth=8cm
\setlength{\columnseprule}{.1pt}
\setlength{\columnsep}{2cm}
\renewcommand{\headrulewidth}{0pt}
\geometry{top=1in, bottom=1in, left=0.7in, right=0.5in}

\pagestyle{fancy}
\fancyhf{}
\fancyfoot[C]{\thepage}

\begin{document}
	
	\noindent\textbf{6FMA95 - Matemática} 
	
	\begin{center}Resolvendo equações do tipo $ax+b = 0~(a \neq 0)$~(Versão estudante)
	\end{center}
	
	\noindent\textbf{Nome:} \underline{\hspace{10cm}}
	\noindent\textbf{Data:} \underline{\hspace{4cm}}
	
	%\section*{Questões de Matemática}
	~ \\ ~
	\begin{multicols}{2}
	\noindent Equações da forma $x = k$ são chamadas equações de solução imediata. Por exemplo, o conjunto verdade da equação $x = 9$ é $V = \{9\}$. \\
	Além disso, é possível resolver mentalmente outras equações. Para resolvermos a equação $x - 1 = 2$, basta pensarmos em um número que, diminuindo de 1, resulta em 2. Dessa forma, $x = 3$ e $V = \{3\}$. \\
	Para resolver uma equação do tipo $ax + b = 0 (a \neq 0)$, devemos escrever outras equações equivalentes a ela, até chegarmos a uma equação de solução imediata, lembrando que: \\
	\begin{itemize}
		\item uma igualdade não se altera quando somamos (ou subtraímos de) a ambos os membros um mesmo número.
		\item uma igualdade não se altera quando multiplicamos (ou dividimos) ambos os membros por um mesmo número diferente de zero.
		\item Vejamos alguns exemplos, sendo $U = \mathbb{Q}$.
		\begin{enumerate}[a)]
			\item \noindent$2x + 5 = 0 \\ \Leftrightarrow 2x + 5 - 5 = 0 - 5 \Leftrightarrow \\ 2x = -5 \Leftrightarrow \frac{2x}{2} = \frac{-5}{2} \Leftrightarrow \\ x = -\frac{5}{2}$ \\
			Logo $V = \left\{-\frac{5}{2}\right\}$
			\item \noindent$-4x + 3 = 0 \Leftrightarrow \\ -4x + 3 - 3 = 0 - 3 \Leftrightarrow \\ -4x = -3 \Leftrightarrow \frac{-4x}{4} = \frac{-3}{-4} \Leftrightarrow x = \frac{3}{4}$ \\
			Logo $V = \left\{\frac{3}{4}\right\}$
			\item \noindent$3x - 7 = 0 \Leftrightarrow \\ 3x -7 + 7 = 0 + 7 \Leftrightarrow \\ 3x = 7 \Leftrightarrow \frac{3x}{3} = \frac{7}{3} \Leftrightarrow x = \frac{7}{3}$ \\
			Logo $V	= \left\{\frac{7}{3}\right\}$		
		\end{enumerate}
	\end{itemize}
	\end{multicols}
\noindent\textsubscript{~-----------------------------------------------------------------------------------------------------------------------------------------------------}
	\begin{multicols}{2}
    	\begin{enumerate}
    		\item Resolva as equações de solução imediata a seguir, sendo $U = \mathbb{Q}$.
    		\begin{enumerate}[a)]
    			\item $x = 9$ \\\\
    			\item $x = - 5$ \\\\\\
    			\item $y = 0$ \\
    			\item $a = \frac{1}{5}$ \\\\\\
    		\end{enumerate}
    		\item Resolva mentalmente as seguintes equações ($U = \mathbb{Q}$).
    		\begin{enumerate}[a)]
    			\item $x + 2 = 8$ \\\\\\
    			\item $x - 3 = 9$ \\\\\\
    			\item $t + 8 = 6$ \\\\\\
    			\item $-k = -7$ \\\\\\
    			\item $4x = 60$ \\\\\\
    			\item $-5x = 20$ \\\\\\
    		\end{enumerate}
    		\item Em cada item a seguir, é dada uma equação do tipo $ax + b = 0$. Apresente $a$ e $b$.
    		\begin{enumerate}[a)]
    			\item $2x + 9 = 0$ \\\\\\
    			\item $3x - 7 = 0$ \\\\\\
    			\item $x - 5 = 0$ \\\\\\
    			\item $4x = 0$ \\\\\\
    			\item $0x + 6 = 0$ \\\\\\
    			\item $0x = 0$ \\\\\\ 
    			\item $-\frac{x}{4}-\frac{1}{5} = 0$ \\\\\\
    		\end{enumerate}
    		\item \begin{enumerate}[label=\Roman*.] % Define a numeração em algarismos romanos
    			\item Resolva as equações ($U = \mathbb{Q})$. Deixe indicadas todas as equivalências. \\
    			\begin{enumerate}[a)]
    				\item $2x + 6 = 0$ \\\\\\\\
    				\item $3x - 12 = 0$ \\\\\\\\
    				\item $4x - 1 = 0$ \\\\\\\\
    				\item $5x + 9 = 0$ \\\\\\\\
    				\item $-6x + 5 = 0$ \\\\\\\\
    				\item $-3x - 18 = 0$ \\\\\\\\
    			\end{enumerate}
    			\item Agora apresente quatro equações da forma $Ax + B = 0$, com $A$ e $B \in \mathbb{Z}$, e, em seguida, resolva-as indicando todas as equivalências.
    		\end{enumerate}
    		\item Resolva as equações de solução imediata a seguir, sendo $U = \mathbb{Q}$.
    		\begin{enumerate}[a)]
    			\item $x = 3$ \\\\\\
    			\item $x = 16$ \\\\\\
    			\item $y = 7$ \\\\\\
    			\item $y = -\frac{3}{7}$ \\\\\\
    			\item $a = \frac{8}{5}$ \\\\\\
    			\item $m = 2075$ \\\\\\ 
    		\end{enumerate}
    		\item Resolva mentalmente as seguintes equações $(U = \mathbb{Q})$.
    		\begin{enumerate}[a)]
    			\item $x + 2 = 8$ \\\\\\
    			\item $x - 4 = 10$ \\\\\\
    			\item $t - 20 = -15$ \\\\\\
    			\item $-k = -7$ \\\\\\
    			\item $4y = -24$ \\\\\\
    			\item $-5y = 10$ \\\\\\ 
    		\end{enumerate}
    		\item Em cada item a seguir, é dada uma equação do tipo $ax + b = 0$. Apresente $a$ e $b$.
    		\begin{enumerate}[a)]
    			\item $3x + 9 = 0$ \\\\\\
    			\item $8y - 1 = 0$ \\\\\\
    			\item $-5t + 7 = 0$ \\\\\\
    			\item $2x + 25 = 0$ \\\\\\
    			\item $x - 15 = 0$ \\\\\\
    			\item $-x - 2 = 0$ \\\\\\
    			\item $-8x = 0$ \\\\\\
    			\item $0x - 4 = 0$ \\\\\\
    			\item $0x = 0$ \\\\\\
    			\item $\frac{3}{5}x - \frac{7}{2} = 0$ \\\\\\
    		\end{enumerate}
    		\item Resolva as equações $(U = \mathbb{Q})$. Deixe indicadas todas as equivalências.
    		\begin{enumerate}[a)]
    			\item $3x - 12 = 0$ \\\\\\
    			\item $-2x + 14 = 0$ \\\\\\
    			\item $8x + 1 = 0$ \\\\\\
    			\item $2x - 3 = 0$ \\\\\\
    			\item $-4x -20 = 0$ \\\\\\
    			\item $-12x - 15 = 0$
    		\end{enumerate}
    	\end{enumerate}
    $~$ \\ $~$ \\ $~$ \\ $~$ \\ $~$ \\ $~$ \\ $~$ \\ $~$ \\ $~$ \\ $~$ 
	\end{multicols}
\end{document}