\documentclass[a4paper,14pt]{article}
\usepackage{float}
\usepackage{extsizes}
\usepackage{amsmath}
\usepackage{amssymb}
\everymath{\displaystyle}
\usepackage{geometry}
\usepackage{fancyhdr}
\usepackage{multicol}
\usepackage{graphicx}
\usepackage[brazil]{babel}
\usepackage[shortlabels]{enumitem}
\usepackage{cancel}
\usepackage{textcomp}
\usepackage{array} % Para melhor formatação de tabelas
\usepackage{longtable}
\usepackage{booktabs}  % Para linhas horizontais mais bonitas
\usepackage{float}   % Para usar o modificador [H]
\usepackage{caption} % Para usar legendas em tabelas
\usepackage{tcolorbox}
\usepackage{wrapfig} % Para usar tabelas e figuras flutuantes

\columnsep=2cm
\hoffset=0cm
\textwidth=8cm
\setlength{\columnseprule}{.1pt}
\setlength{\columnsep}{2cm}
\renewcommand{\headrulewidth}{0pt}
\geometry{top=1in, bottom=1in, left=0.7in, right=0.5in}

\pagestyle{fancy}
\fancyhf{}
\fancyfoot[C]{\thepage}

\begin{document}
	
	\noindent\textbf{6FMA59 - Matemática} 
	
	\begin{center}Elemento neutro e oposto (Versão estudante)
	\end{center}
	
	\noindent\textbf{Nome:} \underline{\hspace{10cm}}
	\noindent\textbf{Data:} \underline{\hspace{4cm}}
	
	%\section*{Questões de Matemática}
	\begin{multicols}{2}
    		\noindent 
    		\begin{itemize}
    			\item \textbf{A3.} $a + 0 = a$ (O zero é elemento neutro da adição.)
    			\item \textbf{A3.} $a + (-a) = 0$ (A soma de um número inteiro com seu oposto é zero.)
    		\end{itemize}
    		\textsubscript{---------------------------------------------------------------------}
    		\begin{enumerate}
    			\item Encontre o oposto dos seguintes números:
    			\begin{enumerate}[a)]
    				\item 8 \\\\\\
    				\item -2 \\\\\\
    				\item -21 \\\\\\
    				\item -(-14) \\\\\\
    				\item 0 \\\\\\
    			\end{enumerate}
    			\item Efetue.
    			\begin{enumerate}[a)]
    				\item 5 + (-5) = \\
    				\item -7 + 7 = \\\\\\
    				\item 2 + (-2) = \\\\\\
    				\item -16 + 16 = \\\\\\
    			\end{enumerate}
    			A partir desses cálculos, podemos concluir alguma coisa? O quê? Tente expressar-se com palavras. Você saberia expressar-se usando letras? \\\\\\\\\\\\\\
    			\item Usando a primeira versão de A4, como você mostraria que -(-6) = 6? \newpage
    			\item Você já estudou a adição no conjunto dos naturais $\mathbb{N} = \{0, 1, 2, ...\}$. Todas as cinco propriedades valem para adição em $\mathbb{N}$? Explicar. \\\\\\
    			\item Assinale \textbf{V} (verdadeiro) ou \textbf{F} (falso).
    			\begin{enumerate}[a)]
    				\item Para todo $x, y \in \mathbb{Z}, x + (-y) = x - y$.
    				\item Para todo $x, y \in \mathbb{Z}, x + y = x -(-y)$.
    				\item Para todo $x$ inteiro, $0 + x = 0 -(-x) = -x$
       			\end{enumerate}
       			\item Determine quantos inteiros há do primeiro ao último inteiro nos itens abaixo.
       			\begin{enumerate}[a)]
       				\item 10 e 100.
       				\item -50 e -20.
       				\item -30 e 30.
       				\item $x$ e $y$ com $x < y$.
       			\end{enumerate}
    		\end{enumerate}
    		$~$ \\ $~$ \\ $~$ \\ $~$ \\ $~$ \\ $~$ \\ $~$ \\ $~$ \\ $~$ \\ $~$ \\ $~$ \\ $~$ \\ $~$ \\ $~$ \\ $~$ \\ $~$ \\ $~$ \\ $~$ \\ $~$ \\ $~$ \\ $~$ \\ $~$ \\ $~$ \\ $~$ \\ $~$ \\ $~$ \\ $~$ \\ $~$ \\ $~$ \\ $~$ \\ $~$ \\ $~$ \\ $~$ \\ $~$ \\ $~$ \\ $~$ \\ $~$ \\ $~$ \\ $~$ \\ $~$ \\ $~$ \\ $~$ \\ $~$ \\ $~$ \\ $~$ \\ $~$ \\ $~$ \\ $~$ \\ $~$ \\ $~$ \\ $~$ \\ $~$ \\ $~$ \\ $~$ \\ $~$
	\end{multicols}
\end{document}