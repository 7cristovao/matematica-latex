\documentclass[a4paper,14pt]{article}
\usepackage{extsizes}
\usepackage{amsmath}
\usepackage{amssymb}
\everymath{\displaystyle}
\usepackage{geometry}
\usepackage{fancyhdr}
\usepackage{multicol}
\usepackage{graphicx}
\usepackage[brazil]{babel}
\usepackage[shortlabels]{enumitem}
\usepackage{cancel}
\columnsep=2cm
\hoffset=0cm
\textwidth=8cm
\setlength{\columnseprule}{.1pt}
\setlength{\columnsep}{2cm}
\renewcommand{\headrulewidth}{0pt}
\geometry{top=1in, bottom=1in, left=0.7in, right=0.5in}

\usepackage{multirow}

\pagestyle{fancy}
\fancyhf{}
\fancyfoot[C]{\thepage}

\begin{document}
	
	\noindent\textbf{7FMA157~-~Matemática} 
	
	\begin{center}Interpretando gráficos (Versão estudante)
	\end{center}
	
	
	\noindent\textbf{Nome:} \underline{\hspace{10cm}}
    \noindent\textbf{Data:} \underline{\hspace{4cm}}
	
	%\section*{Questões de Matemática}
	
	\begin{multicols}{2}
		
				\begin{tabular}{|c|c|}
			\hline
			\multicolumn{2}{|c|}{\textbf{Companhia de eletricidade}}\\
			\hline
			Fornecimento & Valor (R\$) \\
			\hline
			203 kWh $\times$ 0,29518000 & 59,92 \\
			\hline
		\end{tabular}
		\\\\\\
		\noindent
		\begin{tabular}{|c|c|c|c|}
			\hline
			\multicolumn{4}{|c|}{\textbf{Companhia de saneamento}}\\
			\multicolumn{4}{|c|}{\textbf{(Tarifas de água/m³)}}\\
			\hline
			Faixas & Tarifa & Consumo & Valor (R\$) \\
			\hline
			Até 10 & 26,18 & Tarifa mínima & 26,18 \\
			11 a 20 & 4,10 & 7 & 28,70 \\
			21 a 30 & 10,23 & -- & -- \\
			31 a 50 & 10,23 & -- & -- \\
			Acima de 50 & 11,27 & -- & -- \\
			-- & -- & \textbf{Total} & 54,88 \\
			\hline
		\end{tabular}
		
		\begin{enumerate}
			\item 
		\end{enumerate}
    \end{multicols}
\end{document}









