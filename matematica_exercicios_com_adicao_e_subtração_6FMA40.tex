\documentclass[a4paper,14pt]{article}
\usepackage{float}
\usepackage{extsizes}
\usepackage{amsmath}
\usepackage{amssymb}
\everymath{\displaystyle}
\usepackage{geometry}
\usepackage{fancyhdr}
\usepackage{multicol}
\usepackage{graphicx}
\usepackage[brazil]{babel}
\usepackage[shortlabels]{enumitem}
\usepackage{cancel}
\usepackage{textcomp}
\usepackage{array} % Para melhor formatação de tabelas
\usepackage{longtable}
\usepackage{booktabs}  % Para linhas horizontais mais bonitas
\usepackage{float}   % Para usar o modificador [H]
\usepackage{caption} % Para usar legendas em tabelas
\usepackage{tcolorbox}

\columnsep=2cm
\hoffset=0cm
\textwidth=8cm
\setlength{\columnseprule}{.1pt}
\setlength{\columnsep}{2cm}
\renewcommand{\headrulewidth}{0pt}
\geometry{top=1in, bottom=1in, left=0.7in, right=0.5in}

\pagestyle{fancy}
\fancyhf{}
\fancyfoot[C]{\thepage}

\begin{document}
	
	\noindent\textbf{6FMA40 - Matemática} 
	
	\begin{center}Exercícios com adição e subtração (Versão estudante)
	\end{center}
	
	\noindent\textbf{Nome:} \underline{\hspace{10cm}}
	\noindent\textbf{Data:} \underline{\hspace{4cm}}
	
	%\section*{Questões de Matemática}
	~ \\ ~
	\begin{multicols}{2}
		\noindent Quando escrevemos $a + b = c$, dizemos que $a$ e $b$ são parcelas e $c$ é a soma. \\
		A soma de dois números positivos é um número positivo. \\
		A soma de dois números negativos é um número negativo. \\
		A soma de um número positivo com um negativo pode ser positiva, negativa ou nula.
	\textsubscript{---------------------------------------------------------------------}
    	\begin{enumerate}
    		\item Até agora temos usado números com módulos "pequenos" ou de fácil manipulação. Vamos mudar um pouco isso. Efetuar:
    		\begin{enumerate}[a)]
    			\item 4179 + 863 \\\\\\\\\\
    			\item -6758 +(-978) \\\\\\\\\\
    			\item -9856 +(-769) \\\\\\
    			\item -88976 +(-7979) \\\\\\\\\\
    		\end{enumerate}
    		\item A soma de dois números inteiros, um deles positivo e outro negativo, é positiva, negativa ou nula? \\\\\\\\\\\\\\\\\\\\
    		\item Efetuar:
    		\begin{enumerate}[a)]
    			\item 517 +(-1203) \\\\\\\\\\
    			\item -2146 + 352 \\\\\\\\
    			\item -585 + 396 \\\\\\\\\\
    			\item 1509 + (-768) \\\\\\\\\\
    			\item 289 + (-607) \\\\\\\\\\
    		\end{enumerate}
    		\item Resuma as respostas das três seguintes perguntas. \\
    		1)"A soma de dois números positivos é um número positivo ou negativo?" \\
    		2)"A soma de sois números negativos é um número positivo ou negativo?" \\
    		3)"A soma de um número positivo com um negativo é positiva, negativa ou nula?" \\\\\\\\\\\\\\\\\\\\\\
    		\item Efetuar:
    		\begin{enumerate}[a)]
    			\item 2403 + 789 \\\\\\\\\\
    			\item 4079 +(-1586) \\\\\\\\\\
    			\item -4532 +(-1283) \\\\\\\\\\
    			\item -2386 + 1497 \\\\\\\\\\
    			\item 36 + (-27) \\\\\\\\\\
    		\end{enumerate}
    	\end{enumerate}
    	$~$ \\ $~$ \\ $~$ \\ $~$ \\ $~$ \\ $~$ \\ $~$ \\ $~$
	\end{multicols}
\end{document}