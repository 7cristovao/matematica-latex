\documentclass[a4paper,14pt]{article}

\usepackage{comment} % Para comentar várias linhas ao mesmo tempo

%matemática
\usepackage{amsmath}
\usepackage{amssymb}

%diagramação
\usepackage{extsizes}
\everymath{\displaystyle}
\usepackage{geometry}
\usepackage{fancyhdr}
\usepackage{multicol}
\usepackage{graphicx}
\usepackage[brazil]{babel}
\usepackage[shortlabels]{enumitem}
\usepackage{cancel}
\usepackage{textcomp}
\usepackage{tcolorbox}

%tabelas
\usepackage{array} % Para melhor formatação de tabelas
\usepackage{longtable}
\usepackage{booktabs}  % Para linhas horizontais mais bonitas
\usepackage{float}   % Para usar o modificador [H]
\usepackage{caption} % Para usar legendas em tabelas
\usepackage{wrapfig} % Para usar tabelas e figuras flutuantes
\usepackage{xcolor} % Para cores do fundo de tabelas
\usepackage{colortbl} % Para cores do fundo de tabelas

%tikzpicture
\begin{comment}
	\usepackage{tikz}
	\usepackage{scalerel}
	\usepackage{pict2e}
	\usepackage{tkz-euclide}
	\usetikzlibrary{calc}
	\usetikzlibrary{patterns,arrows.meta}
	\usetikzlibrary{shadows}
	\usetikzlibrary{external}
\end{comment}


%pgfplots
\usepackage{pgfplots}
\pgfplotsset{compat=newest}
\usepgfplotslibrary{statistics}
\usepgfplotslibrary{fillbetween}

%colours
\usepackage{xcolor}



\columnsep=2cm
\hoffset=0cm
\textwidth=8cm
\setlength{\columnseprule}{.1pt}
\setlength{\columnsep}{2cm}
\renewcommand{\headrulewidth}{0pt}
\geometry{top=1in, bottom=1in, left=0.7in, right=0.5in}

\pagestyle{fancy}
\fancyhf{}
\fancyfoot[C]{\thepage}

\begin{document}
	
	\noindent\textbf{6FMA137 - Matemática} 
	
	\begin{center}Conjunto dos divisores (Versão estudante)
	\end{center}
	
	\noindent\textbf{Nome:} \underline{\hspace{10cm}}
	\noindent\textbf{Data:} \underline{\hspace{4cm}}
	
	%\section*{Questões de Matemática}
	
	\begin{multicols}{2}
	    \noindent Para $a \in \mathbb{Z}$, temos: \\
	    \begin{itemize}
	    	\item $D(a) = \{x \in \mathbb{Z} : x|a\}$
	    	\item $D_+(a) = \{x \in D(a) : x \geq 0\}$
	    	\item $D_-(a) = \{x \in D(a) : x \leq 0\}$
	    	\item $D_+^*(a) = \{x \in D(a) : x > 0\}$
	    	\item $D_-^*(a) = \{x \in D(a) : x < 0\}$
	    \end{itemize}
		\noindent\textsubscript{--------------------------------------------------------------------------}
		\begin{enumerate} 
			\item Apresentar:
			\begin{enumerate}[a)]
				\item $D(3)$ \\\\\\\\\\\\\\
				\item $D(-3)$ \\\\\\\\\\\\\\
				\item $D(5)$ \\\\\\\\
				\item $D(6)$ \\\\\\\\\\\\\\
				\item $D(1)$ \\\\\\\\\\\\\\
				\item $D(0)$ \\\\\\\\\\\\\\
				\item $D(14)$ \newpage
				\item $D(-20)$ \\\\\\\\\\\\\\
			\end{enumerate}
			\item Apresentar:
			\begin{enumerate}[a)]
				\item $D_+(7)$ \\\\\\\\\\\\\\
				\item $D_-(9)$ \\\\\\\\\\\\\\
				\item $D_+(18)$ \\\\\\\\\\\\\\
				\item $D_-(21)$ \\\\\\\\\\
			\end{enumerate}
			\item Quais são os divisores comuns de 24 e 32? \\\\\\\\\\\\\\\\
			%54
			\item Apresentar:
			\begin{enumerate}[a)]
				\item $D(14)$ \\\\\\\\\\\\\\
				\item $D(6)$ \\\\\\\\\\\\\\
				\item $D(-8)$ \newpage
				\item $D_+(10)$ \\\\\\\\\\\\\\
				\item $D_-(7)$ \\\\\\\\\\\\\\
				\item $\mathbb{Z}_+ \cap D^*(12)$ \\\\\\\\\\\\\\
				\item $M(3)$ \\\\\\\\\\\\\\
				\item $M_-^*(5)$ \\\\\\\\\\\\\\
				\item 3$\mathbb{Z}_+^*$ \\\\\\\\\\\\\\
				\item 4$A$, em que \\ $A = \{-3, -1, 3, 6\}$ \\\\\\\\\\\\\\
			\end{enumerate}
		\end{enumerate}
		$~$ \\ $~$ \\ $~$ \\ $~$ \\ $~$ \\ $~$ \\ $~$ \\ $~$ \\ $~$ \\ $~$ \\ $~$ \\ $~$ \\ $~$ \\ $~$ \\ $~$ \\ $~$ \\ $~$ \\ $~$ \\ $~$ \\ $~$ \\ $~$ \\ $~$ \\ $~$ \\ $~$
	\end{multicols}
\end{document}