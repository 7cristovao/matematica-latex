\documentclass[a4paper,14pt]{article}

\usepackage{comment} % Para comentar várias linhas ao mesmo tempo

%matemática
\usepackage{amsmath}
\usepackage{amssymb}

%diagramação
\usepackage{extsizes}
\everymath{\displaystyle}
\usepackage{geometry}
\usepackage{fancyhdr}
\usepackage{multicol}
\usepackage{graphicx}
\usepackage[brazil]{babel}
\usepackage[shortlabels]{enumitem}
\usepackage{cancel}
\usepackage{textcomp}
\usepackage{tcolorbox}

%tabelas
\usepackage{array} % Para melhor formatação de tabelas
\usepackage{longtable}
\usepackage{booktabs}  % Para linhas horizontais mais bonitas
\usepackage{float}   % Para usar o modificador [H]
\usepackage{caption} % Para usar legendas em tabelas
\usepackage{wrapfig} % Para usar tabelas e figuras flutuantes

\begin{comment}
%tikzpicture
\usepackage{tikz}
\usepackage{scalerel}
\usepackage{pict2e}
\usepackage{tkz-euclide}
\usetikzlibrary{calc}
\usetikzlibrary{patterns,arrows.meta}
\usetikzlibrary{shadows}
\usetikzlibrary{external}
\end{comment}
	
%pgfplots
\usepackage{pgfplots}
\pgfplotsset{compat=newest}
\usepgfplotslibrary{statistics}
\usepgfplotslibrary{fillbetween}

%colours
\usepackage{xcolor}



\columnsep=2cm
\hoffset=0cm
\textwidth=8cm
\setlength{\columnseprule}{.1pt}
\setlength{\columnsep}{2cm}
\renewcommand{\headrulewidth}{0pt}
\geometry{top=1in, bottom=1in, left=0.7in, right=0.5in}

\pagestyle{fancy}
\fancyhf{}
\fancyfoot[C]{\thepage}

\begin{document}
	
	\noindent\textbf{6FMA75 - Matemática} 
	
	\begin{center}Revisão: operações com números decimais e potências de 10 (Versão estudante)
	\end{center}
	
	\noindent\textbf{Nome:} \underline{\hspace{10cm}}
	\noindent\textbf{Data:} \underline{\hspace{4cm}}
	
	%\section*{Questões de Matemática}
	
	\begin{multicols}{2}
		\noindent 
		\begin{itemize}
			\item Para somar ou subtrair números decimais, colocamos vírgula debaixo de vírgula, e se as quantidades de casas decimais forem diferentes, igualamos completando com zeros.
			\item Para multiplicarmos números decimais, usamos o algoritmo da multiplicação que já conhecemos, observando que o número de casas decimais do produto é igual à soma das quantidades de casas decimais dos fatores. \\
			Caso a multiplicação seja por 10, 100, 1 000, ..., deslocamos a vírgula, respectivamente, 1, 2, 3, ..., casas para a direita a partir da posição inicial. Para a divisão, o processo é semelhante, porém, deslocando a vírgula para a esquerda.
		\end{itemize}
		\noindent\textsubscript{-----------------------------------------------------------------------}
    	\begin{enumerate}
   			\item Calcule.
   			\begin{enumerate}[a)]
   				\item 2,7 + 0,8 \\\\\\\\\\\\\\\\\\
   				\item 1,3 + 0,4 + 2,5  \\\\\\\\\\\\\\\\\\
   				\item 3,2 + 4,6 + 2,4  \\\\\\\\\\\\\\\\\\
   				\item 9 - 6,2  \\\\\\\\\\\\\\\\\\
   				\item 15,9 - 8,1 \newpage
   				\item 24,4 - 7,9 \\\\\\\\\\\\\\\\\\
   			\end{enumerate} 
   			\item Calcule.
   			\begin{enumerate}[a)]
   				\item 3,24 $\cdot$ 1,2 \\\\\\\\\\\\\\\\\\\\
   				\item 30,15 $\cdot$ 2,4 \\\\\\\\\\\\\\\\\\\\
   				\item 27,01 $\cdot$ 5,6 \\\\\\\\\\\\\\\\\\\\
   				\item 4,26 $\cdot$ 1 000
   				\\\\\\\\\\\\\\\\\\\\
   				\item 20,105 $\cdot$ 10
   				\\\\\\\\\\\\\\\\\\\\
   				\item 351,809 $\cdot$ 100 \newpage
   				\item 94,302 $\cdot$ 100 \\\\\\\\\\\\\\\\\\\\
   				\item 52,0431 : 100
   				\\\\\\\\\\\\\\\\\\\\
   				\item 2,437 : 1000 \\\\\\\\\\\\\\\\\\\\
   			\end{enumerate} 
   			\item Calcule.
   			\begin{enumerate}[a)]
   				\item 13,1 + 8,7 \\\\\\\\
   				\item 6,3 - 1,4 \\\\\\\\\\\\\\\\\\\\
   				\item 8,3 + 3,5 + 5,7 \\\\\\\\\\\\\\\\\\\\
   				\item 11,2 - 7,6
   				\\\\\\\\\\\\\\\\\\\\
   				\item 8,10 $\cdot$ 2,7
   				\newpage
   				\item 17,23 $\cdot$ 4,6 \\\\\\\\\\\\\\\\\\\\
   				\item 2,031 $\cdot$ 100 \\\\\\\\\\\\\\\\\\\\
   				\item 28,09 $\cdot$ 1 000
   				\\\\\\\\\\\\\\\\\\\\
   				\item 431,23 : 100 \\\\\\\\\\\\
   				\item 781,05 : 1 000
   				\\\\\\\\\\\\\\\\\\\\
   			\end{enumerate} 
   		\end{enumerate}
        $~$ \\ $~$ \\ $~$ \\ $~$ \\ $~$ \\ $~$ \\ $~$ \\ $~$ \\ $~$ \\ $~$ \\ $~$ \\ $~$ \\ $~$ \\ $~$ \\ $~$ \\ $~$ \\ $~$ \\ $~$ \\ $~$ \\ $~$ \\ $~$ \\ $~$ \\ $~$ \\ $~$ \\ $~$ \\ $~$ \\ $~$ \\ $~$ \\ $~$ \\ $~$ 
        \end{multicols}
\end{document}