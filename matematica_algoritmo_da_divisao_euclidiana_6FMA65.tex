\documentclass[a4paper,14pt]{article}
\usepackage{float}
\usepackage{extsizes}
\usepackage{amsmath}
\usepackage{amssymb}
\everymath{\displaystyle}
\usepackage{geometry}
\usepackage{fancyhdr}
\usepackage{multicol}
\usepackage{graphicx}
\usepackage[brazil]{babel}
\usepackage[shortlabels]{enumitem}
\usepackage{cancel}
\usepackage{textcomp}
\usepackage{array} % Para melhor formatação de tabelas
\usepackage{longtable}
\usepackage{booktabs}  % Para linhas horizontais mais bonitas
\usepackage{float}   % Para usar o modificador [H]
\usepackage{caption} % Para usar legendas em tabelas
\usepackage{tcolorbox}
\usepackage{wrapfig} % Para usar tabelas e figuras flutuantes

\columnsep=2cm
\hoffset=0cm
\textwidth=8cm
\setlength{\columnseprule}{.1pt}
\setlength{\columnsep}{2cm}
\renewcommand{\headrulewidth}{0pt}
\geometry{top=1in, bottom=1in, left=0.7in, right=0.5in}

\pagestyle{fancy}
\fancyhf{}
\fancyfoot[C]{\thepage}

\begin{document}
	
	\noindent\textbf{6FMA65 - Matemática} 
	
	\begin{center}Algoritmo da divisão euclidiana(Versão estudante)
	\end{center}
	
	\noindent\textbf{Nome:} \underline{\hspace{10cm}}
	\noindent\textbf{Data:} \underline{\hspace{4cm}}
	
	%\section*{Questões de Matemática}
	\begin{multicols}{2}
		\noindent Na divisão
		\begin{table}[H]
			\centering
			\begin{tabular}{ll}
				\multicolumn{1}{l|}{a} & b \\ \cline{2-2} 
				r                      & z
			\end{tabular}
		\end{table}
		com $a$ e $b$ naturais, temos: \\
		$a$ é o dividendo, $b$ é o divisor, $q$ é o quociente e $r$ é o resto, com $r$ < $b$. \\
		Além disso, temos que $a = b \cdot q + r$. \\
		Exemplo: \begin{table}[H]
			\centering
			\begin{tabular}{ll}
				\multicolumn{1}{l|}{17} & 5 \\ \cline{2-2} 
				2                      & 3 \\\\
			\end{tabular} \\
			\centering $17 = 3 \cdot 5 + 2$
		\end{table} 
		Quando $r = 0$, dizemos que $a$ é divisível por $b$
    	\textsubscript{---------------------------------------------------------------------}
    	\begin{enumerate}
   			\item Complete cada uma das divisões euclidianas abaixo, calculando o quociente e o resto. \\
   			\begin{enumerate}[a)]
   				\item ~ \\
    			\begin{tabular}{ll}
    			\multicolumn{1}{l|}{28} & 6 \\ \cline{2-2} 
    			~                      & ~ \\\\
    			\end{tabular} \\\\
    			\item ~ \\
    			\begin{tabular}{ll}
    				\multicolumn{1}{l|}{10} & 3 \\ \cline{2-2} 
    				~                      & ~ \\\\
    			\end{tabular} \\\\
    			\item ~ \\
    			\begin{tabular}{ll}
    				\multicolumn{1}{l|}{30} & 7 \\ \cline{2-2} 
    				~                      & ~ \\\\
    			\end{tabular} \\\\
    			\item ~ \\
    			\begin{tabular}{ll}
    				\multicolumn{1}{l|}{18} & 3 \\ \cline{2-2} 
    				~                      & ~ \\\\
    			\end{tabular} \\\\
    			\item ~ \\
    			\begin{tabular}{ll}
    				\multicolumn{1}{l|}{53} & 8 \\ \cline{2-2} 
    				~                      & ~ \\\\
    			\end{tabular} \\\\
    			\item ~ \\
    			\begin{tabular}{ll}
    				\multicolumn{1}{l|}{34} & 6 \\ \cline{2-2} 
    				~                      & ~ \\\\
    			\end{tabular} \\\\
    			\item ~ \\
    			\begin{tabular}{ll}
    				\multicolumn{1}{l|}{29} & 4 \\ \cline{2-2} 
    				~                      & ~ \\\\
    			\end{tabular} \\\\
    			\item ~ \\
    			\begin{tabular}{ll}
    				\multicolumn{1}{l|}{45} & 6 \\ \cline{2-2} 
    				~                      & ~ \\\\
    			\end{tabular} \newpage
    			\item ~ \\
    			\begin{tabular}{ll}
    				\multicolumn{1}{l|}{74} & 8 \\ \cline{2-2} 
    				~                      & ~ \\\\
    			\end{tabular} \\\\
   			\end{enumerate}
   			\item Quando dividimos um número natural por 8, quais restos podem aparecer? \\\\\\\\\\\\\\
   			\item Susana comprou dois cachos de uma para fazer um piquenique com suas sete amigas. Ela retirou todas as uvas dos cachos, obtendo 62. Como Susana deve fazer a distribuição das uvas de forma que todas as meninas - inclusive ela - recebam a mesma quantidade, a maior possível? \\\\\\\\\\\\\\
   			\item Dona Suzana tem uma coleção de 61 quadros e quer dá-los a seus 8 sobrinhos. Ela gostaria de dar a mesma quantidade de quadros para cada um, mas como isso não é possível, ela fez o seguinte: distribuiu os quadros da maneira mais justa possível e o que sobrou distribuiu igualmente entre os 5 sobrinhos mais velhos. De que maneira ela fez a distribuição? \\\\\\\\\\\\\\
   			% 55 a 58
   			\item Utilizando o algoritmo da divisão euclidiana, determine o quociente e o resto da divisão de:
   			\begin{enumerate}[a)]
   				\item 243 por 7. \\\\\\\\\\\\\\
   				\item 326 por 8. \\\\\\\\\\\\\\
   				\item 417 por 9. \newpage
   			\end{enumerate}
   			\item Seu Marcelo queria dividir 209 reais para seus três filhos. Como 209 não é divisível por 3, ele calculou o quociente e o resto da divisão, cabendo a cada filho o maior quociente possível. O resto ele usou para comprar biscoitos ao preço de 3 por um real. \\ Quantos biscoitos Seu Marcelo comprou?  \\\\\\\\\\\\\\\\\\\\\\
   			\item Alice está em Londres e decide comprar algumas camisetas. Se ela comprasse 4 camisetas de 16 libras cada uma, lhe sobrariam 8 libras. Mas sua irmã lhe emprestou 22 libras e, com o total, Alice comprou uma camiseta de 16 libras e 3 vestidos longos. Quanto custou cada vestido longo? \\\\\\\\\\\\\\\\\\\\
   			\item Um número dividido por 17 dá quociente 46 e resto 5. Qual é esse número?
        \end{enumerate}
        $~$ \\ $~$ \\ $~$ \\ $~$ \\ $~$ \\ $~$ \\ $~$ \\ $~$ \\ $~$ \\ $~$ \\ $~$ \\ $~$ \\ $~$  \\ $~$ \\ $~$ \\ $~$ \\ $~$ \\ $~$ \\ $~$ \\ $~$ \\ $~$ \\ $~$ \\ $~$ \\ $~$ \\ $~$ \\ $~$ \\ $~$ \\ $~$ \\ $~$ \\ $~$ \\ $~$ \\ $~$ \\ $~$ \\ $~$ \\ $~$ \\ $~$ \\ $~$ \\
	\end{multicols}
\end{document}