\documentclass[a4paper,14pt]{article}
\usepackage{extsizes}
\usepackage{amsmath}
\usepackage{amssymb}
\everymath{\displaystyle}
\usepackage{geometry}
\usepackage{fancyhdr}
\usepackage{multicol}
\usepackage{graphicx}
\usepackage[brazil]{babel}
\usepackage[shortlabels]{enumitem}
\usepackage{cancel}
\columnsep=2cm
\hoffset=0cm
\textwidth=8cm
\setlength{\columnseprule}{.1pt}
\setlength{\columnsep}{2cm}
\renewcommand{\headrulewidth}{0pt}
\geometry{top=1in, bottom=1in, left=0.7in, right=0.5in}

\pagestyle{fancy}
\fancyhf{}
\fancyfoot[C]{\thepage}

\begin{document}
	
	\noindent\textbf{7FMA148~-~Matemática} 
	
	\begin{center}
		\textbf{Potências de expoentes negativos (Versão estudante)}
	\end{center}
	
	
	\noindent\textbf{Nome:} \underline{\hspace{10cm}}
    \noindent\textbf{Data:} \underline{\hspace{4cm}}
	
	%\section*{Questões de Matemática}
	
	\begin{multicols}{2}
		%
	\begin{enumerate}	
		\item Escrever usando radical:
		\begin{enumerate}[a)]
			\item $12^{-\frac{1}{4}}$ \\\\\\
			\item $5^{-\frac{5}{6}}$ \\\\\\
			\item $(-7)^{-\frac{8}{3}}$ \\\\\\
	    \end{enumerate}
        \item Escrever usando expoente racional:
        \begin{enumerate}[a)]
        	\item $\sqrt[-5]{9}$ \\
        	\item $\sqrt[9]{(-4)^{-5}}$ \\\\\\
        	\item $\sqrt[-5]{(-6)^{10}}$ \\\\\\
        	\item $\frac{1}{\sqrt{4^5}}$ \\\\\\
        	\item $\sqrt[7]{9}$ \\\\\\
        \end{enumerate}
        \item Calcule o que for possível com o que sabemos até agora:
        \begin{enumerate}[a)] 
        	\item $64^\frac{1}{2}$ \\\\\\
        	\item $(-8)^\frac{1}{3}$ \\\\\\
        	\item $0^\frac{31}{72}$ \\\\\\
        	\item $1^{-\frac{5}{7}}$ \\\\\\
        \end{enumerate}
        \item Calcule:
        \begin{enumerate}[a)]
        	\item $(5^{-1} + 3^{-1})^{-1}$ \\\\\\
        	\item $\bigg(\bigg(\frac{1}{5}\bigg)^{-1} + \bigg(\frac{1}{3}\bigg)^{-1}\bigg)^{-1}$ \\\\\\
        	\item $\bigg(5^{-\frac{1}{2}} + 3^{-\frac{1}{2}}\bigg)^{-2}$ 
        \end{enumerate}
        \item Escreva usando radical:
        \begin{enumerate}[a)]
        	\item $6^{-\frac{2}{5}}$ \\
        	\item $(-3)^{-\frac{2}{5}}$ \\
        \end{enumerate}
        \item Escreva na forma de expoente racional:
        \begin{enumerate}[a)]
        	\item $\frac{1}{\sqrt[11]{3^4}}$ \\
        	\item $\frac{1}{\sqrt[3]{(-5)^2}}$ \\
        \end{enumerate}
        \item Calcule, usando a definição de expoente racional, o valor de:
        \begin{enumerate}
        	\item $9^\frac{1}{2}$ \\
        	\item $8^\frac{1}{3}$ \\
        	\item $(-27)^\frac{1}{3}$ \\
        	\item $1^\frac{53}{22}$ \\
        	\item $0^\frac{5}{12}$ \\
        	\item $(-1)^\frac{2}{19}$ \\
        	\item $81^\frac{1}{4}$ \\
        	\item $81^{-\frac{1}{4}}$ \\
        	\item $1^{-\frac{13}{47}}$ \\
            \item $5^\frac{3}{2}$ \\
            \item $5^{-\frac{3}{2}}$ \\
            \item $27^\frac{2}{3}$ \\
            \item $(-27)^\frac{2}{3}$ \\
            \item $(-27)^{-\frac{2}{3}}$ \\
            \item $125^{-\frac{2}{3}}$ \\
        \end{enumerate}
        $~$ \\ $~$ \\ $~$ \\ $~$ \\ $~$ \\ $~$ \\ $~$ \\ $~$ \\ $~$ \\ $~$ \\ $~$ \\ $~$ \\ $~$ \\ $~$ \\ $~$ \\ $~$ \\ $~$ \\ $~$ \\ $~$ \\ $~$ \\ $~$ \\ $~$ \\ $~$ \\ $~$ \\ $~$ \\ $~$ \\ $~$ \\ $~$ \\ $~$ \\ $~$ \\ $~$ \\
    \end{enumerate}        
    \end{multicols}
\end{document}