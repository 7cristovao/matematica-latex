\documentclass[a4paper,14pt]{article}
\usepackage{extsizes}
\usepackage{amsmath}
\usepackage{amssymb}
\everymath{\displaystyle}
\usepackage{geometry}
\usepackage{fancyhdr}
\usepackage{multicol}
\usepackage{graphicx}
\usepackage[brazil]{babel}
\usepackage[shortlabels]{enumitem}
\usepackage{cancel}
\columnsep=2cm
\hoffset=0cm
\textwidth=8cm
\setlength{\columnseprule}{.1pt}
\setlength{\columnsep}{2cm}
\renewcommand{\headrulewidth}{0pt}
\geometry{top=1in, bottom=1in, left=0.7in, right=0.5in}

\pagestyle{fancy}
\fancyhf{}
\fancyfoot[C]{\thepage}

\begin{document}
	
	\noindent\textbf{8FMA15~-~Matemática} 
	
	\begin{center}Método da Substituição (Versão estudante)
	\end{center}
	
	
	\noindent\textbf{Nome:} \underline{\hspace{10cm}}
	\noindent\textbf{Data:} \underline{\hspace{4cm}}
	
	%\section*{Questões de Matemática}
	
	\begin{multicols}{2}
	    \begin{enumerate}
	    \item Resolver os seguintes sistemas no universo $U = \mathbb{R^2}$, usando o método da substituição.
		    \begin{enumerate}[a)]
		    	\item ~ 
			    $
			    \begin{cases}
			    	x + y = 5 \\
			    	x - y = 3
			    \end{cases}
		        $\\\\\\\\\\\\\\\\\\\\\\\\\\\\\\\\
		        \item ~ 
		        $
		        \begin{cases}
		        	x + y = -6\\
		        	-x + 3y = 10
		        \end{cases}
		        $\\\\\\\\\\\\\\\\\\\\\\\\\\\\\\\\
		        \item ~ 
		        $
		        \begin{cases}
		        	x - y = 0\\
		        	x + 3y = 16
		        \end{cases}
		        $\\\\\\\\\\\\\\\\\\\\\\\\\\\\\\\\
		        \item ~ 
		        $
		        \begin{cases}
		        	x + y = 1\\
		        	y - 3x = 9
		        \end{cases}
		        $\\\\\\\\\\\\\\\\\\\\\\\\
		        \item ~ 
		        $
		        \begin{cases}
		        	a + 2b = 2\\
		        	4a - 6b = 1
		        \end{cases}
		        $\\\\\\\\\\\\\\\\\\\\\\\\\\\\\\\\\\\\\\\\
		        \item ~ 
		        $
		        \begin{cases}
		        	2x - 5y = 0\\
		        	6x + 4y = 6
		        \end{cases}
		        $\\\\\\\\\\\\\\\\\\\\\\\\\\\\\\\\\\
		    \end{enumerate}
	    \item Resolva os seguintes sistemas no universo $U = \mathbb{R}^2$.
		    \begin{enumerate}[a)]
		    	\item $\begin{cases}
		    		5x - 4y = 0 \\
		    		5x + 3y = 4
		    	\end{cases}$\\\\\\\\\\\\\\\\\\\\\\\\\\\\\\\\\\
	    	    \item $\begin{cases}
	    	    	-3x + 4y = 1 \\
	    	    	3x - 2x = 6
	    	    \end{cases}$\\\\\\\\\\\\\\\\\\\\\\\\\\\\\\\\\\
	            \item $\begin{cases}
	            	3a - 5b = 4 \\
	            	a - 2b = -5
	            \end{cases}$\\\\\\\\\\\\\\\\\\\\\\\\\\\\\\\\
	            \item $\begin{cases}
	            	3x + 7y = -2 \\
	            	-3x + 4y = 6
	            \end{cases}$\\\\\\\\\\\\\\\\\\\\\\\\\\\\\\\\\\\\\\\\
	            \item $\begin{cases}
	            	-x + 3y = -2 \\
	            	-4x - 2y = 7
	            \end{cases}$\\\\\\\\\\\\\\\\\\\\\\\\\\\\\\\\\\\\\\
	            \item $\begin{cases}
	            	-u + 3v = 1 \\
	            	3u - v = 5
	            \end{cases}$\\\\\\\\\\\\\\\\\\\\\\\\\\\\\\\\\\\\
		    \end{enumerate}
	    \item Se $x$ e $y$ são números reais que satisfazem simultaneamente as equações $5x + 6y = 18$ e $3x - 2y = 1$, então $y - x$ é:
	        \begin{enumerate}[a)]
	        	\item $\frac{1}{4}$
	        	\item $\frac{2}{3}$
	        	\item $\frac{5}{2}$
	        	\item $\frac{13}{4}$
	        	\item $\frac{19}{4}$\\\\\\\\\\\\\\\\\\\\\\\\\\\\\\\\\\\\\\\\\\\\\\\\\\\\\\
	        \end{enumerate}
        \item (Enem) Durante uma festa de colégio, um grupo de alunos organizou uma rifa. Oitenta alunos faltaram à festa e não participaram da rifa. Entre os que compareceram, alguns compraram três bilhetes, 45 compraram 2 bilhetes, e muitos compraram apenas um. O total de alunos que compraram um único bilhete foi 20\% do número total de bilhetes vendidos, e o total de bilhetes vendidos excedeu em 33 o número total de alunos do colégio. Quantos alunos compraram somente um bilhete?
	        \begin{enumerate}[a)]
	        	\item $34$
	        	\item $42$
	        	\item $47$
	        	\item $48$
	        	\item $79$
	        \end{enumerate}	
        $~$ \\ $~$ \\ $~$ \\ $~$ \\ $~$ \\ $~$ \\ $~$ \\ $~$ \\ $~$ \\ $~$ \\ $~$ \\ $~$ \\ $~$ \\  $~$ \\  $~$ \\  $~$ \\  $~$ \\
	    \end{enumerate}
    \end{multicols}

\end{document}