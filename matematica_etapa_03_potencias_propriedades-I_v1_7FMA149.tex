\documentclass[a4paper,14pt]{article}
\usepackage{extsizes}
\usepackage{amsmath}
\usepackage{amssymb}
\everymath{\displaystyle}
\usepackage{geometry}
\usepackage{fancyhdr}
\usepackage{multicol}
\usepackage{graphicx}
\usepackage[brazil]{babel}
\usepackage[shortlabels]{enumitem}
\usepackage{cancel}
\columnsep=2cm
\hoffset=0cm
\textwidth=8cm
\setlength{\columnseprule}{.1pt}
\setlength{\columnsep}{2cm}
\renewcommand{\headrulewidth}{0pt}
\geometry{top=1in, bottom=1in, left=0.7in, right=0.5in}

\pagestyle{fancy}
\fancyhf{}
\fancyfoot[C]{\thepage}

\begin{document}
	
	\noindent\textbf{7FMA149~-~Matemática} 
	
	\begin{center}
		\textbf{Potências - Propriedades (I) (Versão estudante)}
	\end{center}
	
	
	\noindent\textbf{Nome:} \underline{\hspace{10cm}}
    \noindent\textbf{Data:} \underline{\hspace{4cm}}
	
	%\section*{Questões de Matemática}
	
	\begin{multicols}{2}
		Para $a > 0$, valem as seguintes propriedades: \\
		P1. $a^m \cdot a^n = a^{m + n}$ (vale também para mais de dois fatores) \\
		P2. $\frac{a^m}{a^n} = a^{m - n}$ \\
		P3. $(a^m)^n = a^{m \cdot n}$ e $\sqrt[n]{\sqrt[m]{a}} = \sqrt[m \cdot n]{a}$ \\
		Se $a < 0$, verificar sinal primeiro.
	\begin{enumerate}
        \item Efetue os produtos a seguir, aplicando a propriedade P1 e escrevendo o resultado na forma de expoente racional:
        \begin{enumerate}[a)]
        	\item $2^\frac{1}{2} \cdot 2^\frac{1}{5}$ \\\\\\
        	\item $(-5)^\frac{1}{4} \cdot (-5)^\frac{1}{3}$ \\\\\\
        	\item $\pi^\frac{2}{5} \cdot \pi^\frac{1}{6}$ \\\\\\
        	\item $\bigg(\frac{1}{3}\bigg)^{\frac{7}{11}} \cdot \bigg(\frac{1}{3}\bigg)^{\frac{1}{11}} \cdot  \bigg(\frac{1}{3}\bigg)^{\frac{3}{11}}$\\\\\\\\\\
        \end{enumerate}
        \item Calcule, obtendo o resultado na forma de $\sqrt{~~}$ e simplificando o que for possível:
        \begin{enumerate}[a)]
        	\item $\sqrt[5]{3} \cdot \sqrt[7]{3}$ \\\\\\
        	\item $\sqrt[3]{-7} \cdot \sqrt[11]{-7}$ \\\\\\
        	\item $\sqrt[9]{-8} \cdot \sqrt[7]{-8} \cdot \sqrt[5]{-8}$ \\\\\\
        	\item $\sqrt[4]{6^5} \cdot \sqrt[4]{6^2}$
        \end{enumerate}
        \item Obtenha os quocientes na forma $a^\frac{m}{n}$.
        \begin{enumerate}[a)]
        	\item $\frac{5^\frac{5}{6}}{5^\frac{11}{12}}$ \\\\\\
        	\item $\frac{(-3)^\frac{7}{5}}{(-3)^\frac{2}{3}}$ \\\\\\
        	\item $\frac{\pi^{-\frac{1}{3}}}{\pi^{-\frac{5}{7}}}$ \\\\\\
        \end{enumerate}
        \item Calcule o valor da expressão a seguir, obtendo o resultado na forma de potência com expoente racional:
        \begin{enumerate}[a)]
        	\item $\frac{2^\frac{2}{3} \cdot 2^\frac{4}{5}}{2^\frac{7}{3} \cdot 2^{-\frac{3}{5}}} \cdot 2^\frac{8}{8}$
        \end{enumerate}
        \item Calcule, obtendo o resultado na forma de raiz:
        \begin{enumerate}[a)]
        	\item $\frac{\sqrt[4]{7}}{\sqrt[5]{7}}$ \\\\\\
        	\item $\frac{\sqrt[8]{3}}{\sqrt[9]{3}}$ \\\\\\
        \end{enumerate}
        \item Calcule:
        \begin{enumerate}[a)]
            \item $\frac{\sqrt[3]{3} \cdot \sqrt[4]{3}}{\sqrt[5]{3^2} \cdot \sqrt[4]{3^3}}$ \\\\\\
            \item Agora é a sua vez. Apresente uma expressão como a do item anterior e depois calcule-a. \\\\\\
        \end{enumerate}
        \item Calcule:
        \begin{enumerate}[a)]
        	\item $\frac{\sqrt[5]{0^{11}} + \sqrt[9]{(-1)^4} + \sqrt{2^4}}{2 \cdot \sqrt[3]{27}- \sqrt[12]{1}}$ \\\\\\
        	\item $\frac{1^\frac{11}{2} - 0^\frac{3}{4} + 16^\frac{1}{2} - 27^\frac{1}{3}}{8^\frac{1}{3} \bigg(125^\frac{1}{3} - 16^\frac{1}{4} \bigg)}$ \\\\\\
        	\item $27^{0,333...} - 4^{0,5}$
       \end{enumerate}
       \item Escreva o resultado na forma de potência de base real e expoente racional:
       \begin{enumerate}[a)]
           \item $8^\frac{1}{2} \cdot 8^\frac{4}{3}$ \\
           \item $9^\frac{5}{4} \cdot 3^\frac{7}{3} \cdot 27^\frac{1}{5}$ \\
           \item $\sqrt{2} \cdot \sqrt[7]{2} \cdot \sqrt[3]{2}$ \\
           \item $\frac{7^\frac{2}{9}}{7^\frac{7}{9}}$ \\
           \item $\frac{\sqrt{10}}{\sqrt[3]{100}}$ \\
           \item $\bigg((-1)^\frac{4}{11}\bigg)^\frac{22}{7}$ \\
           \item $\sqrt[4]{\sqrt[~]{\sqrt[10]{\sqrt[3]{5}}}}$ \\
       \end{enumerate}
       \item Calcule:
       \begin{enumerate}[a)]
           \item $\frac{4^\frac{5}{7} \cdot 4^\frac{3}{4} \cdot \bigg(4^\frac{1}{4} \bigg)^\frac{8}{7}}{\sqrt[4]{4} \cdot \sqrt[7]{4}}$
       \end{enumerate}
       \item Calcule sem passar para expoente fracionário:
       \begin{enumerate}[a)]
       	\item $\sqrt[3]{7} \cdot \sqrt[5]{7}$
       	\item $\sqrt[3]{5^2} \cdot \sqrt[7]{5^6}$
       	\item $\frac{\sqrt[6]{3^{11}}}{\sqrt[4]{3^7}}$
       	\item $\sqrt[5]{6} \cdot \sqrt[3]{6} \cdot \sqrt{6}$
       \end{enumerate}
       \item A expressão $\left\{ \left[ y^\frac{7}{2} \cdot \bigg( \frac{\sqrt[7]{y^3}}{\sqrt[3]{y}} \bigg)^7 \right]^\frac{1}{2} \right\}^\frac{3}{5}$ \\ é igual a:
       \begin{enumerate}[a)]
       	\item $y^5$ \item $y^\frac{37}{10}$ \item $y^\frac{5}{4}$ \item $y^\frac{23}{20}$ \item $y^3$
       \end{enumerate}
       $~$ \\ $~$ \\ $~$ \\ $~$ \\ $~$ \\ $~$ \\ $~$ \\ $~$ \\ $~$ \\ $~$ \\ $~$ \\ $~$ \\ $~$ \\ $~$ \\ $~$ \\ $~$ \\ $~$ \\ $~$ \\ $~$ \\ $~$ \\ $~$ \\ $~$ \\ $~$ \\ $~$ \\ $~$ \\ $~$ \\ $~$ \\ $~$ \\ $~$ \\ $~$ \\ $~$ \\ $~$ \\ $~$ \\ $~$ \\ $~$ \\ $~$ \\ $~$ \\ $~$ \\ $~$ \\ $~$ \\ $~$ \\ $~$ \\ $~$ \\ $~$ \\ $~$ \\ $~$ \\ $~$ \\ $~$ \\ $~$ \\ $~$ \\ $~$ \\ $~$ \\ $~$ \\ $~$ \\ $~$ \\ $~$ \\ $~$ \\ $~$ \\ $~$ \\ $~$ \\ $~$ \\ $~$ \\ $~$ \\ $~$ \\ $~$ \\ $~$ \\ $~$ \\ $~$ \\ 
       
       
    \end{enumerate}        
    \end{multicols}
\end{document}