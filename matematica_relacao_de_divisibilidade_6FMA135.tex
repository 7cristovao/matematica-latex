\documentclass[a4paper,14pt]{article}

\usepackage{comment} % Para comentar várias linhas ao mesmo tempo

%matemática
\usepackage{amsmath}
\usepackage{amssymb}

%diagramação
\usepackage{extsizes}
\everymath{\displaystyle}
\usepackage{geometry}
\usepackage{fancyhdr}
\usepackage{multicol}
\usepackage{graphicx}
\usepackage[brazil]{babel}
\usepackage[shortlabels]{enumitem}
\usepackage{cancel}
\usepackage{textcomp}
\usepackage{tcolorbox}

%tabelas
\usepackage{array} % Para melhor formatação de tabelas
\usepackage{longtable}
\usepackage{booktabs}  % Para linhas horizontais mais bonitas
\usepackage{float}   % Para usar o modificador [H]
\usepackage{caption} % Para usar legendas em tabelas
\usepackage{wrapfig} % Para usar tabelas e figuras flutuantes
\usepackage{xcolor} % Para cores do fundo de tabelas
\usepackage{colortbl} % Para cores do fundo de tabelas

%tikzpicture
\begin{comment}
	\usepackage{tikz}
	\usepackage{scalerel}
	\usepackage{pict2e}
	\usepackage{tkz-euclide}
	\usetikzlibrary{calc}
	\usetikzlibrary{patterns,arrows.meta}
	\usetikzlibrary{shadows}
	\usetikzlibrary{external}
\end{comment}


%pgfplots
\usepackage{pgfplots}
\pgfplotsset{compat=newest}
\usepgfplotslibrary{statistics}
\usepgfplotslibrary{fillbetween}

%colours
\usepackage{xcolor}



\columnsep=2cm
\hoffset=0cm
\textwidth=8cm
\setlength{\columnseprule}{.1pt}
\setlength{\columnsep}{2cm}
\renewcommand{\headrulewidth}{0pt}
\geometry{top=1in, bottom=1in, left=0.7in, right=0.5in}

\pagestyle{fancy}
\fancyhf{}
\fancyfoot[C]{\thepage}

\begin{document}
	
	\noindent\textbf{6FMA135 - Matemática} 
	
	\begin{center}Relação de divisibilidade (Versão estudante)
	\end{center}
	
	\noindent\textbf{Nome:} \underline{\hspace{10cm}}
	\noindent\textbf{Data:} \underline{\hspace{4cm}}
	
	%\section*{Questões de Matemática}
	
	\begin{multicols}{2}
	    \noindent Sejam $a, b$ e $c$ números inteiros. Temos: \\
	    $a | b \Leftrightarrow \exists$$c \in \mathbb{Z}$ t.q. $b = a \cdot c$
		\noindent\textsubscript{--------------------------------------------------------------------------}
		\begin{enumerate} 
			\item Assinale \textbf{V} (verdadeiro) ou \textbf{F} (falso).
			\begin{enumerate}[a)]
				\item (~~) $4 | 12$
				\item (~~) $-4 | 12$
				\item (~~) $4 | -12$
				\item (~~) $-4 | -12$
				\item (~~) $8 | 0$
				\item (~~) $0 | 5$
				\item (~~) $6 | 4$
				\item (~~) $4 \nmid 6$
				\item (~~) $10 | 2$
				\item (~~) $2 | 6$
			\end{enumerate}
			\item Observe a divisão a seguir: \\
			\begin{table}[H]
				\centering
				\begin{tabular}{ll}
					\multicolumn{1}{l|}{$a$} & $b$ \\ \cline{2-2} 
					$r$                      & $q$
				\end{tabular}
			\end{table}
			Você já viu que a nomenclatura correta dos termos é: \\
			$a \rightarrow$ dividendo \\
			$b \rightarrow$ divisor \\
			$q \rightarrow$ quociente \\
			$r \rightarrow$ resto \\
			Identifique esses termos nas divisões a seguir: \\\\
				a) \begin{table}[H]
					\centering
					\begin{tabular}{ll}
						\multicolumn{1}{l|}{22} & 4 \\ \cline{2-2} 
						2                      & 5 
					\end{tabular}
				\end{table}
				b) \begin{table}[H]
					\centering
					\begin{tabular}{ll}
						\multicolumn{1}{l|}{17} & 7 \\ \cline{2-2} 
						3                      & 2 
					\end{tabular}
				\end{table}
				c) \begin{table}[H]
					\centering
					\begin{tabular}{ll}
						\multicolumn{1}{l|}{25} & 6 \\ \cline{2-2} 
						1                      & 4 
					\end{tabular}
				\end{table}
			\item Considerando a divisão do exercício anterior, podemos escrevê-la como $a = b \cdot q + r$ sempre que $0 \leq r < |b|$. Escreva as divisões do exercício anterior dessa maneira. \\\\\\\\\\\\\\\\\\\\\\\\\\
			\item Para todo número inteiro $a$, pode-se afirmar que:
			\begin{enumerate}[a)]
				\item $1 | a$? \\\\\\\\\\\\
				\item $-1 | a$? \\\\\\\\\\\\
				\item $a | a^2$? \\\\\\\\\\\\
				\item $a | (a + 1)$? \\\\\\\\\\\\
			\end{enumerate}
			%45 a 48
			\item Assinale \textbf{V} (verdadeiro) ou \textbf{F} (falso).
			\begin{enumerate}[a)]
				\item (~~) $3 | 9$
				\item (~~) $-4 | 16$
				\item (~~) $12 | 0$
				\item (~~) $-2 | -26$
				\item (~~) $12 | 4$
				\item (~~) $6 | 3$
				\item (~~) $2 \nmid 5$
				\item (~~) $5 \nmid 10$
			\end{enumerate}
			\item Escreva as seguintes expressões na forma de divisão:
			\begin{enumerate}[a)]
				\item $12 = 3 \cdot 4 + 0$ \\\\\\\\\\\\
				\item $15 = 2 \cdot 7 + 1$ \\\\\\\\\\\\
				\item $23 = 7 \cdot 3 + 2$ \\\\\\\\\\\\
				\item $27 = 4 \cdot 6 + 3$ \\\\\\\\\\\\
				\item $2 = 1 \cdot 2 + 0$ \\\\\\\\
				\item $5 = 2 \cdot 2 + 1$ \\\\\\\\\\\\
			\end{enumerate}
			\item Assinale \textbf{V} (verdadeiro) ou \textbf{F} (falso).
			\begin{enumerate}[a)]
				\item (~~) $-2 | -8$
				\item (~~) $-7 | 0$
				\item (~~) $-7 | 28$
				\item (~~) $0 | 5$
				\item (~~) $13 | 53$
				\item (~~) $3 \nmid 7$
				\item (~~) $19 \nmid 115$
				\item (~~) $17 \nmid 2091$
				\item (~~) $-4 \nmid 4^0$
			\end{enumerate}
			\item A seguinte divisão está correta? Justifique.
			\\
			\begin{table}[H]
				\centering
				\begin{tabular}{ll}
					\multicolumn{1}{l|}{169} & 4 \\ \cline{2-2} 
					5                      & 41 
				\end{tabular}
			\end{table}
		\end{enumerate}
		$~$ \\ $~$ \\ $~$ \\ $~$ \\ $~$ \\ $~$ \\ $~$ \\ $~$ \\ $~$ \\ $~$ \\ $~$ \\ $~$ \\ $~$ \\ $~$ \\ $~$ \\ $~$ \\ $~$ \\ $~$ \\ $~$ \\ $~$ \\ $~$ \\ $~$ \\ $~$ \\ $~$ \\ $~$ \\ $~$ \\ $~$ \\ $~$ \\ $~$ \\ $~$ \\ $~$ \\ $~$ \\ $~$ \\ $~$ \\ $~$ \\ $~$ \\ $~$ \\ $~$ \\ $~$ \\ $~$ \\ $~$ \\ $~$ \\ $~$ \\ $~$ \\ $~$ \\ $~$ \\ $~$ \\ $~$ \\ $~$ \\ $~$ \\ $~$ \\ $~$ \\ $~$
	\end{multicols}
\end{document}