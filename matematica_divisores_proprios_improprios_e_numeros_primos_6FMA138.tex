\documentclass[a4paper,14pt]{article}

\usepackage{comment} % Para comentar várias linhas ao mesmo tempo

%matemática
\usepackage{amsmath}
\usepackage{amssymb}

%diagramação
\usepackage{extsizes}
\everymath{\displaystyle}
\usepackage{geometry}
\usepackage{fancyhdr}
\usepackage{multicol}
\usepackage{graphicx}
\usepackage[brazil]{babel}
\usepackage[shortlabels]{enumitem}
\usepackage{cancel}
\usepackage{textcomp}
\usepackage{tcolorbox}

%tabelas
\usepackage{array} % Para melhor formatação de tabelas
\usepackage{longtable}
\usepackage{booktabs}  % Para linhas horizontais mais bonitas
\usepackage{float}   % Para usar o modificador [H]
\usepackage{caption} % Para usar legendas em tabelas
\usepackage{wrapfig} % Para usar tabelas e figuras flutuantes
\usepackage{xcolor} % Para cores do fundo de tabelas
\usepackage{colortbl} % Para cores do fundo de tabelas

%tikzpicture
\begin{comment}
	\usepackage{tikz}
	\usepackage{scalerel}
	\usepackage{pict2e}
	\usepackage{tkz-euclide}
	\usetikzlibrary{calc}
	\usetikzlibrary{patterns,arrows.meta}
	\usetikzlibrary{shadows}
	\usetikzlibrary{external}
\end{comment}


%pgfplots
\usepackage{pgfplots}
\pgfplotsset{compat=newest}
\usepgfplotslibrary{statistics}
\usepgfplotslibrary{fillbetween}

%colours
\usepackage{xcolor}



\columnsep=2cm
\hoffset=0cm
\textwidth=8cm
\setlength{\columnseprule}{.1pt}
\setlength{\columnsep}{2cm}
\renewcommand{\headrulewidth}{0pt}
\geometry{top=1in, bottom=1in, left=0.7in, right=0.5in}

\pagestyle{fancy}
\fancyhf{}
\fancyfoot[C]{\thepage}

\begin{document}
	
	\noindent\textbf{6FMA138 - Matemática} 
	
	\begin{center}Divisores próprios, impróprios e números primos (Versão estudante)
	\end{center}
	
	\noindent\textbf{Nome:} \underline{\hspace{10cm}}
	\noindent\textbf{Data:} \underline{\hspace{4cm}}
	
	%\section*{Questões de Matemática}
	
	\begin{multicols}{2}
	    \noindent 
	    \begin{itemize}
	    	\item Se $a \in \mathbb{Z}$, então 1, -1, $a$ e $-a$ são divisores impróprios de $a$. O restante dos fatores de $a$ são divisores próprios de $a$.
	    	\item Um número $p \in \mathbb{Z}$ é dito \textbf{primo} se possui exatamente quatro divisores inteiros distintos.
	    	\item Seja $a \in \mathbb{Z}$: \\
	    	\centering $a$ é composto \\
	    	$\Updownarrow$ \\
	    	$\begin{cases}
	    		\text{i.}~a~\neq~0~ \text{e}~a~\neq~1 \text{e}~a~\neq~-1 \\
	    		\text{ii.}~a~\text{admite pelo menos um divi-} \\
	    		\text{sor próprio}
	    	\end{cases}$
	    \end{itemize}
		\noindent\textsubscript{--------------------------------------------------------------------------}
		\begin{enumerate} 
			\item Determine o que se pede:
			\begin{enumerate}[a)]
				\item Os divisores próprios de 8. \\\\\\\\\\\\\\\\
				\item Os divisores impróprios de 8.  \\\\\\\\\\\\
				\item Os divisores próprios de 18. \\\\\\\\\\\\\\\\
				\item Os divisores impróprios de 18. \\\\\\\\\\\\\\\\
			\end{enumerate}
			\item Dizer se os números a seguir são primos ou não.
			\begin{enumerate}[a)]
				\item 3 \\\\\\\\
				\item -5 \\\\\\\\
				\item 1 \newpage
				\item 9 \\\\\\\\
				\item 12 \\\\\\\\
				\item -13 \\\\\\\\
				\item 21 \\\\\\\\
				\item 29
			\end{enumerate}
			\item Diga quais são os números pares que são primos. Justifique. \\\\\\\\\\\\\\\\\\\\\\\\\\\\\\\\
			\item Faça uma lista de todos os números primos entre 1 e 100. \textbf{Sugestão:} escreva os números de 1 a 100 na tabela a seguir. Se você preferir, copie a tabela em seu caderno. Em seguida, elimine o número 1 e todos os números divisíveis por 2, 3, 5 e 7 (exceto esses números). Os números que restarem serão todos primos. Iremos explicar mais adiante como esse método funciona. \\
			\begin{table}[H]
				\centering
				\resizebox{8cm}{!}{%
					\begin{tabular}{|>{\rule{0pt}{1.2cm}}c|>{\rule{0pt}{1.2cm}}c|>{\rule{0pt}{1.2cm}}c|>{\rule{0pt}{1.2cm}}c|>{\rule{0pt}{1.2cm}}c|>{\rule{0pt}{1.2cm}}c|>{\rule{0pt}{1.2cm}}c|>{\rule{0pt}{1.2cm}}c|>{\rule{0pt}{1.2cm}}c|>{\rule{0pt}{1.2cm}}c|}
						\hline
						~~~~~ & ~~~~~ & ~~~~~ & ~~~~~ & ~~~~~ & ~~~~~ & ~~~~~ & ~~~~~ & ~~~~~ & ~~~~~ \\
						\hline
						~~ & ~~ & ~~ & ~~ & ~~ & ~~ & ~~ & ~~ & ~~ & ~~ \\
						\hline
						~~ & ~~ & ~~ & ~~ & ~~ & ~~ & ~~ & ~~ & ~~ & ~~ \\
						\hline
						~~ & ~~ & ~~ & ~~ & ~~ & ~~ & ~~ & ~~ & ~~ & ~~ \\
						\hline
						~~ & ~~ & ~~ & ~~ & ~~ & ~~ & ~~ & ~~ & ~~ & ~~ \\
						\hline
						~~ & ~~ & ~~ & ~~ & ~~ & ~~ & ~~ & ~~ & ~~ & ~~ \\
						\hline
						~~ & ~~ & ~~ & ~~ & ~~ & ~~ & ~~ & ~~ & ~~ & ~~ \\
						\hline
						~~ & ~~ & ~~ & ~~ & ~~ & ~~ & ~~ & ~~ & ~~ & ~~ \\
						\hline
						~~ & ~~ & ~~ & ~~ & ~~ & ~~ & ~~ & ~~ & ~~ & ~~ \\
						\hline
						~~ & ~~ & ~~ & ~~ & ~~ & ~~ & ~~ & ~~ & ~~ & ~~ \\
						\hline
						~~ & ~~ & ~~ & ~~ & ~~ & ~~ & ~~ & ~~ & ~~ & ~~ \\
						\hline
					\end{tabular}
				}
			\end{table}
		Quantos números primos existem entre os números 1 e 100? \newpage
		%55 a 61
		\item Quais são os divisores comuns de 12 e 32? \\\\\\\\\\\\\\\\
		\item Assinale \textbf{V} (verdadeiro) ou \textbf{F} (falso).
		\begin{enumerate}[a)]
			\item (~~) 1 é primo.
			\item (~~) 2 é primo.
			\item (~~) 57 não é primo.
			\item (~~) -87 é primo.
			\item (~~) -29 não é primo.
			\item (~~) 3 e 9 são primos entre si.
			\item (~~) -12, 6, -6 e 5 são relativamente primos.
		\end{enumerate}
		\item Quais são os divisores próprios de 23? E os impróprios? Explique o resultado encontrado. \\\\\\\\\\\\\\\\
		\item Apresentar:
		\begin{enumerate}[a)]
			\item $D(30)$
			\item $D_+(16)$
			\item $D_-^*(6)$
			\item $D(14) \cap D_+(56)$
			\item $D(4) \cup D(8)$
			\item $\mathbb{Z}_+^*$
			\item $\mathbb{N}^*$
			\item $D_-^*(18)$
			\item $D_+(10) \cap D_+(20)$
		\end{enumerate}
		\item Quais são os divisores comuns de 16 e 80? \\\\\\\\\\\\\\\\
		\item Apresente o menor inteiro positivo $a$ tal que $D(a)$ tenha 8 elementos. \\\\\\\\\\
		\item Dizemos que um número inteiro positivo é perfeito quando ele é igual à soma de todos os seus divisores positivos, exceto ele mesmo. Entre os números 6, 20, 28, 32 e 36, quais são números perfeitos?
		\end{enumerate}
		$~$ \\ $~$ \\ $~$ \\ $~$ \\ $~$ \\ $~$ \\ $~$
	\end{multicols}
\end{document}