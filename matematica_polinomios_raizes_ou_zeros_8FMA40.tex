\documentclass[a4paper,14pt]{article}
\usepackage{float}
\usepackage{extsizes}
\usepackage{amsmath}
\usepackage{amssymb}
\everymath{\displaystyle}
\usepackage{geometry}
\usepackage{fancyhdr}
\usepackage{multicol}
\usepackage{graphicx}
\usepackage[brazil]{babel}
\usepackage[shortlabels]{enumitem}
\usepackage{cancel}
\columnsep=2cm
\hoffset=0cm
\textwidth=8cm
\setlength{\columnseprule}{.1pt}
\setlength{\columnsep}{2cm}
\renewcommand{\headrulewidth}{0pt}
\geometry{top=1in, bottom=1in, left=0.7in, right=0.5in}

\pagestyle{fancy}
\fancyhf{}
\fancyfoot[C]{\thepage}

\begin{document}
	
	\noindent\textbf{8FMA40~Matemática} 
	
	\begin{center}Valor numérico de um polinômio (Versão estudante)
	\end{center}
	
	\noindent\textbf{Nome:} \underline{\hspace{10cm}}
	\noindent\textbf{Data:} \underline{\hspace{4cm}}
	
	%\section*{Questões de Matemática}
	
    \begin{multicols}{2}
    	Quando o valor numérico de $P(x)$ é igual a zero para $x = \alpha$, dizemos que $\alpha$ é uma raiz ou um zero de $P(x)$. Em outras palavras, $\alpha$ é a raiz de $P(x)$ se, e somente se, $P(\alpha) = 0$.\\
    	Por exemplo, 1 e 4 são raízes de $P(x) = x^2 - 5x + 4$, \\ pois $P(1) = 1^2 - 5 \cdot 1 + 4 = 0$ e $P(4) = 4^2 - 5 \cdot 4 + 4 = 0$. \\
    	Já o número 2 não é raiz de $P(x)$, porque $P(2) = 2^2 - 5 \cdot 2 + 4 = -2 \neq 0$.
    	\begin{enumerate}
    		\item Seja $A(x) = x^2 - 9x + 20$. Quais dos números a seguir são raízes de $A(x)$?\\
    		a) -4~~~~~ b) 1 ~~~~~ c) $\sqrt{2}$ \\
    		d) 5~~~~~~ e) 0 ~~~~~ f) -3 \\
    		g) 4 ~~~~~ h) $\sqrt{5}$\\\\\\\\\\\\\\
    		\item Diga quais são as raízes do polinômio $C(x)$ quando:
    		\begin{enumerate}[a)]
    			\item $C(x) = 3x + 7$ \\\\\\\\\\\\\\
    			\item $C(x) = 4x^2 - 36$ \\\\\\\\\\\\
    			\item $C(x) = (5x - 7)(7x - 5)$\\\\\\\\\\\\
    		\end{enumerate}
    		\item Considere o polinômio \\$Q(x) = x^2 -12x + p$.
    		\begin{enumerate}[a)]
    			\item Calcule o valor de $p$, sabendo que 3 é uma das raízes do polinômio. \\\\\\\\\\\\\\
    			\item Ache a outra raiz do polinômio. \\\\\\\\\\\\
    		\end{enumerate}
    		\item Sendo $P(x) = 4x - 5$ e $Q(x) = 6x + 1$, ache as raízes do polinômio $T(x)$ nos casos em que:
    		\begin{enumerate}[a)]
    			\item $T(x) = P(x) + Q(x)$ \\\\\\\\\\\\\\\\
    			\item $T(x) = P(x) - Q(x)$ \\\\\\\\\\\\\\\\
    			\item $T(x) = P(x) \times Q(x)$ \\\\\\\\\\\\\\\\
    		\end{enumerate}
    	    \item Sendo $P(x) = x - 3$, ache as raízes do polinômio $P + 3P^2 - 4P$.\\\\\\\\\\\\\\\\
    	    \item Seja $T(x) = 3x^2 - 10x + 8$. Quais dos números a seguir são raízes de $T(x)$?
    	    \begin{enumerate}[a)]
    	    	\item 0
    	    	\item 1
    	    	\item 2
    	    	\item 3
    	    	\item -1
    	    	\item -2
    	    	\item -3 \\\\\\\\\\\\\\\\\\\
    	    \end{enumerate}
            \item Determine as raízes reais dos polinômios a seguir:
            \begin{enumerate}[a)]
            	\item $2x + 5$ \\\\\\\\\\\\\\
            	\item $x^2 + 9x + 20$ \\\\\\\\\\\\\\
            	\item $x^2 - 4x -12$ \\\\\\\\\\\\\\
            	\item $x^2 + 2x + 9$ \\\\\\\\\\\\\\
            	\item $(x + 3)(x - 3)$ \\\\\\\\\\\\\\
            	\item $(x + 4)(x^2 + 5x + 6)$ \\\\\\\\\\\\\\
            	\item $(3x - 6)(x + 5)(x - 6)^2$ \\\\\\\\\\\\\\
            	\item $(x^2 - 9)(x^2 - 16)$ \\\\\\\\\\\\\\
            \end{enumerate}
        	\item
        	\begin{enumerate}[a)]
        		\item Seja $P(x) = x^5 - 3x^4 - 8x^3 + 3kx^2 - x + 2$. Qual deve ser o valor de $k$ para que $-2$ seja uma raiz de $P(x)$? \\\\\\\\\\\\\\
        		\item Sabendo-se que $x = 2$ é raiz do polinômio $x^3 - x^2 - ax + 3a + 1$, qual o valor de $a$? \\\\\\\\\\\\\\
        	\end{enumerate}
            \item Sabe-se que 3 é raiz de $P(x)$ e 2 é raiz de $Q(x)$, onde $P(x) = Q(x) + x^2 - 2x$. Determine $P(2) - Q(0) - Q(3) + P(0)$.  \\\\\\\\\\
    	\end{enumerate}
    \end{multicols}
\end{document}