\documentclass[a4paper,14pt]{article}
\usepackage{extsizes}
\usepackage{amsmath}
\usepackage{amssymb}
\everymath{\displaystyle}
\usepackage{geometry}
\usepackage{fancyhdr}
\usepackage{multicol}
\usepackage{graphicx}
\usepackage[brazil]{babel}
\usepackage[shortlabels]{enumitem}
\usepackage{cancel}
\columnsep=2cm
\hoffset=0cm
\textwidth=8cm
\setlength{\columnseprule}{.1pt}
\setlength{\columnsep}{2cm}
\renewcommand{\headrulewidth}{0pt}
\geometry{top=1in, bottom=1in, left=0.7in, right=0.5in}

\pagestyle{fancy}
\fancyhf{}
\fancyfoot[C]{\thepage}

\begin{document}
	
	\noindent\textbf{7FMA151~-~Matemática} 
	
	\begin{center}
		\textbf{Raiz quadrada - Alguns cálculos (Versão estudante)}
	\end{center}
	
	
	\noindent\textbf{Nome:} \underline{\hspace{10cm}}
    \noindent\textbf{Data:} \underline{\hspace{4cm}}
	
	%\section*{Questões de Matemática}
	
	\begin{multicols}{2}
	    \begin{enumerate}
	    	\item Calcule
	    	\begin{enumerate}[a)]
	    		\item $\sqrt{400}$\\\\\\\\\\
	    		\item $\sqrt{676}$\\\\\\\\\\
	    		\item $\sqrt{961}$\\\\\\\\\\
	    		\item $\sqrt{1024}$\\\\\\\\\\
	    		\item $\sqrt{18}$\\\\\\\\\\
	    		\item $\sqrt{28}$\\\\\\\\\\
	    		\item $\sqrt{50}$\\\\\\\\\\
	    	\end{enumerate}
    	    \item Calcule, através do método das aproximações sucessivas, a raiz quadrada dos seguintes números (obtenha $a_3$).
    	    \begin{enumerate}[a)]
    	    	\item 17 \\\\\\\\\\
    	    	\item 62 \\\\\\\\\\
    	    \end{enumerate}
            \item Aplicando a dicotomia, calcule a raiz quadrada, com duas casas decimais, de:
            \begin{enumerate}[a)]
            	\item 46 \\\\\\\\\\
            	\item 24 \\\\\\\\\\
            \end{enumerate}
            \item Calcule a raiz quadrada dos seguintes números através do método geométrico:
            \begin{enumerate}[a)]
            	\item 9 \\\\\\\\\\
            	\item 10 \\\\\\\\\\
            \end{enumerate}
            \item Desafio olímpico: \\Na igualdade verdadeira abaixo, quantas vezes aparece o termo $2020^2$ dentro do radical? \\\\
            $\sqrt{2020^2 + 2020^2 + ... + 2020^2} = 2020^{11}$ \\
            \begin{enumerate}[a)]
            	\item 4
            	\item 20
            	\item $2020^2$
            	\item $2020^{11}$
            	\item $2020^{20}$
            \end{enumerate}
            \item Simplifique:
            \begin{enumerate}[a)]
            	\item $\sqrt{144}$ \\\\\\\\\\
            	\item $\sqrt{256}$ \\\\\\\\\\
            	\item $\sqrt{568}$ \\\\\\\\\\
            	\item $\sqrt{576}$ \\\\\\\\\\
            	\item $\sqrt{684}$ \\\\\\\\\\
            	\item $\sqrt{729}$ \\\\\\\\\\
            	\item $\sqrt{1008}$ \\\\\\\\\\
            	\item $\sqrt{3751}$ \\\\\\\\\\
            \end{enumerate}
            \item Calcule a raiz quadrada, com uma casa decimal de:
            \begin{enumerate}[a)]
            	\item 47, pelo método da dicotomia \\\\\\\\\\
            	\item 13, pelo método geométrico \\\\\\\\\\
            \end{enumerate}
            \item Considerando $x=4$ e $y=\frac{1}{9}$,\\ o valor da expressão $\frac{y^{-1} - x^{-1}}{x + y - \frac{4}{3}}$ é:
            \begin{enumerate}[a)]
            	\item $\frac{21}{9}$
            	\item $\frac{21}{4}$
            	\item $\frac{35}{9}$
            	\item $\frac{17}{4}$
            	\item $\frac{9}{4}$
            \end{enumerate}
            \item Mostre que o método geométrico, com as adaptações mostradas, funciona para $x < 1$.
	    \end{enumerate}
    $~$ \\ $~$ \\ $~$ \\ $~$ \\ $~$ \\ $~$ \\ $~$ \\ $~$ \\ $~$ \\ $~$ \\ $~$ \\ $~$ \\ $~$ \\ $~$ \\ $~$ \\ $~$ \\ $~$ \\ $~$ \\ $~$ \\ $~$ \\ $~$ \\ $~$ \\ $~$ \\ $~$ \\ $~$ \\ $~$ \\ $~$ \\ $~$ \\ $~$ \\ $~$ \\ $~$ \\ $~$ \\ $~$ \\
    \end{multicols}
\end{document}