\documentclass[a4paper,14pt]{article}
\usepackage{float}
\usepackage{extsizes}
\usepackage{amsmath}
\usepackage{amssymb}
\everymath{\displaystyle}
\usepackage{geometry}
\usepackage{fancyhdr}
\usepackage{multicol}
\usepackage{graphicx}
\usepackage[brazil]{babel}
\usepackage[shortlabels]{enumitem}
\usepackage{cancel}
\usepackage{textcomp}
\usepackage{array}
\usepackage{longtable}
\usepackage{booktabs}
\usepackage{float}   % Para usar o modificador [H]

\columnsep=2cm
\hoffset=0cm
\textwidth=8cm
\setlength{\columnseprule}{.1pt}
\setlength{\columnsep}{2cm}
\renewcommand{\headrulewidth}{0pt}
\geometry{top=1in, bottom=1in, left=0.7in, right=0.5in}

\pagestyle{fancy}
\fancyhf{}
\fancyfoot[C]{\thepage}

\begin{document}
	
	\noindent\textbf{6FMA36 - Matemática} 
	
	\begin{center}Porcentagem (Versão estudante)
	\end{center}
	
	\noindent\textbf{Nome:} \underline{\hspace{10cm}}
	\noindent\textbf{Data:} \underline{\hspace{4cm}}
	
	%\section*{Questões de Matemática}
    \begin{multicols}{2}
    	\noindent Porcentagens são expressões formadas por um numeral seguido do símbolo \%. Por exemplo, 24\% lemos vinte e quatro por centro e podemos representar como uma fração $\bigg(\frac{24}{100}\bigg)$\\\\ ou como um número decimal (0,24). \\
    	Para calcularmos a porcentagem de um número, basta escrever a porcentagem na forma de fraçao e multiplicar por tal número. Por exemplo: \\\\
    	10\% de 30 = $\frac{10}{100} \cdot 30 = \frac{1}{10} \cdot 30 = 3$
    	\noindent\textsubscript{~---------------------------------------------------------------------------}
    	\begin{enumerate}
			\item Calcule
			\begin{enumerate}[a)]
				\item 10\% de 70. \\\\\\
				\item 20\% de 20. \\\\\\
				\item 30\% de 90. \\\\\\
				\item 50\% de 102. \\\\\\
				\item 75\% de 60. \\
				\item 84\% de 400. \\\\\\
			\end{enumerate}
			\item Rafael utilizava 75\% do armazenamento total do seu celular. Após retirar todas as fotos, seu armazenamento total passou para 50\%. Se o celular de Rafael tem 16 GB de armazenamento total, quantos GB ele tinha de fotos? \\\\\\\\\\\\
			\item Edgar coleciona camisas de futebol. Ele tem 425 camisas de times nacionais e estrangeiros, sendo 40\% delas de equipes de outros países.
			\begin{enumerate}[a)]
				\item Quantas camisas são de times estrangeiros? \\\\
				\item Quantas camisas são de times nacionais? Que porcentagem isso representa do total? \\\\
			\end{enumerate}
    	\end{enumerate}
    \end{multicols}
\end{document}