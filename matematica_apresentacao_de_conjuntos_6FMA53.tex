\documentclass[a4paper,14pt]{article}
\usepackage{float}
\usepackage{extsizes}
\usepackage{amsmath}
\usepackage{amssymb}
\everymath{\displaystyle}
\usepackage{geometry}
\usepackage{fancyhdr}
\usepackage{multicol}
\usepackage{graphicx}
\usepackage[brazil]{babel}
\usepackage[shortlabels]{enumitem}
\usepackage{cancel}
\usepackage{textcomp}
\usepackage{array} % Para melhor formatação de tabelas
\usepackage{longtable}
\usepackage{booktabs}  % Para linhas horizontais mais bonitas
\usepackage{float}   % Para usar o modificador [H]
\usepackage{caption} % Para usar legendas em tabelas
\usepackage{tcolorbox}

\columnsep=2cm
\hoffset=0cm
\textwidth=8cm
\setlength{\columnseprule}{.1pt}
\setlength{\columnsep}{2cm}
\renewcommand{\headrulewidth}{0pt}
\geometry{top=1in, bottom=1in, left=0.7in, right=0.5in}

\pagestyle{fancy}
\fancyhf{}
\fancyfoot[C]{\thepage}

\begin{document}
	
	\noindent\textbf{6FMA53 - Matemática} 
	
	\begin{center}Apresentação de conjuntos (Versão estudante)
	\end{center}
	
	\noindent\textbf{Nome:} \underline{\hspace{10cm}}
	\noindent\textbf{Data:} \underline{\hspace{4cm}}
	
	%\section*{Questões de Matemática}
	\begin{multicols}{2}
    		\noindent Além do diagrama de Venn, podemos representar um conjunto por meio de propriedades ou condições que seus elementos satisfazem. Por exemplo, o conjunto $A = \{2, 4, 6, 8\}$ pode ser representado da seguinte forma: \\
    		$A = \{x \in \mathbb{N} :$ x é par e $0 < x < 10\}$ \\
    		O símbolo :, em Matemática, pode ser lido \textbf{tal que} e também pode ser representado por $|$ ou simplesmente t.q.\\
    		\textsubscript{---------------------------------------------------------------------}
    		\begin{enumerate}
    			\item Apresentar o conjunto $A$ listando seus elementos.
    			\begin{enumerate}[a)]
    				\item $A = \{x \in \mathbb{N} : x~\text{é par}\}$ \\\\\\\\\\\\
    				\item $A = \{x \in \mathbb{N}~|~ x - 6 = 0\}$ \\\\\\\\\\\\
    				\item $A = \\ \{x \in \mathbb{N} : x \text{~é ímpar e~} x \leq 15\}$ \\\\\\\\
    				\item $A = \{x \in \mathbb{N} \text{~t.q.~} 3 \leq x < 10\}$ \\\\\\\\\\\\
    				\item $A = \{x \in \mathbb{N}~|~x \leq 4 \text{~e~} x > 7\}$ \\\\\\\\\\\\
    			\end{enumerate}
    			\item Escrever o conjunto $A$ usando propriedades (há mais de uma maneira de resolver este exercício).
    			\begin{enumerate}[a)]
    				\item $A = \{1, 3, 5\}$ \\\\\\\\
    				\item $A = \{8, 9, 10\}$ \\\\\\\\
    				\item $A = \{1, 2, 3, 4\}$ \\\\\\\\
    				\item $A = \{42, 43, 44, 45, 46, 47, 48\}$ \\\\\\
    				\item $A = \{0\}$ \\\\\\
    				\item $A = \varnothing$ \\\\\\
    			\end{enumerate}
    			\item Apresentar dois conjuntos finitos e dois conjuntos infinitos na forma sentencial. \\\\\\\\\\\\
    			\item Apresentar o conjunto $A$ listando seus elementos.
    			\begin{enumerate}[a)]
    				\item $A = \{x \in \mathbb{N} : x \text{~é ímpar}\}$ \\\\\\\\\\\\
    				\item $A = \{x \in \mathbb{N} | x - 7 = 0\}$ \\\\\\\\\\\\
    				\item $A = \{x \in \mathbb{N} : x = x\}$ \\\\\\\\\\\\
    				\item $A = \\ \{x \in \mathbb{N} : \text{~x é par e~} x < 15\}$ \\\\\\\\\\\\
    				\item $A = \{x \in \mathbb{N} \text{~t.q.~} 5 < x \leq 15\}$ \\\\\\\\\\\\
    				\item $A = \{x \in \mathbb{N}~|~x + 1 < x + 4\}$ \\\\\\\\\\\\
    			\end{enumerate}
    			\item Escrever o conjunto $A$ usando propriedades (há mais de uma maneira de resolver este exercício).
    			\begin{enumerate}[a)]
    				\item $A = \{10, 12, 14, 16\}$ \\\\\\\\
    				\item $A = \{1, 2, 3, 4\}$ \\\\\\\\\\\\
    				\item $A = \{15, 17, 19\}$ \\\\\\\\\\\\
    				\item $A = \{29, 30, 31, 32\}$ \\\\\\\\\\\\
    				\item $A = \{7\}$ \\\\\\\\\\\\
    				\item $A = \varnothing$ \\\\\\\\\\\\\\\\\\\\\\
    			\end{enumerate}
    			\item Quais dos conjunto abaixo representam o conjunto vazio?
    			\begin{enumerate}[a)]
    				\item $A = \{x \in \mathbb{N}~|~x = x - 3\}$
    				\item $B = \{x \in \mathbb{Z}~|~x < 5$
    				\item $C = \{x \in \mathbb{N}~|~x < -8\}$
    				\item $D = \{x \in \mathbb{N}~|~4x = 0\}$
    			\end{enumerate}
    			\item Escrever os conjuntos:
    			\begin{enumerate}[a)]
    				\item $A = \{x \in \mathbb{N}~|~3 \leq x^2 \leq 16\}$  \\\\\\\\\\\\
    				\item $B = \{x \in \mathbb{N}~|~x \text{~é ímpar e~} x < 10\}$ \\\\\\\\\\\\
    				\item $C = \{x \in \mathbb{N}~\text{~t.q.~} x^2 = 5\}$ \\\\\\\\\\\\
    				\item $D = \{x \in \mathbb{Z}~|~-4 < x < 4\}$ \\\\\\\\\\\\
    			\end{enumerate}
    		\end{enumerate}
    		$~$
	\end{multicols}
\end{document}