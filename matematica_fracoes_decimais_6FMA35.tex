\documentclass[a4paper,14pt]{article}
\usepackage{float}
\usepackage{extsizes}
\usepackage{amsmath}
\usepackage{amssymb}
\everymath{\displaystyle}
\usepackage{geometry}
\usepackage{fancyhdr}
\usepackage{multicol}
\usepackage{graphicx}
\usepackage[brazil]{babel}
\usepackage[shortlabels]{enumitem}
\usepackage{cancel}
\usepackage{textcomp}
\usepackage{array}
\usepackage{longtable}
\usepackage{booktabs}
\usepackage{float}   % Para usar o modificador [H]

\columnsep=2cm
\hoffset=0cm
\textwidth=8cm
\setlength{\columnseprule}{.1pt}
\setlength{\columnsep}{2cm}
\renewcommand{\headrulewidth}{0pt}
\geometry{top=1in, bottom=1in, left=0.7in, right=0.5in}

\pagestyle{fancy}
\fancyhf{}
\fancyfoot[C]{\thepage}

\begin{document}
	
	\noindent\textbf{6FMA35 - Matemática} 
	
	\begin{center}Frações decimais (Versão estudante)
	\end{center}
	
	\noindent\textbf{Nome:} \underline{\hspace{10cm}}
	\noindent\textbf{Data:} \underline{\hspace{4cm}}
	
	%\section*{Questões de Matemática}
    \begin{multicols}{2}
    	\noindent Podemos representar uma fração por um numeral decimal. Por exemplo, $\frac{1}{10}$ pode ser escrito como 0,1 (lê-se "um décimo" ou "zero vírgula um"). \\
    	Frações decimais são aquelas cujo denominador é uma potência de 10, ou seja, 10, 100, 1000, etc. \\
    	Tais frações podem ser facilmente transformadas em numerais decimais. Observe: \\\\
    	$\frac{6}{10} = 0,6; \frac{17}{10} = 1,7; \frac{98}{100} = 0,98.$ \\\\
    	\noindent\textsubscript{~---------------------------------------------------------------------------}
    	\begin{enumerate}
			\item Escreva na forma de numeral decimal:
			\begin{enumerate}[a)]
				\item $\frac{7}{10}$ \\\\
				\item $\frac{28}{10}$ \\\\
				\item $\frac{5}{2}$ \\\\
				\item $\frac{681}{100}$ \\\\
				\item $\frac{374}{1000}$ \\\\
				\item $\frac{32}{1000}$ \\\\
			\end{enumerate}
			\item Escreva como frações decimais:
			\begin{enumerate}[a)]
				\item 0,8 \\\\
				\item 1,7 \\\\
				\item 0,45 \\\\
				\item 5,78 \\\\
				\item 2,034 \\\\
				\item 16,843 \\\\
			\end{enumerate}
			\item Assinale \textbf{V} (verdadeiro) ou \textbf{F} (falso):
			\begin{enumerate}[a)]
				\item (~~~) $\frac{5}{100} = 0,5$
				\item (~~~) $\frac{43}{10} = 0,43$
				\item (~~~) $\frac{13}{10} = 1,3$
				\item (~~~) $\frac{217}{100} = 0,217$
				\item (~~~) $0,81 = \frac{81}{10}$
				\item (~~~) $3,45 = \frac{345}{10}$
				\item (~~~) $72,61 = \frac{7261}{100}$
				\item (~~~) $0,086 = \frac{86}{1000}$
			\end{enumerate}
		    \item Assinale \textbf{V} (verdadeiro) ou \textbf{F} (falso) para as igualdades abaixo.
		    \begin{enumerate}[a)]
		    	\item (~~~) $\frac{18}{10} = 1,8$
		    	\item (~~~) $\frac{824}{10} = 8,24$
		    	\item (~~~) $\frac{18}{100} = 1,80$
		    	\item (~~~) $\frac{18}{100} = 1,08$
		    	\item (~~~) $\frac{18}{100} = 0,18$
		    	\item (~~~) $\frac{824}{100} = 8,24$
		    	\item (~~~) $\frac{11}{10} = 1,1$
		    	\item (~~~) $\frac{618}{10} = 6,18$
		    	\item (~~~) $\frac{45}{100} = 4,50$
		    	\item (~~~) $\frac{11}{100} = 1,01$
		    	\item (~~~) $\frac{11}{100} = 0,11$
		    	\item (~~~) $\frac{618}{100} = 6,18$
		    \end{enumerate}
	        \item Escreva como números decimais:
	        \begin{enumerate}[a)]
	        	\item $\frac{5}{10}$
	        	\item $\frac{9}{10}$
	        	\item $\frac{21}{100}$
	        	\item $\frac{262}{100}$
	        	\item $\frac{4352}{1000}$
	        	\item $\frac{104}{1000}$
	        	\item $\frac{426}{100}$
	        	\item $\frac{23}{10}$
	        	\item $\frac{1264}{1000}$
	        	\item $\frac{34}{100}$
	        	\item $\frac{85}{1000}$
	        	\item $\frac{97}{100}$
	        \end{enumerate}
    	\end{enumerate}
    $~$ \\ $~$ \\ $~$ \\ $~$ \\ $~$ \\ $~$ \\ $~$ \\ $~$ \\ $~$ \\ $~$ \\ $~$ \\ $~$ \\ $~$ \\ $~$ \\ $~$ \\ $~$ \\ $~$ \\ $~$ \\ $~$ \\ $~$ \\ $~$ \\ 
    \end{multicols}
\end{document}