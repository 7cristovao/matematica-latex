\documentclass[a4paper,14pt]{article}

\usepackage{comment} % Para comentar várias linhas ao mesmo tempo

%matemática
\usepackage{amsmath}
\usepackage{amssymb}

%diagramação
\usepackage{extsizes}
\everymath{\displaystyle}
\usepackage{geometry}
\usepackage{fancyhdr}
\usepackage{multicol}
\usepackage{graphicx}
\usepackage[brazil]{babel}
\usepackage[shortlabels]{enumitem}
\usepackage{cancel}
\usepackage{textcomp}
\usepackage{tcolorbox}

%tabelas
\usepackage{array} % Para melhor formatação de tabelas
\usepackage{longtable}
\usepackage{booktabs}  % Para linhas horizontais mais bonitas
\usepackage{float}   % Para usar o modificador [H]
\usepackage{caption} % Para usar legendas em tabelas
\usepackage{wrapfig} % Para usar tabelas e figuras flutuantes
\usepackage{xcolor} % Para cores do fundo de tabelas
\usepackage{colortbl} % Para cores do fundo de tabelas

%tikzpicture
\begin{comment}
	\usepackage{tikz}
	\usepackage{scalerel}
	\usepackage{pict2e}
	\usepackage{tkz-euclide}
	\usetikzlibrary{calc}
	\usetikzlibrary{patterns,arrows.meta}
	\usetikzlibrary{shadows}
	\usetikzlibrary{external}
\end{comment}


%pgfplots
\usepackage{pgfplots}
\pgfplotsset{compat=newest}
\usepgfplotslibrary{statistics}
\usepgfplotslibrary{fillbetween}

%colours
\usepackage{xcolor}



\columnsep=1.2cm
\hoffset=0cm
\textwidth=8cm
\setlength{\columnseprule}{.1pt}
\setlength{\columnsep}{1.2cm}
\renewcommand{\headrulewidth}{0pt}
\geometry{top=1in, bottom=1in, left=0.7in, right=0.5in}

\pagestyle{fancy}
\fancyhf{}
\fancyfoot[C]{\thepage}

\begin{document}
	
	\noindent\textbf{6FMA127 - Matemática} 
	
	\begin{center}Equação produto (Versão estudante)
	\end{center}
	
	\noindent\textbf{Nome:} \underline{\hspace{10cm}}
	\noindent\textbf{Data:} \underline{\hspace{4cm}}
	
	%\section*{Questões de Matemática}
	
	\begin{multicols}{2}
	    \noindent Sejam $F_1, F_2, F_3, ..., F_n$ formas descritivas de números, em que $n$ é natural maior ou igual a 2, então: \\
	    $F_1 \cdot F_2 \cdot F_3 \cdot ... \cdot F_n = 0 \Leftrightarrow \begin{cases} 
	    F_1 = 0 \\
	    \text{~~~ou} \\
	    F_2 = 0 \\
	    \text{~~~ou} \\
	    \vdots \\
	    \text{~~~ou} \\
	    F_n = 0
	    \end{cases}$
		\noindent\textsubscript{--------------------------------------------------------------------------}
		\begin{enumerate} 
			\item Resolver as seguintes equações ($U = \mathbb{Q}$).
			\begin{enumerate}[a)]
			\item $(-x - 3)(3x - 7) = 0$ \\\\\\\\\\\\\\\\\\\\
			\item $(-3z + 4)(z - 5) = 0$ \\\\\\\\\\\\\\\\\\\\
			\item $-x(4x + 9)(-x + 6) = 0$ \\\\\\\\\\\\\\\\\\\\
			\end{enumerate}
		\end{enumerate}
	\end{multicols}
	2. Resolver a equação $(U = \mathbb{Q})$: \\\\
	$\bigg(\frac{7x + 15}{3} - 1 - \frac{4(x + 2)}{6}\bigg) \cdot \bigg(\frac{-3 -x}{2} + \frac{2x}{3}\bigg) = 0$
	\newpage
	\begin{multicols}{2}
		\begin{enumerate}
			\setcounter{enumi}{2} % Define o início da contagem em 3
			\item Resolver as equações ($U = \mathbb{Q}$):
			\begin{enumerate}[a)]
				\item $x^2 = 0$ \\\\\\\\\\
				\item $-5x^3 = 0$ \\\\\\\\\\
				\item $(5x - 6)^2 = 0$ \\\\\\\\\\\\\\\\\\\\
				\item $(-x + 4)^2  (x + 4)^5  (3x - 1)^6 = 0$ \\\\\\\\\\\\\\\\\\\\\\\\\\\\\\
			\end{enumerate}
			%3 a 6
			\item Resolver as seguintes equações ($U = \mathbb{Q}$).
			\begin{enumerate}[a)]
				\item $(-3x + 5)(2x - 9) = 0$ \\\\\\\\\\\\\\\\
				\item $(-y - 2)(-4y - 7) = 0$ \\\\\\\\\\\\\\\\
				\item $x(-x + 2)(x - 4) = 0$ \\\\\\\\\\\\\\\\
				\item $-y(-y - 5)(3y - 8) = 0$ \\\\\\\\\\\\\\\\\\
				\item $x(x - 1)(x + 2)(-x - 1) = 0$ \\\\\\\\\\\\\\\\
				\item $(-3x + 12)(4x - 16)(-x + 4)=0$ \\\\\\\\\\\\\\\\
			\end{enumerate}
		\end{enumerate}
		\end{multicols}
		5. Resolver as equações ($U = \mathbb{Q}$).
		\begin{enumerate}[a)]
			\item $\bigg(\frac{6x - 4}{4} + \frac{7x}{2} - \frac{9x + 4}{2}\bigg) \cdot \bigg(\frac{4x}{5} - \frac{12x}{15}\bigg) = 0$ \\\\\\\\\\\\\\\\\\\\\\\\\\\\\\
			\item $\bigg(\frac{x + 1}{3} - \frac{9x - 2}{9} - \frac{x + 4}{3} + \frac{7x + 9}{7}\bigg) \cdot \bigg(\frac{x + 2}{2} - \frac{2x}{5}\bigg) = 0$ \\\\\\\\\\\\\\\\\\\\\\\\\\\\\\
			\item $\bigg(\frac{2x - 7}{8} + \frac{x + 9}{4} - \frac{x}{2} + \frac{13}{11}\bigg) \cdot \bigg(\frac{4x + 10}{3} - \frac{5}{3} - \frac{15x}{27} - \frac{7x}{9} - \frac{15}{9}\bigg) = 0$ \\\\\\\\\\\\\\\\\\\\\\\\\\\\\\
		\end{enumerate}
		6. Resolver as equações ($U = \mathbb{Q}$).
		\begin{enumerate}[a)]
			\item $y^2 = 0$ \\\\\\\\\\\\\\\\\\\\
			\item $3x^2 = 0$ \\\\\\\\\\\\\\\\\\\\
			\item $-x^2 = 0$ \\\\\\\\\\\\\\\\\\\\
			\item $-x^3 = 0$ \\\\\\\\\\\\\\\\\\\\
			\item $(2x - 5)^2 = 0$ \\\\\\\\\\\\\\\\\\\\
			\item $3(2x - 9)^7 = 0$ \\\\\\\\\\\\\\\\\\\\
			\item $(5y - 6)^{29} = 0$ \\\\\\\\\\\\\\\\\\\\
			\item $(y - 7)^{54} \cdot (6y - 42)^{16} = 0$ \\\\\\\\\\\\\\\\\\\\
			\item $(3y + 5)^6 \cdot (y - 6)^5 \cdot (-y - 6)^4 = 0$ \\\\\\\\\\\\\\\\\\\\
			\item $y^{13}(9y - 1)^{14} \cdot (-y + 2)^{15} = 0$
		\end{enumerate}
\end{document}