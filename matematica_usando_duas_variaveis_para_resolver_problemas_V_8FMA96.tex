\documentclass[a4paper,14pt]{article}
\usepackage{float}
\usepackage{extsizes}
\usepackage{amsmath}
\usepackage{amssymb}
\everymath{\displaystyle}
\usepackage{geometry}
\usepackage{fancyhdr}
\usepackage{multicol}
\usepackage{graphicx}
\usepackage[brazil]{babel}
\usepackage[shortlabels]{enumitem}
\usepackage{cancel}
\usepackage{textcomp}
\usepackage{array}
\usepackage{longtable}
\usepackage{booktabs}
\usepackage{float}   % Para usar o modificador [H]

\columnsep=2cm
\hoffset=0cm
\textwidth=8cm
\setlength{\columnseprule}{.1pt}
\setlength{\columnsep}{2cm}
\renewcommand{\headrulewidth}{0pt}
\geometry{top=1in, bottom=1in, left=0.7in, right=0.5in}

\pagestyle{fancy}
\fancyhf{}
\fancyfoot[C]{\thepage}

\begin{document}
	
	\noindent\textbf{8FMA96 - Matemática} 
	
	\begin{center}Usando duas variáveis para resolver problemas (V) (Versão estudante)
	\end{center}
	
	\noindent\textbf{Nome:} \underline{\hspace{10cm}}
	\noindent\textbf{Data:} \underline{\hspace{4cm}}
	
	%\section*{Questões de Matemática}
    \begin{multicols}{2}
    	\begin{enumerate}
			\item Determine a fração equivalente a $\frac{2}{3}$ que se torna equivalente a $\frac{4}{7}$ quando se subtrai 4 de seu numerador e de seu denominador. \\\\\\\\\\\\\\\\\\\\\\\\\\\\\\
			\item O produto de dois números é 18, e a soma de seus inversos é $\frac{11}{18}$. Determine quais são os números. \\\\\\\\\\\\\\\\\\\\\\\\
			\item Achar todos os números reais $a$ e $b$ para os quais são iguais sua soma, seu produto e o quociente de $a$ por $b$.  \\\\\\\\\\\\\\\\\\\\\\\\\\\\\\
			\item Numa diviso de dois números inteiros, o quociente é 16, e o resto, 14. Sabe-se que a soma do dividendo, do divisor, do quociente e do resto é 350. Encontre o dividendo e o divisor.
    	\end{enumerate}
    $~$ \\ $~$ \\ $~$ \\ $~$ \\ $~$ \\ $~$ \\ $~$ \\ $~$ \\ $~$ \\
    \end{multicols}
\end{document}