\documentclass[a4paper,14pt]{article}
\usepackage{float}
\usepackage{extsizes}
\usepackage{amsmath}
\usepackage{amssymb}
\everymath{\displaystyle}
\usepackage{geometry}
\usepackage{fancyhdr}
\usepackage{multicol}
\usepackage{graphicx}
\usepackage[brazil]{babel}
\usepackage[shortlabels]{enumitem}
\usepackage{cancel}
\columnsep=2cm
\hoffset=0cm
\textwidth=8cm
\setlength{\columnseprule}{.1pt}
\setlength{\columnsep}{2cm}
\renewcommand{\headrulewidth}{0pt}
\geometry{top=1in, bottom=1in, left=0.7in, right=0.5in}

\pagestyle{fancy}
\fancyhf{}
\fancyfoot[C]{\thepage}

\begin{document}
	
	\noindent\textbf{6FMA63~Matemática} 
	
	\begin{center}Potências e propriedades (Versão estudante)
	\end{center}
	
	\noindent\textbf{Nome:} \underline{\hspace{10cm}}
	\noindent\textbf{Data:} \underline{\hspace{4cm}}
	
	%\section*{Questões de Matemática}
	
	
    \begin{multicols}{2}
    	\noindent\textbf{P1.} Para multiplicar potências de mesma base, mantemos a base e somamos os expoentes. \\
    	Exemplo: $7^3 \cdot 7^6 = 7^{3 + 6} = 7^9$. \\
    	\textbf{P2.} Para obter uma potência de outra potência, basta multiplicarmos os expoentes. \\
    	Exemplo: $(8^3)^4 = 8^{3 \cdot 4} = 8^12$.
    	
    	\noindent\textsubscript{---------------------------------------------------------------------------}
    	\begin{enumerate}
    		\item Escreva na forma de potência:
    		\begin{enumerate}[a)]
    			\item $8 \times 8 \times 8 \times 8 \times 8 = $ \\\\
    			\item $4 \times 4 \times 4 \times 4 = $ \\\\
    			\item $16 \cdot 16 \cdot 16 \cdot 16 \cdot 16 \cdot 16 \cdot 16 \cdot 16 \cdot 16 = $ \\\\
    			\item Quinze ao quadrado: \\\\
    			\item Dezoito ao cubo: \\\\
    	    \end{enumerate}
            \item Utilizando a propriedade P1, represente na forma de potência:
            \begin{enumerate}[a)]
            	\item $3^5 \times 3^2 = $ \\\\\\
            	\item $9^4 \cdot 9^6 = $ \\\\\\
            	\item $7^9 \cdot 7 \cdot 7^3 = $ \\\\\\
            	\item $13^3 \times 13^6 \times 13^5 \times 13^8 = $ \\\\\\
            \end{enumerate}
            \item Calcule o valor:
            \begin{enumerate}[a)]
            	\item $4^3 + 3^2 + 2^4 = $ \\\\\\
            	\item $4^3 - 3^2 + 2^4 = $ \\\\\\
            	\item $4^3 + 3^2 - 2^4 = $ \\\\\\
            	\item $4^3 + 2^4 = $ \\\\\\
            	\item $(4 + 3)^3 = $ \\\\\\
            	\item $(4 - 3)^3 = $ \\\\\\
            \end{enumerate}
            \item Efetue os cálculos: \\
            $(3^3)^2 + (4^2)^2 + (5^2)^2 = $ \\\\\\\\
            \item Represente na forma de uma única potência.
            \begin{enumerate}[a)]
            	\item $2^6 \cdot 2^3$
            	\item $4^5 \cdot 4^3$
            	\item $7^2 \cdot 7^4 \cdot 7^3$
            	\item $10^1 \cdot 10^2 \cdot 10^3 \cdot 10^4 \cdot 10^5$
            \end{enumerate}
            \item Escreva na forma de potência.
            \begin{enumerate}[a)]
            	\item $(3^6)^5$
            	\item $(7^2)^8$
            	\item $(6^4)^14$
            \end{enumerate}
            \item Calcule.
            \begin{enumerate}[a)]
            	\item $4^3 + 5^2 + 6^2$
            	\item $4^3 - 5^2 + 6^2$
            	\item $4^3 + 5^2 - 6^2$
            	\item $7^2 + 8^2$
            	\item $(7 + 8)^2$
            	\item $(9 - 5)^2$
            	\item $(3^2)^3 + (4^3)^2 + (5^2)^2$
            	\item $(3^3 - 4^2)^2$
            	\item $(3^4 - 4^3)^2$
            	\item $(4^3 - 7^2)^3$
            	\item $(2^8 - 4^4)^9$
            	\item $(3^3 - 5^2)^{10}$
            \end{enumerate}
            \item Escreva na forma de potência.
            \begin{enumerate}[a)]
            	\item $4 \cdot 4 \cdot 4 \cdot 4 \cdot 4 \cdot 4$
            	\item $5 \cdot 5 \cdot 5 \cdot 5$
            	\item $10 \cdot 10 \cdot 10 \cdot 10 \cdot 10 \cdot 10 \cdot 10 \cdot 10 \cdot 10 \cdot 10$
            	\item Doze ao cubo
            	\item Quarenta ao quadrado 
            \end{enumerate}
        \end{enumerate}
    $~$ \\ $~$ \\ $~$ \\ $~$ \\ $~$ \\ $~$ \\ $~$ \\ $~$ \\ $~$ \\ $~$ \\ $~$ \\ $~$ \\ $~$ \\ $~$ \\ $~$ \\ $~$ \\ $~$ \\ $~$ \\ $~$ \\ $~$ \\ $~$ \\ $~$ \\ $~$ \\ $~$ \\ $~$ \\ $~$ \\ $~$ \\ $~$ \\ $~$ \\ $~$ \\ 
    \end{multicols}
\end{document}