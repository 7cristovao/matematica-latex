\documentclass[a4paper,14pt]{article}
\usepackage{extsizes}
\usepackage{amsmath}
\usepackage{amssymb}
\everymath{\displaystyle}
\usepackage{geometry}
\usepackage{fancyhdr}
\usepackage{multicol}
\usepackage{graphicx}
\usepackage[brazil]{babel}
\usepackage[shortlabels]{enumitem}
\usepackage{cancel}
\columnsep=2cm
\hoffset=0cm
\textwidth=8cm
\setlength{\columnseprule}{.1pt}
\setlength{\columnsep}{2cm}
\renewcommand{\headrulewidth}{0pt}
\geometry{top=1in, bottom=1in, left=0.7in, right=0.5in}

\pagestyle{fancy}
\fancyhf{}
\fancyfoot[C]{\thepage}

\begin{document}
	
	\noindent\textbf{8FMA12~-~Matemática} 
	
	\begin{center}Problemas interdisciplinares de regra de três simples - Exercícios (Versão estudante)
	\end{center}
	
	
	\noindent\textbf{Nome:} \underline{\hspace{10cm}}
	\noindent\textbf{Data:} \underline{\hspace{4cm}}
	
	%\section*{Questões de Matemática}
	\begin{multicols}{2}
		\begin{enumerate}
			\item Tente fazer a seguinte experiência: pegue uma seringa sem a agulha (é claro), puxe o êmbolo e coloque o dedo na ponta por onde sai o ar. Agora, empurre o êmbolo. Note que quanto menor o espaço (volume) que o ar ocupa, maior a pressão que você sente no dedo.
			\begin{enumerate}[a)]
				\item O volume e a pressão são grandezas direta ou inversamente proporcionais? \\\\\\
				\item O volume máximo de uma seringa é de 20 mL. Se reduzirmos tal volume a 5 mL, o que acontecerá com a pressão? \\\\\\\\\\\\
			\end{enumerate}	
		    \item Viajando de São Paulo a Campinas, com uma velocidade de 120 km/h, uma família levou 1 hora e 40 minutos para completar a viagem. Se tal família demorou 20 minutos a mais na volta, qual foi a velocidade nesse trajeto? \\\\\\\\\\\\\\\\\\\\
		    \item Os fios utilizados em instalações elétricas têm uma propriedade chamada resistência, que é medida em ohm$(\Omega)$. Quanto maior a resistência de um fio, menor é a intensidade da corrente que passa por ele, medida em ampère (A), para uma determinada voltagem. Considere um ambiente em que a voltagem é 110 V e dois aparelhos elétricos, um com fio cuja resistência é 30$\Omega$ e o outro, 15$\Omega$. Se a corrente que passa pelo segundo é de 13 A, qual é a intensidade da corrente que passa pelo primeiro? \\\\\\\\\\\\\\\\\\
		    \item Desde o século XVII são realizados estudos sobre o comportamento dos gases e o relacionamento entre si das variáveis de estado: $p$ (pressão em atm), $V$(volume em L), $T$ (temperatura em K) e $n$ (número de mols do gás em mol).
		    Em estudos de um gás ideal, a equação utilizada para relacionar essas variáveis é:
		    \begin{center}
		    	$pV = nRT,$
		    \end{center}
	    	em que $R$ é a constante universal dos gases e com o valor de $R = 0,082~\frac{atm \cdot L}{mol \cdot K}$
	    	\begin{enumerate}[a)]
	    		\item A pressão $p$ e a temperatura $T$ são grandezas direta ou inversamente proporcionais? \\\\\\
	    		\item Em um recipiente de 41 L, há 5 mols de um gás ideal a uma temperatura de 300 K. Determine sua pressão em atm.  \\\\\\\\\\\\\\\\\\\\\\\\\\\\
	    	\end{enumerate}	
    		\item Alguma vez já fizeram a você a pergunta: "O que 'pesa' mais: um quilo de algodão ou um quilo de chumbo?". Muitas pessoas respondem: "Um quilo de chumbo". Isso porque elas pensam em volumes iguais de algodão e de chumbo e imaginam a massa dessa porção. Dessa forma, o chumbo é realmente mais "pesado", pois sua densidade é maior. Porém, se reformularmos a pergunta para: "Qual tem maior volume: um quilo de algodão ou um quilo de chumbo?", a resposta será "Um quilo de algodão". Com base nessas informações, pergunta-se:
    		\begin{enumerate}[a)]
    			\item Densidade e volume são grandezas direta ou inversamente proporcionais? Justifique. \\\\\\\\
    			\item A densidade da água é de 1 kg/L e a densidade do álcool é de 0,8 kg/L. Se pegarmos 1 kg de cada um, o volume de água será de 1 L. Qual será o volume de álcool?  \\\\\\\\\\\\\\\\\\
    		\end{enumerate}	
    	    \textbf{Desafio olímpico} \\\\
    	    (OBMEP) Um reservatório, inicialmente vazio, é abastecido por duas torneiras de vazões diferentes. Se cada torneira for aberta por $\frac{1}{3}$ do tempo necessário para que a outra encha o reservatório, este ficará com $\frac{5}{6}$ de sua capacidade preenchida. Além disso, as duas torneiras juntas enchem o reservatório inicialmente vazio em 2 horas e 30 minutos. Em quanto tempo a torneira de maior vazão enche o reservatório?
    	    \begin{enumerate}[a)]
    	    	\item 3 horas
    	    	\item 3 horas e 15 minutos
    	    	\item 3 horas e 30 minutos
    	    	\item 3 horas e 45 minutos
    	    	\item 4 horas \\\\
    	    	Lembre-se: \\\\vazão = $\frac{volume}{tempo}$
    	    	 \\\\\\\\\\\\\\\\\\\\\\\\\\
    	    \end{enumerate}	
            \item Um paciente toma um remédio ao meio-dia. No período das 14 h às 17 h, a quantidade do remédio presente no sangue do paciente, em mg, é inversamente proporcional ao tempo transcorrido desde o momento que ele toma o remédio. Às 14 h, tal quantidade é de 1,2 mg. Qual a quantidade de medicamento presente no seu sangue as 17 h? \\\\\\\\\\\\\\\\\\\\
            \item Numa editora, a equipe de digitadoras tem 14 profissionais. Trabalhando juntas, digitam um livro em 4 horas. Se 8 delas entrarem em férias, em quanto tempo as restantes digitarão o mesmo livro? \\\\\\\\\\\\\\\\\\\\\\\\
            \item Um motorista de taxi, trabalhando 8 horas por dia durante 10 dias, ganha R\$ 2.104,00. Qual será o seu ganho mensal se trabalhar 5 horas por dia? Considere um mês com 30 dias.
            \begin{enumerate}[a)]
            	\item R\$ 1.945,00
            	\item R\$ 2.534,00
            	\item R\$ 4.113,00
            	\item R\$ 2.106,00
            	\item R\$ 3.945,00 \\\\\\\\\\\\\\\\\\\\\\\\\\\\\\\\\\\\\\\\\\\\\\\\\\\\\\\\
            \end{enumerate}	
        	\item Um atleta, correndo com velocidade constante, completou a matatona em $M$ horas. A fração do percurso que ele correu em 5$M$ minutos foi:
        	\begin{enumerate}[a)]
        		\item $\frac{1}{4}$
        		\item $\frac{1}{12}$
        		\item $\frac{1}{16}$
        		\item $\frac{1}{20}$
        		\item $\frac{1}{27}$
        	\end{enumerate}
   	    \end{enumerate}
       $~$ \\ $~$ \\ $~$ \\ $~$ \\ $~$ \\ $~$ \\ $~$ \\ $~$ \\ $~$ \\ $~$ \\ $~$ \\ $~$ \\ $~$ \\ $~$ \\ $~$ \\ $~$ \\ $~$ \\ $~$ \\ $~$ \\ $~$ \\ $~$ \\ $~$ \\  $~$ \\  $~$ \\  $~$ \\ 
    \end{multicols}

\end{document}