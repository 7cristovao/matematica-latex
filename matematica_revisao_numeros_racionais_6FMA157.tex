\documentclass[a4paper,14pt]{article}

\usepackage{comment} % Para comentar várias linhas ao mesmo tempo

%matemática
\usepackage{amsmath}
\usepackage{amssymb}

%diagramação
\usepackage{extsizes}
\everymath{\displaystyle}
\usepackage{geometry}
\usepackage{fancyhdr}
\usepackage{multicol}
\usepackage{graphicx}
\usepackage[brazil]{babel}
\usepackage[shortlabels]{enumitem}
\usepackage{cancel}
\usepackage{textcomp}
\usepackage{tcolorbox}

%tabelas
\usepackage{array} % Para melhor formatação de tabelas
\usepackage{longtable}
\usepackage{booktabs}  % Para linhas horizontais mais bonitas
\usepackage{float}   % Para usar o modificador [H]
\usepackage{caption} % Para usar legendas em tabelas
\usepackage{wrapfig} % Para usar tabelas e figuras flutuantes
\usepackage{xcolor} % Para cores do fundo de tabelas
\usepackage{colortbl} % Para cores do fundo de tabelas
\usepackage{upgreek} % Para inserir caracteres gregos

%tikzpicture
\begin{comment}
	\usepackage{tikz}
	\usepackage{scalerel}
	\usepackage{pict2e}
	\usepackage{tkz-euclide}
	\usetikzlibrary{calc}
	\usetikzlibrary{patterns,arrows.meta}
	\usetikzlibrary{shadows}
	\usetikzlibrary{external}
\end{comment}


%pgfplots
\usepackage{pgfplots}
\pgfplotsset{compat=newest}
\usepgfplotslibrary{statistics}
\usepgfplotslibrary{fillbetween}

%colours
\usepackage{xcolor}



\columnsep=2cm
\hoffset=0cm
\textwidth=8cm
\setlength{\columnseprule}{.1pt}
\setlength{\columnsep}{2cm}
\renewcommand{\headrulewidth}{0pt}
\geometry{top=1in, bottom=1in, left=0.7in, right=0.5in}

\pagestyle{fancy}
\fancyhf{}
\fancyfoot[C]{\thepage}

\begin{document}
	
	\noindent\textbf{6FMA157 - Matemática} 
	
	\begin{center}Revisão de números racionais (Versão estudante)
	\end{center}
	
	\noindent\textbf{Nome:} \underline{\hspace{10cm}}
	\noindent\textbf{Data:} \underline{\hspace{4cm}}
	
	%\section*{Questões de Matemática}
	
	\begin{multicols}{2}
	    \noindent \begin{itemize}
	    	\item Comparar frações: necessita do mesmo denominador.
	    	\item Uma fração não se altera quando multiplicamos ou dividimos o numerador e o denominador pelo mesmo número.
	    	\item Número decimal: o denominador é uma potência de 10.
	    	\item Para $a, b \in \mathbb{Z}^*$: \\
	    	\small mdc $(a, b)$ = máx. $(D_+ (a) \cap D_+ (b))$
	    	\normalsize \item Para $a, b \in \mathbb{Z}^*$: \\
	    	\small mmc $(a, b)$ = mín. $(M_+^* (a) \cap M_+^* (b))$
	    	\normalsize \item Igualar o denominador: necessita do mmc.
	    \end{itemize}
		\noindent\textsubscript{--------------------------------------------------------------------------}
		\begin{enumerate} 
			\item Coloque as seguintes frações em ordem crescente: \\\\
			$\frac{4}{5}, 1, \frac{5}{12}, \frac{14}{15}, \frac{1}{2}, \frac{7}{10}$ \\\\\\\\\\\\\\\\\\\\
			\item Transforme as frações a seguir em números decimais e os números decimais em frações irredutíveis. \\
			\begin{enumerate}[a)]
				\item 0,12 \\\\\\\\\\\\\\\\\\\\
				\item $\frac{3}{5}$ \\\\\\\\\\\\\\\\\\\\
				\item 1,39 \newpage
				\item $\frac{49}{20}$ \\\\\\\\\\\\\\\\\\\\
			\end{enumerate}
			\item Resolva as expressões a seguir:
			\begin{enumerate}[a)]
				\item $\frac{2}{3} + \frac{1}{5} - \frac{1}{7}$ \\\\\\\\\\\\\\\\\\\\\\\\\\\\\\\\\\\\\\\\\\\\\\\\\\\\
				\item $\bigg(\frac{3}{8} + \frac{1}{9}\bigg) \cdot \frac{72}{7}$ \\\\\\\\\\\\\\\\\\\\\\\\\\\\\\\\\\\\\\\\
				\item $\bigg(\frac{2}{6} \cdot \frac{9}{8} : \frac{3}{2}\bigg)$ \\\\\\\\\\\\\\\\\\\\\\\\\\\\\\\\\\\\
				\item $\bigg[\bigg(\frac{1}{7} + \frac{1}{2} - \frac{1}{9}\bigg) \cdot \frac{3}{5}\bigg] + \frac{2}{3}$ \\\\\\\\\\\\\\\\\\\\\\\\\\\\\\\\\\\\\\\\
			\end{enumerate}
			\textbf{Desafio olímpico} \\\\
			(OBMEP) A professora Luísa observou que o número de meninas de sua turma dividido pelo número de meninos dessa mesma turma é 0,48. Qual é o menor número possível de alunos dessa turma? \\
			\begin{enumerate}[a)]
				\item 24
				\item 37
				\item 40
				\item 45
				\item 48 \newpage
			\end{enumerate}
			%54 a 56
			\item Calcule:
			\begin{enumerate}[a)]
				\item mmc (35, 40) \\\\\\\\\\\\
				\item mdc (138, 54) \\\\\\\\\\\\
			\end{enumerate}
			\item Organize em ordem crescente. \\\\
			$-\frac{9}{4}, \frac{7}{3}, - \frac{6}{5}, \frac{11}{4}, -\frac{8}{3}, \frac{9}{5}$. \\\\\\\\\\\\\\\\\\\\
			\item Transforme em frações irredutíveis:
			\begin{enumerate}[a)]
				\item $4,\overline{1}$ \\\\\\\\\\\\\\\\
				\item $3,0\overline{6}$ \\\\\\\\\\\\
				\item $5,\overline{89}$ \\\\\\\\\\\\
				\item $0,75$ \\\\\\\\\\\\
				\item $2,96$ \\\\\\\\\\\\
				\item $2,\overline{96}$ \\\\\\\\\\\\
			\end{enumerate}
		\end{enumerate}
		$~$ \\ $~$ \\ $~$ \\ $~$ \\ $~$ 
	\end{multicols}
\end{document}