\documentclass[a4paper,14pt]{article}
\usepackage{extsizes}
\usepackage{amsmath}
\usepackage{amssymb}
\everymath{\displaystyle}
\usepackage{geometry}
\usepackage{fancyhdr}
\usepackage{multicol}
\usepackage{graphicx}
\usepackage[brazil]{babel}
\usepackage[shortlabels]{enumitem}
\usepackage{cancel}
\columnsep=2cm
\hoffset=0cm
\textwidth=8cm
\setlength{\columnseprule}{.1pt}
\setlength{\columnsep}{2cm}
\renewcommand{\headrulewidth}{0pt}
\geometry{top=0.5in, bottom=0.5in, left=0.7in, right=0.5in}

\pagestyle{fancy}
\fancyhf{}
\fancyfoot[C]{\thepage}

\begin{document}
	
	\noindent\textbf{7FMA147~-~Matemática} 
	
	\begin{center}
		\textbf{Revisão: Potências de expoentes fracionários (Versão professor)}
	\end{center}
	
	
	\noindent\textbf{Nome:} \underline{\hspace{10cm}}
    \noindent\textbf{Data:} \underline{\hspace{4cm}}
	
	%\section*{Questões de Matemática}
	
	\begin{multicols}{2}
		Se $a^\frac{m}{n} \in \mathbb{R}$, então $a^\frac{m}{n} = \sqrt[n]{a^m}$, para $a \geq 0$. Para $a < 0$, você deve verificar o sinal de $a^\frac{n}{m}$ e $\sqrt[n]{a^m}$
	\begin{enumerate}	
		\item Escreva na forma de raiz quando possível:
		\begin{enumerate}[a)]
			\item $4^\frac{1}{3} = \sqrt[3]{4^1} = \sqrt[3]{4}$ \\
			\item $8^\frac{1}{5} = \sqrt[5]{8^1} = \sqrt[5]{8} $\\
			\item $(-7)^\frac{1}{6} = \sqrt[6]{(-7)^1} = \nexists a < 0$ \\
			\item $-11^\frac{1}{4} = \sqrt[4]{11^1} = \nexists a < 0$ \\
	    \end{enumerate}
        \item Escreva na forma de potência com expoente racional, quando possível:
        \begin{enumerate}[a)]
        	\item $\sqrt[3]{7} = 7^\frac{1}{3}$ \\
        	\item $\sqrt{13} = \sqrt[2]{13^1} = 13^\frac{1}{2}$ \\
        	\item $\sqrt[3]{125} = 125^\frac{1}{3}$ \\ ou $\sqrt[3]{125^1} = \sqrt[3]{5^3} = 5^\frac{3}{3} = 5$ \\
        	\item $\sqrt[4]{81} = \sqrt[4]{3^4} = 3$ \\
        	\item $\sqrt[7]{9} = \sqrt[7]{3^2} = 3^\frac{2}{7}$ \\
        \end{enumerate}
        \item Escrever, quando possível, na forma $\sqrt{~~}$:
        \begin{enumerate}[a)] 
        	\item $3^\frac{2}{4} = 3^\frac{1}{2} = \sqrt[2]{3} = \sqrt{3}$\\
        	\item $13^\frac{3}{5} = \sqrt[5]{13^3}$ \\
        	\item $(-6)^\frac{4}{7} = \nexists a < 0$ \\
        	\item $(-9)^\frac{3}{8} = \nexists a < 0$ \\
        \end{enumerate}
        \item Escrever se possível, com expoente racional:
        \begin{enumerate}[a)]
        	\item $\sqrt[7]{5^6} = 5^\frac{6}{7}$ \\
        	\item $\sqrt[9]{4^{12}} = 4^\frac{12}{9} = 4^\frac{4}{3} = 2^{2^\frac{4}{3}} = 2^{2 \cdot \frac{4}{3}} = 2^\frac{8}{3}$ \\
        	\item $\sqrt[5]{(-11)^3} = \nexists a < 0$ \\
        	\item $\sqrt[11]{(-2)^6} = \nexists a < 0$ \\
        \end{enumerate}
        \item Assinale V(verdadeiro) ou F(falso):
        \begin{enumerate}[a)]
        	\item (V)~$(-1)^\frac{6}{6} = \sqrt[6]{(-1)^6}$\\
        	\item (V)~$\sqrt[5]{(-1)^5} = (-1)^\frac{5}{5}$ \\
        	\item (V)~$(-1)^\frac{4}{4} = (-1)$ \\
        	\item (F)~$(-1)^\frac{16}{12} = (-1)^\frac{4}{3}$ \\
        	\item (F)~$\sqrt{x^2} = |x|$~para todo $x \in \mathbb{R}$ \\
        	\item (F)~$\sqrt[7]{x^7} = x$~para todo $x \in \mathbb{R}$ \\
        \end{enumerate}
                \item Escreva, quando possível, na forma $\sqrt[n]{a}$, para $n \in \mathbb{N^*}$ e $a \in \mathbb{R}$:
        \begin{enumerate}[a)]
        	\item $5^\frac{1}{8} = \sqrt[8]{5^1}$ \\
        	\item $16^\frac{1}{4} = \sqrt[4]{16}$\\
        	\item $(-9)^\frac{1}{5} = \sqrt[5]{(-9)}$ \\
        	\item $(-2)^\frac{1}{6}$ não é possível, pois $(-2)^\frac{1}{6} \notin \mathbb{R}$\\
        \end{enumerate}
        \item Escreva na forma $a^\frac{1}{n}$,$~a \in \mathbb{R}$ e $n \in \mathbb{N^*}$, se possível:
        \begin{enumerate}[a)]
        	\item $\sqrt[6]{7} = 7^\frac{1}{6}$\\
        	\item $\sqrt[5]{-4} = -4^\frac{1}{5}$\\
        	\item $\sqrt[12]{3} = 3^\frac{1}{12}$\\
        	\item $\sqrt[8]{-7}$ não é possível, pois $\sqrt[8]{-7} \notin \mathbb{R}$\\
        \end{enumerate}
        \item Assinale V ou F:
        \begin{enumerate}[a)]
        	\item (F)~$\sqrt[4]{7^5} = 7^\frac{4}{5}$ Falsa, pois $\sqrt[4]{7^5} = 7^\frac{5}{4} \neq 7^\frac{4}{5}$
        	\item (V)~$\sqrt[6]{3^7} = 3^\frac{7}{6}$
        	\item (V)~$\sqrt[4]{27} = 3^\frac{3}{4}$
        	\item (V)~$1^\frac{9}{2} = \sqrt[2]{1^9}$
        	\item (V)~$0^\frac{7}{5} = \sqrt[5]{0^7}$
        	\item (V)~$(-9)^\frac{12}{5} = \sqrt[5]{(-9)^12}$
        	\item (F)~$(-2)^\frac{6}{4} = \sqrt[4]{(-2)^6}$ Falsa, pois $(-2)^\frac{6}{4} = (-2)^\frac{3}{2} \in \mathbb{R}$ e $\sqrt[4]{(-2)^6} = \sqrt[4]{2^6} = 2^\frac{6}{4} = 2^\frac{3}{2} = \sqrt{2^3} = \sqrt{8}$
        	\item (F)~$(-7)^\frac{8}{8} = \sqrt[8]{(-7)^8}$ Falsa, pois $(-7)^\frac{8}{8} = (-7)^1 = -7$ e $\sqrt[8]{(-7)^8} = \sqrt[8]{7^8} = 7$
        	\item (V)~$(-6)^\frac{3}{9} = \sqrt[9]{(-6)^3}$
        	\item (V)~$(-4)^\frac{9}{12} = \sqrt[12]{(-4)^9}$
        \end{enumerate}	
        \item Assinale V ou F. Sendo $a$ um número real e $m$ e $n$ números naturais não nulos, podemos afirmar que:
        \begin{enumerate}[a)]
        	\item (V)~$a^\frac{m}{n} = \sqrt[n]{a}$
        	\item (F)~$\sqrt[n]{a^m} = a^\frac{n}{m}$
        	\item (F)~$\sqrt[n]{a^m} = a^{mn}$
        	\item (V)~$\sqrt[n]{a^m} = a^\frac{m}{n}$
        	\item (V)~$\sqrt[n]{a^m} = a^\frac{m}{n}$, se $a > 0$.
        	\item (F)~$\sqrt[n]{a^m} = a^\frac{m}{n}$, se $n$ for ímpar e $m$ for par.
        	\item (F)~$\sqrt[n]{a^m} = a^\frac{m}{n}$, se $n$ for ímpar.
        	\item (F)~$\sqrt[n]{a^m} = a^\frac{m}{n}$, se $m$ for par.
        	\item (F)~$\sqrt[n]{a^m} = a^\frac{m}{n}$, se $m$ e $n$ forem primos entre si e $\sqrt[n]{a^m} \in \mathbb{R}$
        \end{enumerate}
    \end{enumerate}        
    \end{multicols}    
\end{document}