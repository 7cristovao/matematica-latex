\documentclass[a4paper,14pt]{article}
\usepackage{float}
\usepackage{extsizes}
\usepackage{amsmath}
\usepackage{amssymb}
\everymath{\displaystyle}
\usepackage{geometry}
\usepackage{fancyhdr}
\usepackage{multicol}
\usepackage{graphicx}
\usepackage[brazil]{babel}
\usepackage[shortlabels]{enumitem}
\usepackage{cancel}
\usepackage{textcomp}
\usepackage{array} % Para melhor formatação de tabelas
\usepackage{longtable}
\usepackage{booktabs}  % Para linhas horizontais mais bonitas
\usepackage{float}   % Para usar o modificador [H]
\usepackage{caption} % Para usar legendas em tabelas
\usepackage{tcolorbox}

\columnsep=2cm
\hoffset=0cm
\textwidth=8cm
\setlength{\columnseprule}{.1pt}
\setlength{\columnsep}{2cm}
\renewcommand{\headrulewidth}{0pt}
\geometry{top=1in, bottom=1in, left=0.7in, right=0.5in}

\pagestyle{fancy}
\fancyhf{}
\fancyfoot[C]{\thepage}

\begin{document}
	
	\noindent\textbf{6FMA101 - Matemática} 
	
	\begin{center}Frações: problemas (Versão estudante)
	\end{center}
	
	\noindent\textbf{Nome:} \underline{\hspace{10cm}}
	\noindent\textbf{Data:} \underline{\hspace{4cm}}
	
	%\section*{Questões de Matemática}
	~ \\ ~
	\begin{multicols}{2}
    	\begin{enumerate}
    		\item Daniel comprou um terreno de 6510 m² e destinou $\frac{2}{7}$ dessa área para o plantio de amoras. Qual é a área ocupada por essa plantação? \\\\\\\\\\\\\\\\\\\\\\\\\\\\\\\\\\\\\\\\\\\\\\\\\\\\\\
    		\item André pegou $\frac{21}{24}$ de uma pizza e dividiu o que pegou com seus amigos. Se Bruno ficou com $\frac{3}{8}$ e Carlos com $\frac{4}{16}$ da pizza, qual é a fração da pizza que ficou com André? Considerando que a pizza foi dividida em 8 pedaços, quantos pedaços cada um comeu? \\\\\\\\\\\\\\\\\\\\\\\\\\\\\\\\\\\\\\\\\\\\
   			\item Uma pessoa ganha R\$ 4.256,00 por mês e gasta $\frac{2}{7}$ desse valor em livros e alimentação. Essa pessoa gasta $\frac{1}{5}$ do que sobra em roupas. Quanto ela gasta com roupas? \\\\\\\\\\\\\\\\\\\\\\\\\\\\\\\\
   			\item Em uma loja há uma pilha com 20 tapetes importados, todos com $\frac{3}{5}$ polegadas de espessura. Qual é a altura da pilha? \\\\\\\\\\\\\\\\\\\\\\\\\\\\
   			\item Uma artesã, para fazer alguns lacinhos, cortou uma fita de 16 m em pedaços de $\frac{4}{31}$ m, cada um correspondente a um lacinho. Quantos lacinhos ela fez? \newpage
   			\item Complete o seguinte enunciado com algum número do quadro abaixo e, em seguida, resolva a questão:
   			\begin{center} \begin{tcolorbox}[colback=white, colframe=black, boxrule=0.5mm, width=4cm]
   					12 ~ 8 ~ 14 ~ 10
   				\end{tcolorbox}
   			\end{center}
   			Em uma madeireira, uma das máquinas corta toras de madeira em pequenos pedaços de mesmo tamanho, Se um funcionário iniciar o processo com uma tora de madeira de \underline{~~~~~~~~~~} m, quantos \\\\ pedaços de $\frac{2}{43}$~m haverá no final? \\
   			\textbf{Observação:}~utilize a calculadora para auxiliar nos cálculos.
   			\\\\\\\\\\\\\\\\\\\\\\\\\\\\\\\\\\\\\\\\\\\\\\
   			\textbf{Desafio Olímpico} \\\\
   			Em certo domingo, um oitavo do público de um restaurante era de crianças e $\frac{3}{7}$ dos adultos eram homens. Qual fração do público do restaurante nesse dia era de mulheres adultas?
   			\begin{enumerate}[a)]
   				\item $\frac{3}{8}$
   				\item $\frac{2}{3}$
   				\item $\frac{1}{2}$
   				\item $\frac{3}{4}$
   				\item $\frac{1}{5}$
   			\end{enumerate}
   			\newpage
   			\item Uma agência de viagens de São Paulo (SP) está organizando um pacote turístico com destino à cidade de Foz do Iguaçu (PR) e fretou um avião com 120 lugares. Do total de lugares, reservou $\frac{2}{5}$ das vagas para as pessoas que residem na capital do estado de São Paulo, $\frac{3}{8}$ para as que moram no interior desse estado e o restante para as que residem fora dele. \\
   			Quantas vagas estão reservadas no avião para as pessoas que moram fora do estado de São Paulo?
   			\begin{enumerate}[a)]
   				\item 27
   				\item 40
   				\item 45
   				\item 74
   				\item 81 \\\\\\\\\\\\\\\\\\\\\\\\\\\\\\\\\\\\
   			\end{enumerate}
   			\item Numa certa classe, $\frac{13}{24}$ dos alunos tiveram avaliação abaixo da média em Português e $\frac{7}{12}$ tiveram avaliação abaixo da média em Matemática, sendo que $\frac{3}{8}$ tiveram esse resultado nas duas matérias. Que parte da classe teve avaliação abaixo da média:
   			\begin{enumerate}[a)]
   				\item somente em Matemática? \\\\\\\\\\\\\\\\\\\\\\\\\\\\
   				\item somente em Português? \\\\\\\\\\\\\\\\\\\\\\\\\\\\
   			\end{enumerate}
   			\item Júlia comprou uma barra de chocolate e a dividiu com suas irmãs. Para a mais nova, Júlia deu $\frac{1}{4}$ da barra e para a mais velha, deu a metade de $\frac{7}{8}$ da barra. Qual é a fração que representa a parte da barra de chocolate que restou a Júlia? \\\\\\\\\\\\\\\\\\\\\\\\\\\\
   			\item Complete o enunciado seguinte com algum número do quadro abaixo e, em seguida, resolva o problema.
   			\begin{center} \begin{tcolorbox}[colback=white, colframe=black, boxrule=0.5mm, width=6cm]
   					$\frac{5}{6} ~ -\frac{1}{3} ~~~ \frac{7}{10} ~~~~ \frac{8}{3} ~~~~ \frac{20}{3} \\\\\\ \frac{1}{4} ~~~~ \frac{5}{8} ~~~~ \frac{15}{2} ~~~~ \frac{6}{4} ~~~~ \frac{1}{2}$
   				\end{tcolorbox}
   			\end{center}
   			Daniel comprou pacotes de figurinhas para dividi-las entre seus amigos. André pegou a metade, Breno pegou $\frac{1}{3}$ de ...... das figurinhas e Daniel ficou com as restantes. Qual é a fração que representa a parte de figurinhas destinadas a Daniel? \\\\\\\\\\\\\\\\\\\\\\\\\\\\
   			\item Antônio destina $\frac{2}{7}$ do seu salário para pagar a escola de seu filho, o equivalente a R\$ 980,00. A quarta parte do que sobra ele utiliza para comprar livros. Qual é o valor, em reais, que Antônio gasta com livros?
   			\begin{enumerate}[a)]
   				\item R\$ 731,00.
   				\item R\$ 612,50.
   				\item R\$ 543,00.
   				\item R\$ 420,20.
   				\item R\$ 321,00.
   			\end{enumerate}
    	\end{enumerate}
    $~$ \\ $~$ \\ $~$ \\ $~$ \\ $~$ \\ $~$ \\ $~$ \\ $~$ \\ $~$ \\ $~$
	\end{multicols}
\end{document}