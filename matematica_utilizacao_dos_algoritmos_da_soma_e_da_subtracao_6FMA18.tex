\documentclass[a4paper,14pt]{article}

\usepackage{comment} % Para comentar várias linhas ao mesmo tempo

%matemática
\usepackage{amsmath}
\usepackage{amssymb}

%diagramação
\usepackage{extsizes}
\everymath{\displaystyle}
\usepackage{geometry}
\usepackage{fancyhdr}
\usepackage{multicol}
\usepackage{graphicx}
\usepackage[brazil]{babel}
\usepackage[shortlabels]{enumitem}
\usepackage{cancel}
\usepackage{textcomp}
\usepackage{tcolorbox}

%tabelas
\usepackage{array} % Para melhor formatação de tabelas
\usepackage{longtable}
\usepackage{booktabs}  % Para linhas horizontais mais bonitas
\usepackage{float}   % Para usar o modificador [H]
\usepackage{caption} % Para usar legendas em tabelas
\usepackage{wrapfig} % Para usar tabelas e figuras flutuantes


%tikzpicture
\begin{comment}
	\usepackage{tikz}
	\usepackage{scalerel}
	\usepackage{pict2e}
	\usepackage{tkz-euclide}
	\usetikzlibrary{calc}
	\usetikzlibrary{patterns,arrows.meta}
	\usetikzlibrary{shadows}
	\usetikzlibrary{external}
\end{comment}


%pgfplots
\usepackage{pgfplots}
\pgfplotsset{compat=newest}
\usepgfplotslibrary{statistics}
\usepgfplotslibrary{fillbetween}

%colours
\usepackage{xcolor}



\columnsep=2cm
\hoffset=0cm
\textwidth=8cm
\setlength{\columnseprule}{.1pt}
\setlength{\columnsep}{2cm}
\renewcommand{\headrulewidth}{0pt}
\geometry{top=1in, bottom=1in, left=0.7in, right=0.5in}

\pagestyle{fancy}
\fancyhf{}
\fancyfoot[C]{\thepage}

\begin{document}
	
	\noindent\textbf{6FMA18 - Matemática} 
	
	\begin{center}Utilização dos algoritmos da soma e da subtração (Versão estudante)
	\end{center}
	
	\noindent\textbf{Nome:} \underline{\hspace{10cm}}
	\noindent\textbf{Data:} \underline{\hspace{4cm}}
	
	%\section*{Questões de Matemática}
	
	\begin{multicols}{2}
		\begin{enumerate} 
			\item Calcule
			\begin{enumerate}[a)]
				\item 29 + 64 \\\\\\\\\\
				\item 38 + 47 \\\\\\\\\\
				\item 123 + 56 \\\\\\\\\\
				\item 108 + 283 \\\\\\\\\\
				\item 563 + 347 \\\\\\\\\\\\\\
				\item 2 987 + 725 \\\\\\\\\\
				\item 2 834 + 3 762 \\\\\\\\\\
				\item 15 083 + 7 232 \\\\\\\\\\
				\item 672 + 453 + 1 204 \\\\\\\\\\
			\end{enumerate}
			\item Calcule.
			\begin{enumerate}[a)]
				\item 86 - 32 \\\\\\\\\\\\\\
				\item 118 - 26 \\\\\\\\\\
				\item 308 - 125 \\\\\\\\\\
				\item 3 042 - 963 \\\\\\\\\\
				\item 7 210 - 4065 \\\\\\\\\\
				\item 21 712 - 3 864 \\\\\\\\\\
				\item 43 108 - 25 632 \\\\\\\\\\\\\\\\
			\end{enumerate}
			\textbf{Desafio olímpico} \\\\
			(OBMEP) Ana listou todos os números de três algarismos em que um dos algarismos é par e os outros dois são ímpares e diferentes entre si. Beto fez outra lista com todos os números de três algarismos em que um dos algarismos é ímpar e os outros dois são pares e diferentes entre si. Qual é a maior diferença possível entre um número da lista de Ana e um número da lista de Beto?
			\begin{enumerate}[a)]
				\item 795
				\item 863
				\item 867
				\item 873
				\item 885
			\end{enumerate}
			%64 a 68
			\item Efetue.
			\begin{enumerate}[a)]
				\item 4 + 2 \\\\\\
				\item 13 + 21 \\\\\\
				\item 248 + 129 \newpage
				\item 1 314 + 9 873 \\\\\\\\\\
				\item 23 471 + 18 952 \\\\\\\\\\
				\item 7 809 + 420 \\\\\\\\\\
				\item 2 563 + 1032 \\\\\\\\\\
				\item 1 463 + 2172 \\\\\\\\\\
				\item 9 370 + 3 214 \\\\\\\\\\
				\item 9 999 + 1 \\\\
				\item 807 + 103 \\\\\\\\\\
				\item 5 081 + 1 320 \\\\\\\\\\
				\item 4 324 + 3 527 \\\\\\\\\\
				\item 25 663 + 8 765 \\\\\\\\\\
				\item 17 652 + 32 706 \\\\\\\\\\
				\item 25 762 + 82 133 \newpage
				\item 18 902 + 42 101 \\\\\\\\\\
				\item 50 830 + 42 926 \\\\\\\\\\
				\item 72 813 + 1 876 321 \\\\\\\\\\
				\item 85 213 + 46 796 \\\\\\\\\\
				\item 123 456 789 + 111 113 \\\\\\\\\\
			\end{enumerate}
			\item Calcule.
			\begin{enumerate}[a)]
				\item 42 - 7 \\\\\\\\\\
				\item 37 - 19 \\\\\\\\\\
				\item 83 - 26 \\\\\\\\\\
				\item 115 - 84 \\\\\\\\\\
				\item 321 - 198 \\\\\\\\\\
				\item 3 402 - 789 \\\\\\\\\\
				\item 10 293 - 8 998 \\\\\\\\\\
				\item 200 002 - 78 962 \\\\\\\\\\
				\item 1000 - 83 \\\\\\\\\\
			\end{enumerate}
			\item Calcule a soma de todos os naturais de 50 a 100 \\\\\\\\\\\\\\\\\\\\\\\\\\\\\\
			\item Complete, colocando no lugar das letras os algarismos que tornam corretos os cálculos.
			\begin{enumerate}[a)]
				\item \[
				\begin{array}{cccc}
					~ & 3 & 2 & 1 \\
					+ & 5 & 8 & 9 \\
					\hline
					~ & z & y & x \\
				\end{array}
				\] \\\\
				$x = \underline{~~~~~} ~~~~y = \underline{~~~~~} ~~~~z = \underline{~~~~~} $
				\item \[
				\begin{array}{cccc}
					~ & 1 & 2 & x \\
					+ & y & 0 & 6 \\
					\hline
					~ & 4 & 2 & 8 \\
				\end{array}
				\] \\\\
				$x = \underline{~~~~~} ~~~~y = \underline{~~~~~}$
				\item \[
				\begin{array}{ccccc}
					~ & 3 & 9 & 2 & x \\
					+ & ~ & 1 & y & 7 \\
					\hline
					~ & 4 & z & 6 & 8\\
				\end{array}
				\] \\\\
				$x = \underline{~~~~~} ~~~~y = \underline{~~~~~} ~~~~z = \underline{~~~~~} $
			\end{enumerate}	
			\item Na soma abaixo, calcule os algarismos $x, y$ e $z$.
			 \[
			\begin{array}{ccc}
				~ & x & y \\
				+ & y & x \\
				\hline
				y & y & z \\
			\end{array}
			\] \\\\
			$x = \underline{~~~~~} ~~~~y = \underline{~~~~~} ~~~~z = \underline{~~~~~} $
		\end{enumerate}
		$~$ \\ $~$ \\ $~$ \\ $~$ \\ $~$ \\ $~$ \\ $~$ \\ $~$ \\ $~$ \\ $~$ \\ $~$ \\ $~$ \\ $~$ \\ $~$ \\ $~$ \\ $~$ \\ $~$ \\ $~$ \\ $~$ \\ $~$ \\ $~$ \\ $~$
	\end{multicols}
\end{document}