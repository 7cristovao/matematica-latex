\documentclass[a4paper,14pt]{article}
\usepackage{extsizes}
\usepackage{amsmath}
\usepackage{amssymb}
\everymath{\displaystyle}
\usepackage{geometry}
\usepackage{fancyhdr}
\usepackage{multicol}
\usepackage{graphicx}
\usepackage[brazil]{babel}
\usepackage[shortlabels]{enumitem}
\usepackage{cancel}
\columnsep=2cm
\hoffset=0cm
\textwidth=8cm
\setlength{\columnseprule}{.1pt}
\setlength{\columnsep}{2cm}
\renewcommand{\headrulewidth}{0pt}
\geometry{top=1in, bottom=1in, left=0.7in, right=0.5in}

\pagestyle{fancy}
\fancyhf{}
\fancyfoot[C]{\thepage}

\begin{document}
	
	\noindent\textbf{7FMA6~-~Matemática} 
	
	\begin{center}
		\textbf{Raiz de Índice Par e Ímpar (Versão estudante)}
	\end{center}
	
	\bigskip
	
	\noindent\textbf{Nome:} \underline{\hspace{10cm}}
    \noindent\textbf{Data:} \underline{\hspace{4cm}}
	
	\bigskip
	%\section*{Questões de Matemática}
	
	\begin{multicols}{2}
        \noindent Raiz de índice par positivo de um número negativo não pertence aos racionais. \\
        Exemplos:
    \begin{itemize}
    	\item $2^4 = 16 \Leftrightarrow \sqrt[4]{16} = 2$
    	\item $3^5 = 243 \Leftrightarrow \sqrt[5]{243} = 3$
    	\item $\sqrt[4]{-128} \notin \mathbb{Q}$
    \end{itemize}
	\begin{enumerate}
		
		\item Com base na informação fornecida, completar:
		\begin{enumerate}[a)]
			\item $2^4 = 16 \Leftrightarrow \sqrt[4]{16} = $
			\item $3^6 = 729 \Leftrightarrow \sqrt[6]{729} = $
			\item $4^4 = 256 \Leftrightarrow \sqrt[4]{256} = $
			\item $2^{10} = 1024 \Leftrightarrow \sqrt[10]{~~~~~~~} = 2$
			\item $9^2 = 81 \Leftrightarrow \sqrt[2]{~~~~~~} = 9$
			\item $1^{38} = 1 \Leftrightarrow \sqrt[38]{~~~~~} = 1$
			\item $4^6 = 4096 \Leftrightarrow \sqrt[6]{~~~~~~} = 4$
	    \end{enumerate}
    
        \item Utilizando a calculadora quando necessário, calcule:
		\begin{enumerate}[a)]
        	\item $\sqrt[16]{1} = $
        	\item $\sqrt[4]{81} = $
        	\item $\sqrt[8]{256} = $
        	\item $\sqrt[6]{729} = $
        	\item $\sqrt[12]{4096} = $
        	\item $\sqrt[4]{625} = $
        	\item $\sqrt[6]{~~~~~~~~~~} = 2$
        	\item $\sqrt[4]{~~~~~~~~~~} = 6$
        	\item $\sqrt[14]{~~~~~~~~~~} = 2$
        	\item $\sqrt[6]{~~~~~~~~~~} = 5$
        	\item $\sqrt[4]{~~~~~~~~~~} = 8$
        	\item $\sqrt[8]{~~~~~~~~~~} = 3$
        \end{enumerate}
        \item Com a informação dada, completar:
        \begin{enumerate}[a)]
        	\item $4^5 = 1024 \Leftrightarrow \sqrt[5]{1024} = $
        	\item $(-8)^3 = -512 \Leftrightarrow \sqrt[3]{-512} = $
        	\item $3^7 = 2187 \Leftrightarrow \sqrt[7]{2187} = $
        	\item $(-2)^5 = -32 \Leftrightarrow \sqrt[5]{-32} = $
        	\item $1^{15} = 1 \Leftrightarrow \sqrt[15]{1} = $
        	\item $(-3)^5 = -243 \Leftrightarrow \sqrt[5]{-243} = $
        	\item $(-4)^5 = -1024 \Leftrightarrow \sqrt[5]{~~~} = -4$
        	\item $8^3 = 512 \Leftrightarrow \sqrt[3]{~~~~~~~~~~~~~~} = 8 $
        	\item $(-3)^7 = -2187 \Leftrightarrow \sqrt[7]{~~~} = -3$
        	\item $2^5 = 32 \Leftrightarrow \sqrt[5]{~~~~~~~~~~~~~~} = 2$
        	\item $1^{23} = 1 \Leftrightarrow \sqrt[23]{~~~~~~~~~~~~~~} = 1$
        	\item $3^5 = 243 \Leftrightarrow \sqrt[5]{~~~~~~~~~~~~~~} = 3$
        	\item $(-9)^3 = -729 \Leftrightarrow \sqrt[3]{~~~~} = -9$
        	\item $(-12)^3 = -1728 \Leftrightarrow \sqrt[3]{~~~~~~~~~} = -12$
        \end{enumerate}
        \item Utilizando a calculadora quando necessário, calcule:
        \begin{enumerate}[a)]
        	\item $\sqrt[1]{20} = $
        	\item $\sqrt[9]{512} = $
        	\item $\sqrt[3]{-64} = $
        	\item $\sqrt[5]{243} = $
        	\item $\sqrt[3]{-216} = $
        	\item $\sqrt[7]{2187} = $
        	\item $\sqrt[3]{~~~~~~~~~~~~} = -9$
        	\item $\sqrt[5]{~~~~~~~~~~~~} = 4$
        	\item $\sqrt[7]{~~~~~~~~~~~~} = -2$
        \end{enumerate}
        \item Calcular (U = $\mathbb{Q}$):
        \begin{enumerate}
           	\item a raiz cúbica de -27 mais a raiz sexta de 64.
           	\vspace{8cm}
           	\item a raiz de índice 10 de -1024 menos a raiz de índice 7 de -128.
        \end{enumerate}
        
        
    \end{enumerate}        
    \end{multicols}    

\end{document}