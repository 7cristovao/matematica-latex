\documentclass[a4paper,14pt]{article}
\usepackage{float}
\usepackage{extsizes}
\usepackage{amsmath}
\usepackage{amssymb}
\everymath{\displaystyle}
\usepackage{geometry}
\usepackage{fancyhdr}
\usepackage{multicol}
\usepackage{graphicx}
\usepackage[brazil]{babel}
\usepackage[shortlabels]{enumitem}
\usepackage{cancel}
\columnsep=2cm
\hoffset=0cm
\textwidth=8cm
\setlength{\columnseprule}{.1pt}
\setlength{\columnsep}{2cm}
\renewcommand{\headrulewidth}{0pt}
\geometry{top=1in, bottom=1in, left=0.7in, right=0.5in}

\pagestyle{fancy}
\fancyhf{}
\fancyfoot[C]{\thepage}

\begin{document}
	
	\noindent\textbf{8FMA38~Matemática} 
	
	\begin{center}Problemas com dois móveis (III) (Versão estudante)
	\end{center}
	
	\noindent\textbf{Nome:} \underline{\hspace{10cm}}
	\noindent\textbf{Data:} \underline{\hspace{4cm}}
	
	%\section*{Questões de Matemática}
	
    \begin{multicols}{2}
    	\begin{enumerate}
    		\item Um trem de carga parte da cidade A para a cidade B, desenvolvendo 50 km/h. Três horas depois, um trem de passageiros parte da mesma estação com destino a cidade B, em uma linha paralela a do trem de carga, desenvolvendo 80 km/h. Quanto tempo será necessário para o trem de passageiros alcançar o trem de carga? \\\\\\\\\\\\\\\\
    		\item André e Paulo foram a um acampamento com suas motos. Um dia, André saiu do acampamento para ir à cidade com sua moto 30 minutos depois, Paulo decidiu ir também. Se André viajava a 40 km/h e Paulo a 50 km/h, a que distância do acampamento Paulo alcançou o André? \\\\\\\\\\\\\\
    		\item Dois carros viajam para São Paulo. Um deles está 80 km à frente do outro, na rodovia Castelo Branco. O que está à frente anda 95 km/h, enquanto o outro anda a 120 km/h. Em quanto tempo o carro de trás alcança o que está à frente? \\\\\\\\\\\\\\
    		\item Renato anda de bicicleta até o ponto de ônibus a 12 km/h. Lá ele guarda a bicicleta e toma um ônibus, que viaja a 57 Km por hora até seu trabalho. No ônibus, ele passa 40 minutos a menos que na bicicleta e no percurso de sua casa ao trabalho é de 31 km. Qual é a distância de sua casa ao ponto de ônibus? \\\\\\\\\\\\\\\\\\
    		
    		\textbf{Desafio olímpico}
    		
    		(OBMEP) André partiu de Pirajuba, foi até Quixajuba e voltou sem parar, com velocidade constante. Simultaneamente, e pela mesma estrada, Júlio partiu de Quixajuba, foi até Pirajuba e voltou, também sem parar e com velocidade constante. Eles se encontraram pela primeira vez a 70 km de Quixajuba e uma segunda vez a 40 km de Pirajuba, quando ambos voltavam para sua cidade de origem. 
    		
    		Quantos quilômetros tem a estrada de Quixajuba a Pirajuba?\\\\\\\\\\\\\\\\\\\\\\\\\\\\\\\\\\\\\\\\\\\\\\\\
    		
    		\item Henrique saiu de casa às 11 h e foi para escola andando a 90 passos por minuto; a cada passo, Henrique se desloca 60 cm. Como Henrique esqueceu seu guarda-chuva em casa, seu pai correndo para levá-lo 4,1 minutos depois, a uma velocidade de 18 km/h. Pergunta-se: \\\\
    		a) Quanto tempo levou o pai de Henrique para alcançá-lo? 
\\\\\\\\\\\\\\\\\\
    		
    		b) A que distância de casa Henrique estava? \\\\\\\\\\\\\\\\\\
    		
    		\item André saiu de São Paulo às 7h e viajou em direção a oeste uma velocidade constante de 90 km/h. Guilherme saiu do mesmo lugar às 07h45min e pegou o mesmo caminho que André, viajando a uma velocidade de 120 km/h. \\\\A que horas Guilherme ultrapassou o André? 
    		
    		a) 9h
    		
    		b) 09h25min
    		
    		c) 09h45min
    		
    		d) 10h
    		
    		e) 10h25min\\\\\\\\\\\\\\\\\\\\\\
    		
    		\item Em um estádio esportivo, uma pista circular tem comprimento igual a 160 m. Dois atletas percorrem a pista no mesmo sentido com velocidades constantes a~=~12~m/s e b~=~7~m/s. Ambos passam por o mesmo ponto da data zero. O corredor mais veloz estará com uma volta de vantagem sobre o outro na data: 
    		
    		a) 90 s
    		
    		b) 21 s
    		
    		c) 57 s
    		
    		d) 12 s
    		
    		e) 32 s\\\\\\\\
    		
    		\item Uma moto ultrapassa um caminhão de 12 m de comprimento, que se desloca no mesmo sentido. Se a velocidade da moto é o triplo da velocidade do caminhão, qual é o espaço percorrido pela moto desde o instante em que alcança o caminhão até o instante em que o ultrapassa?
    	\end{enumerate}
    $~$ \\ $~$ \\ $~$ \\ $~$ \\ $~$ \\ $~$ \\ $~$ \\ $~$ \\ $~$ \\ $~$ \\ $~$ \\ $~$ \\ $~$ \\ $~$ \\ $~$ \\ $~$ \\ $~$ \\ $~$ \\ $~$ \\ $~$ \\ $~$ \\ $~$ \\ $~$ \\ $~$ \\ $~$ \\ $~$ \\ $~$ \\ $~$ \\ $~$ \\ $~$ \\ $~$ \\ 
    \end{multicols}
\end{document}