\documentclass[a4paper,14pt]{article}
\usepackage{float}
\usepackage{extsizes}
\usepackage{amsmath}
\usepackage{amssymb}
\everymath{\displaystyle}
\usepackage{geometry}
\usepackage{fancyhdr}
\usepackage{multicol}
\usepackage{graphicx}
\usepackage[brazil]{babel}
\usepackage[shortlabels]{enumitem}
\usepackage{cancel}
\usepackage{textcomp}
\usepackage{array} % Para melhor formatação de tabelas
\usepackage{longtable}
\usepackage{booktabs}  % Para linhas horizontais mais bonitas
\usepackage{float}   % Para usar o modificador [H]
\usepackage{caption} % Para usar legendas em tabelas
\usepackage{tcolorbox}

\columnsep=2cm
\hoffset=0cm
\textwidth=8cm
\setlength{\columnseprule}{.1pt}
\setlength{\columnsep}{2cm}
\renewcommand{\headrulewidth}{0pt}
\geometry{top=1in, bottom=1in, left=0.7in, right=0.5in}

\pagestyle{fancy}
\fancyhf{}
\fancyfoot[C]{\thepage}

\begin{document}
	
	\noindent\textbf{6FMA44 - Matemática} 
	
	\begin{center}Propriedades comutativa e distributiva da multiplicação (Versão estudante)
	\end{center}
	
	\noindent\textbf{Nome:} \underline{\hspace{10cm}}
	\noindent\textbf{Data:} \underline{\hspace{4cm}}
	
	%\section*{Questões de Matemática}
	\begin{multicols}{2}
		\noindent A multiplicação é \textbf{comutativa}, isto é, $a \times b = b \times a$, ou seja, a ordem dos fatores não altera o produto. \\
		Outra propriedade importante é a \textbf{distributiva da multiplicação em relação à adição}, a qual diz que $a \times (b + c) = a \times b + a \times c$ ou, ainda, equivalentemente: \\ $a \cdot (b + c) = a \cdot b + a \cdot c$. \\
		A multiplicação de qualquer número natural por 0 é sempre igual a 0, ou seja, para todo natural $a$, temos $a \times 0 = 0$.
	\textsubscript{---------------------------------------------------------------------}
    	\begin{enumerate}
    		\item Calcule: \\
    		\begin{enumerate}[a)]
    			\item 42 + 8 \\\\\\
    			\item 8 + 42 \\\\\\
    			A adição é comutativa? Explique. \\\\\\
    			\item 7 - 4 \\\\\\
    			\item 4 - 7 \\\\\\
    			A subtração é comutativa? Explique. \\\\\\
    		\end{enumerate}
    		\item Assinale \textbf{V} (verdadeiro) ou \textbf{F} (falso) para as igualdades abaixo:
    		\begin{enumerate}[a)]
    			\item (~~~) $8 \times 2 = 2 \times 8$
    			\item (~~~) $3 \times 7 = 3 \times 4 \times 3$
    			\item (~~~) $2 \times 3 \times 9 = 3 \times 9 \times 2$
    			\item (~~~) $21 \times 5 \times 6 = 5 \times 20 \times 6 \times 1$
    		\end{enumerate}
    		\item Como já vimos, numa expressão em que aparecem as duas operações (adição e multiplicação), devem ser resolvidas primeiro as multiplicações, a não ser que haja sinais indicando outra ordem. Os parênteses são usados exatamente para isso. Coloque parênteses de forma que a igualdade abaixo se torne verdadeira:
    		\begin{center}$3 + 4 \times 7 = 49$ \end{center}
    		\item Calcule o resultado, aplicando antes a distributiva:
    		\begin{center}$(7 + 6) \cdot 4$ \vspace{6cm}\end{center} 
    		\item Use a propriedade distributiva da multiplicação em relação à adição, sem calcular o resultado.
    		\begin{enumerate}[a)]
    			\item $21 \times (12 + 86)$ \\\\\\
    			\item $897 \times (42 + 329)$ \\\\\\
    			\item $17 \cdot (59 + 53)$ \\\\\\
    			\item $46(38 + 1)$ \\\\\\
    		\end{enumerate}
    		\item Quanto vale $0 \times 0$? \\\\\\\\
    		\item Calcule $123 456 789 \cdot 0$. \\\\\\\\
    		\item Coloque parênteses de forma que a igualdade se torne verdadeira. 
    		\begin{enumerate}[a)]
    			\item $5 + 4 \cdot 6 = 54$ 
    			\item $6 \cdot 3 + 4 \cdot 5 = 210$
    		\end{enumerate}
    	\end{enumerate}
    	$~$ \\ $~$ \\ $~$ \\ $~$ \\ $~$ \\ $~$ \\ $~$ \\ $~$ \\ $~$ \\ $~$ \\ $~$ \\ $~$ \\ $~$ \\ $~$ \\ $~$ \\ $~$ \\ $~$ \\ $~$ \\ $~$ \\ $~$ \\ $~$ \\ $~$ \\ $~$ \\ $~$ \\ $~$ \\ $~$ \\ $~$ \\ $~$ \\ $~$ \\ $~$ \\ $~$ \\ $~$ \\ $~$ \\ $~$ \\ $~$ \\ $~$ \\ $~$ \\ $~$ \\ $~$ \\ $~$ \\ $~$ \\ $~$ \\ $~$ \\ $~$ \\
	\end{multicols}
\end{document}