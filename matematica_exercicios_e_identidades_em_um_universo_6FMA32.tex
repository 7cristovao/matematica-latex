\documentclass[a4paper,14pt]{article}
\usepackage{float}
\usepackage{extsizes}
\usepackage{amsmath}
\usepackage{amssymb}
\everymath{\displaystyle}
\usepackage{geometry}
\usepackage{fancyhdr}
\usepackage{multicol}
\usepackage{graphicx}
\usepackage[brazil]{babel}
\usepackage[shortlabels]{enumitem}
\usepackage{cancel}
\usepackage{textcomp}
\usepackage{array}
\usepackage{longtable}
\usepackage{booktabs}
\usepackage{float}   % Para usar o modificador [H]

\columnsep=2cm
\hoffset=0cm
\textwidth=8cm
\setlength{\columnseprule}{.1pt}
\setlength{\columnsep}{2cm}
\renewcommand{\headrulewidth}{0pt}
\geometry{top=1in, bottom=1in, left=0.7in, right=0.5in}

\pagestyle{fancy}
\fancyhf{}
\fancyfoot[C]{\thepage}

\begin{document}
	
	\noindent\textbf{6FMA32 - Matemática} 
	
	\begin{center}Exercícios e identidades em um universo (Versão estudante)
	\end{center}
	
	\noindent\textbf{Nome:} \underline{\hspace{10cm}}
	\noindent\textbf{Data:} \underline{\hspace{4cm}}
	
	%\section*{Questões de Matemática}
	~ \\
    \begin{multicols}{2}
    	\noindent Se o conjunto verdade de uma equação com universo $U$ é o próprio conjunto universo, temos uma \textbf{identidade} em $U$. 
    	\\
    	Por exemplo, a sentença aberta $x = -(-x)$ é uma identidade em $\mathbb{Z}$.
    	\noindent\textsubscript{~------------------------------------------------------------------------}
    	\begin{enumerate}
    		\item \begin{enumerate}[I.]
    			\item Em cada caso, dizer se a sentença aberta é uma equação ou inequação.
    			\item Resolva, isto é, apresente o conjunto verdade, no universo $U = \mathbb{Z}$.
    		\end{enumerate}
    		\begin{enumerate}[a)]
    			\item $1 < x$ \\\\\\\\
    			\item $x = 3$ \\\\\\\\
    			\item $-4 > x$ \\\\\\\\
    			\item $x = x$ \\\\
    			\item $x = -x$ \\\\\\\\
    			\item $-x = 7$ \\\\\\\\
    			\item $-x \geq 0$ \\\\\\\\
    			\item $-x = -8$ \\\\\\\\
    			\item $-12 = x$ \\\\\\\\
    			\item $3 = x$ \\\\\\\\
    			\item $-x = -x$ \\\\\\\\
    			\item $x > 0$ \\\\\\\\
    			\item $x < 0$ \\\\\\\\
    			\item $-x \leq 0$ \\\\\\\\
    			\item $-x = 0$ \\\\\\\\
    			\item $x > x$ \\\\\\\\\\
    		\end{enumerate}
    		\item Aponte, no exercício anterior, as identidades em $\mathbb{Z}$ \\\\\\\\\\\\\\
    		\item Apresente três identidades em $\mathbb{Z}$, diferentes das encontradas no exercício 1. \\\\\\\\\\\\
    		\item Descreva com suas próprias palavras o que é uma identidade. \\\\\\\\\\\\
    		\item Em cada caso: \begin{enumerate}[I.]
    			\item escreva se a sentença aberta é uma equação ou inequação;
    			\item resolva, isto é, apresente um conjunto verdade no universo $U = \mathbb{Z}$;
    			\item represente o conjunto verdade na reta.
    		\end{enumerate}
    			\begin{enumerate}[a)]
    				\item $x = 5$ \\\\\\
    				\item $0 \leq x$ \\\\\\\\
    				\item $-6 = -x$ \\\\
    				\item $x = x$ \\\\\\\\\\
    				\item $x = -x$ \\\\\\\\\\
    				\item $-2 < -x$ \\\\\\\\\\
    				\item $-x > 0$ \\\\\\\\\\
    				\item $-4 = x$ \\\\\\\\
    				\item $-x = -x$ \\\\\\\\\\
    				\item $x \leq x$ \\\\\\\\\\
    				\item $x > -x$ \\\\\\\\\\\\
    				\item $x \leq -x$ \\\\\\\\\\
    				\item $x = -(-x)$ \\\\\\\\
    				\item $0 > -x$ \\\\\\\\
    			\end{enumerate}
    		\item Aponte no exercício acima as identidades em $\mathbb{Z}$.
    		\item Aponte no exercício acima as identidades em $\mathbb{N}$.

    	\end{enumerate}
    $~$ \\ $~$ \\ $~$ \\ $~$ \\ $~$ \\ $~$ \\ $~$ \\ $~$ \\ 
    \end{multicols}
\end{document}