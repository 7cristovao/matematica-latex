\documentclass[a4paper,14pt]{article}
\usepackage{float}
\usepackage{extsizes}
\usepackage{amsmath}
\usepackage{amssymb}
\everymath{\displaystyle}
\usepackage{geometry}
\usepackage{fancyhdr}
\usepackage{multicol}
\usepackage{graphicx}
\usepackage[brazil]{babel}
\usepackage[shortlabels]{enumitem}
\usepackage{cancel}
\usepackage{textcomp}
\columnsep=2cm
\hoffset=0cm
\textwidth=8cm
\setlength{\columnseprule}{.1pt}
\setlength{\columnsep}{2cm}
\renewcommand{\headrulewidth}{0pt}
\geometry{top=1in, bottom=1in, left=0.7in, right=0.5in}

\pagestyle{fancy}
\fancyhf{}
\fancyfoot[C]{\thepage}

\begin{document}
	
	\noindent\textbf{8FMA77 - Matemática} 
	
	\begin{center}Divisão em casos (II) (Versão estudante)
	\end{center}
	
	\noindent\textbf{Nome:} \underline{\hspace{10cm}}
	\noindent\textbf{Data:} \underline{\hspace{4cm}}
	
	%\section*{Questões de Matemática}
	
	
    \begin{multicols}{2}
    	\noindent O princípio multiplicativo não resolve todos os problemas. Pode ser necessário dividir o problema em casos (você percebe isso se em algum momento o número de possibilidades de acontecer alguma coisa depender das escolhas anteriores). Você sempre deve tomar o cuidado de dividir o problema em casos, de modo que dois deles não possam conter uma mesma possibilidade.
    	\noindent\textsubscript{~---------------------------------------------------------------------------}
		\begin{enumerate}
			\item Numa livraria, há 8 livros distintos de terror, 5 livros distintos de poesia e 3 livros distintos de ficção científica. Há também 6 livros duplos distintos contendo um livro de terror e um de poesia e 9 livros duplos distintos contendo um livro de terror e um de ficção científica. De quantas formas podemos fazer uma compra que tenha exatamente um exemplar de cada um dos tipos? \\\\\\\\\\\\\\\\\\\\
			\item No código Morse, as letras, os algarismos e os sinais de pontuação são representados por sequências de traços e pontos. Por exemplo, a letra B é representada por quatro sinais (\textendash $\dotsm$); a letra N, por dois sinais  (\textendash $\cdot$); e a letra E, por um sinal ($\cdot$). Na comunicação usual, necessitamos de aproximadamente 50 caracteres.\\
			Até quantos sinais por sequência seriam necessários para representar todos os caracteres? \\\\\\\\\\\\\\\\\\\\\\\\\\\\\\\\\\\\\\\\\\\\
			\item As placas de automóvel de um certo país são formadas por três letras e quatro algarismos ou quatro letras distintas e três algarismos. Se o alfabeto usado nesse país possui 26 letras, quantos automóveis poderiam ser emplacados, no máximo? \\\\\\\\\\\\\\\\\\\\\\\\\\\\
			\item Com os algarismos 2, 4, 6, 9 e 3, quantos números naturais de três algarismos distintos, divisíveis por 3, podemos formar? \\\\\\\\\\\\\\\\\\\\\\\\\\\\
			\item O nome de usuário de uma empresa não pode ter mais do que 4 caracteres, todos diferentes entre si. Se a empresa permite o uso de 70 caracteres diferentes para compor um nome de usuário, qual é o número de maneiras diferentes de se criar este nome de usuário? \\\\\\\\\\\\\\\\\\\\\\\\\\\\
			\item Utilizando os algarismos 4, 5, 6, 7 e 8 sem repeti-los, quantos números de quatro algarismos maiores do que 5800 podemos escrever? \\\\\\\\\\\\\\\\\\\\\\\\
			\item Ao entrar em uma loja de chocolates, Bernardo notou que há 5 tabletes diferentes da marca $A$, 7 tabletes diferentes da marca $B$ e 4 tabletes diferentes da marca $C$. Há também 8 $kits$ diferentes com 1 tablete da marca $A$ e 1 da marca $C$. De quantas formas Bernardo pode fazer uma compra que tenha exatamente um tablete de cada marca de chocolate? \\\\\\\\\\\\\\\\\\\\\\\\\\\\
        \end{enumerate}
    $~$ \\ $~$ \\ $~$ \\ $~$ \\ $~$ \\ $~$ \\ $~$ \\ $~$ \\ $~$ \\ $~$ \\ $~$ \\ $~$ \\ $~$ \\ $~$ \\ $~$ \\ $~$ \\ $~$ \\ $~$ \\ $~$ \\ $~$ \\ $~$ \\ $~$ \\ $~$ \\ $~$ \\ $~$ \\ $~$ \\ $~$ \\ $~$ \\ $~$ \\ $~$ \\ $~$ \\ $~$ \\ $~$ \\ $~$ \\ $~$ \\ $~$ \\ $~$ \\ $~$ \\ $~$ \\ $~$ \\ $~$ \\$~$ \\ $~$ \\ $~$ \\ $~$ \\ $~$ \\ $~$ \\ $~$ \\ $~$ \\ $~$ \\ $~$ \\ $~$ \\ $~$ \\ $~$ \\ $~$ \\ $~$ \\ 
    \end{multicols}
\end{document}