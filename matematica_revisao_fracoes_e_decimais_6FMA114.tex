\documentclass[a4paper,14pt]{article}

\usepackage{comment} % Para comentar várias linhas ao mesmo tempo

%matemática
\usepackage{amsmath}
\usepackage{amssymb}

%diagramação
\usepackage{extsizes}
\everymath{\displaystyle}
\usepackage{geometry}
\usepackage{fancyhdr}
\usepackage{multicol}
\usepackage{graphicx}
\usepackage[brazil]{babel}
\usepackage[shortlabels]{enumitem}
\usepackage{cancel}
\usepackage{textcomp}
\usepackage{tcolorbox}

%tabelas
\usepackage{array} % Para melhor formatação de tabelas
\usepackage{longtable}
\usepackage{booktabs}  % Para linhas horizontais mais bonitas
\usepackage{float}   % Para usar o modificador [H]
\usepackage{caption} % Para usar legendas em tabelas
\usepackage{wrapfig} % Para usar tabelas e figuras flutuantes
\usepackage{xcolor} % Para cores do fundo de tabelas
\usepackage{colortbl} % Para cores do fundo de tabelas

%tikzpicture
\begin{comment}
	\usepackage{tikz}
	\usepackage{scalerel}
	\usepackage{pict2e}
	\usepackage{tkz-euclide}
	\usetikzlibrary{calc}
	\usetikzlibrary{patterns,arrows.meta}
	\usetikzlibrary{shadows}
	\usetikzlibrary{external}
\end{comment}


%pgfplots
\usepackage{pgfplots}
\pgfplotsset{compat=newest}
\usepgfplotslibrary{statistics}
\usepgfplotslibrary{fillbetween}

%colours
\usepackage{xcolor}



\columnsep=2cm
\hoffset=0cm
\textwidth=8cm
\setlength{\columnseprule}{.1pt}
\setlength{\columnsep}{2cm}
\renewcommand{\headrulewidth}{0pt}
\geometry{top=1in, bottom=1in, left=0.7in, right=0.5in}

\pagestyle{fancy}
\fancyhf{}
\fancyfoot[C]{\thepage}

\begin{document}
	
	\noindent\textbf{6FMA114 - Matemática} 
	
	\begin{center}Revisão: frações e decimais (Versão estudante)
	\end{center}
	
	\noindent\textbf{Nome:} \underline{\hspace{10cm}}
	\noindent\textbf{Data:} \underline{\hspace{4cm}}
	
	%\section*{Questões de Matemática}
	
	\begin{multicols}{2}
		\noindent Você deve lembrar que: \begin{itemize} 
		\item para somar frações de mesmo denominador, devemos somar os numeradores e manter o denominador: \\\\
		$\frac{1}{4} + \frac{2}{4} = \frac{1 + 2}{4} = \frac{3}{4}$ \\\\
		\item para subtrair frações de mesmo denominador, devemos subtrair os numeradores e manter o denominador: \\\\
		$\frac{9}{13} - \frac{6}{13} = \frac{9 - 6}{13} = \frac{3}{13}$ \\\\
		\item para multiplicar frações, multiplicamos os numeradores e os denominadores: \\\\
		$\frac{2}{3} \cdot \frac{5}{7} = \frac{2 \cdot 5}{3 \cdot 7} = \frac{10}{21}$ \\\\
		\item para dividir frações, multiplicamos a primeira fração pelo inverso da segunda fração. \\\\
		$\frac{3}{2} : \frac{7}{5} = \frac{3}{2} \cdot \frac{5}{7} = \frac{15}{14}$ \\\\
		\item para somar ou subtrair frações com denominadores diferentes, devemos primeiro transformá-las em frações equivalentes com o mesmo denominador: \\\\
		$\frac{2}{3} = \frac{3}{4} = \frac{8}{12} + \frac{9}{12} = \frac{17}{12}$
		\end{itemize}
	\end{multicols}
		\noindent\textsubscript{--------------------------------------------------------------------------------------------------------------------------------------------------------------}
	\begin{multicols}{2}
		\begin{enumerate} 
			\item Calcule:
			\begin{enumerate}[a)]
				\item $\frac{2}{5} + \frac{3}{5} + \frac{4}{5}$ \\\\\\\\
				\item $12 \cdot \frac{7}{9}$ \\\\\\\\
				\item $\frac{3}{7} - \frac{2}{3} + \frac{4}{5}$ \\\\\\\\
				\item $\frac{9}{7} : \frac{3}{8}$ \\\\\\\\\\
			\end{enumerate}
			\item Joana fez um bolo para comer com suas três amigas. Ela o repartiu igualmente em 12 pedaços e comeu 2 pedaços.
			\begin{enumerate}[a)]
				\item Que fração do bolo Joana comeu? \\\\\\\\\\\\
				\item As amigas Bianca e Patrícia também comeram 2 pedaços de bolo cada uma e Carol comeu 3. Quantos pedaços do bolo restaram? Qual é a fração que representa este valor em relação à quantidade inicial de pedaços de bolo?  \\\\\\\\\\\\
			\end{enumerate}
			\item Clara comprou 24 balas e distribuiu $\frac{3}{8}$ do total para seus amigos. Quantas balas restaram? \\\\\\\\\\\\\\\\\\\\
			\item Escreva na forma de numeral decimal.
			\begin{enumerate}[a)]
				\item $\frac{26}{10}$ \\\\\\\\\\\\
				\item $\frac{7}{10}$ \\\\\\\\\\\\
				\item $\frac{74}{100}$ \\\\\\\\\\\\
				\item $\frac{321}{100}$ \\\\\\\\\\\\
			\end{enumerate}
			\item Escrever na forma de fração decimal.
			\begin{enumerate}[a)]
				\item 61,7 \\\\\\\\
				\item 9,3 \\\\\\\\\\\\
				\item 2,05 \\\\\\\\\\\\
				\item 4,9 \\\\\\\\\\\\
			\end{enumerate}
			\item Escreva como se lê.
			\begin{enumerate}[a)]
				\item 2,31 \\\\\\\\
				\item $\frac{7}{10}$ \\\\\\\\
				\item $\frac{34}{10}$ \\\\\\\\
				\item 0,16 \\\\\\\\
			\end{enumerate}
			%13 a 17
			\item Escreva na forma de numeral decimal:
			\begin{enumerate}[a)]
				\item $\frac{207}{10}$ \\\\\\\\\\\\
				\item $\frac{1532}{100}$ \\\\\\\\\\\\
				\item $\frac{631}{100}$ \\\\\\\\\\\\
				\item $\frac{39,7}{10}$ \newpage
			\end{enumerate}
			\item Escreva na forma de fração decimal:
			\begin{enumerate}[a)]
				\item 21,2 \\\\\\\\\\
				\item 47,03 \\\\\\\\\\
				\item 5,6 \\\\\\\\\\
				\item 78,49 \\\\\\\\\\
			\end{enumerate}
			\item Complete:
			\begin{enumerate}[a)]
				\item $\frac{3}{5} = ... \cdot \frac{7}{8}$
				\item $\frac{2}{3} + \frac{5}{6} = ... \cdot \frac{7}{2}$
				\item $16 \cdot \frac{5}{128} = ... + \frac{7}{8}$
			\end{enumerate}
			\item Depois de algum tempo construindo uma piscina, Paula decidiu fazer uma festa em comemoração ao fim das obras. Após encher $\frac{91}{420}$ da piscina, seu irmão chegou para ajudá-la. Se a piscina de Paula tem capacidade para 30 000 litros, quantos litros de água Paula e seu irmão ainda precisam colocar para encher a piscina completamente? \\\\\\\\\\\\\\\\\\\\
			\item Renato tem um carro total $flex$. Esse tipo de carro aceita tanto álcool quanto gasolina como combustível e ambos podem ser colocados ao mesmo tempo no tanque. Imagine que o carro de Renato esteja com $\frac{4}{5}$ do tanque com gasolina e que ele gaste um terço dessa quantidade indo trabalhar. Na volta, ele completa o tanque com álcool. Nesse momento, que fração do tanque está com álcool?
			\begin{enumerate}[a)]
				\item $\frac{4}{15}$
				\item $\frac{7}{15}$
				\item 1
				\item $\frac{11}{15}$
				\item $\frac{13}{15}$
			\end{enumerate}
		\end{enumerate}
	\end{multicols}
\end{document}