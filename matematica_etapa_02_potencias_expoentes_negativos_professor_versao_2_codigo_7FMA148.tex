\documentclass[a4paper,14pt]{article}
\usepackage{extsizes}
\usepackage{amsmath}
\usepackage{amssymb}
\everymath{\displaystyle}
\usepackage{geometry}
\usepackage{fancyhdr}
\usepackage{multicol}
\usepackage{graphicx}
\usepackage[brazil]{babel}
\usepackage[shortlabels]{enumitem}
\usepackage{cancel}
\columnsep=2cm
\hoffset=0cm
\textwidth=8cm
\setlength{\columnseprule}{.1pt}
\setlength{\columnsep}{2cm}
\renewcommand{\headrulewidth}{0pt}
\geometry{top=1in, bottom=1in, left=0.7in, right=0.5in}

\pagestyle{fancy}
\fancyhf{}
\fancyfoot[C]{\thepage}

\begin{document}
	
	\noindent\textbf{7FMA148~-~Matemática} 
	
	\begin{center}
		\textbf{Potências de expoentes negativos (Versão professor)}
	\end{center}
	
	
	\noindent\textbf{Nome:} \underline{\hspace{10cm}}
    \noindent\textbf{Data:} \underline{\hspace{4cm}}
	
	%\section*{Questões de Matemática}
	
	\begin{multicols}{2}
		%
	\begin{enumerate}	
		\item Escrever usando radical:
		\begin{enumerate}[a)]
			\item $12^{-\frac{1}{4}} = \frac{1}{12^\frac{1}{4}} = \frac{1}{\sqrt[4]{12}}$ \\\\
			\item $5^{-\frac{5}{6}} = \frac{1}{5^{\frac{5}{6}}} = \frac{1}{\sqrt[6]{5^5}}$ \\\\
			\item $(-7)^{-\frac{8}{3}} = \frac{1}{(-7)^{\frac{8}{3}}} = \frac{1}{\sqrt[3]{(-7)^8}}$ \\\\
	    \end{enumerate}
        \item Escrever usando expoente racional:
        \begin{enumerate}[a)]
        	\item $\sqrt[-5]{9} = 9^{-\frac{1}{5}}$ \\\\
        	\item $\sqrt[9]{(-4)^{-5}} = (-4)^{-\frac{5}{9}} = \frac{1}{(-4)^\frac{5}{9}}$ \\\\
        	\item $\sqrt[-5]{(-6)^{10}} = (-6)^{-\frac{10}{5}} = \frac{1}{(-6)^\frac{10}{5}} = - \frac{1}{6^2}$ \\\\
        	\item $\frac{1}{\sqrt{4^5}} = \frac{1}{4^\frac{5}{2}} = 4^{-\frac{5}{2}} = 2^{2 \cdot -\frac{5}{2}} = 2^{-\frac{10}{2}} = 2^{-5}$ \\
        \end{enumerate}
        \item Calcule o que for possível com o que sabemos até agora:
        \begin{enumerate}[a)] 
        	\item $64^\frac{1}{2} = \sqrt[2]{64^1} = \sqrt{64} = 8$ \\
        	\item $(-8)^\frac{1}{3} = \sqrt[3]{(-8)^1} = \sqrt[3]{-2^3} = -2^\frac{3}{3} = -2$ \\
        	\item $0^\frac{31}{72} = 0$ \\
        	\item $1^{-\frac{5}{7}} = \frac{1}{1^\frac{5}{7}} = \frac{1}{\sqrt[7]{1^5}} = \frac{1}{1} = 1$ \\
        \end{enumerate}
        \item Calcule:
        \begin{enumerate}[a)]
        	\item $(5^{-1} + 3^{-1})^{-1} = \bigg(\frac{1}{5} + \frac{1}{3}\bigg)^{-1} = \frac{1}{\frac{1}{5}+\frac{1}{3}} = \frac{1}{\frac{3}{15}+\frac{5}{15}} = \frac{1}{\frac{8}{15}} = \frac{15}{8}$ \\
        	\item $\bigg(\bigg(\frac{1}{5}\bigg)^{-1} + \bigg(\frac{1}{3}\bigg)^{-1}\bigg)^{-1} = \frac{1}{\frac{1}{\frac{1}{5}}+\frac{1}{\frac{1}{3}}} = (5 + 3)^{-1} = 8^{-1} =  \frac{1}{8}$ \\
        	\item $\big(5^{-\frac{1}{2}} + 3^{-\frac{1}{2}}\big)^{-2} = \\ \bigg(\frac{1}{5^\frac{1}{2}} + \frac{1}{3^\frac{1}{2}}\bigg)^{-2} = \bigg(\frac{1}{5^\frac{1}{2}}\bigg)^{-2} + \bigg(\frac{1}{3^\frac{1}{2}}\bigg)^{-2} = (5^\frac{1}{2})^2 + (3^\frac{1}{2})^2 = 5^{2 \cdot \frac{1}{2}} + 3^{2 \cdot \frac{1}{2}} = 5 + 3 = 8$
        \end{enumerate}
            \item Escreva usando radical:
        \begin{enumerate}[a)]
        	\item $6^{-\frac{2}{5}} = \frac{1}{6^\frac{2}{5}} = \frac{1}{\sqrt[5]{6^2}} = \frac{1}{\sqrt[5]{36}}$ \\
        	\item $(-3)^{-\frac{2}{5}} = \frac{1}{(-3)^\frac{2}{5}} = \frac{1}{\sqrt[5]{(-3)^2}} = \frac{1}{\sqrt[5]{9}}$ \\
        \end{enumerate}
        \item Escreva na forma de expoente racional:
        \begin{enumerate}[a)]
        	\item $\frac{1}{\sqrt[11]{3^4}} = 3^{-\frac{4}{11}}$ \\
        	\item $\frac{1}{\sqrt[3]{(-5)^2}} = (-5)^{-\frac{2}{3}}$ \\
        \end{enumerate}
        \item Calcule, usando a definição de expoente racional, o valor de:
        \begin{enumerate}[a)]
        	\item $9^\frac{1}{2} = \sqrt[2]{3^2} = 3$ \\
        	\item $8^\frac{1}{3} = \sqrt[3]{2^3} = 2$ \\
        	\item $(-27)^\frac{1}{3} = \sqrt[3]{-27} = \sqrt[3]{-3^3} = -3$ \\
        	\item $1^\frac{53}{22} = 1$ \\
        	\item $0^\frac{5}{12} = 0$ \\
        	\item $(-1)^\frac{2}{19} = \sqrt[19]{(-1)^2} = \sqrt[19]{1} \\= 1$ \\
        	\item $81^\frac{1}{4} = \sqrt[4]{81} = \sqrt[4]{3^4} = 3$ \\
        	\item $81^{-\frac{1}{4}} = \frac{1}{\sqrt[4]{81}} = \frac{1}{\sqrt[4]{3^4}} = \frac{1}{3}$ \\
        	\item $1^{-\frac{13}{47}} = \frac{1}{\sqrt[47]{1^13}} = \frac{1}{\sqrt[47]{1}} = \frac{1}{1} \\= 1$ \\
        	\item $5^\frac{3}{2} = \sqrt[2]{5^3} = 5 \sqrt{5}$ \\
        	\item $5^{-\frac{3}{2}} = \frac{1}{\sqrt[2]{5^3}} = \frac{1}{\sqrt{5}}$ \\
        	\item $27^\frac{2}{3} = \sqrt[3]{3^{3^2}} = \sqrt[3]{3^6} = 3^2 = 9$ \\
        	\item $(-27)^\frac{2}{3} = \sqrt[3]{-27^2} = \sqrt[3]{-3^{3^2}} = \sqrt[3]{-3^6} = -3^2 = 9$ \\
        	\item $(-27)^{-\frac{2}{3}} = \frac{1}{\sqrt[3]{-27^2}} = \frac{1}{\sqrt[3]{-3^{3^2}}} = \frac{1}{-3^2} = \frac{1}{9}$ \\
        	\item $125^{-\frac{2}{3}} = \frac{1}{\sqrt[3]{125^2}} = \frac{1}{\sqrt[3]{5^{3^2}}} = \frac{1}{\sqrt[3]{5^6}} = \frac{1}{5^2} = \frac{1}{25}$ \\
        \end{enumerate}
    \end{enumerate}        
    \end{multicols}    
\end{document}