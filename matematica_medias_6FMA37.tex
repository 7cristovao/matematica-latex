\documentclass[a4paper,14pt]{article}
\usepackage{float}
\usepackage{extsizes}
\usepackage{amsmath}
\usepackage{amssymb}
\everymath{\displaystyle}
\usepackage{geometry}
\usepackage{fancyhdr}
\usepackage{multicol}
\usepackage{graphicx}
\usepackage[brazil]{babel}
\usepackage[shortlabels]{enumitem}
\usepackage{cancel}
\usepackage{textcomp}
\usepackage{array}
\usepackage{longtable}
\usepackage{booktabs}
\usepackage{float}   % Para usar o modificador [H]

\columnsep=2cm
\hoffset=0cm
\textwidth=8cm
\setlength{\columnseprule}{.1pt}
\setlength{\columnsep}{2cm}
\renewcommand{\headrulewidth}{0pt}
\geometry{top=1in, bottom=1in, left=0.7in, right=0.5in}

\pagestyle{fancy}
\fancyhf{}
\fancyfoot[C]{\thepage}

\begin{document}
	
	\noindent\textbf{6FMA37 - Matemática} 
	
	\begin{center}Médias (Versão estudante)
	\end{center}
	
	\noindent\textbf{Nome:} \underline{\hspace{10cm}}
	\noindent\textbf{Data:} \underline{\hspace{4cm}}
	
	%\section*{Questões de Matemática}
    \begin{multicols}{2}
    	\noindent A média aritmética de um conjunto com $n$ elementos numéricos é obtida pela razão entre a soma dos valores numéricos dos elementos do conjunto e a quantidade $n$ de elementos. \\
    	Ou seja, dados os números reais $a_1, a_2, ..., a_n$, sua média aritmética é:
    	$m = \frac{a_1 + a_2 + ... + a_n}{n}$
    	\noindent\textsubscript{~---------------------------------------------------------------------------}
    	\begin{enumerate}
			\item Calcule a média aritmética dos números.
			\begin{enumerate}[a)]
				\item 13 e 18. \\\\\\\\
				\item 25 e 31. \\\\\\\\
				\item 47 e 57. \\\\\\\\
				\item 81 e 92. \\\\\\\\
				\item 15, 17 e 19. \\
				\item 21, 46 e 58. \\\\\\\\
				\item 12, 26, 32 e 48. \\\\\\\\
			\end{enumerate}
			\item A média aritmética de $n$ números é 16. Ao retirar o número 4 desse conjunto, qual será a nova média se:
			\begin{enumerate}[a)]
				\item n = 7 \\\\\\\\
				\item n = 11 \\\\\\\\
			\end{enumerate}
		    \item Em uma sala de aula há 14 meninos e 16 meninas. A média aritmética das notas da prova de Matemática dessa turma é igual a 7,3. Sabe-se que a média aritmética das notas das meninas é igual a 8. Qual é a média aritmética da nota dos meninos?
    	\end{enumerate}
    \end{multicols}
\end{document}