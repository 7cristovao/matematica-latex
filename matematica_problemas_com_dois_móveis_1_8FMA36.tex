\documentclass[a4paper,14pt]{article}
\usepackage{float}
\usepackage{extsizes}
\usepackage{amsmath}
\usepackage{amssymb}
\everymath{\displaystyle}
\usepackage{geometry}
\usepackage{fancyhdr}
\usepackage{multicol}
\usepackage{graphicx}
\usepackage[brazil]{babel}
\usepackage[shortlabels]{enumitem}
\usepackage{cancel}
\columnsep=2cm
\hoffset=0cm
\textwidth=8cm
\setlength{\columnseprule}{.1pt}
\setlength{\columnsep}{2cm}
\renewcommand{\headrulewidth}{0pt}
\geometry{top=1in, bottom=1in, left=0.7in, right=0.5in}

\pagestyle{fancy}
\fancyhf{}
\fancyfoot[C]{\thepage}

\begin{document}
	
	\noindent\textbf{8FMA36~Matemática} 
	
	\begin{center}Problemas com dois móveis (I) (Versão estudante)
	\end{center}
	
	\noindent\textbf{Nome:} \underline{\hspace{10cm}}
	\noindent\textbf{Data:} \underline{\hspace{4cm}}
	
	%\section*{Questões de Matemática}
	
    \begin{multicols}{2}
    	\begin{enumerate}
    		\item Um automóvel $A$ parte do Espírito Santo para o Rio de Janeiro no mesmo instante em que um automóvel $B$ parte do Rio de Janeiro para o Espírito Santo. O automóvel $A$ mantém velocidade constante de 96 km/h e o $B$ mantém velocidade constante de 80 km/h. Admita que o percurso entre as duas cidades tem extensão de 704 km/h.
    		\begin{enumerate}[a)]
    			\item A que distância do Espírito Santo irá ocorrer o encontro entre os carros? \\\\\\\\\\\\\\\\\\
    			\item Depois de quanto tempo após a partida irá ocorrer o encontro? \\\\\\\\\\\\\\\\
    			\item Quanto tempo depois do encontro o automóvel $A$ levará até chegar ao Rio de Janeiro? \\\\\\\\\\\\\\\\\\\\\\\\\\
    			\item Os carros chegam em instantes diferentes a seus destinos. Qual deles chega antes? Quanto tempo depois o outro chega a seu destino? \\\\\\\\\\\\\\\\\\\\\\\\\\\\
    		\end{enumerate}
    	    \item Dois barcos partem das margens opostas de um rio, no mesmo instante. Um deles navega a 9 km/h e o outro a 15 km/h. Os dois barcos cruzam-se a 240 m da margem de onde partiu o barco mais lento. Qual é a largura do rio?  \\\\\\\\\\\\\\\\\\\\\\\\
    	    \item Carla faz uma viagem de negócios de sua cidade a Fortaleza em 3 horas. Uma hora depois, ela começa a viagem de volta para sua cidade a uma velocidade 30 km/h menor que a velocidade de ida. Carla retorna a sua cidade 8 horas depois de ter partido. Quantos quilômetros ela percorreu nessa viagem?  \\\\\\\\\\\\\\\\\\\\\\
    	    \item Dois nadadores, $A$ e $B$, pulam de beiradas opostas de uma piscina de 36 metros de comprimento no mesmo momento. O nadador $A$, que desenvolve uma velocidade de 7 m/s, encontra o nadador $B$ quando este já se encontrava a 15 m do seu ponto de partida. Qual é a velocidade do nadador $B$?  \\\\\\\\\\\\\\\\\\\\\\
    	    \item Dois aviões partem do Rio de Janeiro às 9h, um em direção norte, a 700 km/h, e outro em direção sul, a 800 km/h. A que horas esses aviões estarão a 1200 km de distância um do outro?  \\\\\\\\\\\\\\\\\\\\\\\\\\\\
    	    \item O carro $A$ parte de Bauru em direção a Araraquara no mesmo instante em que o carro $B$ parte de Araraquara em direção a Bauru. O percurso entre esses dois pontos, igual para os dois carros, é de 129 km. As velocidades desenvolvidas pelos carros $A$ e $B$ são, respectivamente, de 25 m/s e 35 m/s. A que distância de Bauru os carros irão encontrar-se?  \\\\\\\\\\\\\\\\\\\\\\\\\\\\\\\\\\\\\\\\\\\\\\\\\\\\\\\\\\
    	    \item O motociclista C parte do mesmo sentido que o motociclista $A$, no mesmo instante e do mesmo ponto de partida de um percurso circular com 5,5 km de comprimento. A velocidade do motociclista $C$ é de 260 km/h, e o segundo encontro entre eles ocorre 3 minutos depois da partida. Sabendo que o motociclista $C$ ultrapassa o motociclista $A$, qual é a distância percorrida, do ponto de partida até o segundo encontro, pelo motociclista $A$?
    	\end{enumerate}
    $~$ \\ $~$ \\ $~$ \\ $~$ \\ $~$ \\ $~$ \\ $~$ \\ $~$ \\ $~$ \\ $~$ \\ $~$ \\ $~$ \\ $~$ \\ $~$ \\ $~$ \\ $~$ \\ $~$ \\ $~$ \\ $~$ \\ $~$ \\ $~$ \\ $~$ \\ $~$ \\ $~$ \\ $~$ \\ $~$ \\
    \end{multicols}
\end{document}