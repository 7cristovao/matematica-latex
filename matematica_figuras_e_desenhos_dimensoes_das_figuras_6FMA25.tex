\documentclass[a4paper,14pt]{article}

\usepackage{comment} % Para comentar várias linhas ao mesmo tempo

%matemática
\usepackage{amsmath}
\usepackage{amssymb}

%diagramação
\usepackage{extsizes}
\everymath{\displaystyle}
\usepackage{geometry}
\usepackage{fancyhdr}
\usepackage{multicol}
\usepackage{graphicx}
\usepackage[brazil]{babel}
\usepackage[shortlabels]{enumitem}
\usepackage{cancel}
\usepackage{textcomp}
\usepackage{tcolorbox}

%tabelas
\usepackage{array} % Para melhor formatação de tabelas
\usepackage{longtable}
\usepackage{booktabs}  % Para linhas horizontais mais bonitas
\usepackage{float}   % Para usar o modificador [H]
\usepackage{caption} % Para usar legendas em tabelas
\usepackage{wrapfig} % Para usar tabelas e figuras flutuantes


%tikzpicture
\begin{comment}
	\usepackage{tikz}
	\usepackage{scalerel}
	\usepackage{pict2e}
	\usepackage{tkz-euclide}
	\usetikzlibrary{calc}
	\usetikzlibrary{patterns,arrows.meta}
	\usetikzlibrary{shadows}
	\usetikzlibrary{external}
\end{comment}


%pgfplots
\usepackage{pgfplots}
\pgfplotsset{compat=newest}
\usepgfplotslibrary{statistics}
\usepgfplotslibrary{fillbetween}

%colours
\usepackage{xcolor}



\columnsep=2cm
\hoffset=0cm
\textwidth=8cm
\setlength{\columnseprule}{.1pt}
\setlength{\columnsep}{2cm}
\renewcommand{\headrulewidth}{0pt}
\geometry{top=1in, bottom=1in, left=0.7in, right=0.5in}

\pagestyle{fancy}
\fancyhf{}
\fancyfoot[C]{\thepage}

\begin{document}
	
	\noindent\textbf{6FMA25 - Matemática} 
	
	\begin{center}Figuras e desenhos - Dimensões das figuras (Versão estudante)
	\end{center}
	
	\noindent\textbf{Nome:} \underline{\hspace{10cm}}
	\noindent\textbf{Data:} \underline{\hspace{4cm}}
	
	%\section*{Questões de Matemática}
	
	\begin{multicols}{2}
		\noindent Figuras são imagens que construímos em nossa mente, enquanto desenhos são traços que fazemos no papel, na lousa, no computador, etc. \\
		\begin{itemize}
			\item \textbf{Figura unidimensional}: tem apenas uma dimensão (comprimento). Exemplos: reta, semirreta, segmento de reta, etc.
			\item \textbf{Figura bidimensional}: tem duas dimensões (comprimento e largura). Exemplos: triângulo, quadrilátero, círculo, etc.
			\item \textbf{Figura tridimensional}: tem três dimensões (comprimento, largura e espessura). Exemplos: livro, caixa de sapatos, lata, etc.
		\end{itemize}
		\noindent\textsubscript{-----------------------------------------------------------------------}
		\begin{enumerate} 
			\item Assinale \textbf{V} (verdadeiro) ou \textbf{F} (falso).
			\begin{enumerate}[a)]
				\item (~~) Uma folha de papel é, rigorosamente falando, tridimensional, mas pode ser considerada bidimensional para fins práticos.
				\item (~~) O cubo é tridimensional, mas o desenho de um cubo pode ser considerado bidimensional.
				\item (~~) Todos os desenhos que fazemos na lousa ou no caderno podem ser considerados bidimensionais.
				\item (~~) Uma agulha é, rigorosamente falando, unidimensional.
				\item (~~) Uma agulha é tridimensional, mas pode ser considerada unidimensional para fins práticos.
			\end{enumerate}
			\item Faça, a mão livre, desenhos das figuras a seguir nos espaços correspondentes. Se você não conhecer alguma delas, não se preocupe, pois o professor também fará o desenho na lousa. Ao longo do nosso curso de Geometria, vamos estudar em detalhes todas essas figuras.
			\begin{enumerate}[a)]
				\item ponto \\\\\\\\\\\\\\
				\item reta \\\\\\\\\\\\
				\item hexágono \\\\\\\\\\\\\\
				\item semiplano \\\\\\\\\\\\\\
				\item cilindro \\\\\\\\\\\\\\
				\item segmento de reta \\\\\\\\\\\\\\
				\item triângulo \\\\\\\\\\\\\\
				\item circunferência \\\\\\\\\\\\\\
				\item paralelepípedo \\ (ou bloco retangular) \\\\\\\\\\\\\\
				\item pirâmide \\\\\\\\\\\\\\
				\item quadrilátero \\\\\\\\\\\\\\
				\item círculo \\\\\\\\\\\\
				\item cone \\\\\\\\\\\\\\
				\item semirreta \\\\\\\\\\\\\\
				\item pentágono \\\\\\\\\\\\\\
				\item plano \\\\\\\\\\\\\\
				\item esfera \\\\\\\\\\\\\\
			\end{enumerate}	
			\item Cite pelo menos três exemplos de figuras:
			\begin{enumerate}[a)]
				\item unidimensionais \\\\\\\\\\
				\item bidimensionais \\\\\\\\\\
				\item tridimensionais \\\\\\\\\\
			\end{enumerate}
			\item A agulha é tridimensional, mas pode ser considerada unidimensional para fins práticos. Que outras coisas você conhece que são tridimensionais, mas também podem ser consideradas unidimensionais?
		\end{enumerate}
		$~$ \\ $~$ \\ $~$ \\ $~$ \\ $~$ \\ $~$ \\ $~$ \\ $~$ \\ $~$ \\ $~$ \\ 
	\end{multicols}
\end{document}