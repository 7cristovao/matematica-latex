\documentclass[a4paper,14pt]{article}
\usepackage{float}
\usepackage{extsizes}
\usepackage{amsmath}
\usepackage{amssymb}
\everymath{\displaystyle}
\usepackage{geometry}
\usepackage{fancyhdr}
\usepackage{multicol}
\usepackage{graphicx}
\usepackage[brazil]{babel}
\usepackage[shortlabels]{enumitem}
\usepackage{cancel}
\usepackage{textcomp}
\usepackage{array}
\usepackage{longtable}
\usepackage{booktabs}
\usepackage{float}   % Para usar o modificador [H]

\columnsep=2cm
\hoffset=0cm
\textwidth=8cm
\setlength{\columnseprule}{.1pt}
\setlength{\columnsep}{2cm}
\renewcommand{\headrulewidth}{0pt}
\geometry{top=1in, bottom=1in, left=0.7in, right=0.5in}

\pagestyle{fancy}
\fancyhf{}
\fancyfoot[C]{\thepage}

\begin{document}
	
	\noindent\textbf{8FMA93 - Matemática} 
	
	\begin{center}Usando duas variáveis para resolver alguns problemas (II) (Versão estudante)
	\end{center}
	
	\noindent\textbf{Nome:} \underline{\hspace{10cm}}
	\noindent\textbf{Data:} \underline{\hspace{4cm}}
	
	%\section*{Questões de Matemática}
    \begin{multicols}{2}
    	\begin{enumerate}
			\item Júlio e Bianca colecionam jogos. Se Júlio desse $\frac{1}{5}$ dos seus jogos para Bianca, ela ficaria com uma quantidade de jogos igual ao que lhe restaria. Por outro lado, se Bianca desse 12 dos seus jogos para Júlio, ele ficaria com 7 vezes a quantidade de jogos que lhe restaria. Quantos jogos cada um possui? \\\\\\\\\\\\\\\\
			\item Caio diz a Marcelo: "Dê-me seis dentre os seus livros e eu terei o triplo dos que lhe sobrarão". \\
			Marcelo responde: "É melhor você me dar dois de seus livros e assim eu terei tantos quantos lhe sobrarão". \\
			Quantos livros tinham inicialmente Caio e Marcelo? \\\\\\\\\\\\\\
			\item Em um instituto de adoção de animais, tinha-se um certo número de cães e gatos. Em um evento, foram adotados 10 gatos e nenhum cão no primeiro dia, ficando o mesmo número de cães e gatos. No dia seguinte, foram adotados 20 cães e nenhum gato, ficando o número de gatos igual ao quíntuplo do número de cães. \\
			Qual era o total de animais (cães e gatos) no instituto antes desse evento? \newpage
			\item Joana tem um certo número de barras de chocolate branco e de chocolate preto. Ao comprar 3 barras de chocolate, ela ficou com a quantidade igual ao dobro do número de barras de chocolate preto. Logo após voltar do mercado, ela deu 6 barras de chocolate preto aos seus pais, ficando com a quantidade de barras de chocolate preto igual à quinta parte da quantidade inicial de barras de chocolate branco. Quantas barras de chocolate, no total, ela tinha inicialmente?
    	\end{enumerate}
    $~$ \\ $~$ \\ $~$ \\ $~$ \\ $~$ \\ $~$ \\ $~$ \\ $~$ \\ $~$ \\ $~$ \\ $~$ \\ $~$ \\ $~$ \\ $~$ \\ $~$ \\ $~$ \\ $~$ \\ $~$ \\ $~$ \\ $~$ \\ $~$ \\ $~$ \\ $~$ \\ $~$ \\ $~$ \\ $~$ \\ $~$ \\ $~$ \\ $~$ \\ $~$ \\ $~$ \\ $~$ \\ $~$ \\ $~$ \\ $~$ \\ $~$ \\ $~$ \\ $~$ \\ $~$ \\ $~$ \\ $~$ \\ $~$ \\ $~$ \\ $~$ \\ $~$ \\ $~$ \\ $~$ \\ $~$ \\ $~$ \\ $~$ \\ $~$ \\ $~$ \\ $~$ \\ $~$ \\ $~$ \\ $~$ \\ $~$ \\ $~$ \\ $~$ \\ $~$ \\ $~$ \\ $~$ \\ $~$ \\ $~$ \\ $~$ \\ 
    \end{multicols}
\end{document}