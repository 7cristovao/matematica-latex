\documentclass[a4paper,14pt]{article}
\usepackage{extsizes}
\usepackage{amsmath}
\usepackage{amssymb}
\everymath{\displaystyle}
\usepackage{geometry}
\usepackage{fancyhdr}
\usepackage{multicol}
\usepackage{graphicx}
\usepackage[brazil]{babel}
\usepackage[shortlabels]{enumitem}
\usepackage{cancel}
\columnsep=2cm
\hoffset=0cm
\textwidth=8cm
\setlength{\columnseprule}{.1pt}
\setlength{\columnsep}{2cm}
\renewcommand{\headrulewidth}{0pt}
\geometry{top=1in, bottom=1in, left=0.7in, right=0.5in}

\pagestyle{fancy}
\fancyhf{}
\fancyfoot[C]{\thepage}

\begin{document}
	
	\noindent\textbf{8FMA18~-~Matemática} 
	
	\begin{center}Sistemas com equações produto (Versão estudante)
	\end{center}
	
	
	\noindent\textbf{Nome:} \underline{\hspace{10cm}}
	\noindent\textbf{Data:} \underline{\hspace{4cm}}
	
	%\section*{Questões de Matemática}
	\begin{multicols}{2}
		\begin{enumerate}
			\item Resolver os seguintes sistemas no universo $U = \mathbb{R}^2$, usando o método da substituição.
			\begin{enumerate}[a)]
				\item $\begin{cases}
					(x - 2)y = 0 \\
					2x - 5y = 2
				\end{cases}$ \\\\\\\\\\\\\\\\\\\\
			    \item $\begin{cases}
			    	(x - 3)(y + 1) = 2 \\
			    	xy = 0
			    \end{cases}$ \\\\\\\\\\\\\\\\\\\\
		        \item $\begin{cases}
		        	(u - 2)(v + 1) = 0 \\
		        	u - v = 4
		        \end{cases}$ \\\\\\\\\\\\\\\\\\\\
	            \item $\begin{cases}
	            	xy = 0 \\
	            	x^2 + 4xy - y = 4
	            \end{cases}$ \\\\\\\\\\\\\\\\\\\\
                \item $\begin{cases}
                	ab = 1 \\
                	(a - 3)(b + 4) = 0
                \end{cases}$ \\\\\\\\\\\\\\\\\\\\\\\\\\\\\\\\\\
			\end{enumerate}
		    \item Resolva os seguintes sistemas ($U = \mathbb{R}^2$):
		    \begin{enumerate}[a)]
		    	\item $\begin{cases}
		    		(x - 3)y = 0 \\
		    		2x - 5y = 3
		    	\end{cases}$ \\\\\\\\\\\\\\\\\\\\\\\\
	    		\item $\begin{cases}
	    			(x - 4)(y + 3) = 1 \\
	    			xy = 0
	    		\end{cases}$ \\\\\\\\\\\\\\\\\\\\\\\\
    		    \item $\begin{cases}
    		    	(u - 3)(v + 2) = 0 \\
    		    	u - v = 3
    		    \end{cases}$ \\\\\\\\\\\\\\\\\\\\\\\\
    	        \item $\begin{cases}
    	        	xy = 0 \\
    	        	x^2 + 6xy - y = 3
    	        \end{cases}$ \\\\\\\\\\\\\\\\\\\\\\\\
                \item $\begin{cases}
                	(x - 3)(y - 2) = 0 \\
                	x^2 + y^2 - 4xy + x - y = 1
                \end{cases}$ \\\\\\\\\\\\\\\\\\\\\\\\
            	\item $\begin{cases}
            		ab = 1 \\
            		(a - 4)(b + 3) = 0
            	\end{cases}$ \\\\\\\\\\\\\\\\
		    \end{enumerate}
	    	\item O conjunto verdade do sistema $\begin{cases}
	    		(x - 2)(y + 3) = 0 \\
	    		x - y = k
	    	\end{cases}$ \\tem somente um elemento. \\Determine $k$.
		\end{enumerate}
	$~$ \\ $~$ \\ $~$ \\ $~$ \\ $~$ \\ $~$ \\ $~$ \\ $~$ \\ $~$ \\ $~$ \\ $~$ \\ $~$ \\ $~$ \\ $~$ \\ $~$ \\ $~$ \\ $~$ \\ $~$ \\ $~$ \\ $~$ \\ $~$ \\ $~$ \\ $~$ \\ $~$ \\ $~$ \\ $~$ \\ $~$ \\ $~$ \\ $~$ \\ $~$ \\ $~$ \\ $~$ \\ $~$ \\ $~$ \\ $~$ \\ $~$ \\ $~$ \\ $~$ \\ $~$ \\ $~$ \\ $~$ \\ $~$ \\ $~$ \\ $~$ \\ $~$ \\ $~$ \\ $~$ \\ $~$ \\ $~$ \\ $~$ \\ $~$ \\ $~$ \\ $~$ \\ $~$ \\ $~$ \\ $~$ \\ $~$ \\ $~$ \\ $~$ \\ $~$ \\ $~$ \\ $~$ \\ $~$ \\ $~$ \\ $~$ \\ $~$ \\ $~$ \\ $~$ \\ $~$ \\ $~$ \\ $~$ \\ $~$ \\ $~$ \\ $~$ \\ $~$ \\ $~$ \\ $~$ \\ 
    \end{multicols}

\end{document}