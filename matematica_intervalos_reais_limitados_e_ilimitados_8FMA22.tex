\documentclass[a4paper,14pt]{article}
\usepackage{float}
\usepackage{extsizes}
\usepackage{amsmath}
\usepackage{amssymb}
\everymath{\displaystyle}
\usepackage{geometry}
\usepackage{fancyhdr}
\usepackage{multicol}
\usepackage{graphicx}
\usepackage[brazil]{babel}
\usepackage[shortlabels]{enumitem}
\usepackage{cancel}
\columnsep=2cm
\hoffset=0cm
\textwidth=8cm
\setlength{\columnseprule}{.1pt}
\setlength{\columnsep}{2cm}
\renewcommand{\headrulewidth}{0pt}
\geometry{top=1in, bottom=1in, left=0.7in, right=0.5in}

\pagestyle{fancy}
\fancyhf{}
\fancyfoot[C]{\thepage}

\begin{document}
	
	\noindent\textbf{8FMA22~Matemática} 
	
	\begin{center}Intervalos reais limitados e ilimitados (Versão estudante)
	\end{center}
	
	\noindent\textbf{Nome:} \underline{\hspace{10cm}}
	\noindent\textbf{Data:} \underline{\hspace{4cm}}
	
	%\section*{Questões de Matemática}
 
	\begin{multicols}{2}
		\begin{enumerate}
			\item Pela notação usada anteriormente:
			\begin{center}
				$[1;2] = \{x \in \mathbb{\textbf{R}}:1 \leq x \leq 2 \}$ \\
				$[-1;0[ = \{x \in \mathbb{\textbf{R}}:-1 \leq x < 0\}$
			\end{center}
			Escreva a que conjunto corresponde cada um dos intervalos, como nos exemplos dados.
			\begin{enumerate}[a)]
				\item $[2; 4]$ \\\\\\\\
				\item $]-5; 2]$ \\\\\\\\
				\item $[0; 6[$ \\\\\\\\
				\item $]-5; 0[$ \\\\\\\\
				\item $[5; 9]$ \\\\\\\\
				\item $[-7; -2]$ \\\\\\\\
			\end{enumerate}	
		    Represente na reta real os intervalos dados no exercício \textbf{1}.\\\\\\\\\\\\\\\\\\\\\\\\\\\\\\\\\\\\\\\\\\\\\\\\\\\\\\
		    \item Pela notação usada, $[2;+\infty[=\{x \in \mathbb{\textbf{R}}:x \geq 2\}$, escreva a que conjunto corresponde cada um dos intervalos, como no exemplo dado.
		    \begin{enumerate}[a)]
		    	\item $[3; +\infty[$ \\\\\\\\\\
		    	\item $]0; +\infty[$ \\\\\\\\\\
		    	\item $]-\infty; +\infty[$ \\\\\\\\\\
		    	\item $]-\infty; -2]$ \\\\\\\\
		    	\item $]-\infty; 7[$ \\\\\\\\
		    	\item $[-4; +\infty[$ \\\\\\\\\\
		    \end{enumerate}	
	    	\item Represente na reta real os intervalos do exercício \textbf{3}.
	    			$~$ \\ $~$ \\ $~$ \\ $~$ \\ $~$ \\ $~$ \\ $~$ \\ $~$ \\ $~$ \\ $~$ \\ $~$ \\ $~$ \\ $~$ \\ $~$ \\ $~$ \\ $~$ \\ $~$ \\ $~$ \\ $~$ \\ $~$ \\ $~$ \\ $~$ \\ $~$ \\ $~$ \\ $~$ \\ $~$ \\ $~$ \\ $~$ \\ $~$ \\ $~$ \\ $~$ \\ $~$ \\ $~$ \\ $~$ \\ $~$ \\ $~$ \\ $~$ \\ $~$ \\ $~$ \\
	    	\item Escreva a que conjunto corresponde cada um dos intervalos e, em seguida, represente-os na reta real.
	    	\begin{enumerate}[a)]
	    		\item $[-3;4]$ \\\\\\\\\\\\
	    		\item $]0;10]$ \\\\\\\\\\\\
	    		\item $[-7;-2[$ \\\\\\\\\\\\
	    		\item $[21;22[$ \\\\\\\\\\\\
	    		\item $-\infty;4]$ \\\\\\\\\\\\
	    		\item $[-1; +\infty[$ \\\\\\\\\\\\
	    	\end{enumerate}
    	    \item Escreva o intervalo correspondente a cada um dos conjuntos e, em seguida, represente-os na reta real.
    	    \begin{enumerate}[a)]
    	        \item $\{x \in \mathbb{\textbf{R}}:x \leq 3\}$ \\\\\\\\\\\\
    	        \item $\{x \in \mathbb{\textbf{R}}:7 \leq x < 9\}$ \\\\\\\\\\\\
    	        \item $\{x \in \mathbb{\textbf{R}}:-6 \leq x \leq -4\}$ \\\\\\\\\\\\
    	        \item $\{x \in \mathbb{\textbf{R}}|x < 0\}$ \\\\\\\\\\\\
    	        \item $\{x \in \mathbb{\textbf{R}}$ tal que $2 > x > -2\}$ \\\\\\\\\\\\
    	        \item $\{x \in \mathbb{\textbf{R}}$ t.q. $-7 < x \leq 13\}$ \\\\\\\\\\\\
    	    \end{enumerate}
            \item Os valores de $x$ que pertencem ao intervalo 2 a 4 fechado à esquerda e aberto à direita são:
             \begin{enumerate}[a)]
             	\item $2 < x \leq 4$ 
             	\item $2 \leq x \leq 4$
             	\item $2 < x < 4$
             	\item $2 \leq x < 4$
             	\item $x < 2$ ou $x > 4$
             \end{enumerate}
             \item Assinale \textbf{V}(verdadeiro) ou \textbf{F}(falso) e justifique:
             \begin{enumerate}[a)]
             	\item (~~~)$\{5\} \in [4; 8]$
             	\item (~~~)$2 \subset ]-2; 6]$
             	\item (~~~)$\{\pi\} \subset ]-5; 9[$
             	\item (~~~)$-4 \in [-3; 3]$
             	\item (~~~)$3^0 \in [-4; 4[$
            \end{enumerate}	
		\end{enumerate}
	$~$ \\ $~$ \\ $~$ \\ $~$ \\ $~$ \\ $~$ \\ $~$ \\ $~$ \\ $~$ \\ $~$ \\ $~$ \\ $~$ \\ $~$ \\ $~$ \\ $~$ \\ $~$ \\ $~$ \\ $~$ \\ $~$ \\ $~$ \\ $~$ \\ $~$ \\ $~$ \\ $~$ \\ $~$ \\ $~$ \\ $~$ \\ $~$ \\ $~$ \\ $~$ \\ $~$ \\ $~$ \\ $~$ \\ $~$ \\ $~$ \\ $~$ \\ $~$ \\ $~$ \\ $~$ \\ $~$ \\ $~$ \\ $~$ \\ $~$ \\ $~$ \\ $~$ \\ $~$ \\ $~$ \\ $~$ \\
    \end{multicols}
\end{document}