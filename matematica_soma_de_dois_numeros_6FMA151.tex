\documentclass[a4paper,14pt]{article}

\usepackage{comment} % Para comentar várias linhas ao mesmo tempo

%matemática
\usepackage{amsmath}
\usepackage{amssymb}

%diagramação
\usepackage{extsizes}
\everymath{\displaystyle}
\usepackage{geometry}
\usepackage{fancyhdr}
\usepackage{multicol}
\usepackage{graphicx}
\usepackage[brazil]{babel}
\usepackage[shortlabels]{enumitem}
\usepackage{cancel}
\usepackage{textcomp}
\usepackage{tcolorbox}

%tabelas
\usepackage{array} % Para melhor formatação de tabelas
\usepackage{longtable}
\usepackage{booktabs}  % Para linhas horizontais mais bonitas
\usepackage{float}   % Para usar o modificador [H]
\usepackage{caption} % Para usar legendas em tabelas
\usepackage{wrapfig} % Para usar tabelas e figuras flutuantes
\usepackage{xcolor} % Para cores do fundo de tabelas
\usepackage{colortbl} % Para cores do fundo de tabelas
\usepackage{upgreek} % Para inserir caracteres gregos

%tikzpicture
\begin{comment}
	\usepackage{tikz}
	\usepackage{scalerel}
	\usepackage{pict2e}
	\usepackage{tkz-euclide}
	\usetikzlibrary{calc}
	\usetikzlibrary{patterns,arrows.meta}
	\usetikzlibrary{shadows}
	\usetikzlibrary{external}
\end{comment}


%pgfplots
\usepackage{pgfplots}
\pgfplotsset{compat=newest}
\usepgfplotslibrary{statistics}
\usepgfplotslibrary{fillbetween}

%colours
\usepackage{xcolor}



\columnsep=2cm
\hoffset=0cm
\textwidth=8cm
\setlength{\columnseprule}{.1pt}
\setlength{\columnsep}{2cm}
\renewcommand{\headrulewidth}{0pt}
\geometry{top=1in, bottom=1in, left=0.7in, right=0.5in}

\pagestyle{fancy}
\fancyhf{}
\fancyfoot[C]{\thepage}

\begin{document}
	
	\noindent\textbf{6FMA151 - Matemática} 
	
	\begin{center}Soma de dois números (Versão estudante)
	\end{center}
	
	\noindent\textbf{Nome:} \underline{\hspace{10cm}}
	\noindent\textbf{Data:} \underline{\hspace{4cm}}
	
	%\section*{Questões de Matemática}
	
	\begin{multicols}{2}
	    \noindent \textbf{Exemplo: } \\
	    A soma de dois números é 21. Cinco vezes um deles é seis mais o quádruplo do outro. Quais são os números? \\
	    \noindent \textbf{resolução: } \\
	    \noindent \textbf{Uma maneira: } \\
	    Sejam $x$ um número e 21 - $x$ o outro número. \\
	    Podemos ter: \\
	    $5x = 6 + 4(21 - x)$ \\
	    Resolvendo essa equação, encontramos $x = 10$. \\
	    Logo, o outro número é 21 - 10 = 11. \\
	    Os números são 10 e 11. \\
	    \noindent \textbf{Outra maneira: } \\
	    Seja $x$ um número e 21 - $x$ o outro número. \\
	    Podemos ter: \\
	    $5(21 - x) = 6 + 4x$ \\
	    Resolvendo essa equação, encontramos $x = 11$. \\
	    Logo, o outro número é 21 - 11 = 10. \\
	    Os números são 10 e 11. \\
		\noindent\textsubscript{--------------------------------------------------------------------------}
		\begin{enumerate} 
			\item A soma de dois números é 36. Sabendo-se que um é o dobro do outro, quais são os números? \\\\\\\\\\\\
			\item A soma de dois números é 2. Um é o triplo do outro. Quais são os números? \\\\\\\\\\\\\\\\\\\\\\\\\\\\\\\\
			\item Somando-se dois números, encontramos -63. Sabe-se que um dos números é 8 vezes o outro. Quais são os números? \newpage
			\item Separar 89 em duas partes, de tal modo que uma parte exceda a outra em 5. \\
			\textbf{Observação: } "Separar 89 em duas partes" significa que queremos dois números tais que a soma deles seja 89. \\\\\\\\\\\\\\\\\\\\\\\\\\\\\\\\\\\\\\
			\item A soma de dois números é 52 e a diferença é 6. Quais são os números? \\\\\\\\\\\\\\\\\\\\\\\\
			\textbf{Desafio olímpico} \\\\
			(OBMEP) Para ir com Maria ao cinema, João pode escolher dois caminhos. No primeiro, ele passa pela casa de Maria e os dois vão juntos até o cinema; nesse caso, ele anda sozinho $\frac{2}{3}$ do caminho. No segundo, ele vai sozinho e encontra Maria na frente do cinema; nesse caso, ele anda 1 km a menos que no primeiro caminho, mais o dobro do que Maria terá que caminhar. Qual é a distância entre a casa de Maria e o cinema? \\
			\begin{enumerate}[a)]
				\item 1 km
				\item 2 km
				\item 3 km
				\item 4 km
				\item 6 km \newpage
			\end{enumerate}
			%20 a 26
			\item A soma de dois números é 51. Sabendo-se que um é o dobro do outro, quais são os números? \\\\\\\\\\\\\\\\\\\\
			\item A soma de dois números é 16. Um é o triplo do outro. Quais são os números? \\\\\\\\\\\\\\\\\\\\
			\item Somando-se dois números encontramos 12. Sabe-se que um dos números é 8 vezes o outro. Quais são os números? \\\\\\\\\\\\\\\\\\\\
			\item A soma de dois números é 26. Três vezes um deles é oito, mais quatro vezes o outro. Quais são os números? \\\\\\\\\\\\\\\\\\\\
			\item A soma de dois números é 77. Um deles é menor uma unidade que três vezes o outro. Quais são os números? \\\\\\\\\\\\\\\\\\\\
			\item A soma de dois números é 27. Um deles é 5 vezes o outro mais 3. Quais são os números? \\\\\\\\\\\\\\\\
			\item A soma de dois números é 27. Um deles é seis vezes o outro, menos nove. Quais são os números? \\\\\\\\\\\\\\\\\\\\
		\end{enumerate}
		 $~$ \\ $~$ \\ $~$ \\ $~$ \\ $~$ \\ $~$ \\ $~$ \\ $~$ \\ $~$ \\ $~$ \\ $~$ \\ $~$ \\ $~$ \\ $~$ \\ $~$ \\ $~$ \\ $~$ \\ $~$ \\ $~$ \\ $~$ \\ $~$ \\ $~$ \\ $~$ \\ $~$ \\ $~$ \\ $~$ \\ $~$ \\ $~$ \\ $~$ \\ $~$ \\ $~$ \\ $~$ \\ $~$ \\ $~$ \\ $~$ \\ $~$ \\ $~$ \\ $~$ \\ $~$ \\ $~$ \\ $~$ \\ $~$ \\ $~$ \\ $~$ \\ $~$ \\ $~$ \\ $~$ \\ $~$ \\ $~$ \\ $~$ \\ $~$ \\ $~$ \\ $~$ \\ $~$ \\ $~$ \\ $~$ \\ $~$ \\ $~$ \\ $~$ \\ $~$ \\ $~$ \\ $~$ \\ $~$ \\ $~$ \\ $~$ \\ $~$ \\ $~$ \\ $~$ \\ $~$
	\end{multicols}
\end{document}