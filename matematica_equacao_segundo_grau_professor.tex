\documentclass[a4paper,14pt]{article}
\usepackage{extsizes}
\usepackage{amsmath}
\usepackage{cancel}
\everymath{\displaystyle}
\usepackage{geometry}
\usepackage{fancyhdr}
\usepackage{multicol}
\usepackage{graphicx}
\usepackage[brazil]{babel}
\usepackage{enumitem}
\columnsep=2cm
\hoffset=0cm
\textwidth=8cm
\setlength{\columnseprule}{.1pt}
\setlength{\columnsep}{2cm}
\renewcommand{\headrulewidth}{0pt}
\geometry{top=1in, bottom=1in, left=1in, right=1in}

\pagestyle{fancy}
\fancyhf{}
\fancyfoot[C]{\thepage}

\begin{document}
	
	\noindent\textbf{EF09MA211-A~-~Matemática} 
	
	\begin{center}
		\textbf{Reconhecendo uma equação quadrática (Versão professor)}
	\end{center}
	
	\bigskip
	
	\noindent\textbf{Nome:} \underline{\hspace{10cm}}
	\noindent\textbf{Data:} \underline{\hspace{4cm}}
	
	\bigskip
%	\section*{Questões de Matemática}
	
	\begin{multicols}{2}
	\begin{enumerate}
		\item Indique quais são os coeficientes: \\
		\\
		$3x^2 + 4x +1 = 0$ \\ \\
		a = 3 \\ \\
		b = 4 \\ \\
		c = 1 \\ \\
		$5 - 7x^2 = 0$ \\ \\
		a = - 7 \\ \\
		b = 0 \\ \\
		c = 5 \\ \\
		$6x^2 + 18$ \\ \\
		a = 6 \\ \\
		b = 0 \\ \\
		c = 18 \\ \\
		$x^2 + 5x + 2,5 = 0$ \\ \\
		a = 1 \\ \\
		b = 5 \\ \\
		c = 2,5 \\ \\
		$2x^2 - 32 = 0$ \\ \\
		a = 2 \\ \\
		b = 0 \\ \\
		c = - 32 \\ \\
		$-x^2 - 4x = 0$ \\ \\
		a = - 1 \\ \\
		b = -4 \\ \\
		c = 0 \\ \\
		$7x = 0$ \\ \\
		a = 7 \\ \\
		b = 0 \\ \\
		c = 0 \\ \\
		\\ \\
		$6x - 3 + x^2 = 1$ \\ \\
		$6x - 4 + x^2 = 0$ \\ \\
		$1x^2 + 6x - 4 = 0$ \\ \\
		a = 1 \\ \\
		b = 6 \\ \\
		c = - 4 \\ \\
		\\
		$1 + 2x = x^2$ \\ \\
		$-x^2 + 2x + 1 = 0$ \\ \\
		a = -1 \\ \\
		b = 2 \\ \\
		c = 1 \\ \\
		\\
		\item Resolva estas equações incompletas do segundo grau : \\
		\\
		$x^2 - 25 = 0$ \\ 
		a = 1; b = 0 ; c = 25 \\
		$x^2 = 25$ \\
		$x = \sqrt{25}$ \\
		$x = \pm{5}$ \\ \\
		
        $4x^2 - 36 = 0$  \\ 
        a = 4; b = 0; c = -36 \\
        $4x^2 = 36$ \\
        $x^2 = \frac{36}{4}$ \\
        $x^2 = 9$ \\
        $x = \sqrt{9} $ \\
        $x = \pm{3} $ \\ \\
        
        $x^2 - 64x = 0$  \\ 
        a = 1; b = -64; c = 0 \\
        $x \cdot (x - 64) = 0$ \\
        $x = 0$ ou $x-64 = 0$ \\
        $x - 64 = 0$ \\
        $x = 64$ \\ \\
        
        $2x^2 - 8x = 0$   \\ 
        a = 2; b = - 8; c = 0 \\
        $2x \cdot (x - 4) = 0$ \\ 
        $2x = 0$ ou $x - 4 = 0$ \\
        $\frac{\cancel{2} \cdot x}{\cancel{2}} = \frac{0}{2}$ \\
        $x = 0$ \\
        $x - 4 = 0$ \\
        $x = 4$ \\
        
        \item Também resolva esta equação incompleta do segundo grau.
        
        $4x \cdot 6x = 6144$ \\
        $24x^2 = 6144$ \\
        $\frac{\cancel{24}x^2}{\cancel{24}} = \frac{6144}{24}$ \\
        $x^2 = \frac{6144}{24}$ \\
        $x^2 = 256$ \\
        $x = \sqrt{256}$ \\
        $x = \pm{16}$
        
        \newpage
        
        \item Encontre a solução desta equação do segundo grau completa.
        \\
        $4x^2 - 3x -1 = 0$ \\
        a = 4; b = -3; c = -1 \\
        $\Delta = b^2 - 4 \cdot a \cdot c$ \\
        $\Delta = (-3)^2 - 4 \cdot 4 \cdot (-1)$ \\
        $\Delta = 9 + 16$ \\
        $\Delta = 25$ \\ \\
        $x = \frac{-b \pm{\sqrt{\Delta}}}{2 \cdot a}$ \\
        $x = \frac{-(-3) \pm{\sqrt{25}}}{2 \cdot 4}$ \\
        $x = \frac{3 \pm{5}}{8}$ \\
        $x_1 = \frac{3 + 5}{8} = \frac{8}{8} = 1$ \\
        $x_2 = \frac{3 - 5}{8} = \frac{-2}{8} = \frac{-1}{4}$\\ \\
        $S = \left \{\frac{-1}{4}; 1 \right \} $
        \vspace{8cm}
        
    \end{enumerate}
	\end{multicols}

\end{document}