\documentclass[a4paper,14pt]{article}

\usepackage{comment} % Para comentar várias linhas ao mesmo tempo

%matemática
\usepackage{amsmath}
\usepackage{amssymb}

%diagramação
\usepackage{extsizes}
\everymath{\displaystyle}
\usepackage{geometry}
\usepackage{fancyhdr}
\usepackage{multicol}
\usepackage{graphicx}
\usepackage[brazil]{babel}
\usepackage[shortlabels]{enumitem}
\usepackage{cancel}
\usepackage{textcomp}
\usepackage{tcolorbox}

%tabelas
\usepackage{array} % Para melhor formatação de tabelas
\usepackage{longtable}
\usepackage{booktabs}  % Para linhas horizontais mais bonitas
\usepackage{float}   % Para usar o modificador [H]
\usepackage{caption} % Para usar legendas em tabelas
\usepackage{wrapfig} % Para usar tabelas e figuras flutuantes
\usepackage{xcolor} % Para cores do fundo de tabelas
\usepackage{colortbl} % Para cores do fundo de tabelas

%tikzpicture
\begin{comment}
	\usepackage{tikz}
	\usepackage{scalerel}
	\usepackage{pict2e}
	\usepackage{tkz-euclide}
	\usetikzlibrary{calc}
	\usetikzlibrary{patterns,arrows.meta}
	\usetikzlibrary{shadows}
	\usetikzlibrary{external}
\end{comment}


%pgfplots
\usepackage{pgfplots}
\pgfplotsset{compat=newest}
\usepgfplotslibrary{statistics}
\usepgfplotslibrary{fillbetween}

%colours
\usepackage{xcolor}



\columnsep=2cm
\hoffset=0cm
\textwidth=8cm
\setlength{\columnseprule}{.1pt}
\setlength{\columnsep}{2cm}
\renewcommand{\headrulewidth}{0pt}
\geometry{top=1in, bottom=1in, left=0.7in, right=0.5in}

\pagestyle{fancy}
\fancyhf{}
\fancyfoot[C]{\thepage}

\begin{document}
	
	\noindent\textbf{6FMA83 - Matemática} 
	
	\begin{center}Adição com números racionais (Versão estudante)
	\end{center}
	
	\noindent\textbf{Nome:} \underline{\hspace{10cm}}
	\noindent\textbf{Data:} \underline{\hspace{4cm}}
	
	%\section*{Questões de Matemática}
	
	\begin{multicols}{2}
		\noindent Os números racionais são todos aqueles que podem ser escritos na forma $\frac{a}{b}$, em que $a$ e $b$ são inteiros e $b$ é diferente de zero. \\
		Podemos representar um número racional por uma fração ou por um numeral decimal. Por exemplo, $\frac{3}{2} = 1,5$. \\
		Para somar duas frações com o mesmo denominador, conservamos este e somamos os numeradores. Por exemplo: $\frac{1}{5} + \frac{3}{5} = \frac{1 + 3}{5} = \frac{4}{5}$ \\
		\noindent\textsubscript{-----------------------------------------------------------------------}
		\begin{enumerate} 
			\item O número 1 é racional? Explique. \\\\\\\\\\\\\\\\\\
			\item O zero é racional? Explique. \\\\\\\\\\\\\\\\
			\item Calcule as somas abaixo:
			\noindent\begin{enumerate}[a)]
				\item $\frac{1}{10} + \frac{1}{10} + \frac{1}{10} + \frac{1}{10} + \\\\ \frac{1}{10} + \frac{1}{10} + \frac{1}{10}$ \\\\\\\\\\\\\\\\\\
				\item $0,1 + 0,1 + 0,1$ \\\\\\\\\\\\\\\\\\\\
			\end{enumerate}
			\item Escreva, com suas palavras, o que devemos fazer para somar frações que têm o mesmo denominador. \\\\\\\\\\\\\\
			\item Calcule.
			\begin{enumerate}[a)]
				\item $0,2 + 0,6$ \\\\\\\\\\\\\\
				\item $\frac{1}{10} + \frac{4}{10} + \frac{1}{10} + \frac{2}{10} + \frac{6}{10}$ \\\\\\\\\\\\\\
			\end{enumerate}
			\item Determine $x$ na igualdade abaixo: \\ $x = 0,9 + 0,2 + 1,7$ \\\\\\\\\\\\\\
			\item Quanto vale $\frac{7}{10} + 0,5$? (Dê a resposta na forma de fração.) \\\\\\\\\\\\\\\\
			\item Em um mês Carlos gastou $\frac{9}{21}$ de sua mesada para de divertir indo ao cinema, teatro ou à praia, $\frac{7}{21}$ para pagar seu curso de inglês e $\frac{3}{21}$ com o material do curso. O resto ele guardou. \\
			\begin{enumerate}[a)]
				\item Que fração de sua mesada ele gastou com estudo? \\\\\\\\\\\\\\\\
				\item Que fração de sua mesada ele guardou? \\\\\\\\\\\\\\\\
			\end{enumerate}
			%33 a 38
			\item Efetue as operações indicadas a seguir:
			\begin{enumerate}[a)]
				\item $\frac{3}{9} + \frac{5}{9}$ \\\\\\\\\\\\\\
				\item $\frac{12}{31} + \frac{17}{31} + \frac{9}{31}$ \\\\\\\\\\\\\\\\
				\item $\frac{87}{143} + \frac{124}{143} + \frac{75}{143}$ \\\\\\\\\\\\\\\\
			\end{enumerate}
			\item Calcule, dando o resultado na forma de fração:
			\begin{enumerate}[a)]
				\item 0,4 + 0,1 + 0,8 \\\\\\\\\\\\\\\\
				\item $0,2 + 1,6 + \frac{7}{10}$ \\\\\\\\\\\\\\\\
				\item $\frac{3}{10} + 0,6 + \frac{1}{10} + 0,9$ \\\\\\\\\\\\\\\\
			\end{enumerate}
			\item Eduardo gasta $\frac{5}{17}$ de sua mesada indo ao cinema, $\frac{8}{17}$ para pagar as aulas de natação e $\frac{2}{17}$ comprando livros e revistas. O que sobra ele guarda.
			\begin{enumerate}[a)]
				\item Que fração de sua mesada Eduardo gasta com atividades culturais (cinema, livros e revistas)? \\\\\\\\\\\\\\\\\\\\
				\item Que fração de sua mesada ele guarda? \\\\\\\\\\\\\\
			\end{enumerate}
			\item Explique por que, numa adição de frações de mesmo denominador, somamos os numeradores e conservamos o denominador. \\\\\\\\\\\\\\\\
			\item Determine $y$, sabendo que $y + \frac{9}{11} = \frac{34}{11}$. \\\\\\\\\\\\\\\\
			\item Alexandre afirmou que todos os números inteiros são racionais. Roberto disse que existem racionais que não são inteiros. \\
			Quem está certo? Por quê?
		\end{enumerate}
		$~$ \\ $~$ \\ $~$ \\ $~$ \\ $~$ \\ $~$ \\ $~$ \\ $~$ \\ $~$ \\ $~$ \\ $~$ \\ $~$ \\ $~$ \\ $~$ \\ $~$ \\ $~$ \\ $~$ \\ $~$ \\ $~$ \\ $~$ \\ $~$ \\ $~$ \\ $~$ \\ $~$ \\ $~$ \\ $~$ \\ $~$ \\ $~$ \\ $~$ \\ $~$ \\ $~$ \\ $~$ \\ $~$ \\ $~$ \\ $~$ \\ $~$ \\ $~$ \\ $~$ \\ $~$ \\ $~$ \\ $~$ \\ $~$ \\ $~$ \\ $~$ \\ $~$ \\ $~$ \\ $~$ \\ $~$ \\ $~$ \\ $~$ \\ $~$ \\ $~$ \\ $~$ \\ $~$
	\end{multicols}
\end{document}