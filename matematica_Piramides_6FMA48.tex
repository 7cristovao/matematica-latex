\documentclass[a4paper,14pt]{article}
\usepackage{float}
\usepackage{extsizes}
\usepackage{amsmath}
\usepackage{amssymb}
\everymath{\displaystyle}
\usepackage{geometry}
\usepackage{fancyhdr}
\usepackage{multicol}
\usepackage{graphicx}
\usepackage[brazil]{babel}
\usepackage[shortlabels]{enumitem}
\usepackage{cancel}
\usepackage{textcomp}
\columnsep=2cm
\hoffset=0cm
\textwidth=8cm
\setlength{\columnseprule}{.1pt}
\setlength{\columnsep}{2cm}
\renewcommand{\headrulewidth}{0pt}
\geometry{top=1in, bottom=1in, left=0.7in, right=0.5in}

\pagestyle{fancy}
\fancyhf{}
\fancyfoot[C]{\thepage}

\begin{document}
	
	\noindent\textbf{6FMA48 - Matemática} 
	
	\begin{center}Pirâmides (Versão estudante)
	\end{center}
	
	\noindent\textbf{Nome:} \underline{\hspace{10cm}}
	\noindent\textbf{Data:} \underline{\hspace{4cm}}
	
	%\section*{Questões de Matemática}
	
	
    \begin{multicols}{2}
    	\noindent A face de apoio da pirâmide é chamada \textbf{base}, e as demais são chamadas \textbf{faces laterais}. O vértice de cima, que não pertence à base, é o \textbf{vértice da pirâmide}, e os segmentos ligando o vértice da pirâmide e os vértices da base são as \textbf{arestas laterais}. \\
    	Quando as arestas laterais têm a mesma medida $a$ e a base é um polígono regular (todas as arestas possuem a mesma medida, não necessariamente igual a $a$, e todos os ângulos internos têm a mesma medida), dizemos que a pirâmide é \textbf{regular}, caso contrário, a pirâmide é \textbf{oblíqua}. \\
    	A altura de uma pirâmide é a distância do seu vértice à base. \\
    	Uma pirâmide de base triangular também é chamada \textbf{tetraedro}.
    	\noindent\textsubscript{~---------------------------------------------------------------------------}
		\begin{enumerate}
			\item \begin{enumerate}[a)]
				\item Quantas arestas tem uma pirâmide de base hexagonal? \\\\\\\\\\
				\item Quantos vértices tem essa pirâmide? \\\\\\			\end{enumerate}
		    \item Quantas faces, quantas arestas e quantos vértices tem uma pirâmide de base triangular? \\\\\\\\\\\\\\
		    \item Cite dois ou mais objetos com forma de pirâmide ou que tenham partes com a forma de pirâmide. \\\\\\\\\\\\\\
		    \item Uma pirâmide tem como base um pentágono regular cujos lados medem 6 cm. As arestas laterais dessa pirâmide têm medida igual a 9 cm. Desenhe essa pirâmide e responda: ela pode ser uma pirâmide oblíqua? \\\\\\\\\\\\
		    \item Uma pirâmide tem 14 faces. Quantas arestas e quantos vértices tem essa pirâmide? \\\\\\\\\\\\\\
		    \item Mostre que em um cubo qualquer vale a relação $V - A + F = 2$, em que $V$ é o número de vértices, $A$ é o número de arestas e $F$ é o número de faces do cubo (essa relação é conhecida como Fórmula de Euler). \\\\\\\\\\\\\\
		    \item Considere uma pirâmide cuja base é um undecágono (polígono de onze lados).
		    \begin{enumerate}[a)]
		    	\item Quantas faces laterais possui a pirâmide? \\\\\\\\
		    	\item Quantos vértices ela possui? \\\\\\\\\\
		    \end{enumerate}
	        \item Uma pirâmide possui seis vértices.
	        \begin{enumerate}[a)]
	        	\item Qual é a forma da base dessa pirâmide? \\\\\\\\\\
	        	\item Quantas arestas a pirâmide possui? \\\\\\\\\\
	        \end{enumerate}
            \item Quantas arestas tem uma pirâmide de base triangular? E uma de base hexagonal? \\\\\\\\\\\\\\\\\\\\
            \item As faces laterais de uma pirâmide de base pentagonal são triângulos equiláteros de lado 6 cm. Quanto vale a soma dos comprimentos das arestas dessa pirâmide? \\\\\\\\
            \item Os três pés de uma mesa de centro medem 50 cm e formam com o chão uma pirâmide cuja base é um triângulo equilátero de lados medindo 50 cm. Desenhe essa pirâmide e responda:
            \begin{enumerate}[a)]
            	\item Ela pode ser uma pirâmide oblíqua? \\\\\\\\\\
            	\item Quantos vértices tem essa pirâmide? \\\\\\\\\\
            \end{enumerate}
        \end{enumerate}
    $~$ \\ $~$ \\ $~$ \\ $~$ \\ $~$ \\ $~$ \\ $~$ \\ $~$ \\ $~$ \\ $~$ \\ $~$ \\ $~$ \\ $~$ \\ $~$ \\ $~$ \\ $~$ \\ $~$ \\ $~$ \\ $~$ \\ $~$ \\ $~$ \\ $~$ \\ $~$ \\ $~$ \\ $~$ \\ $~$ \\ $~$ \\ $~$ \\ $~$ \\ $~$ \\ $~$ \\ $~$ \\ $~$ \\ $~$ \\ $~$ \\ $~$ \\ $~$ \\ $~$ \\ $~$ \\ $~$ \\ $~$ \\ $~$ \\ $~$ \\ $~$ \\ $~$ \\ $~$ \\ $~$ \\ $~$ \\ $~$ \\ $~$ \\ $~$ \\ $~$ \\ $~$ \\ $~$ \\ $~$ \\ $~$ \\ $~$ \\ $~$ \\ $~$ \\ $~$ \\ $~$ \\ $~$
    \end{multicols}
\end{document}