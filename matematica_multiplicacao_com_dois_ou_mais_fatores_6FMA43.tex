\documentclass[a4paper,14pt]{article}
\usepackage{float}
\usepackage{extsizes}
\usepackage{amsmath}
\usepackage{amssymb}
\everymath{\displaystyle}
\usepackage{geometry}
\usepackage{fancyhdr}
\usepackage{multicol}
\usepackage{graphicx}
\usepackage[brazil]{babel}
\usepackage[shortlabels]{enumitem}
\usepackage{cancel}
\usepackage{textcomp}
\usepackage{array} % Para melhor formatação de tabelas
\usepackage{longtable}
\usepackage{booktabs}  % Para linhas horizontais mais bonitas
\usepackage{float}   % Para usar o modificador [H]
\usepackage{caption} % Para usar legendas em tabelas
\usepackage{tcolorbox}

\columnsep=2cm
\hoffset=0cm
\textwidth=8cm
\setlength{\columnseprule}{.1pt}
\setlength{\columnsep}{2cm}
\renewcommand{\headrulewidth}{0pt}
\geometry{top=1in, bottom=1in, left=0.7in, right=0.5in}

\pagestyle{fancy}
\fancyhf{}
\fancyfoot[C]{\thepage}

\begin{document}
	
	\noindent\textbf{6FMA43 - Matemática} 
	
	\begin{center}Multiplicação com dois ou mais fatores (Versão estudante)
	\end{center}
	
	\noindent\textbf{Nome:} \underline{\hspace{10cm}}
	\noindent\textbf{Data:} \underline{\hspace{4cm}}
	
	%\section*{Questões de Matemática}
	\begin{multicols}{2}
		\noindent Quando tivermos que calcular a soma de dois ou mais números iguais, usamos a multiplicação. \\
		Os números que compõem uma multiplicação são chamados \textbf{fatores} e o resultado chama-se \textbf{produto}. \\
		Quando temos uma multiplicação com mais dois fatores, por exemplo, $2 \times 3 \times 5$, podemos calcular primeiro $2 \times 3 = 6$ e depois multiplicar este por 5, obtendo $6 \times 5 = 30$, ou podemos calcular $3 \times 5 = 15$ e multiplicar por 2, obtendo $2 \times 15 = 30$. Essa é a propriedade \textbf{associativa} da multiplicação.
	\textsubscript{---------------------------------------------------------------------}
    	\begin{enumerate}
    		\item Represente como multiplicação (sem calcular).
    		\begin{enumerate}[a)]
    			\item 213 + 213 + 213 \\\\\\\\
    			\item 7103 + 7103 + 7103 + 7103 + 7103 + 7103 + 7103 + 7103 \\\\\\\\
    		\end{enumerate}
    		\item Ache o quádruplo do número 21. \\\\\\\\\\\\
    		\item Determine o quíntuplo de 8. \\\\\\\\\\\\
    		\item Preencha as tabelas abaixo. \\\\
    		\begin{tabular}{ccc}
    			$6 \times 1 = $ & ~~~~~~ & $7 \times 1 = $ \\\\
    			$6 \times 2 = $ & ~~~~~~ & $7 \times 2 = $  \\\\
    			$6 \times 3 = $	 & ~~~~~~ & $7 \times 3 = $  \\\\
    			$6 \times 4 = $ & ~~~~~~ & $7 \times 4 = $ \\\\
    			$6 \times 5 = $ & ~~~~~~ & $7 \times 5 = $  \\\\
    			$6 \times 6 = $	 & ~~~~~~ & $7 \times 6 = $  \\\\
    			$6 \times 7 = $ & ~~~~~~ & $7 \times 7 = $ \\\\
    			$6 \times 8 = $ & ~~~~~~ & $7 \times 8 = $  \\\\
    			$6 \times 9 = $	 & ~~~~~~ & $7 \times 9 = $  \\\\
    		\end{tabular} \newpage
    		\begin{tabular}{ccc}
    			$8 \times 1 = $ & ~~~~~~ & $9 \times 1 = $ \\\\
    			$8 \times 2 = $ & ~~~~~~ & $9 \times 2 = $  \\\\
    			$8 \times 3 = $	 & ~~~~~~ & $9 \times 3 = $  \\\\
    			$8 \times 4 = $ & ~~~~~~ & $9 \times 4 = $ \\\\
    			$8 \times 5 = $ & ~~~~~~ & $9 \times 5 = $  \\\\
    			$8 \times 6 = $	 & ~~~~~~ & $9 \times 6 = $  \\\\
    			$8 \times 7 = $ & ~~~~~~ & $9 \times 7 = $ \\\\
    			$8 \times 8 = $ & ~~~~~~ & $9 \times 8 = $  \\\\
    			$8 \times 9 = $	 & ~~~~~~ & $9 \times 9 = $  \\\\
    		\end{tabular}
			\item Quanto vale $4 \cdot 2 \cdot 7$? \\\\\\
			\item Calcule:
			\begin{enumerate}[a)]
				\item $2 \times 3 \times 6$ \\\\\\\\\\
				\item $4 \times 2 \times 5 \times 3$ \\\\\\\\\\
				\item $9 \times 3 \times 6$ \\\\\\\\\\
				\item $3 \times 5 \times 7 \times 8$ \\\\\\\\\\
				\item $2 \times 6 \times 4 \times 9$ \\\\\\\\\\
			\end{enumerate}
			\item Mudando apenas a ordem dos fatores, escreva todas as maneiras possíveis à multiplicação $2 \cdot 4 \cdot 6$. 
    	\end{enumerate}
    	$~$ \\ $~$ \\ $~$ \\ $~$ \\ $~$ \\ $~$ \\ $~$ \\ $~$ \\ $~$ \\ $~$ \\ $~$ \\ $~$ \\ $~$ \\ $~$ \\ $~$ \\ $~$ \\ $~$ \\ $~$ \\ $~$ \\ $~$ \\ $~$ \\
	\end{multicols}
\end{document}