\documentclass[a4paper,14pt]{article}

\usepackage{comment} % Para comentar várias linhas ao mesmo tempo

%matemática
\usepackage{amsmath}
\usepackage{amssymb}

%diagramação
\usepackage{extsizes}
\everymath{\displaystyle}
\usepackage{geometry}
\usepackage{fancyhdr}
\usepackage{multicol}
\usepackage{graphicx}
\usepackage[brazil]{babel}
\usepackage[shortlabels]{enumitem}
\usepackage{cancel}
\usepackage{textcomp}
\usepackage{tcolorbox}

%tabelas
\usepackage{array} % Para melhor formatação de tabelas
\usepackage{longtable}
\usepackage{booktabs}  % Para linhas horizontais mais bonitas
\usepackage{float}   % Para usar o modificador [H]
\usepackage{caption} % Para usar legendas em tabelas
\usepackage{wrapfig} % Para usar tabelas e figuras flutuantes


%tikzpicture
\usepackage{tikz}
\usepackage{scalerel}
\usepackage{pict2e}
\usepackage{tkz-euclide}
\usetikzlibrary{calc}
\usetikzlibrary{patterns,arrows.meta}
\usetikzlibrary{shadows}
\usetikzlibrary{external}

%pgfplots
\usepackage{pgfplots}
\pgfplotsset{compat=newest}
\usepgfplotslibrary{statistics}
\usepgfplotslibrary{fillbetween}

%colours
\usepackage{xcolor}



\columnsep=2cm
\hoffset=0cm
\textwidth=8cm
\setlength{\columnseprule}{.1pt}
\setlength{\columnsep}{2cm}
\renewcommand{\headrulewidth}{0pt}
\geometry{top=1in, bottom=1in, left=0.7in, right=0.5in}

\pagestyle{fancy}
\fancyhf{}
\fancyfoot[C]{\thepage}

\begin{document}
	
	\noindent\textbf{6FMA01 - Matemática} 
	
	\begin{center}Pertinência (Versão estudante)
	\end{center}
	
	\noindent\textbf{Nome:} \underline{\hspace{10cm}}
	\noindent\textbf{Data:} \underline{\hspace{4cm}}
	
	%\section*{Questões de Matemática}
	
	\begin{multicols}{2}
		\noindent Seja o conjunto $A = \{1, \{1\}, 2, 3\}$. Esse conjunto é apresentado com chaves, isto é, uma chave à esquerda \{ e uma à direita \}. O que está escrito entre as chaves são os elementos do conjunto. Assim, o conjunto $A$ tem 4 elementos distintos, que são os números 1, 2 e 3 e o conjunto \{1\} (observemos que 1 $\neq$ \{1\}). Para dizer que 2 é elemento de $A$, indicamos $2 \in A$ (ou $A \ni 2$). \\
		Um conjunto que não possui elementos é chamado \textbf{vazio} e é indicado por $\varnothing$ ou \{ \}. Usamos no nosso curso apenas a notação $\varnothing$.
		\noindent\textsubscript{-----------------------------------------------------------------------}
		\begin{enumerate}
			\item Dados os conjuntos abaixo, escreva quantos elementos tem cada conjunto e quais são seus elementos.
			\begin{enumerate}[a)]
				\item $C = \{1, 2, 3, 4\}$ \\\\\\\\
				\item $M = \{1, 2, \{2\}, \{3\}\}$ \\\\\\\\
				\item $E = \{\varnothing, 0, \{0\}\}$ \\\\
				\item $S = \{\varnothing, \{\varnothing\}, \{\{\varnothing\}\}\}$ \\\\\\\\
			\end{enumerate}
			\item Sendo $A = \{1, \{3\}, 5, \{7\}\}$:
			\begin{enumerate}[a)]
				\item quantos elementos distintos tem o conjunto $A$? \\\\\\\\
				\item quais são os elementos de $A$? \\\\\\\\
				Para os itens de $c$ a $h$, assinale \textbf{V} (verdadeiro) ou \textbf{F} (falso).
				\item (~~) $\{1\} \in A$
				\item (~~) $3 \in A$
				\item (~~) $\{5\} \in A$
				\item (~~) $7 \in A$
				\item (~~) $\{3\} \in A$
				\item (~~) $\varnothing \in A$
			\end{enumerate}
			\item Reescreva em linguagem simbólica.
			\begin{enumerate}[a)]
				\item 8 não é elemento de $C$. \\\\
				\item 13 é elemento de $M$. \\\\\\\\
				\item O conjunto unitário de 15 não é elemento de $S$. \\\\\\\\
				\item O conjunto cujos elementos são os números 1 e 2 é elemento de $E$. \\\\\\\\
				\item O conjunto vazio é elemento de $A$.
			\end{enumerate}
			%1 a 4
			\item Em cada um dos conjuntos, escreva os seus elementos.
			\begin{enumerate}[a)]
				\item $A = \{\{3\}\}$ \\\\\\\\
				\item $B = \{\varnothing, \{\varnothing\}\}$ \\\\\\\\
				\item $C = \{1, 2, \{1\}\}$ \\\\\\\\
				\item $D = \{7\}$ \\\\\\\\
				\item $E = \{2, 4\}$ \\\\\\\\
				\item $F = \{3, 5, 7\}$ \\\\\\\\
			\end{enumerate}	
			\item Assinale \textbf{V} (verdadeiro) ou \textbf{F} (falso).
			\begin{enumerate}[a)]
				\item (~~) $2 \in \{1, 3, 5\}$
				\item (~~) $3 \in \{3\}$
				\item (~~) $\{1\} \in \{1, 2\}$
				\item (~~) $2 \in \{\{2\}\}$
				\item (~~) $\{5\} \in \{\{5\}, 6\}$
				\item (~~) $\varnothing \in \{1, 0, \{ \varnothing \}\}$
				\item (~~) $12 \in \{12, 13\}$
				\item (~~) $4 \notin \{4, 5\}$
				\item (~~) $3 \notin \{1, 2\}$
				\item (~~) $\{6, 8\} \ni 8$
				\item (~~) $\{5, 9\} \not\ni 7$
				\item (~~) $\{\{1\}, 5\} \ni 1$
			\end{enumerate}	
			\item Seja $A = \{2, \{2\}, 4, \{6\}\}$. Assinale \textbf{V} (verdadeiro) ou \textbf{F} (falso).
			\begin{enumerate}[a)]
				\item (~~) $2 \in A$
				\item (~~) $\{\{2\}\} \in A$
				\item (~~) $4 \in A$
				\item (~~) $\{2\} \in A$
				\item (~~) $\{\{6\}\} \in A$
				\item (~~) $\{2, 4\} \in A$
				\item (~~) $\{\{2\}, \{6\}\} \in A$
				\item (~~) $\{2, \{2\}\} \in A$
				\item (~~) $\varnothing \in A$
				\item (~~) $\{4\} \notin A$
			\end{enumerate}
		\end{enumerate}
		$~$ \\ $~$ \\ $~$ \\ $~$ \\ $~$ \\ $~$ \\ $~$ \\ $~$ \\ $~$ \\ $~$ \\ $~$ \\ $~$ \\ $~$ \\ $~$ \\ $~$ \\ $~$ \\ $~$ \\ $~$ \\ $~$ \\ $~$ \\ $~$ \\ $~$ \\ $~$ \\ $~$ \\ $~$ \\ $~$ \\ $~$ \\ $~$ \\ $~$ \\ $~$ \\ $~$ \\ $~$ \\ $~$ \\ $~$ \\ $~$ \\ $~$ \\ $~$ \\ $~$ \\ $~$ \\ $~$ \\ $~$ \\ $~$ \\ $~$ \\ $~$ \\ $~$ \\ $~$ \\ $~$ \\ $~$ \\ $~$ \\ $~$ \\ $~$ \\ $~$ \\ $~$ \\ $~$ \\ $~$ \\ $~$ \\ $~$ \\ $~$ \\ $~$ \\ $~$ \\ $~$ \\ $~$ \\ $~$ \\ $~$ \\ $~$ \\ $~$ \\ $~$ \\ $~$ \\ $~$ \\ $~$ \\ $~$ \\ $~$ \\ $~$ \\ $~$ \\ 
	\end{multicols}
\end{document}