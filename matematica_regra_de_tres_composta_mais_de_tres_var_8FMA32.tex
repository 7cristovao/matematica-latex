\documentclass[a4paper,14pt]{article}
\usepackage{float}
\usepackage{extsizes}
\usepackage{amsmath}
\usepackage{amssymb}
\everymath{\displaystyle}
\usepackage{geometry}
\usepackage{fancyhdr}
\usepackage{multicol}
\usepackage{graphicx}
\usepackage[brazil]{babel}
\usepackage[shortlabels]{enumitem}
\usepackage{cancel}
\columnsep=2cm
\hoffset=0cm
\textwidth=8cm
\setlength{\columnseprule}{.1pt}
\setlength{\columnsep}{2cm}
\renewcommand{\headrulewidth}{0pt}
\geometry{top=1in, bottom=1in, left=0.7in, right=0.5in}

\pagestyle{fancy}
\fancyhf{}
\fancyfoot[C]{\thepage}

\begin{document}
	
	\noindent\textbf{8FMA32~Matemática} 
	
	\begin{center}Regra de três composta - Mais de três variáveis (Versão estudante)
	\end{center}
	
	\noindent\textbf{Nome:} \underline{\hspace{10cm}}
	\noindent\textbf{Data:} \underline{\hspace{4cm}}
	
	%\section*{Questões de Matemática}
	
    \begin{multicols}{2}
    	Considere as variáveis proporcionais $x, y_i(i=1, 2, 3, ..., k)$ e $z_j(j=1, 2, 3, ..., l)$. Sendo $x$ diretamente proporcional a $y_i$ e inversamente proporcional a $z_j$ temos $\frac{x \cdot z_1 \cdot z_2 \cdot z_3 \cdot ... \cdot z_l}{y_1 \cdot y_2 \cdot y_3 \cdot ... \cdot y_k}$ constante. \\\\
		\begin{enumerate}
			\item Oito pintores pintam uma parede de 15 metros de comprimento por 4 metros de altura em 3 dias de trabalho. Quantos dias seriam necessários para 4 pintores pintarem uma parede de 10 m de comprimento por 6 metros de altura? (Considere que todos os pintores possuem o mesmo rendimento.) \\\\\\\\\\\\\\\\\\\\\\\\\\\\\\\\
			\item Se 14 máquinas, funcionando 5 horas por dia, durante 81 dias, produzem 180000 peças, em quantos dias 18 dessas mesmas máquinas, funcionando 7 horas por dia, produzirão 236000 peças? \\\\\\\\\\\\\\\\\\\\
			\item Numa gráfica, 4 impressoras idênticas imprimem 650000 folhas em 3 dias, trabalhando 5 horas por dia. Quantas impressoras iguais às anteriores seriam necessárias para imprimir 3900000 folhas em 12 dias, trabalhando 6 horas por dia? \\\\\\\\\\\\\\\\\\\\
			\item Daniel recebeu R\$ 2590,00 após ter trabalhado durante 3 semanas, 4 dias por semana, 7 horas por dia. Se ele tivesse trabalhado 6 horas por dia, 3 dias por semana, durante 5 semanas, quanto teria recebido? \\\\\\\\\\\\\\\\\\\\\\\\
			\item Uma turbina de avião consome 400 litros de combustível em 6 horas de funcionamento a 2400 rotações por minuto. Quantos litros consumiria, em 5 horas, um avião com 3 turbinas idênticas à anterior, cada uma funcionando a 1800 rotações por minuto? \\\\\\\\\\\\\\\\\\\\\\\\\\\\
			\textbf{Desafio olímpico}\\
			(OBMEP) Os médicos recomendam, para um adulto, 800 mg de cálcio por dia. Sabe-se que 200 mL de leite contêm 296 mg de cálcio. Quando um adulto bebe 200 mL de leite, qual é o percentual da dose diária recomendada de cálcio que ele está ingerindo?\\\\
			a) 17\% b) 27\% c) 37\% d) 47\% 57\% \\\\\\\\\\\\\\\\\\\\\\\\\\\\\\\\\\\\\\\\\\\\\\\\\\\\\\\\
			\item Num hotel de quatro andares, cada um com sete quartos, são necessárias oito camareiras para arrumar todos os quartos entre as 12 h e as 14 h. Num outro hotel da mesma rede e com quartos do mesmo tamanho, há sete andares com dez quartos em cada. Quantas camareiras são necessárias para arrumar todos os quartos entre as 11h30min e as 14 h? \\
			Considere que todas as camareiras têm o mesmo rendimento.  \\\\\\\\\\\\\\\\\\\\\\\\\\\\\\\\\\\\\\\\\\\\\\\\\\\\
			\item Num recipiente de 250 mL, há $2,4 \cdot 10^{24}$ moléculas de um certo gás perfeito a uma temperatura de $40^\circ$C e uma pressão de 4 atm. Num outro recipiente de 500 mL, o mesmo tipo de gás tem temperatura de $50^\circ$C e pressão de 3 atm. Quantas moléculas de gás há no segundo recipiente, sabendo-se que o número de moléculas é diretamente proporcional ao volume e à pressão, e inversamente proporcional à temperatura? \\\\\\\\\\\\\\\\\\\\\\\\\\\\\\\\\\\\\\\\\\\\\\\\\\\\
			\item Duas digitadoras, trabalhando no mesmo ritmo durante oito horas por dia, de segunda a sexta, digitam um livro de 820 páginas em três semanas. Quantos livros, de 615 páginas cada, seis digitadoras, com o mesmo ritmo de trabalho das primeiras, digitam em quatro semanas, trabalhando cinco horas por dia, de segunda a sábado? \\\\\\\\\\\\\\\\\\\\\\\\\\\\\\\\\\\\\\\\\\\\\\\\\\\\\\\\\\\\
			\item Cinco operários, trabalhando 6 horas por dia, constroem um muro de 15 metros de comprimento por 4 metros de altura em 6 dias. Quantos operários, trabalhando 5 horas por dia, durante 3 dias, serão necessários para construir um muro de 8 metros de comprimento por 5 metros de altura?\\
			Considere que todos os operários têm o mesmo ritmo de trabalho. \\\\\\\\\\\\\\\\\\\\\\\\\\\\\\\\\\\\\\
		\end{enumerate}
	$~$ \\ $~$ \\ $~$ \\ $~$ \\ $~$ \\ $~$ \\ $~$ \\ $~$ \\ $~$ \\
    \end{multicols}
\end{document}