\documentclass[a4paper,14pt]{article}
\usepackage{float}
\usepackage{extsizes}
\usepackage{amsmath}
\usepackage{amssymb}
\everymath{\displaystyle}
\usepackage{geometry}
\usepackage{fancyhdr}
\usepackage{multicol}
\usepackage{graphicx}
\usepackage[brazil]{babel}
\usepackage[shortlabels]{enumitem}
\usepackage{cancel}
\usepackage{textcomp}
\usepackage{array} % Para melhor formatação de tabelas
\usepackage{longtable}
\usepackage{booktabs}  % Para linhas horizontais mais bonitas
\usepackage{float}   % Para usar o modificador [H]
\usepackage{caption} % Para usar legendas em tabelas

\columnsep=2cm
\hoffset=0cm
\textwidth=8cm
\setlength{\columnseprule}{.1pt}
\setlength{\columnsep}{2cm}
\renewcommand{\headrulewidth}{0pt}
\geometry{top=1in, bottom=1in, left=0.7in, right=0.5in}

\pagestyle{fancy}
\fancyhf{}
\fancyfoot[C]{\thepage}

\begin{document}
	
	\noindent\textbf{8FMA117, 8FMA118 - Matemática} 
	
	\begin{center}Inequação produto (Versão estudante)
	\end{center}
	
	\noindent\textbf{Nome:} \underline{\hspace{10cm}}
	\noindent\textbf{Data:} \underline{\hspace{4cm}}
	
	%\section*{Questões de Matemática}	
    \begin{multicols}{2}
    	\noindent Ao estudarmos o sinal de uma expressão da forma $A \cdot B$, primeiramente observam-se os sinais das expressões $A$ e $B$ separadamente para depois analisarmos o sinal de $A \cdot B$. \\
    	\noindent\textsubscript{~---------------------------------------------------------------------------}
    	\begin{enumerate}
    		\item Resolver as seguintes inequações:
    		\begin{enumerate}[a)]
    			\item $-x(x^2 + 2x)(x^2 + 7x + 12) \geq 0$ \\\\\\\\\\\\\\\\\\\\\\\\
    			\item $(x^2 - 5x + 4)(-x^2 + 16)\\(8x + 1) \geq 0$
    			\\\\\\\\\\\\\\\\\\\\\\\\
    			\item $(x^2 + x + 4)(x^2 - 6x + 9)\\(x + 1) < 0$ 
    			\\\\\\\\\\\\\\\\\\\\
    			\item $(x - 3)^2 < 2x - 3$
    			\\\\\\\\\\\\\\\\\\\\
    			\item $x^2(x^2 - 4)(x^2 - 7) \geq 0$
    			\\\\\\\\\\\\\\\\\\\\\\
    			\item $(-x^2 + 6x - 25)(x^2 + 7) < 0$
    			\\\\\\\\\\\\\\\\\\\\\\\\
    			\item $(x^2 - 3x + 6)(x^2 - 5x + 4)\\(x^2 - 25) \leq 0$
    			\\\\\\\\\\\\\\\\\\\\\\\\
    			\item $(x^2 - 16)(x^2 - 8x) > 0$
    			\\\\\\\\\\\\\\\\\\\\\\\\
    		\end{enumerate}
    		\item Resolver as inequações em $U = \mathbb{R}$
    		\begin{enumerate}[a)]
    			\item $x(x^2 - 1)(x^2 - 7x + 12) > 0$ 
    			\\\\\\\\\\\\\\\\\\\\\\\\
    			\item $(x^2 - 11x + 10)(-x^2 + 9)\\(2x + 16)(x^2 + 7x) \leq 0$
    			\\\\\\\\\\\\\\\\\\\\\\\\
    			\item $-x^2(x^2 - 2x + 1)(x^2 + 5) < 0$
    			\\\\\\\\\\\\\\\\\\\\\\
    			\item $-x(x^2 + 8)(-x^2 + 9x - 20)\\(x - 1)(-x^2 + 5x) \geq 0$
    			\\\\\\\\\\\\\\\\\\\\
    			\item $x^2(x^2 - 2)(x + 1)\\(x^2 + 5x - 6) \leq 0$
    			\\\\\\\\\\\\\\\\\\\\\\\\
    		\end{enumerate}
    		\item Apresentar o intervalo dos $x$ reais tais que \\
    		$(x + 1) \cdot (-x + 6) < x^2 - 1$.
    		\\\\\\\\\\\\\\\\\\\\\\
    		\item No universo dos reais, a solução de \\
    		$(-x^2 + 11x - 18) \cdot (x^2 + 3) \leq 0$ é:
    		\begin{enumerate}[a)]
    			\item $V = ]-\infty; 3] \cup [18; +\infty[$ 
    			\\\\\\\\\\\\\\
    			\item $V = [2; 3] \cup [3; 18]$
    			\\\\\\\\\\\\\\
    			\item $V = ]-\infty; 2] \cup [9; +\infty[$
    			\\\\\\\\\\\\\\
    			\item $V = [2; 9] \cup [11; +\infty[$
    			\\\\\\\\\\\\\\
    			\item $V = ]-3; 2] \cup [3; 11]$
    		\end{enumerate}
    	\end{enumerate}
    $~$ \\ $~$ \\ $~$
    \end{multicols}
\end{document}