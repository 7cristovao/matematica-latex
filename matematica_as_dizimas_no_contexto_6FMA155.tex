\documentclass[a4paper,14pt]{article}

\usepackage{comment} % Para comentar várias linhas ao mesmo tempo

%matemática
\usepackage{amsmath}
\usepackage{amssymb}

%diagramação
\usepackage{extsizes}
\everymath{\displaystyle}
\usepackage{geometry}
\usepackage{fancyhdr}
\usepackage{multicol}
\usepackage{graphicx}
\usepackage[brazil]{babel}
\usepackage[shortlabels]{enumitem}
\usepackage{cancel}
\usepackage{textcomp}
\usepackage{tcolorbox}

%tabelas
\usepackage{array} % Para melhor formatação de tabelas
\usepackage{longtable}
\usepackage{booktabs}  % Para linhas horizontais mais bonitas
\usepackage{float}   % Para usar o modificador [H]
\usepackage{caption} % Para usar legendas em tabelas
\usepackage{wrapfig} % Para usar tabelas e figuras flutuantes
\usepackage{xcolor} % Para cores do fundo de tabelas
\usepackage{colortbl} % Para cores do fundo de tabelas
\usepackage{upgreek} % Para inserir caracteres gregos

%tikzpicture
\begin{comment}
	\usepackage{tikz}
	\usepackage{scalerel}
	\usepackage{pict2e}
	\usepackage{tkz-euclide}
	\usetikzlibrary{calc}
	\usetikzlibrary{patterns,arrows.meta}
	\usetikzlibrary{shadows}
	\usetikzlibrary{external}
\end{comment}


%pgfplots
\usepackage{pgfplots}
\pgfplotsset{compat=newest}
\usepgfplotslibrary{statistics}
\usepgfplotslibrary{fillbetween}

%colours
\usepackage{xcolor}



\columnsep=2cm
\hoffset=0cm
\textwidth=8cm
\setlength{\columnseprule}{.1pt}
\setlength{\columnsep}{2cm}
\renewcommand{\headrulewidth}{0pt}
\geometry{top=1in, bottom=1in, left=0.7in, right=0.5in}

\pagestyle{fancy}
\fancyhf{}
\fancyfoot[C]{\thepage}

\begin{document}
	
	\noindent\textbf{6FMA155 - Matemática} 
	
	\begin{center}As dízimas no contexto (Versão estudante)
	\end{center}
	
	\noindent\textbf{Nome:} \underline{\hspace{10cm}}
	\noindent\textbf{Data:} \underline{\hspace{4cm}}
	
	%\section*{Questões de Matemática}
	
	\begin{multicols}{2}
	    \noindent Para $n \in \mathbb{N}^*$: \\
	    1 = 0,9999... \\
	    2 = 1,9999... \\
	    $\vdots$ \\
	    $n = (n - 1),9999...$ \\
		\noindent\textsubscript{--------------------------------------------------------------------------}
		\begin{enumerate} 
			\item Escreva na forma de dízima:
			\begin{enumerate}[a)]
				\item 1 000 \\\\\\\\\\\\\\\\
				\item -1 \\\\\\\\\\\\\\\\
				\item $\frac{1}{4}$ \\\\\\\\\\\\\\\\
				\item $-\frac{2}{5}$ \\\\\\\\\\\\\\\\
			\end{enumerate}
			\item Sabe-se que $a$ e $b$ são inteiros primos entre si e que $\frac{a}{b} = 0,73$. Determine $a$ e $b$. \newpage
			\item Calcule o valor da expressão: \\
			$\frac{2,888... + 0,888...}{4,12555... - 2,31222...}$ \\\\\\\\\\\\\\\\\\\\\\\\\\\\\\\\
			\item Se $\frac{x - 0,666...}{4} = 0,222...$, qual é o valor de $x$? \\\\\\\\\\\\\\\\\\\\\\\\\\\\\\\\\\\\\\
			\item Calcule: \\
			\footnotesize $\bigg(\frac{0,555... - 0,333...}{2 + 0,777...}\bigg)^2 - 0,444... \cdot \bigg(\frac{5}{1 + 0,777...}\bigg)$ \\\\\\\\\\\\\\\\\\\\\\\\\\\\\\\\
			%43 a 50
			\normalsize\item A dízima 1,999... é representada pelo número real $x$. Podemos afirmar que:
			\begin{enumerate}[a)]
				\item $x$ é um número negativo.
				\item $x$ é maior que 2.
				\item $x$ é um número ímpar.
				\item $x$ é um pouco maior que 2.
				\item $x$ é igual a 2. \newpage
			\end{enumerate}
			\item A expressão $(0,\overline{12} + 0,\overline{03}) \cdot 0,24$ equivale a:
			\begin{enumerate}[a)]
				\item $0,0\overline{36}$
				\item $0,0\overline{54}$
				\item $0,0\overline{72}$
				\item $0,\overline{09}$
				\item $0,1\overline{09}$ \\\\\\\\\\\\\\\\\\\\
			\end{enumerate}
			\item A fração irredutível $\frac{a}{b}$, com $a, b \in \mathbb{Z}$, é igual a $2,1\overline{35}$. \\
			Então $a - b$ é igual a:
			\begin{enumerate}[a)]
				\item 891
				\item 704
				\item 673
				\item 562
				\item 437 \\\\\\\\\\\\\\\\\\\\
			\end{enumerate}
			\item Mostre que não existe uma fração de denominador 10 equivalente à fração $\frac{1}{3}$. Em seguida, prove que não existe nenhuma fração de denominador 10 que possa originar a dízima 0,333...33... \newpage
			\item O valor da expressão $\frac{1}{0,8333...} : \bigg(0,2 - \frac{1}{7}\bigg)$ é:
			\begin{enumerate}[a)]
				\item 7
				\item 11
				\item 13
				\item 17
				\item 21 \\\\\\\\\\\\\\\\\\\\
			\end{enumerate}
			\item Simplificando a expressão $\dfrac{1 - \dfrac{1}{4}}{\dfrac{3}{2} + \dfrac{3}{8}} - 3,999...$, obtemos: 
			\begin{enumerate}[a)]
				\item -3
				\item $\frac{15}{2}$
				\item $-\frac{18}{5}$
				\item $\frac{1}{3}$
				\item -4 \\\\\\\\\\
			\end{enumerate}
			\item Calcule a expressão numérica $\bigg(\frac{1}{3} + \frac{3}{5} : 3\bigg):0,3777...$ \\\\\\\\\\\\\\\\\\\\\\\\\\\\
			\item Ao calcular a expressão numérica $\dfrac{\dfrac{1}{6} + \dfrac{3}{4} \cdot 0,999...}{\dfrac{8}{5}:1,7\overline{45}}$
		\end{enumerate}
		$~$ \\ $~$ \\ $~$ \\ $~$ \\ $~$ \\ $~$ \\ $~$ \\ $~$ \\ $~$ \\ $~$ \\ $~$ \\ $~$ \\ $~$ \\ $~$ \\ $~$ \\ $~$ \\ $~$ \\ $~$ \\ $~$ \\
	\end{multicols}
\end{document}