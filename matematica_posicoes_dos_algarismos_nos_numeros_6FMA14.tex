\documentclass[a4paper,14pt]{article}

\usepackage{comment} % Para comentar várias linhas ao mesmo tempo

%matemática
\usepackage{amsmath}
\usepackage{amssymb}

%diagramação
\usepackage{extsizes}
\everymath{\displaystyle}
\usepackage{geometry}
\usepackage{fancyhdr}
\usepackage{multicol}
\usepackage{graphicx}
\usepackage[brazil]{babel}
\usepackage[shortlabels]{enumitem}
\usepackage{cancel}
\usepackage{textcomp}
\usepackage{tcolorbox}

%tabelas
\usepackage{array} % Para melhor formatação de tabelas
\usepackage{longtable}
\usepackage{booktabs}  % Para linhas horizontais mais bonitas
\usepackage{float}   % Para usar o modificador [H]
\usepackage{caption} % Para usar legendas em tabelas
\usepackage{wrapfig} % Para usar tabelas e figuras flutuantes


%tikzpicture
\begin{comment}
	\usepackage{tikz}
	\usepackage{scalerel}
	\usepackage{pict2e}
	\usepackage{tkz-euclide}
	\usetikzlibrary{calc}
	\usetikzlibrary{patterns,arrows.meta}
	\usetikzlibrary{shadows}
	\usetikzlibrary{external}
\end{comment}


%pgfplots
\usepackage{pgfplots}
\pgfplotsset{compat=newest}
\usepgfplotslibrary{statistics}
\usepgfplotslibrary{fillbetween}

%colours
\usepackage{xcolor}



\columnsep=2cm
\hoffset=0cm
\textwidth=8cm
\setlength{\columnseprule}{.1pt}
\setlength{\columnsep}{2cm}
\renewcommand{\headrulewidth}{0pt}
\geometry{top=1in, bottom=1in, left=0.7in, right=0.5in}

\pagestyle{fancy}
\fancyhf{}
\fancyfoot[C]{\thepage}

\begin{document}
	
	\noindent\textbf{6FMA14 - Matemática} 
	
	\begin{center}Posições dos algarismos nos números (Versão estudante)
	\end{center}
	
	\noindent\textbf{Nome:} \underline{\hspace{10cm}}
	\noindent\textbf{Data:} \underline{\hspace{4cm}}
	
	%\section*{Questões de Matemática}
	
	\begin{multicols}{2}
		\noindent A última casa (à direita) de um número é a unidade: a penúltima é a dezena: a antepenúltima é a centena, etc. \\
		O papel de cada algarismo depende da posição que este ocupa no número.
		\noindent\textsubscript{-----------------------------------------------------------------------}
		\begin{enumerate} 
			\item Represente como numerais:
			\begin{enumerate}[a)]
				\item cinco milhares e oito unidades \\\\\\
				\item quatro milhares, sei centenas, nove dezenas e quatro unidades \\\\\\
				\item 3 000 + 600 + 50 + 3 \\\\\\
				\item $6 \cdot 1 000 + 8 \cdot 100 + 4 \cdot 10 + 2$ (cuidado: faça primeiro as multiplicações) \\\\\\
				\item $8 \cdot 1 000 + 4 \cdot 10$ \\\\\\
			\end{enumerate}
			\item Represente somente como somas:
			\begin{enumerate}[a)]
				\item 276 \\\\\\
				\item 681 \\\\\\
				\item 2 135 \\\\\\
				\item 5 072 \\\\\\
			\end{enumerate}
			\item Represente através de uma expressão numérica com somas e produtos os números:
			\begin{enumerate}[a)]
				\item 1 073 265 \\\\\\
				\item 2 316 903 \\\\\\
				\item 5 807 412 \\\\\\
				\item 6 312 003 \\\\\\
				\item 1 007 900 \\\\\\
			\end{enumerate}
			\item Escreva os números abaixo como somas de unidades, dezenas e centenas.
			\begin{enumerate}[a)]
				\item 310 \\\\\\
				\item 243 \\\\\\
				\item 596 \\\\\\
				\item 405 \\\\\\
				\item 871 \\\\\\
				\item 709 \\\\\\
			\end{enumerate}
			\item Represente os números abaixo como somas de unidades, dezenas e centenas.
			\begin{enumerate}[a)]
				\item 148 \\\\\\
				\item 762 \\\\\\
				\item 548 \\\\\\
				\item 217 \\\\\\
				\item 631 \\\\\\
				\item 807 \\\\\\
			\end{enumerate}
			\item Represente através de uma expressão numérica com somas e produtos:
			\begin{enumerate}[a)]
				\item 214 736 \\\\\\
				\item 5 632 001 \\\\\\
				\item 5 218 703 \\
				\item 9 307 263 \\\\\\
				\item 21 742 302 \\\\\\
				\item 6 750 021 \\\\\\
			\end{enumerate}
		\end{enumerate}
		$~$ \\ $~$ \\ $~$ \\ $~$ \\ $~$ \\ $~$ \\ $~$ \\ $~$ \\ $~$ \\ $~$ \\ $~$ \\ $~$ \\ $~$ \\ $~$ \\ $~$ \\ $~$ \\ $~$ \\ $~$ \\ $~$ \\ $~$ \\ $~$ \\ $~$ \\ $~$ \\ $~$ \\ $~$ \\ $~$ \\ $~$ \\ $~$ \\ $~$ \\ $~$ \\ $~$ \\ $~$ \\ $~$ \\ $~$ \\ $~$ \\ $~$ \\ $~$ \\ $~$ \\ $~$ \\ $~$ \\ $~$ \\ $~$ \\ $~$ \\ $~$ \\ $~$ \\ $~$ \\ $~$ \\ $~$ \\ $~$ \\ $~$ \\ $~$ \\ $~$ \\ $~$ \\ $~$ \\ $~$ \\ $~$ \\ $~$ \\ $~$ \\ $~$ \\ $~$ \\ $~$ \\ $~$ \\ $~$ \\ $~$ \\ $~$ \\ $~$ \\ $~$ \\ $~$ \\ $~$ \\ $~$ \\ 
	\end{multicols}
\end{document}