\documentclass[a4paper,14pt]{article}

\usepackage{comment} % Para comentar várias linhas ao mesmo tempo

%matemática
\usepackage{amsmath}
\usepackage{amssymb}

%diagramação
\usepackage{extsizes}
\everymath{\displaystyle}
\usepackage{geometry}
\usepackage{fancyhdr}
\usepackage{multicol}
\usepackage{graphicx}
\usepackage[brazil]{babel}
\usepackage[shortlabels]{enumitem}
\usepackage{cancel}
\usepackage{textcomp}
\usepackage{tcolorbox}

%tabelas
\usepackage{array} % Para melhor formatação de tabelas
\usepackage{longtable}
\usepackage{booktabs}  % Para linhas horizontais mais bonitas
\usepackage{float}   % Para usar o modificador [H]
\usepackage{caption} % Para usar legendas em tabelas
\usepackage{wrapfig} % Para usar tabelas e figuras flutuantes

\begin{comment}
%tikzpicture
\usepackage{tikz}
\usepackage{scalerel}
\usepackage{pict2e}
\usepackage{tkz-euclide}
\usetikzlibrary{calc}
\usetikzlibrary{patterns,arrows.meta}
\usetikzlibrary{shadows}
\usetikzlibrary{external}
\end{comment}
	
%pgfplots
\usepackage{pgfplots}
\pgfplotsset{compat=newest}
\usepgfplotslibrary{statistics}
\usepgfplotslibrary{fillbetween}

%colours
\usepackage{xcolor}



\columnsep=2cm
\hoffset=0cm
\textwidth=8cm
\setlength{\columnseprule}{.1pt}
\setlength{\columnsep}{2cm}
\renewcommand{\headrulewidth}{0pt}
\geometry{top=1in, bottom=1in, left=0.7in, right=0.5in}

\pagestyle{fancy}
\fancyhf{}
\fancyfoot[C]{\thepage}

\begin{document}
	
	\noindent\textbf{6FMA72 - Matemática} 
	
	\begin{center}Conjunto das partes de um conjunto (Versão estudante)
	\end{center}
	
	\noindent\textbf{Nome:} \underline{\hspace{10cm}}
	\noindent\textbf{Data:} \underline{\hspace{4cm}}
	
	%\section*{Questões de Matemática}
	
	\begin{multicols}{2}
		\noindent Dado um conjunto $A$, o conjunto das partes de $A, P(A)$, é o conjunto formado por todos os subconjuntos de $A$. \\
		Exemplo: \\
		$A = \{2, 4, 6\}$ \\
		$P(A) = \{\varnothing, \{2\}, \{4\}, \{6\}, \{2, 4\}, \{2, 6\}, \\ \{4, 6\}, \{2, 4, 6\}\}$ \\
		Se um conjunto $A$ possui $n$ elementos, então $P(A)$ tem $2^{n}$ elementos. \\
		\noindent\textsubscript{--------------------------------------------------------------------------}
    	\begin{enumerate}
   			\item Em cada item, apresente o conjunto das partes do conjunto dado.
   			\begin{enumerate}[a)]
   				\item $A = \{3\}$ \\\\\\\\\\
   				\item $B = \{2, 4\}$ \\\\\\\\\\
   				\item $D = \varnothing$ \\\\\\\\\\
   				\item $E = \{-3, 0, 3\}$ \\\\\\\\\\
   				\item $F = \{0, 2, 4, 6\}$ \\\\\\\\\\
   			\end{enumerate}
   			\item Nos itens abaixo é dado o número de elementos de $P(A)$. Apresente o número de elementos de $A$.
   			\begin{enumerate}[a)]
   				\item 1 \\\\\\\\\\
   				\item 32 \\\\\\\\\\
   				\item 4 \\\\\\\\\\\\\\
   				\item 8 \\\\\\\\\\
   				\item 64 \\\\\\\\\\
   				\item 16 \\\\\\\\\\
   				\item 128 \\\\\\\\\\
   				\item 256 \\\\\\\\\\
   				\item 512 \\\\\\\\\\\\\\\\\\
   			\end{enumerate}
   			%2 a 4
   			\item Determine o conjunto das partes de cada um dos conjuntos a seguir.
   			\begin{enumerate}[a)]
   				\item $A = \{1, 5\}$
   				\item $B = \{2\}$
   				\item $C = \{1, 2, 8, 9\}$
   				\item $D = \varnothing$
   				\item $E = \{\varnothing\}$
   			\end{enumerate}
   			\item O conjunto das partes de um conjunto $A$ tem 32 elementos. Sabendo-se que $\varnothing \in P(A), \{0, 3, 8\} \in P(A)$ e $\{\{4\},\{7\}\} \subset P(A)$, determine $A$. \\\\\\\\\\\\\\\\\\\\
   			\item Explique como é possível calcular o número de elementos de um conjunto $A$ quando é dado o número de elementos de $P(A)$.
	    \end{enumerate}
        $~$ \\ $~$ \\ $~$ \\ $~$ \\ $~$ \\ $~$ \\ $~$ \\ $~$ \\ $~$ \\ $~$
	\end{multicols}
\end{document}