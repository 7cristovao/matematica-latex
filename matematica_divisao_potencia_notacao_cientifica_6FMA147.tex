\documentclass[a4paper,14pt]{article}

\usepackage{comment} % Para comentar várias linhas ao mesmo tempo

%matemática
\usepackage{amsmath}
\usepackage{amssymb}

%diagramação
\usepackage{extsizes}
\everymath{\displaystyle}
\usepackage{geometry}
\usepackage{fancyhdr}
\usepackage{multicol}
\usepackage{graphicx}
\usepackage[brazil]{babel}
\usepackage[shortlabels]{enumitem}
\usepackage{cancel}
\usepackage{textcomp}
\usepackage{tcolorbox}

%tabelas
\usepackage{array} % Para melhor formatação de tabelas
\usepackage{longtable}
\usepackage{booktabs}  % Para linhas horizontais mais bonitas
\usepackage{float}   % Para usar o modificador [H]
\usepackage{caption} % Para usar legendas em tabelas
\usepackage{wrapfig} % Para usar tabelas e figuras flutuantes
\usepackage{xcolor} % Para cores do fundo de tabelas
\usepackage{colortbl} % Para cores do fundo de tabelas
\usepackage{upgreek} % Para inserir caracteres gregos

%tikzpicture
\begin{comment}
	\usepackage{tikz}
	\usepackage{scalerel}
	\usepackage{pict2e}
	\usepackage{tkz-euclide}
	\usetikzlibrary{calc}
	\usetikzlibrary{patterns,arrows.meta}
	\usetikzlibrary{shadows}
	\usetikzlibrary{external}
\end{comment}


%pgfplots
\usepackage{pgfplots}
\pgfplotsset{compat=newest}
\usepgfplotslibrary{statistics}
\usepgfplotslibrary{fillbetween}

%colours
\usepackage{xcolor}



\columnsep=2cm
\hoffset=0cm
\textwidth=8cm
\setlength{\columnseprule}{.1pt}
\setlength{\columnsep}{2cm}
\renewcommand{\headrulewidth}{0pt}
\geometry{top=1in, bottom=1in, left=0.7in, right=0.5in}

\pagestyle{fancy}
\fancyhf{}
\fancyfoot[C]{\thepage}

\begin{document}
	
	\noindent\textbf{6FMA147 - Matemática} 
	
	\begin{center}Divisão (Versão estudante)
	\end{center}
	
	\noindent\textbf{Nome:} \underline{\hspace{10cm}}
	\noindent\textbf{Data:} \underline{\hspace{4cm}}
	
	%\section*{Questões de Matemática}
	
	\begin{multicols}{2}
	    \noindent Para $m, n \in \mathbb{N}$, valem as seguintes propriedades: \\
	    \begin{itemize}
	    	\item $\frac{10^m}{10^n} = 10^{m - n}$
	    	\item $10^{-n} = \frac{1}{10^n}$
	    \end{itemize}
		\noindent\textsubscript{--------------------------------------------------------------------------}
		\begin{enumerate} 
			\item Escreva na forma mais simples:
			\begin{enumerate}[a)]
				\item $\frac{10^{14}}{10^{23}}$ \\\\\\
				\item $10^{47} : 10^7$ \\\\\\
				\item $\frac{10^{32}}{(10^9 : 10^6)}$ \\\\\\
			\end{enumerate}
			\item Calcule:
			\begin{enumerate}[a)]
				\item $\frac{10^8 \cdot 10^{-21}}{10^{14}}$ \\\\\\\\
				\item $[(10^{37} \cdot 10^6) : 10^4] \cdot 10^{-12}$ \\\\
				\item $\frac{10^7 \cdot 10^6 \cdot 10^{-5}}{10^2 \cdot 10^9 \cdot 10^{-8}}$ \\\\\\\\
				\item $(10^6 : 10^4) \cdot 10^{-2} \cdot 10^6$ \\\\\\\\
				\item $\frac{10^8 : 10^5}{10^7 : (10^{-2} \cdot 10^4)}$ \\\\\\\\
			\end{enumerate}
			\item Para $A = 4,2 \cdot 10^{-4}, B = 2 \cdot 10^6, C = 10^{-9}$ e $D = 8 \cdot 10^5$, calcule e dê sua resposta em notação científica:
			\begin{enumerate}[a)]
				\item $\frac{A \cdot B}{C}$ \\\\\\\\
				\item $(A \cdot D) : B$ \\\\\\\\
				\item $\frac{B \cdot D}{C}$ \newpage
				\item $\frac{A \cdot D}{B \cdot C}$ \\\\\\\\
			\end{enumerate}
			% 5 a 8
			\item Simplifique:
			\begin{enumerate}[a)]
				\item $\frac{10^{21}}{10^3}$ \\\\\\\\
				\item $10^6 : 10^4$ \\\\\\\\
				\item $\frac{10^4 : 10^2}{(10^2 : 10^5) : 10^2}$ \\\\\\\\
				\item $\frac{10^7}{(10^6 : 10^3)} : 10^6$ \\\\\\\\
				\item $\dfrac{10^7}{\dfrac{10^4}{10^2}} : 10^8$ \\\\\\\\
				\item $\frac{10^2}{(10^4 : 10^6)}$ \\\\\\\\
			\end{enumerate}
			\item Calcule:
			\begin{enumerate}[a)]
				\item $\frac{10^3 \cdot 10^{-6}}{10^7 \cdot 10^2}$ \\\\\\\\
				\item $\frac{[(10^4 \cdot 10^{-6}) : 10^6] \cdot 10^2}{10^9}$ \\\\\\\\
				\item $\frac{10^{-2} \cdot 10^3 \cdot 10^6}{10^4 \cdot 10^3 \cdot 10^4} : 10^6$ \\\\\\\\
				\item $10^4 : [(10^3 : 10^4) \cdot 10^2 \cdot 10^8]$ \\\\\\\\
				\item $\frac{10^2 : (10^5 \cdot 10^2)}{(10^{-2} \cdot 10^7) : 10^4} : 10^9$ \\\\\\\\
				\item $\frac{[10^4 : (10^3 \cdot 10^4)] : 10^6}{10^7 \cdot (10^4 : 10^6)}$ \\\\\\\\
				\item $10^2 : \frac{10^4 \cdot 10^2}{10^6 : 10^7}$ \\\\\\\\
			\end{enumerate}
			\item Para $A = 3 \cdot 10^3, B = 9 \cdot 10^6, \\ C = 10^{-8}$ e $D = 3 \cdot 10^{-2}$, calcule:
			\begin{enumerate}[a)]
				\item $A \cdot \frac{B : D}{C}$ \\\\\\\\
				\item $[A \cdot (D \cdot C) : B]$ \\\\\\\\
				\item $B : (C \cdot A)$ \\\\\\\\
				\item $\frac{C \cdot A}{B : D}$ \\\\\\\\
			\end{enumerate}
			\item Escreva na forma de potência:
			\begin{enumerate}[a)]
				\item $\frac{10^{14}}{1000}$ \\\\\\\\
				\item $10 : \frac{10^2}{10^6}$ \\\\\\\\
				\item $10^7 : (10^5 \cdot 10^4)$ \\\\
				\item $\frac{10^{-6} \cdot 10^7}{10^9 : 10^3} : 10^5$ \\\\\\\\
				\item $(10^7 : 10^4) : 10^6$ \\\\\\\\
				\item $\frac{10^9 : 10^6}{10^4 \cdot 10^2}$ \\\\\\\\
			\end{enumerate}
		\end{enumerate}
		 $~$ \\ $~$ \\ $~$ \\ $~$ \\ $~$ \\ $~$ \\ $~$ \\ $~$ \\ $~$ \\ $~$ \\ $~$ \\ $~$ \\ $~$ \\ $~$ \\ $~$ \\ $~$ \\ $~$ \\ $~$ \\ $~$ \\ $~$ \\ $~$ \\ $~$ \\ $~$ \\ $~$ \\
	\end{multicols}
\end{document}