\documentclass[a4paper,14pt]{article}

\usepackage{comment} % Para comentar várias linhas ao mesmo tempo

%matemática
\usepackage{amsmath}
\usepackage{amssymb}

%diagramação
\usepackage{extsizes}
\everymath{\displaystyle}
\usepackage{geometry}
\usepackage{fancyhdr}
\usepackage{multicol}
\usepackage{graphicx}
\usepackage[brazil]{babel}
\usepackage[shortlabels]{enumitem}
\usepackage{cancel}
\usepackage{textcomp}
\usepackage{tcolorbox}

%tabelas
\usepackage{array} % Para melhor formatação de tabelas
\usepackage{longtable}
\usepackage{booktabs}  % Para linhas horizontais mais bonitas
\usepackage{float}   % Para usar o modificador [H]
\usepackage{caption} % Para usar legendas em tabelas
\usepackage{wrapfig} % Para usar tabelas e figuras flutuantes
\usepackage{xcolor} % Para cores do fundo de tabelas
\usepackage{colortbl} % Para cores do fundo de tabelas
\usepackage{upgreek} % Para inserir caracteres gregos

%tikzpicture
\begin{comment}
	\usepackage{tikz}
	\usepackage{scalerel}
	\usepackage{pict2e}
	\usepackage{tkz-euclide}
	\usetikzlibrary{calc}
	\usetikzlibrary{patterns,arrows.meta}
	\usetikzlibrary{shadows}
	\usetikzlibrary{external}
\end{comment}


%pgfplots
\usepackage{pgfplots}
\pgfplotsset{compat=newest}
\usepgfplotslibrary{statistics}
\usepgfplotslibrary{fillbetween}

%colours
\usepackage{xcolor}



\columnsep=2cm
\hoffset=0cm
\textwidth=8cm
\setlength{\columnseprule}{.1pt}
\setlength{\columnsep}{2cm}
\renewcommand{\headrulewidth}{0pt}
\geometry{top=1in, bottom=1in, left=0.7in, right=0.5in}

\pagestyle{fancy}
\fancyhf{}
\fancyfoot[C]{\thepage}

\begin{document}
	
	\noindent\textbf{6FMA152 - Matemática} 
	
	\begin{center}Decimais e frações (Versão estudante)
	\end{center}
	
	\noindent\textbf{Nome:} \underline{\hspace{10cm}}
	\noindent\textbf{Data:} \underline{\hspace{4cm}}
	
	%\section*{Questões de Matemática}
	
	\begin{multicols}{2}
	    \noindent \textbf{Dízimas periódicas} são os números decimais infinitos originados de frações. Exemplos: \\
	    $\frac{4}{9} = 0,444... = 0,\overline{4}$ \\\\
	    $\frac{13}{99} = 0,131313... = 0,\overline{13}$ \\
	    É importante lembrar que: \\
	    \begin{itemize}
	    	\item $0,1 = \frac{1}{10}$
	    	\item $0,01 = \frac{1}{100}$
	    	\item $0,001 = \frac{1}{1000}$
	    	\item $0,0001 = \frac{1}{10 000}$
	    	\item \textbf{Dízima periódica simples: } o período começa logo após a vírgula.
	    	\item \textbf{Dízima periódica composta: } o período começa imediatamente após a vírgula. \\
	    \end{itemize}
		\noindent\textsubscript{--------------------------------------------------------------------------}
		\begin{enumerate} 
			\item Quais dos exemplos a seguir são dízimas simples e quais são compostas?
			\begin{enumerate}[a)]
				\item $0,\overline{83}$ \\\\\\
				\item $1,\overline{402}$ \\\\\\
				\item $2,6\overline{54}$ \\\\\\
				\item $0,\overline{8}$ \\\\\\
			\end{enumerate}
			\item Apresente uma dízima periódica cuja parte inteira é 183 e cuja parte decimal tem período com três algarismos distintos e a parte não periódica com dois algarismos distintos. Essa dízima é simples ou composta? \newpage
			\item Escreva na forma simplificada cada uma das dízimas a seguir, classificando-a como simples ou composta:
			\begin{enumerate}[a)]
				\item 0,25555... \\\\\\
				\item 0,213213...213... \\\\\\
				\item 0,91323232... \\\\\\
			\end{enumerate}
			\item Diga se é verdadeira ou falsa cada uma das afirmações a seguir:
			\begin{enumerate}[a)]
				\item $1,3\dot{6} = 1,\overline{36}$ \\\\
				\item $1,\dot{4} + 0,\dot{2} = 1,\dot{6}$ \\\\
				\item $0,\dot{3} + 0,\dot{4} = 0,7$ \\\\
				\item $2,1\dot{5} = 2,1\overline{55}$ \\\\
			\end{enumerate}
			\item Transforme em frações irredutíveis:
			\begin{enumerate}[a)]
				\item 0,46 \\\\\\
				\item 0,52 \\\\\\
				\item 1,5 \\\\\\
			\end{enumerate}
			%27 a 29
			\item Transforme em numerais decimais:
			\begin{enumerate}[a)]
				\item $\frac{5}{6}$
				\item $\frac{11}{9}$
				\item $\frac{16}{9}$
				\item $\frac{25}{6}$
			\end{enumerate}
			\item Classifique cada uma das dízimas obtidas no exercício anterior. \\\\\\\\\\
			\item Assinale \textbf{V} (verdadeiro) ou \textbf{F} (falso):
			\begin{enumerate}[a)]
				\item (~~) $\frac{7}{9} = 0,777...$
				\item (~~) $1,5 = 1,555...$
				\item (~~) $2,36 = 2,3666...$
				\item (~~) $3,999 = 4$
				\item (~~) $1,324324324... = 1,3\overline{243}$
			\end{enumerate}
		\end{enumerate}
		$~$ \\ $~$ \\ $~$ \\
	\end{multicols}
\end{document}