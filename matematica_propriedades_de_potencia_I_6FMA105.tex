\documentclass[a4paper,14pt]{article}
\usepackage{float}
\usepackage{extsizes}
\usepackage{amsmath}
\usepackage{amssymb}
\everymath{\displaystyle}
\usepackage{geometry}
\usepackage{fancyhdr}
\usepackage{multicol}
\usepackage{graphicx}
\usepackage[brazil]{babel}
\usepackage[shortlabels]{enumitem}
\usepackage{cancel}
\usepackage{textcomp}
\usepackage{array} % Para melhor formatação de tabelas
\usepackage{longtable}
\usepackage{booktabs}  % Para linhas horizontais mais bonitas
\usepackage{float}   % Para usar o modificador [H]
\usepackage{caption} % Para usar legendas em tabelas
\usepackage{tcolorbox}

\columnsep=2cm
\hoffset=0cm
\textwidth=8cm
\setlength{\columnseprule}{.1pt}
\setlength{\columnsep}{2cm}
\renewcommand{\headrulewidth}{0pt}
\geometry{top=1in, bottom=1in, left=0.7in, right=0.5in}

\pagestyle{fancy}
\fancyhf{}
\fancyfoot[C]{\thepage}

\begin{document}
	
	\noindent\textbf{6FMA105 - Matemática} 
	
	\begin{center}Propriedades de potência (I) (Versão estudante)
	\end{center}
	
	\noindent\textbf{Nome:} \underline{\hspace{10cm}}
	\noindent\textbf{Data:} \underline{\hspace{4cm}}
	
	%\section*{Questões de Matemática}
	~ \\ ~
	\begin{multicols}{2}
		\noindent Sendo $a \in \mathbb{Z},~m,~n,~m_1,~m_2,~...,~m_k~ \\ \in~\mathbb{N}$, valem as seguintes propriedades: \\
		\begin{itemize}
			\item \textbf{P1}. $a^m \cdot a^n = a^{m + n}$
			\item \textbf{P1*}. $a^m_1 \cdot a^m_2 \cdot ... \cdot a^m_k = a^{m_1~+~m_2~+~...~+~m^k}$
			\item \textbf{P2}. $\frac{a^m}{a^n} = a^{m - n}$ ~~~~~~~~~~ $(a \neq 0)$
			\item \textbf{P2*}. $a^{-n} = \frac{1}{a^n}$ ~~~~~~~~~~ $(a \neq 0)$
		\end{itemize}
	\textsubscript{---------------------------------------------------------------------}
    	\begin{enumerate}
    		\item Aplicando P1, escreva na forma $a^n$
    		\begin{enumerate}[a)]
    			\item $(-8)^{37} \cdot (-8)^9$ \\\\\\\\\\
    			\item $x^43 \cdot x^18$ \\\\\\\\\\
    			\item $(-42)^0 \cdot (-42)^7$ \\\\
 				\item $y^64 \cdot y^37$ \\\\\\\\\\
    		\end{enumerate}
    		\item Aplique P1*.
    		\begin{enumerate}[a)]
    			\item $3^5 \cdot 3^2 \cdot 3^7$ \\\\\\\\\\
    			\item $(-6)^8 \cdot (-6)^3 \cdot (-6)^6 \cdot (-6)^{12}$ \\\\\\\\\\
    			\item $18^7 \cdot 18^5 \cdot 18^9 \cdot 18^4$ \\\\\\\\\\
    			\item $(-21)^8 \cdot (-21)^24 \cdot (-21)^6 \cdot (-21)^3 \cdot (-21)^{13}$ \newpage
    		\end{enumerate}
    		\item Aplicando P2 ou P2*, escreva na forma $a^n$ ou $\frac{1}{a^n}$.
    		\begin{enumerate}[a)]
    			\item $\frac{(-7)^{12}}{(-7)^{10}}$ \\\\\\\\\\\\\\\\\\\\\\
    			\item $\frac{y^{32}}{y^{31}}$ \\\\\\\\\\\\\\\\\\\\\\
    			\item $\frac{11^{12}}{11^7}$ \\\\\\\\\\\\\\\\\\\\\\
    			\item $\frac{(-3)^4}{(-3)^7}$ \\\\\\\\\\\\\\\\\\\\
    			\item $\frac{x^{21}}{x^{43}}$ \\\\\\\\\\\\\\\\\\\\
    		\end{enumerate}
    		\item Calcular o resultado.
    		\begin{enumerate}[a)]
    			\item $\frac{(-3)^{81}}{(-3)^{78}}$ \\\\\\\\\\\\\\\\\\\\
    			\item $\frac{5^{16}}{5^{18}}$ \newpage
    			\item $7^{22} : 7^{25}$ \\\\\\\\\\\\\\\\\\\\
    			\item $(-4)^{23} : (-4)^{19}$ \\\\\\\\\\\\\\\\\\\\
    		\end{enumerate}
    		\item Escrever na forma $a^n$
    		\begin{enumerate}[a)]
    			\item $\frac{(-3)^{21} \cdot (-3)^{12}}{(-3)^{15} \cdot (-3)^9}$ \\\\\\\\\\\\\\\\\\\\\\\\\\\\
    			\item $\frac{x^{11} \cdot x^{16}}{x^{21} \cdot x^3}$  \\\\\\\\\\\\\\\\\\\\
    			\item $\frac{5^{13} \cdot 5^{15}}{5^{10} \cdot 5^{17} \cdot 5^{16}}$ \\\\\\\\\\\\\\\\\\\\
    			\item $\frac{y^{13} \cdot y^{17} \cdot y^{22}}{y^{23} \cdot y^{41}}$ \newpage
    		\end{enumerate}
    		\item Calcular o valor da expressão $\frac{(-2)^6 \cdot (-2)^{19} \cdot (-2)^2 \cdot (-2)^7}{(-2)^{13} \cdot (-2)^{14} \cdot (-2)^{16}}$ \\\\\\\\\\\\\\\\\\\\
    		\item Escreva na forma $a^n$.
    		\begin{enumerate}[a)]
    			\item $(-5)^6 \cdot (-5)^2$ \\\\\\\\\\\\\\\\\\\\
    			\item $2^2 \cdot 2^8 \cdot 2^{1053}$ \\\\\\\\\\\\\\\\\\\\\\\\\\
    			\item $\frac{(-3)^{15}}{(-3)^6}$ \\\\\\\\\\\\\\\\\\\\
    			\item $\frac{4^7}{4^{12}}$ \\\\\\\\\\\\\\\\\\\\
    			\item $(-1)^8 \cdot 2^8 \cdot (-7)^8$ \\\\\\\\\\\\\\\\\\\\
    			\item $13^6 \cdot 0^6 \cdot (-45)^6 \cdot 2018^6$ \newpage
    			\item $(((-6)^4)^6)^2$ \\\\\\\\\\\\\\\\\\\\
    			\item $(((1^3)^5)^7)^2$ \\\\\\\\\\\\\\\\\\\\
    			\item $5^{2^5}$ \\\\\\\\\\\\\\\\\\\\
    			\item $(-4)^{3^{4}}$ \\\\\\\\\\\\
    		\end{enumerate}
    		\item Escreva na forma $a^n$.
    		\begin{enumerate}[a)]
    			\item $\frac{(-6)^{17} \cdot (-6)^5}{(-6)^8 \cdot (-6)^{12}}$ \\\\\\\\\\\\\\\\\\\\
    			\item $\frac{y^{15} \cdot y^{16}}{y^{18} \cdot y^9}$ \\\\\\\\\\\\\\\\\\\\
    			\item $\frac{9^6 \cdot 9^3 \cdot 9^7}{9^4 \cdot 9^5}$ \newpage
    			\item $\frac{x^{23} \cdot x^4 \cdot x^8}{x^{13} \cdot x^{11}}$ \\\\\\\\\\\\\\\\\\\\
    			\item $\frac{(-5)^{14} \cdot (-5)^0}{(-5)^8 \cdot (-5)^3}$ \\\\\\\\\\\\\\\\\\\\
    		\end{enumerate}
    		\item Usando a propriedade P2, mostre que, para todo inteiro $a \neq 0$, temos $a^0 = 1$ \\\\\\\\\\\\\\\\\\\\
    	\end{enumerate}
    	$~$ \\ $~$ \\ $~$ \\ $~$ \\ $~$ \\ $~$ \\ $~$ \\ $~$ \\ $~$ \\ $~$ \\ $~$ \\ $~$ \\ $~$ \\ $~$ \\ $~$ \\ $~$ \\ $~$ \\ $~$ \\ $~$ \\ $~$ \\ $~$ \\ $~$ \\ $~$ \\ $~$ \\ $~$ \\ $~$ \\ $~$ \\ $~$ \\ $~$ \\ $~$ \\ $~$ \\ $~$ \\ $~$ \\ $~$ \\ $~$ \\ $~$ \\ $~$ \\ $~$ \\ $~$ \\ $~$ \\ $~$ \\ $~$ \\ $~$ \\ $~$ \\ $~$ \\ 
	\end{multicols}
\end{document}