\documentclass[a4paper,14pt]{article}

\usepackage{comment} % Para comentar várias linhas ao mesmo tempo

%matemática
\usepackage{amsmath}
\usepackage{amssymb}

%diagramação
\usepackage{extsizes}
\everymath{\displaystyle}
\usepackage{geometry}
\usepackage{fancyhdr}
\usepackage{multicol}
\usepackage{graphicx}
\usepackage[brazil]{babel}
\usepackage[shortlabels]{enumitem}
\usepackage{cancel}
\usepackage{textcomp}
\usepackage{tcolorbox}

%tabelas
\usepackage{array} % Para melhor formatação de tabelas
\usepackage{longtable}
\usepackage{booktabs}  % Para linhas horizontais mais bonitas
\usepackage{float}   % Para usar o modificador [H]
\usepackage{caption} % Para usar legendas em tabelas
\usepackage{wrapfig} % Para usar tabelas e figuras flutuantes


%tikzpicture
\usepackage{tikz}
\usepackage{scalerel}
\usepackage{pict2e}
\usepackage{tkz-euclide}
\usetikzlibrary{calc}
\usetikzlibrary{patterns,arrows.meta}
\usetikzlibrary{shadows}
\usetikzlibrary{external}

%pgfplots
\usepackage{pgfplots}
\pgfplotsset{compat=newest}
\usepgfplotslibrary{statistics}
\usepgfplotslibrary{fillbetween}

%colours
\usepackage{xcolor}



\columnsep=2cm
\hoffset=0cm
\textwidth=8cm
\setlength{\columnseprule}{.1pt}
\setlength{\columnsep}{2cm}
\renewcommand{\headrulewidth}{0pt}
\geometry{top=1in, bottom=1in, left=0.7in, right=0.5in}

\pagestyle{fancy}
\fancyhf{}
\fancyfoot[C]{\thepage}

\begin{document}
	
	\noindent\textbf{6FMA04 - Matemática} 
	
	\begin{center}Pertinência, inclusão e igualdade (Versão estudante)
	\end{center}
	
	\noindent\textbf{Nome:} \underline{\hspace{10cm}}
	\noindent\textbf{Data:} \underline{\hspace{4cm}}
	
	%\section*{Questões de Matemática}
	
	\begin{multicols}{2}
		\noindent Dois conjuntos são iguais se têm os mesmos elementos. \\
		Consideremos os conjuntos: \\
		$A = \{1, 1 + 1, 2, 2, 3 - 1, 2 + 0, 2, 3\}$ e \\
		$B = \{2 - 1, 2, 1 + 1 + 1, 3, 5 - 2, 2 + 1, 3\}$ \\
		Temos $A = B$, pois os elementos de $A$ são 1, 2 e 3 e os elementos de $B$ também são 1, 2 e 3. \\
		$A = B$ se, e somente se, $A \subset B$ e $B \subset A$.
		\noindent\textsubscript{-----------------------------------------------------------------------}
		\begin{enumerate}
			\item Assinale \textbf{V} (verdadeiro) ou \textbf{F} (falso).
			\begin{enumerate}[a)]
				\item (~~) O conjunto vazio é elemento de qualquer conjunto.
				\item (~~) O conjunto vazio é subconjunto de qualquer conjunto.
				\item (~~) Todo conjunto é subconjunto de si mesmo.
				\item (~~) $\{1\} \in \{1\}$
				\item (~~) $\{1\} \subset \{1\}$
				\item (~~) $1 \subset \{1\}$
				\item (~~) $1 \in \{1\}$
				\item (~~) $\{\varnothing\} \subset \{1, 2, \varnothing\}$
				\item (~~) $\varnothing \in \{1, 2, \varnothing\}$
				\item (~~) Para quaisquer conjuntos $A$ e $B$, se $A \subset B$ e $B \subset A$, então $A = B$.
				\item (~~) Para quaisquer conjuntos $A$ e $B$, se $A = B$, então $A \subset B$ e $B \subset A$.
				\item (~~) \\ $\{2\} = \left\{1 + 1,3 - 1, \frac{4}{2}, 7 - 6 + 1\right\}$
			\end{enumerate}
			\item Apresente todos os subconjuntos de $A = \{1, 2\}$. \\\\\\\\\\\\\\\\\\\\
			\item Quantos subconjuntos tem o conjunto $A = \{a, b\}$, com $a \neq b$? \\\\\\\\\\\\\\\\\\\\
			\item Apresente todos os subconjuntos de $A = \{1, 2, 3\}$. \\\\\\\\\\
			\item Quantos subconjuntos tem o conjunto $A = \{a, b, c\}$, com $a \neq b$, $a \neq c$ e $b \neq c$? \\\\\\\\\\\\\\\\\\\\
			%10 a 15
			\item (FCC) Sendo $A = \{\varnothing, a, \{b\}\}$, com $\{b\} \neq a \neq b \neq \varnothing$, então:
			\begin{enumerate}[a)]
				\item $\{\varnothing, \{b\}\} \subset A$
				\item $\{\varnothing, b\} \subset A$
				\item $\{\varnothing, \{a\}\} \subset A$
				\item $\{a, b\} \subset A$
				\item $\{\{a\}, \{b\}\} \subset A$
			\end{enumerate}
			\item Escreva todos os subconjuntos do conjunto $A = \{1, 2, \{3\}\}$. \\\\\\\\\\\\\\\\\\\\
			\item Assinale \textbf{V} (verdadeiro) ou \textbf{F} (falso).
			\begin{enumerate}[a)]
				\item (~~) $3 = 3$
				\item (~~) $\{12, 16\} = \{16, 12, 12, 16, 16\}$
				\item (~~) $\{21\} = \{\{21\}\}$
				\item (~~) $14 \not\in \varnothing$
				\item (~~) $\varnothing \in \varnothing$
				\item (~~) $\varnothing \subset \varnothing$
				\item (~~) $10 \in \{2\}$
				\item (~~) $\{17\} \subset \{17\}$
				\item (~~) $\{5, \{17\}\} = \{5, 17\}$
				\item (~~) $\{3, 12\} = \{10 + 13, 9 + 3\}$
			\end{enumerate}
			\item Assinale \textbf{V} (verdadeiro) ou \textbf{F} (falso).
			\begin{enumerate}[a)]
				\item (~~) $\{11, 16\} = \{11, 11, 11, 16, 16\}$
				\item (~~) $\{18\} = \{18, 17 + 1, 16 + 2, 15 + 3, 14 + 4\}$
				\item (~~) $\{7, 5 + 1\} = \{8 - 1, 7 - 2, 3 + 4, 7, 6 + 1\}$
				\item (~~) $\{2, 2, 3\} = \{2, 2, 2, 2, 2 + 3\}$
				\item (~~) $\{0, 10, 100, 1 000\} = \{10, 10 - 10, 10 + 10, 10 + 100\}$
			\end{enumerate}
			\item Escreva todos os subconjuntos de $A = \{1, 3, \{5\}, 9\}$. \\\\\\\\\\\\\\
			\item Quantos subconjuntos tem o conjunto $B = \{1, \{1\}, 2, \{2\}\}$? \\\\\\\\\\\\\\\\
		\end{enumerate}
	\end{multicols}
\end{document}