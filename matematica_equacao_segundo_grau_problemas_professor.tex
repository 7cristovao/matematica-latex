\documentclass[a4paper,14pt]{article}
\usepackage{extsizes}
\usepackage{amsmath}
\usepackage{cancel}
\everymath{\displaystyle}
\usepackage{geometry}
\usepackage{fancyhdr}
\usepackage{multicol}
\usepackage{graphicx}
\usepackage[brazil]{babel}
\usepackage{enumitem}
\columnsep=2cm
\hoffset=0cm
\textwidth=8cm
\setlength{\columnseprule}{.1pt}
\setlength{\columnsep}{2cm}
\renewcommand{\headrulewidth}{0pt}
\geometry{top=1in, bottom=1in, left=1in, right=1in}

\pagestyle{fancy}
\fancyhf{}
\fancyfoot[C]{\thepage}

\begin{document}
	
	\noindent\textbf{EF09MA215-A~-~Matemática} 
	
	\begin{center}
		\textbf{Resolução de problemas de equação quadrática \\(Versão professor)}
	\end{center}
	
	\bigskip
	
	\noindent\textbf{Nome:} \underline{\hspace{10cm}}
	\noindent\textbf{Data:} \underline{\hspace{4cm}}
	
	\bigskip
	
	%	\section*{Questões de Matemática}
	
	\begin{enumerate}
		\item Para construir um jardim, uma pessoa ocupou uma superfície quadrada de $225~m^2$. Qual a medida dos lados desse jardim?
		\\ \\
		$area = lado \cdot lado$ \\ \\
		$x' \cdot x' = 225$ \\
		$x^2 = 225~m^2$ \\
		$x = \sqrt{225}$ \\
		$x = \pm{15} $ \\
		\\
		$(-15) \cdot (-15) = 225$ \\
		$(15) \cdot (15) = 225$ \\
		\\
		$x = 15~m$
		\\
		\vspace{0cm}
		
		\item Uma pessoa vai fazer uma toalha retangular cujo comprimento terá 40cm a mais que a largura. Sabendo que serão utilizados 1200 $cm^2$ de tecido, calcule as dimensões dessa toalha.
		\\ \\
		$x \cdot x + 40x = 1200$ \\
		$x^2 + 40x = 1200$ \\
		$x^2 + 40x - 1200 = 0$ \\
		a = 1; b = 40; c = - 1200 \\
		$\Delta = b^2 - 4 \cdot a \cdot c$ \\
		$\Delta = 40^2 - 4 \cdot 1 \cdot (-1200)$ \\
		$\Delta = 1600 + 4800$ \\
		$\Delta = 6400$ \\
		$x = \frac{-b \pm{\sqrt{\Delta}}}{2 \cdot a}$ \\
		$x = \frac{-40 \pm{\sqrt{6400}}}{2 \cdot 1}$ \\
		$x = \frac{-40 \pm{80}}{2}$ \\
		$x_1 = \frac{-40 + 80}{2} = \frac{40}{2} = 20$ \\
		$x_2 = \frac{-40 - 80}{2} = \frac{-120}{2} = -60$ \\
		Portanto utilizar a raiz positiva que resultou em: \\
		x = 20~cm \\
		Além disso x + 40 = 60~cm \\

		
		\item Uma pessoa vai construir duas casas, uma delas em um terreno retangular e a outra em um terreno quadrado. Sabendo que os terrenos tem a mesma área, quais são as dimensões de cada um?    
		Dados:  \\ 
		Lados do retângulo são (x+8) $\cdot$ m e 4~m. \\ 
		Lados do quadrado são x e x. \\
		\\ 
		$4 \cdot (x+8) = x^2$ \\
		$4x + 32 = x^2$ \\
		$-x^2 +4x +32 = 0$ \\
		a = -1; b = 4; c = 32 \\
		$\Delta = b^2 - 4 \cdot a \cdot c$ \\
		$\Delta = 4^2 - 4 \cdot (-1) \cdot 32$ \\
		$\Delta = 16 + 128$ \\
		$\Delta = 144$ \\
		$x = \frac{-b \pm{\sqrt{\Delta}}}{2 \cdot a}$ \\
		$x = \frac{-4 \pm{\sqrt{144}}}{2 \cdot (-1)}$ \\
		$x = \frac{-4 \pm{12}}{-2}$ \\
		$x_1 = \frac{-4 + 12}{-2} = \frac{8}{-2} = -4$ \\
		$x_2 = \frac{-4 - 12}{-2} = \frac{-16}{-2} = 8$ \\
		\\
		Portanto as dimensões do terreno retangular seriam x + 8 = 16~m \\
		E as dimensões do outro terreno seriam de 8~m por 8~m 
		\vspace{10cm}
	\end{enumerate}
	
\end{document}

